% !Mode:: "TeX:UTF-8"
\documentclass[twoside,nofonts,fancyhdr,openany,UTF8]{ctexbook}
\usepackage{natbib}

% CJK related
\setCJKmainfont[AutoFakeBold=true]{Songti SC}
\setCJKsansfont{Heiti SC}
\setCJKmonofont{FangSong}
\CTEXsetup[format={\raggedright}]{chapter}
\CTEXsetup[format={\Large\bfseries}]{section}
\CTEXsetup[format={\large\bfseries}]{subsection}
\CTEXsetup[format={\normalsize\bfseries}]{subsubsection}

% Needed for some foreign characters
\usepackage[T1]{fontenc}

\usepackage{amsmath}
\usepackage{subfigure}
\usepackage{amsfonts}
\usepackage{amsthm}
\usepackage{multirow}
\usepackage{colortbl}
\usepackage{booktabs}
% This allows us to cite chapters by name, which was useful for making the
% acknowledgements page
\usepackage{nameref}
\usepackage{breakcites}

\usepackage[tocindentauto]{tocstyle}
\usetocstyle{standard}

\usepackage{bm}
\usepackage{float}
\newcommand{\boldindex}[1]{\textbf{\hyperpage{#1}}}
\usepackage{makeidx}\makeindex
% Make bibliography and index appear in table of contents
\usepackage[nottoc]{tocbibind}
\usepackage[font=small]{caption}

\usepackage[section]{placeins}
\usepackage[chapter]{algorithm}
\usepackage{algorithmic}
% Include chapter number in algorithm number
\renewcommand{\thealgorithm}{\arabic{chapter}.\arabic{algorithm}}
\makeatletter
\renewcommand*{\ALG@name}{算法}
\makeatother

\usepackage[pdfpagelabels=true,
pdffitwindow=false,
pdfview=FitH,
pdfstartview=FitH,
pagebackref=true,
breaklinks=true,
colorlinks=false,
bookmarks=true,
hidelinks=true,
plainpages=true]{hyperref}

\usepackage{zref-abspage}
\setcounter{secnumdepth}{3}


% my page
\usepackage[vcentering,dvips]{geometry}
\geometry{papersize={7in,9in},bottom=3pc,top=5pc,left=5pc,right=5pc,bmargin=4.5pc,footskip=18pt,headsep=25pt}
\setlength\emergencystretch{1.5em}

% my command
\newcommand{\firstgls}[1]{\textbf{\gls{#1}}(\glsdesc{#1})}
\newcommand{\firstacr}[1]{\textbf{\gls{#1}}~(\glssymbol{#1})}
\newcommand{\glsacr}[1]{\gls{#1}~(\glssymbol{#1})}
\newcommand{\firstall}[1]{\textbf{\gls{#1}}~(\glsdesc{#1}, \glssymbol{#1})}
\newcommand{\ENNAME}[1]{\text{#1}}
\newcommand{\NUMTEXT}[1]{\text{#1}}
\newcommand{\figref}[1]{图\ref{#1}}
\newcommand{\chapref}[1]{第\ref{#1}章}
\newcommand{\secref}[1]{第\ref{#1}节}
\newcommand{\algref}[1]{算法\ref{#1}}
\newcommand{\eqnref}[1]{式\eqref{#1}}


% Draft
\usepackage{draftwatermark}
\SetWatermarkText{DRAFT}
\SetWatermarkLightness{0.9}
\usepackage{background}
\SetBgContents{仅供学习使用,不得用于商业目的。
\url{https://github.com/exacity/deeplearningbook-chinese}}
\SetBgScale{1}
\SetBgAngle{0}
\SetBgOpacity{1}
\SetBgColor{red}
\SetBgPosition{current page.north}
\SetBgVshift{-0.5cm}

\newif\ifOpenSource
\OpenSourcetrue

% http://tex.stackexchange.com/questions/198140/glossaries-and-custom-section-headings-broken  \glsentrytext!
\usepackage[nomain,acronym,xindy,toc,nopostdot]{glossaries}
\makeglossaries
\usepackage[xindy]{imakeidx}
\makeindex


% symbol and 
% !Mode:: "TeX:UTF-8"
\newcommand{\argmax}{\arg\max}
\newcommand{\argmin}{\arg\min}
\newcommand{\sigmoid}{\text{sigmoid}}
\newcommand{\norm}[1]{\left\lVert#1\right\rVert}
\newcommand{\Tr}{\text{Tr}}

\newcommand{\Var}{\text{Var}}
\newcommand{\Cov}{\text{Cov}}
\newcommand{\plim}{\text{plim}}
\newcommand{\Tsp}{\top}

% Scala
\newcommand{\Sa}{\mathit{a}}
\newcommand{\Sb}{\mathit{b}}
\newcommand{\Sc}{\mathit{c}}
\newcommand{\Sd}{\mathit{d}}
\newcommand{\Se}{\mathit{e}}
\newcommand{\Sf}{\mathit{f}}
\newcommand{\Sg}{\mathit{g}}
\newcommand{\Sh}{\mathit{h}}
\newcommand{\Si}{\mathit{i}}
\newcommand{\Sj}{\mathit{j}}
\newcommand{\Sk}{\mathit{k}}
\newcommand{\Sl}{\mathit{l}}
\newcommand{\Sm}{\mathit{m}}
\newcommand{\Sn}{\mathit{n}}
\newcommand{\So}{\mathit{o}}
\newcommand{\Sp}{\mathit{p}}
\newcommand{\Sq}{\mathit{q}}
\newcommand{\Sr}{\mathit{r}}
\newcommand{\Ss}{\mathit{s}}
\newcommand{\St}{\mathit{t}}
\newcommand{\Su}{\mathit{u}}
\newcommand{\Sv}{\mathit{v}}
\newcommand{\Sw}{\mathit{w}}
\newcommand{\Sx}{\mathit{x}}
\newcommand{\Sy}{\mathit{y}}
\newcommand{\Sz}{\mathit{z}}

\newcommand{\SA}{\mathit{A}}
\newcommand{\SB}{\mathit{B}}
\newcommand{\SC}{\mathit{C}}
\newcommand{\SD}{\mathit{D}}
\newcommand{\SE}{\mathit{E}}
\newcommand{\SF}{\mathit{F}}
\newcommand{\SG}{\mathit{G}}
\newcommand{\SH}{\mathit{H}}
\newcommand{\SJ}{\mathit{J}}
\newcommand{\SK}{\mathit{K}}
\newcommand{\SI}{\mathit{L}}
\newcommand{\SM}{\mathit{M}}
\newcommand{\SN}{\mathit{N}}
\newcommand{\SO}{\mathit{O}}
\newcommand{\SP}{\mathit{P}}
\newcommand{\SQ}{\mathit{Q}}
\newcommand{\SR}{\mathit{R}}
\newcommand{\ST}{\mathit{T}}
\newcommand{\SU}{\mathit{U}}
\newcommand{\SV}{\mathit{V}}
\newcommand{\SW}{\mathit{W}}
\newcommand{\SX}{\mathit{X}}
\newcommand{\SY}{\mathit{Y}}
\newcommand{\SZ}{\mathit{Z}}



% Vector
\newcommand{\Va}{\boldsymbol{\mathit{a}}}
\newcommand{\Vb}{\boldsymbol{\mathit{b}}}
\newcommand{\Vc}{\boldsymbol{\mathit{c}}}
\newcommand{\Vd}{\boldsymbol{\mathit{d}}}
\newcommand{\Ve}{\boldsymbol{\mathit{e}}}
\newcommand{\Vf}{\boldsymbol{\mathit{f}}}
\newcommand{\Vg}{\boldsymbol{\mathit{g}}}
\newcommand{\Vh}{\boldsymbol{\mathit{h}}}
\newcommand{\Vi}{\boldsymbol{\mathit{i}}}
\newcommand{\Vj}{\boldsymbol{\mathit{j}}}
\newcommand{\Vk}{\boldsymbol{\mathit{k}}}
\newcommand{\Vl}{\boldsymbol{\mathit{l}}}
\newcommand{\Vm}{\boldsymbol{\mathit{m}}}
\newcommand{\Vn}{\boldsymbol{\mathit{n}}}
\newcommand{\Vo}{\boldsymbol{\mathit{o}}}
\newcommand{\Vp}{\boldsymbol{\mathit{p}}}
\newcommand{\Vq}{\boldsymbol{\mathit{q}}}
\newcommand{\Vr}{\boldsymbol{\mathit{r}}}
\newcommand{\Vs}{\boldsymbol{\mathit{s}}}
\newcommand{\Vt}{\boldsymbol{\mathit{t}}}
\newcommand{\Vu}{\boldsymbol{\mathit{u}}}
\newcommand{\Vv}{\boldsymbol{\mathit{v}}}
\newcommand{\Vw}{\boldsymbol{\mathit{w}}}
\newcommand{\Vx}{\boldsymbol{\mathit{x}}}
\newcommand{\Vy}{\boldsymbol{\mathit{y}}}
\newcommand{\Vz}{\boldsymbol{\mathit{z}}}

% Matrix
\newcommand{\MA}{\boldsymbol{\mathit{A}}}
\newcommand{\MB}{\boldsymbol{\mathit{B}}}
\newcommand{\MC}{\boldsymbol{\mathit{C}}}
\newcommand{\MD}{\boldsymbol{\mathit{D}}}
\newcommand{\ME}{\boldsymbol{\mathit{E}}}
\newcommand{\MF}{\boldsymbol{\mathit{F}}}
\newcommand{\MG}{\boldsymbol{\mathit{G}}}
\newcommand{\MH}{\boldsymbol{\mathit{H}}}
\newcommand{\MI}{\boldsymbol{\mathit{I}}}
\newcommand{\MJ}{\boldsymbol{\mathit{J}}}
\newcommand{\MK}{\boldsymbol{\mathit{K}}}
\newcommand{\ML}{\boldsymbol{\mathit{L}}}
\newcommand{\MM}{\boldsymbol{\mathit{M}}}
\newcommand{\MN}{\boldsymbol{\mathit{N}}}
\newcommand{\MO}{\boldsymbol{\mathit{O}}}
\newcommand{\MP}{\boldsymbol{\mathit{P}}}
\newcommand{\MQ}{\boldsymbol{\mathit{Q}}}
\newcommand{\MR}{\boldsymbol{\mathit{R}}}
\newcommand{\MS}{\boldsymbol{\mathit{S}}}
\newcommand{\MT}{\boldsymbol{\mathit{T}}}
\newcommand{\MU}{\boldsymbol{\mathit{U}}}
\newcommand{\MV}{\boldsymbol{\mathit{V}}}
\newcommand{\MW}{\boldsymbol{\mathit{W}}}
\newcommand{\MX}{\boldsymbol{\mathit{X}}}
\newcommand{\MY}{\boldsymbol{\mathit{Y}}}
\newcommand{\MZ}{\boldsymbol{\mathit{Z}}}


%Tensor
\newcommand{\TSA}{\textsf{\textbf{A}}}
\newcommand{\TSB}{\textsf{\textbf{B}}}
\newcommand{\TSC}{\textsf{\textbf{C}}}
\newcommand{\TSD}{\textsf{\textbf{D}}}
\newcommand{\TSE}{\textsf{\textbf{E}}}
\newcommand{\TSF}{\textsf{\textbf{F}}}
\newcommand{\TSG}{\textsf{\textbf{G}}}
\newcommand{\TSH}{\textsf{\textbf{H}}}
\newcommand{\TSI}{\textsf{\textbf{I}}}
\newcommand{\TSJ}{\textsf{\textbf{J}}}
\newcommand{\TSK}{\textsf{\textbf{K}}}
\newcommand{\TSL}{\textsf{\textbf{L}}}
\newcommand{\TSM}{\textsf{\textbf{M}}}
\newcommand{\TSN}{\textsf{\textbf{N}}}
\newcommand{\TSO}{\textsf{\textbf{O}}}
\newcommand{\TSP}{\textsf{\textbf{P}}}
\newcommand{\TSQ}{\textsf{\textbf{Q}}}
\newcommand{\TSR}{\textsf{\textbf{R}}}
\newcommand{\TSS}{\textsf{\textbf{S}}}
\newcommand{\TST}{\textsf{\textbf{T}}}
\newcommand{\TSU}{\textsf{\textbf{U}}}
\newcommand{\TSV}{\textsf{\textbf{V}}}
\newcommand{\TSW}{\textsf{\textbf{W}}}
\newcommand{\TSX}{\textsf{\textbf{X}}}
\newcommand{\TSY}{\textsf{\textbf{Y}}}
\newcommand{\TSZ}{\textsf{\textbf{Z}}}

% Tensor Element
\newcommand{\TEA}{\textit{\textsf{A}}}
\newcommand{\TEB}{\textit{\textsf{B}}}
\newcommand{\TEC}{\textit{\textsf{C}}}
\newcommand{\TED}{\textit{\textsf{D}}}
\newcommand{\TEE}{\textit{\textsf{E}}}
\newcommand{\TEF}{\textit{\textsf{F}}}
\newcommand{\TEG}{\textit{\textsf{G}}}
\newcommand{\TEH}{\textit{\textsf{H}}}
\newcommand{\TEI}{\textit{\textsf{I}}}
\newcommand{\TEJ}{\textit{\textsf{J}}}
\newcommand{\TEK}{\textit{\textsf{K}}}
\newcommand{\TEL}{\textit{\textsf{L}}}
\newcommand{\TEM}{\textit{\textsf{M}}}
\newcommand{\TEN}{\textit{\textsf{N}}}
\newcommand{\TEO}{\textit{\textsf{O}}}
\newcommand{\TEP}{\textit{\textsf{P}}}
\newcommand{\TEQ}{\textit{\textsf{Q}}}
\newcommand{\TER}{\textit{\textsf{R}}}
\newcommand{\TES}{\textit{\textsf{S}}}
\newcommand{\TET}{\textit{\textsf{T}}}
\newcommand{\TEU}{\textit{\textsf{U}}}
\newcommand{\TEV}{\textit{\textsf{V}}}
\newcommand{\TEW}{\textit{\textsf{W}}}
\newcommand{\TEX}{\textit{\textsf{X}}}
\newcommand{\TEY}{\textit{\textsf{Y}}}
\newcommand{\TEZ}{\textit{\textsf{Z}}}

% Random Scala
\newcommand{\RSa}{\mathrm{a}}
\newcommand{\RSb}{\mathrm{b}}
\newcommand{\RSc}{\mathrm{c}}
\newcommand{\RSd}{\mathrm{d}}
\newcommand{\RSe}{\mathrm{e}}
\newcommand{\RSf}{\mathrm{f}}
\newcommand{\RSg}{\mathrm{g}}
\newcommand{\RSh}{\mathrm{h}}
\newcommand{\RSi}{\mathrm{i}}
\newcommand{\RSj}{\mathrm{j}}
\newcommand{\RSk}{\mathrm{k}}
\newcommand{\RSl}{\mathrm{l}}
\newcommand{\RSm}{\mathrm{m}}
\newcommand{\RSn}{\mathrm{n}}
\newcommand{\RSo}{\mathrm{o}}
\newcommand{\RSp}{\mathrm{p}}
\newcommand{\RSq}{\mathrm{q}}
\newcommand{\RSr}{\mathrm{r}}
\newcommand{\RSs}{\mathrm{s}}
\newcommand{\RSt}{\mathrm{t}}
\newcommand{\RSu}{\mathrm{u}}
\newcommand{\RSv}{\mathrm{v}}
\newcommand{\RSw}{\mathrm{w}}
\newcommand{\RSx}{\mathrm{x}}
\newcommand{\RSy}{\mathrm{y}}
\newcommand{\RSz}{\mathrm{z}}


% Random Vector
\newcommand{\RVa}{\mathbf{a}}
\newcommand{\RVb}{\mathbf{b}}
\newcommand{\RVc}{\mathbf{c}}
\newcommand{\RVd}{\mathbf{d}}
\newcommand{\RVe}{\mathbf{e}}
\newcommand{\RVf}{\mathbf{f}}
\newcommand{\RVg}{\mathbf{g}}
\newcommand{\RVh}{\mathbf{h}}
\newcommand{\RVi}{\mathbf{i}}
\newcommand{\RVj}{\mathbf{j}}
\newcommand{\RVk}{\mathbf{k}}
\newcommand{\RVl}{\mathbf{l}}
\newcommand{\RVm}{\mathbf{m}}
\newcommand{\RVn}{\mathbf{n}}
\newcommand{\RVo}{\mathbf{o}}
\newcommand{\RVp}{\mathbf{p}}
\newcommand{\RVq}{\mathbf{q}}
\newcommand{\RVr}{\mathbf{r}}
\newcommand{\RVs}{\mathbf{s}}
\newcommand{\RVt}{\mathbf{t}}
\newcommand{\RVu}{\mathbf{u}}
\newcommand{\RVv}{\mathbf{v}}
\newcommand{\RVw}{\mathbf{w}}
\newcommand{\RVx}{\mathbf{x}}
\newcommand{\RVy}{\mathbf{y}}
\newcommand{\RVz}{\mathbf{z}}

% Random Matrix
% will be added later
\newcommand{\RMX}{\boldsymbol{\mathrm{X}}}
\newcommand{\RMA}{\boldsymbol{\mathrm{A}}}

\newcommand{\Valpha}{\boldsymbol{\alpha}}
\newcommand{\Vbeta}{\boldsymbol{\beta}}
\newcommand{\Vtheta}{\boldsymbol{\theta}}
\newcommand{\Vlambda}{\boldsymbol{\lambda}}
\newcommand{\VLambda}{\boldsymbol{\Lambda}}
\newcommand{\Vepsilon}{\boldsymbol{\epsilon}}
\newcommand{\Vmu}{\boldsymbol{\mu}}
\newcommand{\VPhi}{\boldsymbol{\Phi}}
\newcommand{\Vsigma}{\boldsymbol{\sigma}}
\newcommand{\VSigma}{\boldsymbol{\Sigma}}
\newcommand{\Vrho}{\boldsymbol{\rho}}
\newcommand{\Vgamma}{\boldsymbol{\gamma}}
\newcommand{\Vomega}{\boldsymbol{\omega}}
\newcommand{\Vpsi}{\boldsymbol{\psi}}
\newcommand{\Vzeta}{\boldsymbol{\zeta}}
\newcommand{\Vone}{\boldsymbol{1}}


\newcommand{\CalB}{\mathcal{B}}
\newcommand{\CalC}{\mathcal{C}}
\newcommand{\CalG}{\mathcal{G}}
\newcommand{\CalH}{\mathcal{H}}
\newcommand{\CalL}{\mathcal{L}}
\newcommand{\CalM}{\mathcal{M}}
\newcommand{\CalN}{\mathcal{N}}
\newcommand{\CalO}{\mathcal{O}}
\newcommand{\CalD}{\mathcal{D}}
\newcommand{\CalU}{\mathcal{U}}
\newcommand{\CalF}{\mathcal{F}}
\newcommand{\CalT}{\mathcal{T}}

% Set
\newcommand{\SetA}{\mathbb{A}}
\newcommand{\SetB}{\mathbb{B}}
\newcommand{\SetD}{\mathbb{D}}
\newcommand{\SetE}{\mathbb{E}}
\newcommand{\SetG}{\mathbb{G}}
\newcommand{\SetL}{\mathbb{L}}
\newcommand{\SetN}{\mathbb{N}}
\newcommand{\SetR}{\mathbb{R}}
\newcommand{\SetS}{\mathbb{S}}
\newcommand{\SetT}{\mathbb{T}}
\newcommand{\SetV}{\mathbb{V}}
\newcommand{\SetX}{\mathbb{X}}
\newcommand{\SetY}{\mathbb{Y}}

% !Mode:: "TeX:UTF-8"
\newglossaryentry{DL}
{
  name=深度学习,
  description={deep learning},
  sort={deep learning},
}

\newglossaryentry{knowledge_base}
{
  name=知识库,
  description={knowledge base},
  sort={knowledge base},
}

\newglossaryentry{ML}
{
  name=机器学习,
  description={machine learning},
  sort={machine learning},
}

\newglossaryentry{ML_model}
{
  name=机器学习模型,
  description={machine learning model},
  sort={machine learning model},
}

\newglossaryentry{ebv}
{
  name=基本单位向量,
  description={elementary basis vectors},
  sort={elementary basis vectors},
}

\newglossaryentry{logistic_regression}
{
  name=逻辑回归,
  description={logistic regression},
  sort={logistic regression},
}

\newglossaryentry{regression}
{
  name=回归,
  description={regression},
  sort={regression},
}

\newglossaryentry{AI}
{
  name=人工智能,
  description={artificial intelligence},
  sort={artificial intelligence},
  symbol={AI}
}

\newglossaryentry{naive_bayes}
{
  name=朴素贝叶斯,
  description={naive Bayes},
  sort={naive Bayes},
}

\newglossaryentry{representation}
{
  name=表示,
  description={representation},
  sort={representation},
}

\newglossaryentry{representation_learning}
{
  name=表示学习,
  description={representation learning},
  sort={representation learning},
}

\newglossaryentry{AE}
{
  name=自编码器,
  description={autoencoder},
  sort={autoencoder},
}

\newglossaryentry{encoder}
{
  name=编码器,
  description={encoder},
  sort={encoder},
}

\newglossaryentry{decoder}
{
  name=解码器,
  description={decoder},
  sort={decoder},
}

\newglossaryentry{MLP}
{
  name=多层感知机,
  description={multilayer perceptron},
  sort={multilayer perceptron},
  symbol={MLP}
}

\newglossaryentry{cybernetics}
{
  name=控制论,
  description={cybernetics},
  sort={cybernetics},
}

\newglossaryentry{connectionism}
{
  name=联结主义,
  description={connectionism},
  sort={connectionism},
}

\newglossaryentry{ANN}
{
  name=人工神经网络,
  description={artificial neural network},
  sort={artificial neural network},
  symbol={ANN}
}

\newglossaryentry{NN}
{
  name=神经网络,
  description={neural network},
  sort={neural network},
}

\newglossaryentry{SGD}
{
  name=随机梯度下降,
  description={stochastic gradient descent},
  sort={stochastic gradient descent},
  symbol={SGD}
}

\newglossaryentry{linear_model}
{
  name=线性模型,
  description={linear model},
  sort={linear model},
}

\newglossaryentry{mode}
{
  name=峰值,
  description={mode},
  sort={mode},
}

\newglossaryentry{unimodal}
{
  name=单峰值,
  description={unimodal},
  sort={unimodal},
}

\newglossaryentry{modality}
{
  name=模态,
  description={modality},
  sort={modality},
}

\newglossaryentry{multimodal}
{
  name=多峰值,
  description={multimodal},
  sort={multimodal},
}

\newglossaryentry{linear_regression}
{
  name=线性回归,
  description={linear regression},
  sort={linear regression},
}

\newglossaryentry{ReLU}
{
  name=整流线性单元,
  description={rectified linear unit},
  sort={rectified linear unit},
  symbol={ReLU}
}

\newglossaryentry{distributed_representation}
{
  name=分布式表示,
  description={distributed representation},
  sort={distributed representation},
}

\newglossaryentry{nondistributed_representation}
{
  name=非分布式表示,
  description={nondistributed representation},
  sort={nondistributed representation},
}

\newglossaryentry{nondistributed}
{
  name=非分布式,
  description={nondistributed},
  sort={nondistributed},
}

\newglossaryentry{hidden_unit}
{
  name=隐藏单元,
  description={hidden unit},
  sort={hidden unit},
}

\newglossaryentry{LSTM}
{
  name=长短期记忆,
  description={long short-term memory},
  sort={long short-term memory},
  symbol={LSTM}
}

\newglossaryentry{DBN}
{
  name=深度信念网络,
  description={deep belief network},
  sort={deep belief network},
  symbol={DBN}
}

\newglossaryentry{RNN}
{
  name=循环神经网络,
  description={recurrent neural network},
  sort={recurrent neural network},
  symbol={RNN}
}

\newglossaryentry{recurrence}
{
  name=循环,
  description={recurrence},
  sort={recurrence},
}

\newglossaryentry{RL}
{
  name=强化学习,
  description={reinforcement learning},
  sort={reinforcement learning},
}

\newglossaryentry{inference}
{
  name=推断,
  description={inference},
  sort={inference},
}

\newglossaryentry{overflow}
{
  name=上溢,
  description={overflow},
  sort={overflow},
}

\newglossaryentry{underflow}
{
  name=下溢,
  description={underflow},
  sort={underflow},
}

\newglossaryentry{softmax}
{
  name=softmax函数,
  description={softmax function},
  sort={softmax function},
}

\newglossaryentry{softmax_chap15}
{
  name=softmax,
  description={softmax},
  sort={softmax},
}

\newglossaryentry{underestimation}
{
  name=欠估计,
  description={underestimation},
  sort={underestimation},
}

\newglossaryentry{overestimation}
{
  name=过估计,
  description={overestimation},
  sort={overestimation},
}

\newglossaryentry{softmax_unit}
{
  name=softmax单元,
  description={softmax unit},
  sort={softmax unit},
}

\newglossaryentry{softmax_chap11}
{
  name=softmax,
  description={softmax},
  sort={softmax},
}

\newglossaryentry{multinoulli}
{
  name=Multinoulli分布,
  description={multinoulli distribution},
  sort={multinoulli distribution},
}

\newglossaryentry{poor_conditioning}
{
  name=病态条件,
  description={poor conditioning},
  sort={poor conditioning},
}

\newglossaryentry{objective_function}
{
  name=目标函数,
  description={objective function},
  sort={objective function},
}

\newglossaryentry{objective}
{
	name=目标,
	description={objective},
	sort={objective},
}

\newglossaryentry{criterion}
{
  name=准则,
  description={criterion},
  sort={criterion},
}

\newglossaryentry{cost_function}
{
  name=代价函数,
  description={cost function},
  sort={cost function},
}

\newglossaryentry{cost}
{
  name=代价,
  description={cost},
  sort={cost},
}

\newglossaryentry{loss_function}
{
  name=损失函数,
  description={loss function},
  sort={loss function},
}

\newglossaryentry{prcurve}
{
  name=PR曲线,
  description={PR curve},
  sort={PR curve},
}

\newglossaryentry{fscore}
{
  name=F分数,
  description={F-score},
  sort={F-score},
}

\newglossaryentry{loss}
{
  name=损失,
  description={loss},
  sort={loss},
}

\newglossaryentry{error_function}
{
  name=误差函数,
  description={error function},
  sort={error function},
}

\newglossaryentry{GD}
{
  name=梯度下降,
  description={gradient descent},
  sort={gradient descent},
}

\newglossaryentry{local_descent}
{
  name=局部下降,
  description={local descent},
  sort={local descent},
}

\newglossaryentry{steepest}
{
  name=最陡下降,
  description={steepest descent},
  sort={steepest descent},
}

\newglossaryentry{GA}
{
  name=梯度上升,
  description={gradient ascent},
  sort={gradient ascent},
}

\newglossaryentry{derivative}
{
  name=导数,
  description={derivative},
  sort={derivative},
}

\newglossaryentry{critical_points}
{
  name=临界点,
  description={critical point},
  sort={critical point},
}

\newglossaryentry{stationary_point}
{
  name=驻点,
  description={stationary point},
  sort={stationary point},
}

\newglossaryentry{local_minimum}
{
  name=局部极小点,
  description={local minimum},
  sort={local minimum},
}

\newglossaryentry{minimum}
{
  name=极小点,
  description={minimum},
  sort={minimum},
}

\newglossaryentry{local_minima}
{
  name=局部极小值,
  description={local minima},
  sort={local minima},
}

\newglossaryentry{minima}
{
  name=极小值,
  description={minima},
  sort={minima},
}

\newglossaryentry{global_minima}
{
  name=全局极小值,
  description={global minima},
  sort={global minima},
}

\newglossaryentry{local_maxima}
{
  name=局部极大值,
  description={local maxima},
  sort={local maxima},
}

\newglossaryentry{maxima}
{
  name=极大值,
  description={maxima},
  sort={maxima},
}

\newglossaryentry{local_maximum}
{
  name=局部极大点,
  description={local maximum},
  sort={local maximum},
}

\newglossaryentry{saddle_points}
{
  name=鞍点,
  description={saddle point},
  sort={saddle point},
}

\newglossaryentry{global_minimum}
{
  name=全局最小点,
  description={global minimum},
  sort={global minimum},
}

\newglossaryentry{partial_derivatives}
{
  name=偏导数,
  description={partial derivative},
  sort={partial derivative},
}

\newglossaryentry{gradient}
{
  name=梯度,
  description={gradient},
  sort={gradient},
}

\newglossaryentry{identifiable}
{
  name=可辨认的,
  description={identifiable},
  sort={identifiable},
}

\newglossaryentry{directional_derivative}
{
  name=方向导数,
  description={directional derivative},
  sort={directional derivative},
}

\newglossaryentry{line_search}
{
  name=线搜索,
  description={line search},
  sort={line search},
}

\newglossaryentry{example}
{
  name=样本,
  description={example},
  sort={example},
}

\newglossaryentry{hill_climbing}
{
  name=爬山,
  description={hill climbing},
  sort={hill climbing},
}

\newglossaryentry{ill_conditioning}
{
  name=病态,
  description={ill conditioning},
  sort={ill conditioning},
}

\newglossaryentry{jacobian}
{
  name=Jacobian,
  description={Jacobian},
  sort={Jacobian},
}

\newglossaryentry{hessian}
{
  name=Hessian,
  description={Hessian},
  sort={Hessian},
}

\newglossaryentry{second_derivative}
{
  name=二阶导数,
  description={second derivative},
  sort={second derivative},
}

\newglossaryentry{curvature}
{
  name=曲率,
  description={curvature},
  sort={curvature},
}

\newglossaryentry{taylor}
{
  name=泰勒,
  description={taylor},
  sort={taylor},
}

\newglossaryentry{second_derivative_test}
{
  name=二阶导数测试,
  description={second derivative test},
  sort={second derivative test},
}

\newglossaryentry{newton_method}
{
  name=牛顿法,
  description={Newton's method},
  sort={Newton's method},
}

\newglossaryentry{second_order_method}
{
  name=二阶方法,
  description={second-order method},
  sort={second-order method},
}

\newglossaryentry{first_order_method}
{
  name=一阶方法,
  description={first-order method},
  sort={first-order method},
}

\newglossaryentry{lipschitz}
{
  name=Lipschitz,
  description={Lipschitz},
  sort={Lipschitz},
}

\newglossaryentry{lipschitz_continuous}
{
  name=Lipschitz连续,
  description={Lipschitz continuous},
  sort={Lipschitz continuous},
}

\newglossaryentry{lipschitz_constant}
{
  name=Lipschitz常数,
  description={Lipschitz constant},
  sort={Lipschitz constant},
}

\newglossaryentry{convex_optimization}
{
  name=凸优化,
  description={Convex optimization},
  sort={Convex optimization},
}

\newglossaryentry{nonconvex}
{
  name=非凸,
  description={nonconvex},
  sort={nonconvex},
}

\newglossaryentry{nume_optimization}
{
  name=数值优化,
  description={numerical optimization},
  sort={numerical optimization},
}

\newglossaryentry{constrained_optimization}
{
  name=约束优化,
  description={constrained optimization},
  sort={constrained optimization},
}

\newglossaryentry{feasible}
{
  name=可行,
  description={feasible},
  sort={feasible},
}

\newglossaryentry{KKT}
{
  name=Karush–Kuhn–Tucker,
  description={Karush–Kuhn–Tucker},
  sort={Karush–Kuhn–Tucker},
  symbol={KKT}
}

\newglossaryentry{generalized_lagrangian}
{
  name=广义Lagrangian,
  description={generalized Lagrangian},
  sort={generalized Lagrangian},
}

\newglossaryentry{generalized_lagrange_function}
{
  name=广义Lagrange函数,
  description={generalized Lagrange function},
  sort={generalized Lagrange function},
}

\newglossaryentry{equality_constraints}
{
  name=等式约束,
  description={equality constraint},
  sort={equality constraint},
}

\newglossaryentry{inequality_constraints}
{
  name=不等式约束,
  description={inequality constraint},
  sort={inequality constraint},
}

\newglossaryentry{regularization}
{
  name=正则化,
  description={regularization},
  sort={regularization},
}

\newglossaryentry{regularizer}
{
  name=正则化项,
  description={regularizer},
  sort={regularizer},
}

\newglossaryentry{regularize}
{
  name=正则化,
  description={regularize},
  sort={regularize},
}

\newglossaryentry{generalization}
{
  name=泛化,
  description={generalization},
  sort={generalization},
}

\newglossaryentry{generalize}
{
  name=泛化,
  description={generalize},
  sort={generalize},
}

\newglossaryentry{underfitting}
{
  name=欠拟合,
  description={underfitting},
  sort={underfitting},
}

\newglossaryentry{overfitting}
{
  name=过拟合,
  description={overfitting},
  sort={overfitting},
}

\newglossaryentry{bias_sta}
{
  name=偏差,
  description={bias in statistics},
  sort={bias in statistics},
}

\newglossaryentry{BIAS}
{
  name=偏差,
  description={biass},
  sort={biass},
}

\newglossaryentry{bias_aff}
{
  name=偏置,
  description={bias in affine function},
  sort={bias in affine function},
}

\newglossaryentry{variance}
{
  name=方差,
  description={variance},
  sort={variance},
}

\newglossaryentry{ensemble}
{
  name=集成,
  description={ensemble},
  sort={ensemble},
}

\newglossaryentry{estimator}
{
  name=估计,
  description={estimator},
  sort={estimator},
}

\newglossaryentry{weight_decay}
{
  name=权重衰减,
  description={weight decay},
  sort={weight decay},
}

\newglossaryentry{ridge_regression}
{
  name=岭回归,
  description={ridge regression},
  sort={ridge regression},
}

\newglossaryentry{tikhonov_regularization}
{
  name=Tikhonov正则,
  description={Tikhonov regularization},
  sort={Tikhonov regularization},
}

\newglossaryentry{covariance}
{
  name=协方差,
  description={covariance},
  sort={covariance},
}

\newglossaryentry{sparse}
{
  name=稀疏,
  description={sparse},
  sort={sparse},
}

\newglossaryentry{feature_selection}
{
  name=特征选择,
  description={feature selection},
  sort={feature selection},
}

\newglossaryentry{feature_extractor}
{
  name=特征提取器,
  description={feature extractor},
  sort={feature extractor},
}

\newglossaryentry{MAP}
{
  name=最大后验,
  description={Maximum A Posteriori},
  sort={Maximum A Posteriori},
  symbol={MAP}
}

\newglossaryentry{pooling}
{
  name=池化,
  description={pooling},
  sort={pooling},
}

\newglossaryentry{dropout}
{
  name=Dropout,
  description={Dropout},
  sort={dropout},
}

\newglossaryentry{monte_carlo}
{
  name=蒙特卡罗,
  description={Monte Carlo},
  sort={Monte Carlo},
}

\newglossaryentry{early_stopping}
{
  name=提前终止,
  description={early stopping},
  sort={early stopping},
}

\newglossaryentry{CNN}
{
  name=卷积神经网络,
  description={convolutional neural network},
  sort={convolutional neural network},
  symbol={CNN}
}

\newglossaryentry{mcmc}
{
  name=马尔可夫链蒙特卡罗,
  description={Markov Chain Monte Carlo},
  symbol={MCMC},
  sort={Markov Chain Monte Carlo},
}

\newglossaryentry{tempering_transition}
{
  name=回火转移,
  description={tempered transition},
  sort={tempered transition},
}

\newglossaryentry{markov_chain}
{
  name=马尔可夫链,
  description={Markov Chain},
  sort={Markov Chain},
}

\newglossaryentry{harris_chain}
{
  name=哈里斯链,
  description={Harris Chain},
  sort={Harris Chain},
}

\newglossaryentry{minibatch}
{
  name=小批量,
  description={minibatch},
  sort={minibatch},
}

\newglossaryentry{importance_sampling}
{
  name=重要采样,
  description={Importance Sampling},
  sort={Importance Sampling},
}

\newglossaryentry{undirected_model}
{
  name=无向模型,
  description={undirected Model},
  sort={undirected Model},
}

\newglossaryentry{partition_function}
{
  name=配分函数,
  description={Partition Function},
  sort={Partition Function},
}

\newglossaryentry{law_of_large_numbers}
{
  name=大数定理,
  description={Law of large number},
  sort={Law of large number},
}

\newglossaryentry{central_limit_theorem}
{
  name=中心极限定理,
  description={central limit theorem},
  sort={central limit theorem},
}

\newglossaryentry{energy_based_model}
{
  name=基于能量的模型,
  description={Energy-based model},
  symbol={EBM},
  sort={Energy-based model},
}

\newglossaryentry{tempering}
{
  name=回火,
  description={tempering},
  sort={tempering},
}

\newglossaryentry{biased_importance_sampling}
{
  name=有偏重要采样,
  description={biased importance sampling},
  sort={biased importance sampling},
}

\newglossaryentry{VAE}
{
  name=变分自编码器,
  description={variational auto-encoder},
  sort={variational auto-encoder},
  symbol={VAE},
}

\newglossaryentry{CV}
{
  name=计算机视觉,
  description={Computer Vision},
  sort={Computer Vision},
}

\newglossaryentry{SR}
{
  name=语音识别,
  description={Speech Recognition},
  sort={Speech Recognition},
}

\newglossaryentry{NLP}
{
  name=自然语言处理,
  description={Natural Language Processing},
  sort={Natural Language Processing},
  symbol={NLP}
}

\newglossaryentry{RBM}
{
  name=受限玻尔兹曼机,
  description={Restricted Boltzmann Machine},
  sort={Restricted Boltzmann Machine},
  symbol={RBM}
}

\newglossaryentry{discriminative_RBM}
{
  name=判别RBM,
  description={discriminative RBM},
  sort={discriminative RBM},
}

\newglossaryentry{Boltzmann}
{
  name=玻尔兹曼,
  description={Boltzmann},
  sort={Boltzmann},
}

\newglossaryentry{BM}
{
  name=玻尔兹曼机,
  description={Boltzmann Machine},
  sort={Boltzmann Machine},
}

\newglossaryentry{DBM}
{
  name=深度玻尔兹曼机,
  description={Deep Boltzmann Machine},
  sort={Deep Boltzmann Machine},
  symbol={DBM}
}

\newglossaryentry{CBM}
{
  name=卷积玻尔兹曼机,
  description={Convolutional Boltzmann Machine},
  sort={Convolutional Boltzmann Machine},
  symbol={CBM}
}

\newglossaryentry{directed_model}
{
  name=有向模型,
  description={Directed Model},
  sort={Directed Model},
}

\newglossaryentry{ancestral_sampling}
{
  name=原始采样,
  description={Ancestral Sampling},
  sort={Ancestral Sampling},
}

\newglossaryentry{stochastic_matrix}
{
  name=随机矩阵,
  description={Stochastic Matrix},
  sort={Stochastic Matrix},
}

\newglossaryentry{stationary_distribution}
{
  name=平稳分布,
  description={Stationary Distribution},
  sort={Stationary Distribution},
}

\newglossaryentry{equilibrium_distribution}
{
  name=均衡分布,
  description={Equilibrium Distribution},
  sort={Equilibrium Distribution},
}

\newglossaryentry{index}
{
  name=索引,
  description={index of matrix},
  sort={index of matrix},
}

\newglossaryentry{burn_in}
{
  name=磨合,
  description={Burning-in},
  sort={Burning-in},
}

\newglossaryentry{mixing_time}
{
  name=混合时间,
  description={Mixing Time},
  sort={Mixing Time},
}

\newglossaryentry{mixing}
{
  name=混合,
  description={Mixing},
  sort={Mixing},
}

\newglossaryentry{gibbs_sampling}
{
  name=Gibbs采样,
  description={Gibbs Sampling},
  sort={Gibbs Sampling},
}

\newglossaryentry{block_gibbs_sampling}
{
  name=块吉布斯采样,
  description={block Gibbs Sampling},
  sort={block Gibbs Sampling},
}

\newglossaryentry{gibbs_steps}
{
  name=吉布斯步数,
  description={Gibbs steps},
  sort={Gibbs steps},
}

\newglossaryentry{bagging}
{
  name=Bagging,
  description={bootstrap aggregating},
  sort={bagging},
}

\newglossaryentry{mask}
{
  name=掩码,
  description={mask},
  sort={mask},
}

\newglossaryentry{batch_normalization}
{
  name=批标准化,
  description={batch normalization},
  sort={batch normalization},
}

\newglossaryentry{Batch_normalization}
{
  name=批标准化,
  description={Batch normalization},
  sort={Batch normalization},
}

\newglossaryentry{parameter_sharing}
{
  name=参数共享,
  description={parameter sharing},
  sort={parameter sharing},
}

\newglossaryentry{KL}
{
  name=KL散度,
  description={KL divergence},
  sort={KL},
}

\newglossaryentry{temperature}
{
  name=温度,
  description={temperature},
  sort={temperature},
}

\newglossaryentry{critical_temperatures}
{
  name=临界温度,
  description={critical temperatures},
  sort={critical temperatures},
}

\newglossaryentry{parallel_tempering}
{
  name=并行回火,
  description={parallel tempering},
  sort={parallel tempering},
}

\newglossaryentry{ASR}
{
  name=自动语音识别,
  description={Automatic Speech Recognition},
  sort={Automatic Speech Recognition},
  symbol={ASR}
}

\newglossaryentry{GP_GPU}
{
  name=通用GPU,
  description={general purpose GPU},
  sort={general purpose GPU},
}

\newglossaryentry{coalesced}
{
  name=级联,
  description={coalesced},
  sort={coalesced},
}

\newglossaryentry{warp}
{
  name=warp,
  description={warp},
  sort={warp},
}

\newglossaryentry{data_parallelism}
{
  name=数据并行,
  description={data parallelism},
  sort={data parallelism},
}

\newglossaryentry{model_parallelism}
{
  name=模型并行,
  description={model parallelism},
  sort={model parallelism},
}

\newglossaryentry{ASGD}
{
  name=异步随机梯度下降,
  description={Asynchoronous Stochastic Gradient Descent},
  sort={Asynchoronous Stochastic Gradient Descent},
}

\newglossaryentry{parameter_server}
{
  name=参数服务器,
  description={parameter server},
  sort={parameter server},
}

\newglossaryentry{model_compression}
{
  name=模型压缩,
  description={model compression},
  sort={model compression},
}

\newglossaryentry{dynamic_structure}
{
  name=动态结构,
  description={dynamic structure},
  sort={dynamic structure},
}

\newglossaryentry{conditional_computation}
{
  name=条件计算,
  description={conditional computation},
  sort={conditional computation},
}

\newglossaryentry{sphering}
{
  name=sphering,
  description={sphering},
  sort={sphering},
}

\newglossaryentry{GCN}
{
  name=全局对比度归一化,
  description={Global contrast normalization},
  sort={Global contrast normalization},
  symbol={GCN}
}

\newglossaryentry{LCN}
{
  name=局部对比度归一化,
  description={local contrast normalization},
  symbol={LCN},
  sort={local contrast normalization},
}

\newglossaryentry{HMM}
{
  name=隐马尔可夫模型,
  description={Hidden Markov Model},
  sort={Hidden Markov Model},
  symbol={HMM}
}

\newglossaryentry{GMM}
{
  name=高斯混合模型,
  description={Gaussian Mixture Model},
  sort={Gaussian Mixture Model},
  symbol={GMM}
}

\newglossaryentry{transcribe}
{
  name=转录,
  description={transcribe},
  sort={transcribe},
}

\newglossaryentry{PCA}
{
  name=主成分分析,
  description={principal components analysis},
  sort={principal components analysis},
  symbol={PCA}
}

\newglossaryentry{FA}
{
  name=因子分析,
  description={factor analysis},
  sort={factor analysis},
}

\newglossaryentry{ICA}
{
  name=独立成分分析,
  description={independent component analysis},
  sort={independent component analysis},
  symbol={ICA}
}

\newglossaryentry{tICA}
{
  name=地质ICA,
  description={topographic ICA},
  sort={topo independent component analysis},
}

\newglossaryentry{sparse_coding}
{
  name=稀疏编码,
  description={sparse coding},
  sort={sparse coding},
}

\newglossaryentry{fixed_point_arithmetic}
{
  name=定点运算,
  description={fixed-point arithmetic},
  sort={fixed-point arithmetic},
}

\newglossaryentry{float_point_arithmetic}
{
  name=浮点运算,
  description={float-point arithmetic},
  sort={float-point arithmetic},
}

\newglossaryentry{GPU}
{
  name=图形处理器,
  description={Graphics Processing Unit},
  sort={Graphics Processing Unit},
  symbol={GPU}
}

\newglossaryentry{generative_model}
{
  name=生成模型,
  description={generative model},
  sort={generative model},
}

\newglossaryentry{dataset_augmentation}
{
  name=数据集增强,
  description={dataset augmentation},
  sort={dataset augmentation},
}

\newglossaryentry{whitening}
{
  name=白化,
  description={whitening},
  sort={whitening},
}

\newglossaryentry{DNN}
{
  name=深度神经网络,
  description={DNN},
  sort={DNN},
}

\newglossaryentry{end_to_end}
{
  name=端到端的,
  description={end-to-end},
  sort={end-to-end},
}

\newglossaryentry{structured_probabilistic_models}
{
  name=结构化概率模型,
  description={structured probabilistic model},
  sort={structured probabilistic model},
}

\newglossaryentry{graphical_models}
{
  name=图模型,
  description={graphical model},
  sort={graphical model},
}

\newglossaryentry{directed_graphical_model}
{
  name=有向图模型,
  description={directed graphical model},
  sort={directed graphical model},
}

\newglossaryentry{dependency}
{
  name=依赖,
  description={dependency},
  sort={dependency},
}

\newglossaryentry{bayesian_network}
{
  name=贝叶斯网络,
  description={Bayesian network},
  sort={Bayesian network},
}

\newglossaryentry{model_averaging}
{
  name=模型平均,
  description={model averaging},
  sort={model averaging},
}

\newglossaryentry{boosting}
{
  name=Boosting,
  description={Boosting},
  sort={Boosting},
}

\newglossaryentry{weight_scaling_inference_rule}
{
  name=权重比例推断规则,
  description={weight scaling inference rule},
  sort={weight scaling inference rule},
}

\newglossaryentry{statement}
{
  name=声明,
  description={statement},
  sort={statement},
}

\newglossaryentry{quantum_mechanics}
{
  name=量子力学,
  description={quantum mechanics},
  sort={quantum mechanics},
}

\newglossaryentry{subatomic}
{
  name=亚原子,
  description={subatomic},
  sort={subatomic},
}

\newglossaryentry{fidelity}
{
  name=逼真度,
  description={fidelity},
  sort={fidelity},
}

\newglossaryentry{degree_of_belief}
{
  name=信任度,
  description={degree of belief},
  sort={degree of belief},
}

\newglossaryentry{frequentist_probability}
{
  name=频率派概率,
  description={frequentist probability},
  sort={frequentist probability},
}

\newglossaryentry{subsample}
{
  name=子采样,
  description={subsample},
  sort={subsample},
}

\newglossaryentry{bayesian_probability}
{
  name=贝叶斯概率,
  description={Bayesian probability},
  sort={Bayesian probability},
}

\newglossaryentry{likelihood}
{
  name=似然,
  description={likelihood},
  sort={likelihood},
}

\newglossaryentry{RV}
{
  name=随机变量,
  description={random variable},
  sort={random variable},
}

\newglossaryentry{PD}
{
  name=概率分布,
  description={probability distribution},
  sort={probability distribution},
}

\newglossaryentry{PMF}
{
  name=概率质量函数,
  description={probability mass function},
  sort={probability mass function},
  symbol={PMF}
}

\newglossaryentry{joint_probability_distribution}
{
  name=联合概率分布,
  description={joint probability distribution},
  sort={joint probability distribution},
}

\newglossaryentry{normalized}
{
  name=归一化的,
  description={normalized},
  sort={normalized},
}

\newglossaryentry{uniform_distribution}
{
  name=均匀分布,
  description={uniform distribution},
  sort={uniform distribution},
}

\newglossaryentry{PDF}
{
  name=概率密度函数,
  description={probability density function},
  sort={probability density function},
  symbol={PDF}
}

\newglossaryentry{marginal_probability_distribution}
{
  name=边缘概率分布,
  description={marginal probability distribution},
  sort={marginal probability distribution},
}

\newglossaryentry{sum_rule}
{
  name=求和法则,
  description={sum rule},
  sort={sum rule},
}

\newglossaryentry{conditional_probability}
{
  name=条件概率,
  description={conditional probability},
  sort={conditional probability},
}

\newglossaryentry{intervention_query}
{
  name=干预查询,
  description={intervention query},
  sort={intervention query},
}

\newglossaryentry{causal_modeling}
{
  name=因果模型,
  description={causal modeling},
  sort={causal modeling},
}

\newglossaryentry{causal_factor}
{
  name=因果因子,
  description={causal factor},
  sort={causal factor},
}

\newglossaryentry{chain_rule}
{
  name=链式法则,
  description={chain rule},
  sort={chain rule},
}

\newglossaryentry{product_rule}
{
  name=乘法法则,
  description={product rule},
  sort={product rule},
}

\newglossaryentry{independent}
{
  name=相互独立的,
  description={independent},
  sort={independent},
}

\newglossaryentry{conditionally_independent}
{
  name=条件独立的,
  description={conditionally independent},
  sort={conditionally independent},
}

\newglossaryentry{expectation}
{
  name=期望,
  description={expectation},
  sort={expectation},
}

\newglossaryentry{expected_value}
{
  name=期望值,
  description={expected value},
  sort={expected value},
}

\newglossaryentry{example:chap5}
{
  name=样本,
  description={example},
  sort={example},
}

\newglossaryentry{feature}
{
  name=特征,
  description={feature},
  sort={feature},
}

\newglossaryentry{accuracy}
{
  name=准确率,
  description={accuracy},
  sort={accuracy},
}

\newglossaryentry{error_rate}
{
  name=错误率,
  description={error rate},
  sort={error rate},
}

\newglossaryentry{training_set}
{
  name=训练集,
  description={training set},
  sort={training set},
}

\newglossaryentry{explanatory_factor}
{
  name=解释因子,
  description={explanatory factort},
  sort={explanatory factor},
}

\newglossaryentry{underlying}
{
  name=潜在,
  description={underlying},
  sort={underlying},
}

\newglossaryentry{underlying_cause}
{
	name=潜在成因,
	description={underlying cause},
	sort={underlying cause},
}

\newglossaryentry{test_set}
{
  name=测试集,
  description={test set},
  sort={test set},
}

\newglossaryentry{performance_measures}
{
  name=性能度量,
  description={performance measures},
  sort={performance measures},
}

\newglossaryentry{experience}
{
  name=经验,
  description={experience, E},
  sort={experience, E},
}

\newglossaryentry{unsupervised}
{
  name=无监督,
  description={unsupervised},
  sort={unsupervised},
}

\newglossaryentry{supervised}
{
  name=监督,
  description={supervised},
  sort={supervised},
}

\newglossaryentry{semi_supervised}
{
  name=半监督,
  description={semi-supervised},
  sort={semi-supervised},
}

\newglossaryentry{supervised_learning}
{
  name=监督学习,
  description={supervised learning},
  sort={supervised learning},
}

\newglossaryentry{unsupervised_learning}
{
  name=无监督学习,
  description={unsupervised learning},
  sort={unsupervised learning},
}

\newglossaryentry{dataset}
{
  name=数据集,
  description={dataset},
  sort={dataset},
}

\newglossaryentry{setting}
{
  name=情景,
  description={setting},
  sort={setting},
}

\newglossaryentry{data_points}
{
  name=数据点,
  description={data point},
  sort={data point},
}

\newglossaryentry{label}
{
  name=标签,
  description={label},
  sort={label},
}

\newglossaryentry{labeled}
{
  name=标注,
  description={labeled},
  sort={labeled},
}

\newglossaryentry{unlabeled}
{
  name=无标签的,
  description={unlabeled},
  sort={unlabeled},
}

\newglossaryentry{target}
{
  name=目标,
  description={target},
  sort={target},
}

\newglossaryentry{reinforcement_learning}
{
  name=强化学习,
  description={reinforcement learning},
  sort={reinforcement learning},
}

\newglossaryentry{design_matrix}
{
  name=设计矩阵,
  description={design matrix},
  sort={design matrix},
}

\newglossaryentry{parameters}
{
  name=参数,
  description={parameter},
  sort={parameter},
}

\newglossaryentry{weights}
{
  name=权重,
  description={weight},
  sort={weight},
}

\newglossaryentry{mean_squared_error}
{
  name=均方误差,
  description={mean squared error},
  sort={mean squared error},
  symbol={MSE}
}

\newglossaryentry{normal_equations}
{
  name=正规方程,
  description={normal equation},
  sort={normal equation},
}

\newglossaryentry{training_error}
{
  name=训练误差,
  description={training error},
  sort={training error},
}

\newglossaryentry{generalization_error}
{
  name=泛化误差,
  description={generalization error},
  sort={generalization error},
}

\newglossaryentry{test_error}
{
  name=测试误差,
  description={test error},
  sort={test error},
}

\newglossaryentry{hypothesis_space}
{
  name=假设空间,
  description={hypothesis space},
  sort={hypothesis space},
}

\newglossaryentry{capacity}
{
  name=容量,
  description={capacity},
  sort={capacity},
}

\newglossaryentry{representational_capacity}
{
  name=表示容量,
  description={representational capacity},
  sort={representational capacity},
}

\newglossaryentry{effective_capacity}
{
  name=有效容量,
  description={effective capacity},
  sort={effective capacity},
}

\newglossaryentry{linear_threshold_units}
{
  name=线性阀值单元,
  description={linear threshold units},
  sort={linear threshold units},
}

\newglossaryentry{nonparametric}
{
  name=非参数,
  description={non-parametric},
  sort={non-parametric},
}

\newglossaryentry{nearest_neighbor_regression}
{
  name=最近邻回归,
  description={nearest neighbor regression},
  sort={nearest neighbor regression},
}

\newglossaryentry{nearest_neighbor}
{
  name=最近邻,
  description={nearest neighbor},
  sort={nearest neighbor},
}

\newglossaryentry{bayes_error}
{
  name=贝叶斯误差,
  description={Bayes error},
  sort={Bayes error},
}

\newglossaryentry{no_free_lunch_theorem}
{
  name=没有免费午餐定理,
  description={no free lunch theorem},
  sort={no free lunch theorem},
}

\newglossaryentry{validation_set}
{
  name=验证集,
  description={validation set},
  sort={validation set},
}

\newglossaryentry{benchmarks}
{
  name=基准,
  description={bechmark},
  sort={bechmark},
}

\newglossaryentry{baseline}
{
  name=基准,
  description={baseline},
  sort={beseline}
}

\newglossaryentry{point_estimator}
{
  name=点估计,
  description={point estimator},
  sort={point estimator},
}

\newglossaryentry{estimator:chap5}
{
  name=估计量,
  description={estimator},
  sort={estimator},
}

\newglossaryentry{statistics}
{
  name=统计量,
  description={statistics},
  sort={statistics},
}

\newglossaryentry{unbiased}
{
  name=无偏,
  description={unbiased},
  sort={unbiased},
}

\newglossaryentry{biased}
{
  name=有偏,
  description={biased},
  sort={biased},
}

\newglossaryentry{asynchronous}
{
  name=异步,
  description={asynchronous},
  sort={asynchronous},
}

\newglossaryentry{asymptotically_unbiased}
{
  name=渐近无偏,
  description={asymptotically unbiased},
  sort={asymptotically unbiased},
}

\newglossaryentry{sample_mean}
{
  name=样本均值, %采样均值?
  description={sample mean},
  sort={sample mean},
}

\newglossaryentry{sample_variance}
{
  name=样本方差, %采样方差?
  description={sample variance},
  sort={sample variance},
}

\newglossaryentry{unbiased_sample_variance}
{
  name=无偏样本方差, %采样方差?
  description={unbiased sample variance},
  sort={unbiased sample variance},
}

\newglossaryentry{standard_error}
{
  name=标准误差,
  description={standard error},
  symbol={SE},
  sort={standard error},
}

\newglossaryentry{consistency}
{
  name=相合性,
  description={consistency},
  sort={consistency},
}

\newglossaryentry{almost_sure}
{
  name=几乎必然,
  description={almost sure},
  sort={almost sure},
}

\newglossaryentry{almost_sure_convergence}
{
  name=几乎必然收敛,
  description={almost sure convergence},
  sort={almost sure convergence},
}

\newglossaryentry{statistical_efficiency}
{
  name=统计效率,
  description={statistic efficiency},
  sort={statistic efficiency},
}

\newglossaryentry{parametric_case}
{
  name=有参情况,
  description={parametric case},
  sort={parametric case},
}

\newglossaryentry{frequentist_statistics}
{
  name=频率派统计,
  description={frequentist statistics},
  sort={frequentist statistics},
}

\newglossaryentry{bayesian_statistics}
{
  name=贝叶斯统计,
  description={Bayesian statistics},
  sort={Bayesian statistics},
}

\newglossaryentry{prior_probability_distribution}
{
  name=先验概率分布,
  description={prior probability distribution},
  sort={prior probability distribution},
}

\newglossaryentry{maximum_a_posteriori}
{
  name=最大后验,
  description={maximum a posteriori},
  sort={maximum a posteriori},
}

\newglossaryentry{maximum_likelihood_estimation}
{
  name=最大似然估计,
  description={maximum likelihood estimation},
  sort={maximum likelihood estimation},
}

\newglossaryentry{maximum_likelihood}
{
  name=最大似然,
  description={maximum likelihood},
  sort={maximum likelihood},
}

\newglossaryentry{kernel_trick}
{
  name=核技巧,
  description={kernel trick},
  sort={kernel trick},
}

\newglossaryentry{kernel}
{
  name=核函数,
  description={kernel function},
  sort={kernel function},
}

\newglossaryentry{gaussian_kernel}
{
  name=高斯核,
  description={Gaussian kernel},
  sort={Gaussian kernel},
}

\newglossaryentry{kernel_machines}
{
  name=核机器,
  description={kernel machine},
  sort={kernel machine},
}

\newglossaryentry{kernel_methods}
{
  name=核方法,
  description={kernel method},
  sort={kernel method},
}

\newglossaryentry{support_vectors}
{
  name=支持向量,
  description={support vector},
  sort={support vector},
}

\newglossaryentry{SVM}
{
  name=支持向量机,
  description={support vector machine},
  symbol={SVM}
}

\newglossaryentry{phoneme}
{
  name=音素,
  description={phoneme},
  sort={phoneme},
}

\newglossaryentry{acoustic}
{
  name=声学,
  description={acoustic},
  sort={acoustic},
}

\newglossaryentry{phonetic}
{
  name=语音,
  description={phonetic},
  sort={phonetic},
}

\newglossaryentry{mixture_of_experts}
{
  name=专家混合体,
  description={mixture of experts},
  sort={mixture of experts},
}

\newglossaryentry{gauss_mixture}
{
  name=高斯混合体,
  description={Gaussian mixtures},
  sort={Gaussian mixtures},
}

\newglossaryentry{hard_mixture_of_experts}
{
  name=硬专家混合体, 
  description={hard mixture of experts},
  sort={hard mixture of experts},
}

\newglossaryentry{gater}
{
  name=选通器,
  description={gater},
  sort={gater},
}

\newglossaryentry{expert_network}
{
  name=专家网络,
  description={expert network},
  sort={expert network},
}

\newglossaryentry{attention_mechanism}
{
  name=注意力机制,
  description={attention mechanism},
  sort={attention mechanism},
}

\newglossaryentry{fast_dropout}
{
  name=快速Dropout,
  description={fast dropout},
  sort={fast dropout},
}

\newglossaryentry{dropout_boosting}
{
  name=Dropout Boosting,
  description={Dropout Boosting},
  sort={dropout boosting},
}

\newglossaryentry{adversarial_example}
{
  name=对抗样本,
  description={adversarial example},
  sort={adversarial example},
}

\newglossaryentry{adversarial}
{
  name=对抗,
  description={adversarial},
  sort={adversarial},
}

\newglossaryentry{virtual_adversarial_example}
{
  name=虚拟对抗样本,
  description={virtual adversarial example},
  sort={virtual adversarial example},
}

\newglossaryentry{adversarial_training}
{
  name=对抗训练,
  description={adversarial training},
  sort={adversarial training},
}

\newglossaryentry{virtual_adversarial_training}
{
  name=虚拟对抗训练,
  description={virtual adversarial training},
  sort={virtual adversarial training}
}

\newglossaryentry{tangent_distance}
{
  name=切面距离,
  description={tangent distance},
  sort={tangent distance},
}

\newglossaryentry{tangent_prop}
{
  name=正切传播,
  description={tangent prop},
  sort={tangent prop},
}

\newglossaryentry{tangent_propagation}
{
  name=正切传播,
  description={tangent propagation}
}

\newglossaryentry{double_backprop}
{
  name=双反向传播,
  description={double backprop},
  sort={double backprop},
}

\newglossaryentry{EM}
{
  name=期望最大化,
  description={expectation maximization},
  sort={expectation maximization},
  symbol={EM}
}

\newglossaryentry{mean_field}
{
  name=均值场,
  description={mean-field},
  sort={mean-field},
}

\newglossaryentry{ELBO}
{
  name=证据下界,
  description={evidence lower bound},
  sort={evidence lower bound},
  symbol={ELBO}
}

\newglossaryentry{variational_free_energy}
{
  name=变分自由能,
  description={variational free energy},
  sort={variational free energy},
}

\newglossaryentry{structured_variational_inference}
{
  name=结构化变分推断,
  description={structured variational inference},
  sort={structured variational inference},
}

\newglossaryentry{variational_inference}
{
  name=变分推断,
  description={variational inference},
  sort={variational inference},
}

\newglossaryentry{binary_sparse_coding}
{
  name=二值稀疏编码,
  description={binary sparse coding},
  sort={binary sparse coding},
}

\newglossaryentry{feedforward_network}
{
  name=前馈网络,
  description={feedforward network},
  sort={feedforward network},
}

\newglossaryentry{transition}
{
  name=转移,
  description={transition},
  sort={transition},
}

\newglossaryentry{reconstruction}
{
  name=重构,
  description={reconstruction},
  sort={reconstruction},
}

\newglossaryentry{GSN}
{
  name=生成随机网络,
  description={generative stochastic network},
  sort={generative stochastic network},
  symbol={GSN}
}

\newglossaryentry{score_matching}
{
  name=得分匹配,
  description={score matching},
  sort={score matching},
}

\newglossaryentry{factorial}
{
  name=因子,
  description={factorial},
  sort={factorial},
}

\newglossaryentry{factorized}
{
  name=分解的,
  description={factorized},
  sort={factorized},
}

\newglossaryentry{meanfield}
{
  name=均匀场,
  description={meanfield},
  sort={meanfield},
}

\newglossaryentry{MLE}
{
  name=最大似然估计,
  description={maximum likelihood estimation},
  sort={maximum likelihood estimation},
}

\newglossaryentry{PPCA}
{
  name=概率PCA,
  description={probabilistic PCA},
  sort={probabilistic PCA},
  symbol={PPCA}
}

\newglossaryentry{SGA}
{
  name=随机梯度上升,
  description={Stochastic Gradient Ascent},
  sort={Stochastic Gradient Ascent},
}

\newglossaryentry{clique}
{
  name=团,
  description={clique},
  sort={clique},
}

\newglossaryentry{dirac_distribution}
{
  name=Dirac分布,
  description={dirac distribution},
  sort={dirac distribution},
}

\newglossaryentry{fixed_point_equation}
{
  name=不动点方程,
  description={fixed point equation},
  sort={fixed point equation},
}

\newglossaryentry{calculus_of_variations}
{
  name=变分法,
  description={calculus of variations},
  sort={calculus of variations},
}

\newglossaryentry{wake_sleep}
{
  name=wake sleep,
  description={wake sleep},
  sort={wake sleep},
}

\newglossaryentry{BN}
{
  name=信念网络,
  description={belief network},
  sort={belief network},
}

\newglossaryentry{MRF}
{
  name=马尔可夫随机场,
  description={Markov random field},
  sort={Markov random field},
  symbol={MRF}
}

\newglossaryentry{markov_network}
{
  name=马尔可夫网络,
  description={Markov network},
  sort={Markov network},
}

\newglossaryentry{log_linear_model}
{
  name=对数线性模型,
  description={log-linear model},
  sort={log-linear model},
}

\newglossaryentry{product_of_expert}
{
  name=专家之积,
  description={product of expert},
  sort={product of expert},
}

\newglossaryentry{free_energy}
{
  name=自由能,
  description={free energy},
  sort={free energy},
}

\newglossaryentry{harmony}
{
  name=harmony,
  description={harmony},
  sort={harmony},
}

\newglossaryentry{separation}
{
  name=分离,
  description={separation},
  sort={separation},
}

\newglossaryentry{separate}
{
  name=分离的,
  description={separate},
  sort={separate},
}

\newglossaryentry{dseparation}
{
  name=d-分离,
  description={d-separation},
  sort={d-separation},
}

\newglossaryentry{local_conditional_probability_distribution}
{
  name=局部条件概率分布,
  description={local conditional probability distribution},
  sort={local conditional probability distribution},
}

\newglossaryentry{conditional_probability_distribution}
{
  name=条件概率分布,
  description={conditional probability distribution}
}

\newglossaryentry{boltzmann_distribution}
{
  name=玻尔兹曼分布,
  description={Boltzmann distribution},
  sort={Boltzmann distribution},
}

\newglossaryentry{gibbs_distribution}
{
  name=吉布斯分布,
  description={Gibbs distribution},
  sort={Gibbs distribution},
}

\newglossaryentry{energy_function}
{
  name=能量函数,
  description={energy function},
  sort={energy function},
}

\newglossaryentry{immorality}
{
  name=不道德,
  description={immorality},
  sort={immorality},
}

\newglossaryentry{moralization}
{
  name=道德化,
  description={moralization},
  sort={moralization},
}

\newglossaryentry{moralized_graph}
{
  name=道德图,
  description={moralized graph},
  sort={moralized graph},
}

\newglossaryentry{standard_deviation}
{
  name=标准差,
  description={standard deviation},
  symbol={SD},
  sort={standard deviation},
}

\newglossaryentry{correlation}
{
  name=相关系数,
  description={correlation},
  sort={correlation},
}

\newglossaryentry{standard_normal_distribution}
{
  name=标准正态分布,
  description={standard normal distribution},
  sort={standard normal distribution},
}

\newglossaryentry{covariance_matrix}
{
  name=协方差矩阵,
  description={covariance matrix},
  sort={covariance matrix},
}

\newglossaryentry{bernoulli_distribution}
{
  name=Bernoulli分布,
  description={Bernoulli distribution},
  sort={Bernoulli distribution},
}

\newglossaryentry{bernoulli_output_distribution}
{
  name=Bernoulli输出分布,
  description={Bernoulli output distribution},
  sort={Bernoulli output distribution},
}

\newglossaryentry{multinoulli_distribution}
{
  name=Multinoulli分布,
  description={multinoulli distribution},
  sort={multinoulli distribution},
}

\newglossaryentry{multinoulli_output_distribution}
{
  name=Multinoulli输出分布,
  description={multinoulli output distribution},
  sort={multinoulli output distribution},
}

\newglossaryentry{categorical_distribution}
{
  name=范畴分布,
  description={categorical distribution},
  sort={categorical distribution},
}

\newglossaryentry{multinomial_distribution}
{
  name=多项式分布,
  description={multinomial distribution},
  sort={multinomial distribution},
}

\newglossaryentry{normal_distribution}
{
  name=正态分布,
  description={normal distribution},
  sort={normal distribution},
}

\newglossaryentry{gaussian_distribution}
{
  name=高斯分布,
  description={Gaussian distribution},
  sort={Gaussian distribution},
}

\newglossaryentry{precision}
{
  name=精度,
  description={precision},
  sort={precision},
}

\newglossaryentry{multivariate_normal_distribution}
{
  name=多维正态分布,
  description={multivariate normal distribution},
  sort={multivariate normal distribution},
}

\newglossaryentry{precision_matrix}
{
  name=精度矩阵,
  description={precision matrix},
  sort={precision matrix},
}

\newglossaryentry{isotropic}
{
  name=各向同性,
  description={isotropic},
  sort={isotropic},
}

\newglossaryentry{exponential_distribution}
{
  name=指数分布,
  description={exponential distribution},
  sort={exponential distribution},
}

\newglossaryentry{indicator_function}
{
  name=指示函数,
  description={indicator function},
  sort={indicator function},
}

\newglossaryentry{laplace_distribution}
{
  name=Laplace分布,
  description={Laplace distribution},
  sort={Laplace distribution},
}

\newglossaryentry{dirac_delta_function}
{
  name=Dirac delta函数,
  description={Dirac delta function},
  sort={Dirac delta function},
}

\newglossaryentry{generalized_function}
{
  name=广义函数,
  description={generalized function},
  sort={generalized function},
}

\newglossaryentry{empirical_distribution}
{
  name=经验分布,
  description={empirical distribution},
  sort={empirical distribution},
}

\newglossaryentry{empirical_frequency}
{
  name=经验频率,
  description={empirical frequency},
  sort={empirical frequency},
}

\newglossaryentry{mixture_distribution}
{
  name=混合分布,
  description={mixture distribution},
  sort={mixture distribution},
}

\newglossaryentry{latent_variable}
{
  name=潜变量,
  description={latent variable},
  sort={latent variable},
}

\newglossaryentry{hidden_variable}
{
  name=隐藏变量,
  description={hidden variable},
  sort={hidden variable},
}

\newglossaryentry{prior_probability}
{
  name=先验概率,
  description={prior probability},
  sort={prior probability},
}

\newglossaryentry{posterior_probability}
{
  name=后验概率,
  description={posterior probability},
  sort={posterior probability},
}

\newglossaryentry{universal_approximator}
{
  name=万能近似器,
  description={universal approximator},
  sort={universal approximator},
}

\newglossaryentry{universal_function_approximator}
{
  name=万能函数近似器,
  description={universal function approximator},
  sort={universal function approximator},
}

\newglossaryentry{logistic_sigmoid}
{
  name=logistic sigmoid,
  description={logistic sigmoid},
  sort={logistic sigmoid},
}

\newglossaryentry{sigmoid}
{
  name=sigmoid,
  description={sigmoid}
}

\newglossaryentry{saturate}
{
  name=饱和,
  description={saturate},
  sort={saturate},
}

\newglossaryentry{softplus_function}
{
  name=softplus函数,
  description={softplus function},
  sort={softplus function},
}

\newglossaryentry{logit}
{
  name=分对数,
  description={logit},
  sort={logit},
}

\newglossaryentry{positive_part_function}
{
  name=正部函数,
  description={positive part function},
  sort={positive part function},
}

\newglossaryentry{negative part function}
{
  name=负部函数,
  description={negative part function},
  sort={negative part function},
}

\newglossaryentry{bayes_rule}
{
  name=贝叶斯规则,
  description={Bayes' rule},
  sort={Bayes' rule},
}

\newglossaryentry{measure_theory}
{
  name=测度论,
  description={measure theory},
  sort={measure theory},
}

\newglossaryentry{measure_zero}
{
  name=零测度,
  description={measure zero},
  sort={measure zero},
}

\newglossaryentry{almost_everywhere}
{
  name=几乎处处,
  description={almost everywhere},
  sort={almost everywhere},
}

\newglossaryentry{jacobian_matrix}
{
  name=Jacobi矩阵,
  description={Jacobian matrix},
  sort={Jacobian matrix},
}

\newglossaryentry{self_information}
{
  name=自信息,
  description={self-information},
  sort={self-information},
}

\newglossaryentry{nats}
{
  name=奈特,
  description={nats},
  sort={nats},
}

\newglossaryentry{bits}
{
  name=比特,
  description={bit},
  sort={bit},
}

\newglossaryentry{shannons}
{
  name=香农,
  description={shannons},
  sort={shannons},
}

\newglossaryentry{Shannon_entropy}
{
  name=香农熵,
  description={Shannon entropy},
  sort={Shannon entropy},
}

\newglossaryentry{differential_entropy}
{
  name=微分熵,
  description={differential entropy},
  sort={differential entropy},
}

\newglossaryentry{differential_equation}
{
  name=微分方程,
  description={differential equation},
  sort={differential equation},
}

\newglossaryentry{KL_divergence}
{
  name=KL散度,
  description={Kullback-Leibler (KL) divergence},
  sort={Kullback-Leibler (KL) divergence},
}

\newglossaryentry{cross_entropy}
{
  name=交叉熵,
  description={cross-entropy},
  sort={cross-entropy},
}

\newglossaryentry{entropy}
{
  name=熵,
  description={entropy},
  sort={entropy},
}

\newglossaryentry{factorization}
{
  name=分解,
  description={factorization},
  sort={factorization},
}

\newglossaryentry{structured_probabilistic_model}
{
  name=结构化概率模型,
  description={structured probabilistic model},
  sort={structured probabilistic model},
}

\newglossaryentry{graphical_model}
{
  name=图模型,
  description={graphical model},
  sort={graphical model},
}

\newglossaryentry{backoff}
{
  name=回退,
  description={back-off},
  sort={back-off},
}

\newglossaryentry{directed}
{
  name=有向,
  description={directed},
  sort={directed},
}

\newglossaryentry{undirected}
{
  name=无向,
  description={undirected},
  sort={undirected},
}

\newglossaryentry{undirected_graphical_model}
{
  name=无向图模型,
  description={undirected graphical model}
}

\newglossaryentry{proportional}
{
  name=成比例,
  description={proportional},
  sort={proportional},
}

\newglossaryentry{description}
{
  name=描述,
  description={description},
  sort={description},
}

\newglossaryentry{decision_tree}
{
  name=决策树,
  description={decision tree},
  sort={decision tree},
}

\newglossaryentry{factor_graph}
{
  name=因子图,
  description={factor graph},
  sort={factor graph},
}

\newglossaryentry{structure_learning}
{
  name=结构学习,
  description={structure learning},
  sort={structure learning},
}

\newglossaryentry{loopy_belief_propagation}
{
  name=环状信念传播,
  description={loopy belief propagation},
  sort={loopy belief propagation},
}

\newglossaryentry{harmonium}
{
  name=簧风琴,
  description={harmonium},
  sort={harmonium},
}

\newglossaryentry{convolutional_network}
{
  name=卷积网络,
  description={convolutional network},
  sort={convolutional network},
}

\newglossaryentry{convolutional_net}
{
  name=卷积网络,
  description={convolutional net},
  sort={convolutional net},
}

\newglossaryentry{main_diagonal}
{
  name=主对角线,
  description={main diagonal},
  sort={main diagonal},
}

\newglossaryentry{transpose}
{
  name=转置,
  description={transpose},
  sort={transpose},
}

\newglossaryentry{broadcasting}
{
  name=广播,
  description={broadcasting},
  sort={broadcasting},
}

\newglossaryentry{matrix_product}
{
  name=矩阵乘积,
  description={matrix product},
  sort={matrix product},
}

\newglossaryentry{adagrad}
{
  name=AdaGrad,
  description={AdaGrad},
  sort={AdaGrad},
}

\newglossaryentry{element_wise_product}
{
  name=元素对应乘积,
  description={element-wise product},
  sort={element-wise product},
}

\newglossaryentry{hadamard_product}
{
  name=Hadamard乘积,
  description={Hadamard product},
  sort={Hadamard product},
}

\newglossaryentry{clique_potential}
{
  name=团势能,
  description={clique potential},
  sort={clique potential},
}

\newglossaryentry{factor}
{
  name=因子,
  description={factor},
  sort={factor},
}

\newglossaryentry{unnormalized_probability_function}
{
  name=未归一化概率函数,
  description={unnormalized probability function},
  sort={unnormalized probability function},
}

\newglossaryentry{recurrent_network}
{
  name=循环网络,
  description={recurrent network},
  sort={recurrent network},
}

\newglossaryentry{vanish_explode_gradient}
{
  name=梯度消失与爆炸问题,
  description={vanishing and exploding gradient problem},
  sort={vanishing and exploding gradient problem },
}

\newglossaryentry{vanish_gradient}
{
  name=梯度消失,
  description={vanishing gradient},
  sort={vanishing gradient},
}

\newglossaryentry{explode_gradient}
{
  name=梯度爆炸,
  description={exploding gradient},
  sort={exploding gradient},
}

\newglossaryentry{computational_graph}
{
  name=梯度消失与爆炸问题,
  description={vanishing and exploding gradient problem},
  sort={vanishing and exploding gradient problem },
}

\newglossaryentry{vanish_gradient}
{
  name=梯度消失,
  description={vanishing gradient},
  sort={vanishing gradient},
}

\newglossaryentry{explode_gradient}
{
  name=梯度爆炸,
  description={exploding gradient},
  sort={exploding gradient},
}

\newglossaryentry{computational_graph}
{
  name=计算图,
  description={computational graph},
  sort={computational graph},
}

\newglossaryentry{unfolding}
{
  name=展开,
  description={unfolding},
  sort={unfolding},
}

\newglossaryentry{time_step}
{
  name=时间步,
  description={time step},
  sort={time step},
}

\newglossaryentry{n_gram}
{
  name=$n$-gram,
  description={n-gram},
  sort={n-gram},
}

\newglossaryentry{curse_of_dimensionality}
{
  name=维数灾难,
  description={curse of dimensionality},
  sort={curse of dimensionality},
}

\newglossaryentry{smoothness_prior}
{
  name=平滑先验,
  description={smoothness prior},
  sort={smoothness prior},
}

\newglossaryentry{local_constancy_prior}
{
  name=局部不变性先验,
  description={local constancy prior},
  sort={local constancy prior},
}

\newglossaryentry{local_kernel}
{
  name=局部核,
  description={local kernel},
  sort={local kernel},
}

\newglossaryentry{manifold}
{
  name=流形,
  description={manifold},
  sort={manifold},
}

\newglossaryentry{manifold_tangent_classifier}
{
  name=流形正切分类器,
  description={manifold tangent classifier}
}

\newglossaryentry{manifold_learning}
{
  name=流形学习,
  description={manifold learning},
  sort={manifold learning},
}

\newglossaryentry{manifold_hypothesis}
{
  name=流形假设,
  description={manifold hypothesis},
  sort={manifold hypothesis},
}

\newglossaryentry{loop}
{
  name=环,
  description={loop},
  sort={loop},
}

\newglossaryentry{chord}
{
  name=弦,
  description={chord},
  sort={chord},
}

\newglossaryentry{chordal_graph}
{
  name=弦图,
  description={chordal graph},
  sort={chordal graph},
}

\newglossaryentry{triangulated_graph}
{
  name=三角形化图,
  description={triangulated graph},
  sort={triangulated graph},
}

\newglossaryentry{triangulate}
{
  name=三角形化,
  description={triangulate},
  sort={triangulate},
}

\newglossaryentry{risk}
{
  name=风险,
  description={risk},
  sort={risk},
}

\newglossaryentry{empirical_risk}
{
  name=经验风险,
  description={empirical risk},
  sort={empirical risk},
}

\newglossaryentry{empirical_risk_minimization}
{
  name=经验风险最小化,
  description={empirical risk minimization},
  sort={empirical risk minimization},
}

\newglossaryentry{surrogate_loss_function}
{
  name=代理损失函数,
  description={surrogate loss function},
  sort={surrogate loss function},
}

\newglossaryentry{batch}
{
  name=batch,
  description={batch},
  sort={batch},
}

\newglossaryentry{deterministic}
{
  name=确定性,
  description={deterministic},
  sort={deterministic},
}

\newglossaryentry{stochastic}
{
  name=随机,
  description={stochastic},
  sort={stochastic},
}

\newglossaryentry{online}
{
  name=在线,
  description={online},
  sort={online},
}

\newglossaryentry{minibatch_stochastic}
{
  name=小批量随机, %小批量
  description={minibatch stochastic},
  sort={minibatch stochastic},
}

\newglossaryentry{stream}
{
  name=流,
  description={stream},
  sort={stream},
}

\newglossaryentry{model_identifiability}
{
  name=模型可辨识性,
  description={model identifiability},
  sort={model identifiability},
}

\newglossaryentry{weight_space_symmetry}
{
  name=权重空间对称性,
  description={weight space symmetry},
  sort={weight space symmetry},
}

\newglossaryentry{saddle_free_newton_method}
{
  name=无鞍牛顿法,
  description={saddle-free Newton method},
  sort={saddle-free Newton method},
}

\newglossaryentry{gradient_clipping} 
{
  name=梯度截断,
  description={gradient clipping},
  sort={gradient clipping},
}

\newglossaryentry{power_method}
{
  name=幂方法,
  description={power method},
  sort={power method},
}

\newglossaryentry{linear_factor}
{
  name=线性因子模型,
  description={linear factor model},
  sort={linear factor model},
}

\newglossaryentry{forward_propagation}
{
  name=前向传播,
  description={forward propagation},
  sort={forward propagation},
}

\newglossaryentry{backward_propagation}
{
  name=反向传播,
  description={backward propagation},
  sort={backward propagation},
}

\newglossaryentry{unfolded_graph}
{
  name=展开图,
  description={unfolded graph},
  sort={unfolded graph},
}

\newglossaryentry{BPTT}
{
  name=通过时间反向传播,
  description={back-propagation through time},
  sort={back-propagation through time},
  symbol={BPTT}
}

\newglossaryentry{teacher_forcing}
{
  name=Teacher Forcing,
  description={Teacher Forcing},
  sort={Teacher Forcing},
}

\newglossaryentry{stationary}
{
  name=平稳的,
  description={stationary},
  sort={stationary},
}

\newglossaryentry{deep_feedforward_network}
{
  name=深度前馈网络,
  description={deep feedforward network},
  sort={deep feedforward network},
}

\newglossaryentry{feedforward_neural_network}
{
  name=前馈神经网络,
  description={feedforward neural network},
  sort={feedforward neural network},
}

\newglossaryentry{feedforward}
{
  name=前向,
  description={feedforward},
  sort={feedforward},
}

\newglossaryentry{feedback}
{
  name=反馈,
  description={feedback},
  sort={feedback},
}

\newglossaryentry{network}
{
  name=网络,
  description={network},
  sort={network},
}

\newglossaryentry{first_layer}
{
  name=第一层,
  description={first layer},
  sort={first layer},
}

\newglossaryentry{second_layer}
{
  name=第二层,
  description={second layer},
  sort={second layer},
}

\newglossaryentry{depth}
{
  name=深度,
  description={depth},
  sort={depth},
}

\newglossaryentry{output_layer}
{
  name=输出层,
  description={output layer},
  sort={output layer},
}

\newglossaryentry{hidden_layer}
{
  name=隐藏层,
  description={hidden layer},
  sort={hidden layer},
}

\newglossaryentry{width}
{
  name=宽度,
  description={width},
  sort={width},
}

\newglossaryentry{unit}
{
  name=单元,
  description={unit},
  sort={unit},
}

\newglossaryentry{activation_function}
{
  name=激活函数,
  description={activation function},
  sort={activation function},
}

\newglossaryentry{dbd}
{
  name=delta-bar-delta,
  description={delta-bar-delta},
  sort={delta-bar-delta},
}

\newglossaryentry{gaussian_output_distribution}
{
  name=高斯输出分布,
  description={Gaussian output distribution},
  sort={Gaussian output distribution},
}

\newglossaryentry{back_propagation}
{
  name=反向传播,
  description={back propagation},
  sort={back propagation},
}

\newglossaryentry{back_propagate}
{
  name=反向传播,
  description={back propagate},
  sort={back propagate},
}

\newglossaryentry{BP}
{
  name=反向传播,
  description={backprop},
  sort={backprop},
  symbol={BP}
}

\newglossaryentry{functional}
{
  name=泛函,
  description={functional},
  sort={functional},
}

\newglossaryentry{mean_absolute_error}
{
  name=平均绝对误差,
  description={mean absolute error},
  sort={mean absolute error},
}

\newglossaryentry{winner_take_all}
{
  name=赢者通吃,
  description={winner-take-all},
  sort={winner-take-all},
}

\newglossaryentry{heteroscedastic}
{
  name=异方差,
  description={heteroscedastic},
  sort={heteroscedastic},
}

\newglossaryentry{mixture_density_network}
{
  name=混合密度网络,
  description={mixture density network},
  sort={mixture density network},
}

\newglossaryentry{clip_gradients}
{
  name=梯度截断,
  description={clip gradient},
  sort={clip gradient},
}

\newglossaryentry{absolute_value_rectification}
{
  name=绝对值整流,
  description={absolute value rectification},
  sort={absolute value rectification},
}

\newglossaryentry{leaky_ReLU}   %有没有更好的翻译
{
  name=渗漏整流线性单元,
  description={Leaky ReLU},
  sort={Leaky ReLU},
  symbol={Leaky ReLU}
}

\newglossaryentry{PReLU}
{
  name=参数化整流线性单元,
  description={parametric ReLU},
  sort={parametric ReLU},
  symbol={PReLU}
}

\newglossaryentry{maxout_unit}
{
  name=maxout单元,
  description={maxout unit},
  sort={maxout unit},
}

\newglossaryentry{maxout}
{
  name=maxout,
  description={maxout},
  symbol={maxout}
}

\newglossaryentry{catastrophic_forgetting}
{
  name=灾难遗忘,
  description={catastrophic forgetting},
  sort={catastrophic forgetting},
}

\newglossaryentry{RBF}
{
  name=径向基函数,
  description={radial basis function},
  sort={radial basis function},
  symbol={RBF}
}

\newglossaryentry{softplus}
{
  name=softplus,
  description={softplus},
  sort={softplus},
}

\newglossaryentry{hard_tanh}
{
  name=硬双曲正切函数,
  description={hard tanh},
  sort={hard tanh},
}

\newglossaryentry{architecture}
{
  name=结构,
  description={architecture},
  sort={architecture},
}

\newglossaryentry{universal_approximation_theorem}
{
  name=万能近似定理, %万能逼近定理?
  description={universal approximation theorem},
  sort={universal approximation theorem},
}

\newglossaryentry{operation}
{
  name=操作,
  description={operation},
  sort={operation},
}

\newglossaryentry{symbol}
{
  name=符号,
  description={symbol},
  sort={symbol},
}

\newglossaryentry{numeric_value}
{
  name=数值,
  description={numeric value},
  sort={numeric value},
}

\newglossaryentry{dynamic_programming}
{
  name=动态规划,
  description={dynamic programming},
  sort={dynamic programming},
}

\newglossaryentry{automatic_differentiation}
{
  name=自动微分,
  description={automatic differentiation},
  sort={automatic differentiation},
}

\newglossaryentry{reverse_mode_accumulation}
{
  name=反向模式累加,
  description={reverse mode accumulation},
  sort={reverse mode accumulation},
}

\newglossaryentry{forward_mode_accumulation}
{
  name=前向模式累加,
  description={forward mode accumulation},
  sort={forward mode accumulation},
}

\newglossaryentry{Krylov_methods}
{
  name=Krylov方法,
  description={Krylov method},
  sort={Krylov method},
}

\newglossaryentry{parallel_distributed_processing}
{
  name=并行分布式处理,
  description={Parallel Distributed Processing},
  sort={Parallel Distributed Processing},
}

\newglossaryentry{sparse_activation}
{
  name=稀疏激活,
  description={sparse activation},
  sort={sparse activation},
}

\newglossaryentry{damping}
{
  name=衰减,
  description={damping},
  sort={damping},
}

\newglossaryentry{learned}
{
  name=学成,
  description={learned},
  sort={learned},
}

\newglossaryentry{message_passing}
{
  name=信息传输,
  description={message passing},
  sort={message passing},
}

\newglossaryentry{functional_derivative}
{
  name=泛函导数,
  description={functional derivative},
  sort={functional derivative},
}

\newglossaryentry{variational_derivative}
{
  name=变分导数,
  description={variational derivative},
  sort={variational derivative},
}

\newglossaryentry{excess_error}
{
  name=额外误差,
  description={excess error},
  sort={excess error},
}

\newglossaryentry{momentum}
{
  name=动量,
  description={momentum},
  sort={momentum},
}

\newglossaryentry{nmomentum}
{
  name=Nesterov 动量,
  description={Nesterov momentum},
  sort={Nesterov momentum},
}

\newglossaryentry{chaos}
{
  name=混沌,
  description={chaos},
  sort={chaos},
}

\newglossaryentry{normalized_initialization}
{
  name=标准初始化, % ??
  description={normalized initialization},
  sort={normalized initialization},
}

\newglossaryentry{sparse_initialization}
{
  name=稀疏初始化,
  description={sparse initialization},
  sort={sparse initialization},
}

\newglossaryentry{conjugate_directions}
{
  name=共轭方向,
  description={conjugate directions},
  sort={conjugate directions},
}

\newglossaryentry{conjugate}
{
  name=共轭,
  description={conjugate},
  sort={conjugate},
}

\newglossaryentry{conditional_independent}
{
  name=条件独立,
  description={conditionally independent},
  sort={conditionally independent},
}

\newglossaryentry{ensemble_learning}
{
  name=集成学习,
  description={ensemble learning},
  sort={ensemble learning},
}

\newglossaryentry{NICE}
{
  name=非线性独立成分估计,
  description={nonlinear independent components estimation},
  sort={nonlinear independent components estimation},
  symbol={NICE}
}

\newglossaryentry{ISA}
{
  name=独立子空间分析,
  description={independent subspace analysis},
  sort={independent subspace analysis},
}

\newglossaryentry{SFA}
{
  name=慢特征分析,
  description={slow feature analysis},
  sort={slow feature analysis},
  symbol={SFA}
}

\newglossaryentry{slow_principle}
{
  name=慢性原则,
  description={slowness principle},
  sort={slowness principle},
}

\newglossaryentry{rectified_linear}
{
  name=整流线性,
  description={rectified linear},
  sort={rectified linear},
}

\newglossaryentry{rectified_linear_transformation}
{
  name=整流线性变换,
  description={rectified linear transformation},
  sort={rectified linear transformation},
}

\newglossaryentry{rectifier_network}
{
  name=整流网络,
  description={rectifier network},
  sort={rectifier network},
}

\newglossaryentry{coordinate_descent}
{
  name=坐标下降,
  description={coordinate descent},
  sort={coordinate descent},
}

\newglossaryentry{coordinate_ascent}
{
  name=坐标上升,
  description={coordinate ascent},
  sort={coordinate ascent},
}

\newglossaryentry{block_coordinate_descent}
{
  name=块坐标下降,
  description={block coordinate descent},
  sort={block coordinate descent},
}

\newglossaryentry{pretraining}
{
  name=预训练,
  description={pretraining},
  sort={pretraining},
}

\newglossaryentry{unsupervised_pretraining}
{
  name=无监督预训练,
  description={unsupervised pretraining},
  sort={unsupervised pretraining},
}

\newglossaryentry{greedy_layer_wise_unsupervised_pretraining}
{
  name=贪心逐层无监督预训练,
  description={greedy layer-wise unsupervised pretraining},
  sort={greedy layer-wise unsupervised pretraining},
}

\newglossaryentry{greedy_layer_wise_pretraining}
{
  name=贪心逐层预训练,
  description={greedy layer-wise pretraining},
  sort={greedy layer-wise pretraining},
}

\newglossaryentry{layer_wise}
{
  name=逐层的,
  description={layer-wise},
  sort={layer-wise},
}

\newglossaryentry{greedy_algorithm}
{
  name=贪心算法,
  description={greedy algorithm},
  sort={greedy algorithm},
}

\newglossaryentry{greedy}
{
  name=贪心,
  description={greedy},
  sort={greedy},
}

\newglossaryentry{fine_tuning}
{
  name=精调,
  description={fine-tuning},
  sort={fine-tuning},
}

\newglossaryentry{greedy_supervised_pretraining}
{
  name=贪心监督预训练,
  description={greedy supervised pretraining},
  sort={greedy supervised pretraining},
}

\newglossaryentry{continuation_method}
{
  name=延拓法,
  description={continuation method},
  sort={continuation method},
}

\newglossaryentry{curriculum_learning}
{
  name=课程学习,
  description={curriculum learning},
  sort={curriculum learning},
}

\newglossaryentry{shaping}
{
  name=塑造, % ? 整形
  description={shaping},
  sort={shaping},
}

\newglossaryentry{stochastic_curriculum}
{
  name=随机课程,
  description={stochastic curriculum},
  sort={stochastic curriculum},
}

\newglossaryentry{recall}
{
  name=召回率,
  description={recall},
  sort={recall},
}

\newglossaryentry{coverage}
{
  name=覆盖,
  description={coverage},
  sort={coverage},
}

\newglossaryentry{hyperparameter_optimization}
{
  name=超参数优化,
  description={hyperparameter optimization},
  sort={hyperparameter optimization},
}

\newglossaryentry{hyperparameter}
{
  name=超参数,
  description={hyperparameter},
  sort={hyperparameter}
}

\newglossaryentry{grid_search}
{
  name=网格搜索,
  description={grid search},
  sort={grid search},
}

\newglossaryentry{finite_difference}
{
  name=有限差分,
  description={finite difference},
  sort={finite difference},
}

\newglossaryentry{centered_difference}
{
  name=中央差分,
  description={centered difference},
  sort={centered difference},
}

\newglossaryentry{e_step}
{
  name=E步,
  description={expectation step},
  sort={expectation step},
  symbol={E step}
}

\newglossaryentry{m_step}
{
  name=M步,
  description={maximization step},
  sort={maximization step},
  symbol={M step}
}

\newglossaryentry{euler_lagrange_eqn}
{
  name=欧拉-拉格朗日方程,
  description={Euler-Lagrange Equation},
  sort={Euler-Lagrange Equation},
}

\newglossaryentry{lagrange_multi}
{
  name=拉格朗日乘子,
  description={Lagrange multiplier},
  sort={Lagrange multiplier},
}

\newglossaryentry{ESN}
{
  name=回声状态网络,
  description={echo state network},
  sort={echo state network},
  symbol={ESN}
}

\newglossaryentry{liquid_state_machines}
{
  name=流体状态机,
  description={liquid state machine},
  sort={liquid state machine},
}

\newglossaryentry{reservoir_computing}
{
  name=储层计算,
  description={reservoir computing},
  sort={reservoir computing},
}

\newglossaryentry{spectral_radius}
{
  name=谱半径,
  description={spectral radius},
  sort={spectral radius},
}

\newglossaryentry{contractive}
{
  name=收缩,
  description={contractive},
  sort={contractive},
}

\newglossaryentry{long_term_dependency}
{
  name=长期依赖,
  description={long-term dependency},
  sort={long-term dependency},
}

\newglossaryentry{skip_connection}
{
  name=跳跃连接,
  description={skip connection},
  sort={skip connection},
}

\newglossaryentry{leaky_unit}
{
  name=渗漏单元,
  description={leaky unit},
  sort={leaky unit},
}

\newglossaryentry{gated_rnn}
{
  name=门控RNN,
  description={gated RNN},
  sort={gated RNN},
}

\newglossaryentry{gated}
{
  name=门控,
  description={gated},
  sort={gated},
}

\newglossaryentry{convolution}
{
  name=卷积,
  description={convolution},
  sort={convolution},
}

\newglossaryentry{input}
{
  name=输入,
  description={input},
  sort={input},
}

\newglossaryentry{input_distribution}
{
  name=输入分布,
  description={input distribution},
  sort={input distribution},
}

\newglossaryentry{output}
{
  name=输出,
  description={output},
  sort={output},
}

\newglossaryentry{feature_map}
{
  name=特征映射,
  description={feature map},
  sort={feature map},
}

\newglossaryentry{flip}
{
  name=翻转,
  description={flip},
  sort={flip},
}

\newglossaryentry{cross_correlation}
{
  name=互相关函数,
  description={cross-correlation},
  sort={cross-correlation},
}

\newglossaryentry{Toeplitz_matrix}
{
  name=Toeplitz矩阵,
  description={Toeplitz matrix},
  sort={Toeplitz matrix},
}

\newglossaryentry{doubly_block_circulant_matrix}
{
  name=双重块循环矩阵,
  description={doubly block circulant matrix},
  sort={doubly block circulant matrix},
}

\newglossaryentry{sparse_interactions}
{
  name=稀疏交互,
  description={sparse interactions},
  sort={sparse interactions},
}

\newglossaryentry{equivariant_representations}
{
  name=等变表示,
  description={equivariant representations},
  sort={equivariant representations},
}

\newglossaryentry{sparse_connectivity}
{
  name=稀疏连接,
  description={sparse connectivity},
  sort={sparse connectivity},
}

\newglossaryentry{sparse_weights}
{
  name=稀疏权重,
  description={sparse weights},
  sort={sparse weights},
}

\newglossaryentry{receptive_field}
{
  name=接受域,
  description={receptive field},
  sort={receptive field},
}

\newglossaryentry{tied_weights}
{
  name=绑定的权值,
  description={tied weights},
  sort={tied weights},
}

\newglossaryentry{equivariance}
{
  name=等变,
  description={equivariance},
  sort={equivariance},
}

\newglossaryentry{detector_stage}
{
  name=探测阶,
  description={detector stage},
  sort={detector stage},
}

\newglossaryentry{symbolic_representation}
{
  name=符号表示,
  description={symbolic representation},
  sort={symbolic representation},
}

\newglossaryentry{pooling_funciton}
{
  name=池化函数,
  description={pooling function},
  sort={pooling function},
}

\newglossaryentry{max_pooling}
{
  name=最大池化,
  description={max pooling},
  sort={max pooling},
}

\newglossaryentry{pool}
{
  name=池,
  description={pool},
  sort={max pool},
}

\newglossaryentry{invariant}
{
  name=不变,
  description={invariant},
  sort={invariant},
}

\newglossaryentry{permutation_invariant}
{
  name=置换不变性,
  description={permutation invariant},
  sort={permutation invariant},
}

\newglossaryentry{stride}
{
  name=步幅,
  description={stride},
  sort={stride},
}

\newglossaryentry{downsampling}
{
  name=降采样,
  description={downsampling},
  sort={downsampling},
}

\newglossaryentry{valid}
{
  name=有效,
  description={valid},
  sort={valid},
}

\newglossaryentry{same}
{
  name=相同,
  description={same},
  sort={same},
}

\newglossaryentry{full}
{
  name=全,
  description={full},
  sort={full},
}

\newglossaryentry{unshared_convolution}
{
  name=非共享卷积,
  description={unshared convolution},
  sort={unshared convolution},
}

\newglossaryentry{tiled_convolution}
{
  name=拼贴卷积,
  description={tiled convolution},
  sort={tiled convolution},
}

\newglossaryentry{recurrent_convolutional_network}
{
  name=循环卷积网络,
  description={recurrent convolutional network},
  sort={recurrent convolutional network},
}

\newglossaryentry{Fourier_transform}
{
  name=傅立叶变换,
  description={Fourier transform},
  sort={Fourier transform},
}

\newglossaryentry{separable}
{
  name=可分离的,
  description={separable},
  sort={separable},
}

\newglossaryentry{primary_visual_cortex}
{
  name=初级视觉皮层,
  description={primary visual cortex},
  sort={primary visual cortex},
}

\newglossaryentry{simple_cells}
{
  name=简单细胞,
  description={simple cell},
  sort={simple cell},
}

\newglossaryentry{complex_cells}
{
  name=复杂细胞,
  description={complex cell},
  sort={complex cell},
}

\newglossaryentry{fovea}
{
  name=中央凹,
  description={fovea},
  sort={fovea},
}

\newglossaryentry{saccade}
{
  name=扫视,
  description={saccade},
  sort={saccade},
}

\newglossaryentry{TDNNs}
{
  name=时延神经网络,
  description={time delay neural network},
  sort={time delay neural network},
  symbol={TDNN}
}

\newglossaryentry{reverse_correlation}
{
  name=反向相关,
  description={reverse correlation},
  sort={reverse correlation},
}

\newglossaryentry{Gabor_function}
{
  name=Gabor函数,
  description={Gabor function},
  sort={Gabor function},
}

\newglossaryentry{quadrature_pair}
{
  name=象限对,
  description={quadrature pair},
  sort={quadrature pair},
}

\newglossaryentry{gated_recurrent_unit}
{
  name=门控循环单元,
  description={gated recurrent unit},
  sort={gated recurrent unit},
  symbol={GRU}
}

\newglossaryentry{gated_recurrent_net}
{
  name=门控循环网络,
  description={gated recurrent net},
  sort={gated recurrent net}, 
}

\newglossaryentry{forget_gate}
{
  name=遗忘门,
  description={forget gate},
  sort={forget gate},
}

\newglossaryentry{clipping_gradient}
{
  name=截断梯度,
  description={clipping the gradient},
  sort={clipping the gradient},
}

\newglossaryentry{memory_network}
{
  name=记忆网络,
  description={memory network},
  sort={memory network},
}

\newglossaryentry{NTM}
{
  name=神经网络图灵机,
  description={neural Turing machine},
  sort={neural Turing machine},
  symbol={NTM}
}

\newglossaryentry{fine_tune}
{
  name=精调,
  description={fine-tune},
  sort={fine-tune},
}

\newglossaryentry{explaining_away}
{
  name=explaining away,
  description={explaining away},
  sort={explaining away},
}

\newglossaryentry{explaining_away_effect}
{
  name=explaining away作用,
  description={explaining away effect},
  sort={explaining away effect},
}

\newglossaryentry{common_cause}
{
  name=共因,
  description={common cause},
  sort={common cause},
}

\newglossaryentry{code}
{
  name=编码,
  description={code},
  sort={code},
}

\newglossaryentry{recirculation}
{
  name=再循环,
  description={recirculation},
  sort={recirculation},
}

\newglossaryentry{undercomplete}
{
  name=欠完备,
  description={undercomplete},
  sort={undercomplete},
}

\newglossaryentry{complete_graph}
{
  name=完全图,
  description={complete graph},
  sort={complete graph},
}

\newglossaryentry{underdetermined}
{
  name=欠定的,
  description={underdetermined},
  sort={underdetermined},
}

\newglossaryentry{overcomplete}
{
  name=过完备,
  description={overcomplete},
  sort={overcomplete},
}

\newglossaryentry{denoising}
{
  name=去噪,
  description={denoising},
  sort={denoising},
}

\newglossaryentry{denoise}
{
  name=去噪,
  description={denoise}
}

\newglossaryentry{DAE}
{
  name=去噪自编码器,
  description={denoising autoencoder},
  sort={denoising autoencoder},
  symbol={DAE}
}

\newglossaryentry{CAE}
{
  name=收缩自编码器,
  description={contractive autoencoder},
  sort={contractive autoencoder},
  symbol={CAE}
}

\newglossaryentry{reconstruction_error}
{
  name=重构误差,
  description={reconstruction error},
  sort={reconstruction error},
}

\newglossaryentry{gradient_field}
{
  name=梯度场,
  description={gradient field},
  sort={gradient field},
}

\newglossaryentry{denoising_score_matching}
{
  name=去噪得分匹配,
  description={denoising score matching},
  sort={denoising score matching},
}

\newglossaryentry{score}
{
  name=得分,
  description={score},
  sort={score},
}

\newglossaryentry{tangent_plane}
{
  name=切平面,
  description={tangent plane},
  sort={tangent plane},
}

\newglossaryentry{nearest_neighbor_graph}
{
  name=最近邻图,
  description={nearest neighbor graph},
  sort={nearest neighbor graph},
}

\newglossaryentry{embedding}
{
  name=嵌入,
  description={embedding},
  sort={embedding},
}

\newglossaryentry{PSD}
{
  name=预测稀疏分解,
  description={predictive sparse decomposition},
  sort={predictive sparse decomposition},
  symbol={PSD}
}

\newglossaryentry{learned_approximate_inference}
{
  name=学习近似推断,
  description={learned approximate inference},
  sort={learned approximate inference},
}

\newglossaryentry{approximate_inference}
{
  name=近似推断,
  description={approximate inference},
  sort={approximate inference},
}

\newglossaryentry{information_retrieval}
{
  name=信息检索,
  description={information retrieval},
  sort={information retrieval},
}

\newglossaryentry{semantic_hashing}
{
  name=语义哈希,
  description={semantic hashing},
  sort={semantic hashing},
}

\newglossaryentry{dimensionality_reduction}
{
  name=降维,
  description={dimensionality reduction},
  sort={dimensionality reduction},
}

\newglossaryentry{helmholtz_machine}
{
  name=Helmholtz机,
  description={Helmholtz machine},
  sort={Helmholtz machine},
}

\newglossaryentry{contrastive_divergence}
{
  name=对比散度,
  description={contrastive divergence},
  symbol={CD},
  sort={contrastive divergence},
}

\newglossaryentry{language_model}
{
  name=语言模型,
  description={language model},
  sort={language model},
}

\newglossaryentry{token}
{
  name=标记,
  description={token},
  sort={token},
}

\newglossaryentry{unigram}
{
  name=一元语法,
  description={unigram},
  sort={unigram},
}

\newglossaryentry{bigram}
{
  name=二元语法,
  description={bigram},
  sort={bigram},
}

\newglossaryentry{trigram}
{
  name=三元语法,
  description={trigram},
  sort={trigram},
}

\newglossaryentry{smoothing}
{
  name=平滑,
  description={smoothing},
  sort={smoothing},
}

\newglossaryentry{NLM}
{
  name=神经语言模型,
  description={Neural Language Model},
  sort={Neural Language Model},
  symbol={NLM}
}

\newglossaryentry{cascade}
{
  name=级联,
  description={cascade},
  sort={cascade},
}

\newglossaryentry{policy_gradient}
{
  name=策略梯度,
  description={policy gradient},
  sort={policy gradient},
}

\newglossaryentry{DGM}
{
  name=深度生成模型,
  description={deep generative model},
  sort={deep generative model},
}

\newglossaryentry{model}
{
  name=模型,
  description={model},
  sort={model},
}

\newglossaryentry{layer}
{
  name=层,
  description={layer},
  sort={layer},
}

\newglossaryentry{greedy_unsupervised_pretraining}
{
  name=贪心无监督预训练,
  description={greedy unsupervised pretraining},
  sort={greedy unsupervised pretraining},
}

\newglossaryentry{semi_supervised_learning}
{
  name=半监督学习,
  description={semi-supervised learning},
  sort={semi-supervised learning},
}

\newglossaryentry{supervised_model}
{
  name=监督模型,
  description={supervised model},
  sort={supervised model},
}

\newglossaryentry{word_embeddings}
{
  name=词嵌入,
  description={word embedding},
  sort={word embedding},
}

\newglossaryentry{one_hot}
{
  name=独热码,
  description={one-hot},
  sort={one-hot},
}

\newglossaryentry{supervised_pretraining}
{
  name=监督预训练,
  description={supervised pretraining},
  sort={supervised pretraining},
}

\newglossaryentry{transfer_learning}
{
  name=迁移学习,
  description={transfer learning},
  sort={transfer learning},
}

\newglossaryentry{learner}
{
  name=学习器,
  description={learner},
  sort={learner},
}

\newglossaryentry{multitask_learning}
{
  name=多任务学习,
  description={multitask learning},
  sort={multitask learning},
}

\newglossaryentry{domain_adaption}
{
  name=领域自适应,
  description={domain adaption}  ,
  sort={domain adaption}  ,
}

\newglossaryentry{concept_drift}
{
  name=概念漂移,
  description={concept drift},
  sort={concept drift},
}

\newglossaryentry{one_shot_learning}
{
  name=一次学习,
  description={one-shot learning},
  sort={one-shot learning},
}

\newglossaryentry{zero_shot_learning}
{
  name=零次学习,
  description={zero-shot learning},
  sort={zero-shot learning},
}

\newglossaryentry{zero_data_learning}
{
  name=零数据学习,
  description={zero-data learning},
  sort={zero-data learning},
}

\newglossaryentry{multimodal_learning}
{
  name=多模态学习,
  description={multimodal learning},
  sort={multimodal learning},
}

\newglossaryentry{generative_adversarial_networks}
{
  name=生成式对抗网络,
  description={generative adversarial network},
  sort={generative adversarial network},
  symbol={GAN}
}

\newglossaryentry{generative_adversarial_framework}
{
  name=生成式对抗框架,
  description={generative adversarial framework},
  sort={generative adversarial framework},
}

\newglossaryentry{GAN}
{
  name=生成式对抗网络,
  description={generative adversarial network},
  sort={generative adversarial network},
  symbol={GAN}
}

\newglossaryentry{feedforward_classifier}
{
  name=前馈分类器,
  description={feedforward classifier},
  sort={feedforward classifier},
}

\newglossaryentry{sum_product_network}
{
  name=和-积网络,
  description={sum-product network},
  sort={sum-product network},
  symbol={SPN}
}

\newglossaryentry{deep_circuit}
{
  name=深度电路,
  description={deep circuit},
  sort={deep circuit},
}

\newglossaryentry{shadow_circuit}
{
  name=浅度电路,
  description={shadow circuit},
  sort={shadow circuit},
}

\newglossaryentry{linear_classifier}
{
  name=线性分类器,
  description={linear classifier},
  sort={linear classifier},
}

\newglossaryentry{positive_phase}
{
  name=正相,
  description={positive phase},
  sort={positive phase},
}

\newglossaryentry{negative_phase}
{
  name=负相,
  description={negative phase},
  sort={negative phase},
}

\newglossaryentry{leibniz_rule}
{
  name=莱布尼兹法则,
  description={Leibniz's rule},
  sort={Leibniz's rule},
}

\newglossaryentry{lebesgue_integrable}
{
  name=勒贝格可积,
  description={Lebesgue-integrable},
  sort={Lebesgue-integrable},
}

\newglossaryentry{spurious_modes}
{
  name=虚假模态,
  description={spurious modes},
  sort={spurious modes},
}

\newglossaryentry{SML}
{
  name=随机最大似然,
  symbol={SML},
  description={stochastic maximum likelihood},
  sort={stochastic maximum likelihood},
}

\newglossaryentry{persistent_contrastive_divergence}
{
  name=持续性对比散度,
  description={persistent contrastive divergence},
  sort={persistent contrastive divergence},
  symbol={PCD}
}

\newglossaryentry{FPCD}
{
  name=快速持续性对比散度,
  symbol={FPCD},
  description={fast persistent contrastive divergence},
  sort={fast persistent contrastive divergence},
}

\newglossaryentry{pseudolikelihood}
{
  name=伪似然,
  description={pseudolikelihood},
  sort={pseudolikelihood},
}

\newglossaryentry{generalized_pseudolikelihood_estimator}
{
  name=广义伪似然估计,
  description={generalized pseudolikelihood estimator},
  sort={generalized pseudolikelihood estimator},
}

\newglossaryentry{GSM}
{
  name=广义得分匹配,
  description={generalized score matching},
  sort={generalized score matching},
  symbol={GSM}
}

\newglossaryentry{ratio_matching}
{
  name=比率匹配,
  description={ratio matching},
  sort={ratio matching},
}

\newglossaryentry{NCE}
{
  name=噪声对比估计,
  description={noise-contrastive estimation},
  sort={noise-contrastive estimation},
  symbol={NCE}
}

\newglossaryentry{noise_distribution}
{
  name=噪声分布,
  description={noise distribution},
  sort={noise distribution},
}

\newglossaryentry{noise}
{
  name=噪声,
  description={noise},
  sort={noise},
}

\newglossaryentry{self_contrastive_estimation}
{
  name=自对比估计,
  description={self-contrastive estimation},
  sort={self-contrastive estimation},
}

\newglossaryentry{iid_chap18}
{
  name=独立同分布,
  description={independent identically distributed},
  sort={independent identically distributed},
  symbol={i.i.d.}
}

\newglossaryentry{AIS}
{
  name=退火重要采样,
  description={annealed importance sampling},
  sort={annealed importance sampling},
  symbol={AIS}
}

\newglossaryentry{bridge_sampling}
{
  name=桥式采样,
  description={bridge sampling},
  sort={bridge sampling},
}

\newglossaryentry{linked_importance_sampling}
{
  name=链接重要采样,
  description={linked importance sampling},
  sort={linked importance sampling},
}

\newglossaryentry{ASIC}
{
  name=专用集成电路,
  description={application-specific integrated circuit},
  sort={application-specific integrated circuit},
  symbol={ASIC}
}

\newglossaryentry{FPGA}
{
  name=现场可编程门阵列,
  description={field programmable gated array},
  sort={field programmable gated array},
  symbol={FPGA}
}

\newglossaryentry{scalar}
{
  name=标量,
  description={scalar},
  sort={scalar},
}

\newglossaryentry{vector}
{
  name=向量,
  description={vector},
  sort={vector},
}

\newglossaryentry{matrix}
{
  name=矩阵,
  description={matrix},
  sort={matrix},
}

\newglossaryentry{tensor}
{
  name=张量,
  description={tensor},
  sort={tensor},
}

\newglossaryentry{dot_product}
{
  name=点积,
  description={dot product},
  sort={dot product},
}

\newglossaryentry{inner_product}
{
  name=内积,
  description={inner product},
  sort={inner product}
}

\newglossaryentry{square}
{
  name=方阵,
  description={square},
  sort={square},
}

\newglossaryentry{singular}
{
  name=奇异的,
  description={singular},
  sort={singular},
}

\newglossaryentry{norm}
{
  name=范数,
  description={norm},
  sort={norm},
}

\newglossaryentry{triangle_inequality}
{
  name=三角不等式,
  description={triangle inequality},
  sort={triangle inequality},
}

\newglossaryentry{euclidean_norm}
{
  name=欧几里得范数,
  description={Euclidean norm},
  sort={Euclidean norm},
}

\newglossaryentry{max_norm}
{
  name=max 范数,
  description={max norm},
  sort={max norm},
}

\newglossaryentry{frobenius_norm}
{
  name=Frobenius 范数,
  description={Frobenius norm},
  sort={Frobenius norm},
}

\newglossaryentry{diagonal_matrix}
{
  name=对角矩阵,
  description={diagonal matrix},
  sort={diagonal matrix},
}

\newglossaryentry{symmetric}
{
  name=对称,
  description={symmetric},
  sort={symmetric},
}

\newglossaryentry{unit_vector}
{
  name=单位向量,
  description={unit vector},
  sort={unit vector},
}

\newglossaryentry{unit_norm}
{
  name=单位范数,
  description={unit norm},
  sort={unit norm},
}

\newglossaryentry{orthogonal}
{
  name=正交,
  description={orthogonal},
  sort={orthogonal},
}

\newglossaryentry{orthogonal_matrix}
{
  name=正交矩阵,
  description={orthogonal matrix},
  sort={orthogonal matrix},
}

\newglossaryentry{orthonormal}
{
  name=标准正交,
  description={orthonormal},
  sort={orthonormal},
}

\newglossaryentry{eigendecomposition}
{
  name=特征分解,
  description={eigendecomposition},
  sort={eigendecomposition},
}

\newglossaryentry{eigenvector}
{
  name=特征向量,
  description={eigenvector},
  sort={eigenvector},
}

\newglossaryentry{eigenvalue}
{
  name=特征值,
  description={eigenvalue},
  sort={eigenvalue},
}

\newglossaryentry{decompose}
{
  name=分解,
  description={decompose},
  sort={decompose},
}

\newglossaryentry{P_D}
{
  name=正定,
  description={positive definite},
  sort={positive definite},
}

\newglossaryentry{ND}
{
  name=负定,
  description={negative definite},
  sort={negative definite},
}

\newglossaryentry{NSD}
{
  name=半负定,
  description={negative semidefinite},
  sort={negative semidefinite},
}

\newglossaryentry{P_SD}
{
  name=半正定,
  description={positive semidefinite},
  sort={positive semidefinite},
}

\newglossaryentry{SVD}
{
  name=奇异值分解,
  description={singular value decomposition},
  sort={singular value decomposition},
  symbol={SVD}
}

\newglossaryentry{Svalue}
{
  name=奇异值,
  description={singular value},
  sort={singular value},
}

\newglossaryentry{Svector}
{
  name=奇异向量,
  description={singular vector},
  sort={singular vector},
}

\newglossaryentry{left_Svector}
{
  name=左奇异向量,
  description={left singular vector},
  sort={left singular vector},
}

\newglossaryentry{right_Svector}
{
  name=右奇异向量,
  description={right singular vector},
  sort={right singular vector},
}

\newglossaryentry{left_Evector}
{
  name=左特征向量,
  description={left eigenvector},
  sort={left eigenvector},
}

\newglossaryentry{right_Evector}
{
  name=右特征向量,
  description={right eigenvector},
  sort={right eigenvector},
}

\newglossaryentry{Moore}
{
  name=Moore-Penrose 伪逆,
  description={Moore-Penrose pseudoinverse},
  sort={Moore-Penrose pseudoinverse},
}

\newglossaryentry{identity_matrix}
{
  name=单位矩阵,
  description={identity matrix},
  sort={identity matrix},
}

\newglossaryentry{matrix_inverse}
{
  name=矩阵逆,
  description={matrix inversion},
  sort={matrix inversion},
}

\newglossaryentry{origin}
{
  name=原点,
  description={origin},
  sort={origin},
}

\newglossaryentry{linear_combination}
{
  name=线性组合,
  description={linear combination},
  sort={linear combination},
}

\newglossaryentry{column_space}
{
  name=列空间,
  description={column space},
  sort={column space},
}

\newglossaryentry{range}
{
  name=值域,
  description={range},
  sort={range},
}

\newglossaryentry{linear_depend}
{
  name=线性相关,
  description={linear dependency},
  sort={linear dependency},
}

\newglossaryentry{linear_dependence}
{
  name=线性相关,
  description={linear dependence},
  sort={linear dependence}
}

\newglossaryentry{linearly_independent}
{
  name=线性无关,
  description={linearly independent},
  sort={linearly independent}
}

\newglossaryentry{column}
{
  name=列,
  description={column},
  sort={column},
}

\newglossaryentry{row}
{
  name=行,
  description={row},
  sort={row},
}

\newglossaryentry{span}
{
  name=生成子空间,
  description={span},
  sort={span},
}

\newglossaryentry{SLT}
{
  name=统计学习理论,
  description={statistical learning theory},
  sort={statistical learning theory},
}

\newglossaryentry{DGP}
{
  name=数据生成过程,
  description={data generating process},
  sort={data generating process},
}

\newglossaryentry{iid}
{
  name=独立同分布假设,
  description={i.i.d. assumption},
  sort={i.i.d. assumption},
}

\newglossaryentry{id}
{
  name=同分布的,
  description={identically distributed},
  sort={identically distributed},
}

\newglossaryentry{DGD}
{
  name=数据生成分布,
  description={data generating distribution},
  sort={data generating distribution},
}

\newglossaryentry{large_learning_step}
{
  name=大学习步骤,
  description={large learning step},
  sort={large learning step},
}

\newglossaryentry{OR}
{
  name=奥卡姆剃刀,
  description={Occam's razor},
  sort={Occam's razor},
}

\newglossaryentry{VC}
{
  name=Vapnik-Chervonenkis维度,
  description={Vapnik-Chervonenkis dimension},
  sort={Vapnik-Chervonenkis dimension},
  symbol={VC}
}

\newglossaryentry{unsupervised_learning_algorithm}
{
  name=无监督学习算法,
  description={unsupervised learning algorithm},
  sort={unsupervised learning algorithm},
}

\newglossaryentry{supervised_learning_algorithm}
{
  name=监督学习算法,
  description={supervised learning algorithm},
  sort={supervised learning algorithm},
}

\newglossaryentry{word_embedding}
{
  name=词嵌入,
  description={word embedding},
  sort={word embedding},
}

\newglossaryentry{shortlist}
{
  name=短列表,
  description={shortlist},
  sort={shortlist},
}

\newglossaryentry{NMT}
{
  name=神经机器翻译,
  description={Neural Machine Translation},
  sort={Neural Machine Translation},
  symbol={NMT}
}

\newglossaryentry{machine_translation}
{
  name=机器翻译,
  description={machine translation},
  sort={machine translation}
}

\newglossaryentry{recommender_system}
{
  name=推荐系统,
  description={recommender system},
  sort={recommender system},
}

\newglossaryentry{proposal_distribution}
{
  name=提议分布,
  description={proposal distribution},
  sort={proposal distribution},
}

\newglossaryentry{bag_of_words}
{
  name=词袋,
  description={bag of words},
  sort={bag of words},
}

\newglossaryentry{collaborative_filtering}
{
  name=协同过滤,
  description={collaborative filtering},
  sort={collaborative filtering},
}

\newglossaryentry{exploration}
{
  name=探索,
  description={exploration},
  sort={exploration},
}

\newglossaryentry{exploitation}
{
  name=开发,
  description={exploitation},
  sort={exploitation},
}

\newglossaryentry{bandit}
{
  name=bandit ,
  description={bandit},
  sort={bandit},
}

\newglossaryentry{contextual_bandit}
{
  name=contextual bandit ,
  description={contextual bandit},
  sort={contextual bandit},
}

\newglossaryentry{policy}
{
  name=策略,
  description={policy},
  sort={policy},
}

\newglossaryentry{relation}
{
  name=关系,
  description={relation},
  sort={relation},
}

\newglossaryentry{binary_relation}
{
  name=二元关系,
  description={binary relation},
  sort={binary relation},
}

\newglossaryentry{attribute}
{
  name=属性,
  description={attribute},
  sort={attribute},
}

\newglossaryentry{relational_database}
{
  name=关系型数据库,
  description={relational database},
  sort={relational database},
}

\newglossaryentry{link_prediction}
{
  name=链接预测,
  description={link prediction},
  sort={link prediction},
}

\newglossaryentry{word_sense_disambiguation}
{
  name=词义消歧,
  description={word-sense disambiguation},
  sort={word-sense disambiguation},
}

\newglossaryentry{error_metric}
{
  name=误差度量,
  description={error metric},
  sort={error metric},
}

\newglossaryentry{performance_metrics}
{
  name=性能度量,
  description={performance metrics},
  sort={performance metrics},
}

\newglossaryentry{transcription_system}
{
  name=转录系统,
  description={transcription system},
  sort={transcription system},
}

\newglossaryentry{BFGS}
{
  name=BFGS,
  description={BFGS},
  sort={BFGS},
}

\newglossaryentry{LBFGS}
{
  name=L-BFGS,
  description={L-BFGS},
  sort={L-BFGS},
}

\newglossaryentry{CG}
{
  name=共轭梯度,
  description={conjugate gradient},
  sort={conjugate gradient},
  symbol={CG}
}

\newglossaryentry{nonlinear_CG}
{
  name=非线性共轭梯度,
  description={nonlinear conjugate gradients},
  sort={nonlinear conjugate gradients},
}

\newglossaryentry{bayesian_inference}
{
  name=贝叶斯推断,
  description={Bayesian inference},
  sort={Bayesian inference},
}

\newglossaryentry{online_learning}
{
  name=在线学习,
  description={online learning},
  sort={online learning},
}

\newglossaryentry{layer_wise_pretraining}
{
  name=逐层预训练,
  description={layer-wise pretraining},
  sort={layer-wise pretraining},
}

\newglossaryentry{MPDBM}
{
  name=多预测深度玻尔兹曼机,
  description={multi-prediction deep Boltzmann machine},
  sort={multi-prediction deep Boltzmann machine},
  symbol={MP-DBM}
}

\newglossaryentry{GBRBM}
{
  name=Gaussian-Bernoulli RBM,
  description={Gaussian-Bernoulli RBM},
  sort={Gaussian-Bernoulli RBM},
  symbol={GB-RBM}
}

\newglossaryentry{gaussian_rbm}
{
  name=Gaussian RBM,
  description={Gaussian RBM},
  sort={Gaussian RBM}
}

\newglossaryentry{mcrbm}
{
  name=均值和协方差RBM,
  description={mean and covariance RBM},
  sort={mean and covariance RBM},
  symbol={mcRBM},
}

\newglossaryentry{mcrbm2}
{
  name=均值-协方差RBM,
  description={mean-covariance restricted Boltzmann machine},
  sort={mean-covariance restricted Boltzmann machine},
  symbol={mcRBM2},
}

\newglossaryentry{crbm}
{
  name=协方差RBM,
  description={covariance RBM},
  sort={covariance RBM},
  symbol={cRBM},
}

\newglossaryentry{mpot}
{
  name=学生$t$分布均值乘积,
  description={mean product of Student t-distribution},
  sort={mean product of Student t-distribution},
  symbol={mPoT},
}

\newglossaryentry{ssrbm}
{
  name=尖峰和平板RBM,
  description={spike and slab RBM},
  sort={spike and slab RBM},
  symbol={ssRBM},
}

\newglossaryentry{gamma_distribution}
{
  name=Gamma分布,
  description={Gamma distribution},
  sort={Gamma distribution},
}

\newglossaryentry{convolutional_bm}
{
  name=卷积玻尔兹曼机 ,
  description={convolutional Boltzmann machine},
  sort={convolutional Boltzmann machine},
}

\newglossaryentry{reparametrization_trick}
{
  name=重参数化技巧,
  description={reparametrization trick},
  sort={reparametrization trick},
}

\newglossaryentry{variance_reduction}
{
  name=方差减小,
  description={variance reduction},
  sort={variance reduction},
}

\newglossaryentry{sigmoid_bn}
{
  name=sigmoid信念网络,
  description={sigmoid Belief Network},
  sort={sigmoid Belief Network},
}

\newglossaryentry{auto_regressive_network}
{
  name=自回归网络,
  description={auto-regressive network},
  sort={auto-regressive network}
}

\newglossaryentry{generator_network}
{
  name=生成器网络,
  description={generator network},
  sort={generator network}
}

\newglossaryentry{discriminator_network}
{
  name=判别器网络,
  description={discriminator network},
  sort={discriminator network},
}

\newglossaryentry{generative_moment_matching_network}
{
  name=生成矩匹配网络,
  description={generative moment matching network},
  sort={generative moment matching network},
}

\newglossaryentry{moment_matching}
{
  name=矩匹配,
  description={moment matching},
  sort={moment matching},
}

\newglossaryentry{moment}
{
  name=矩,
  description={moment},
  sort={moment},
}

\newglossaryentry{MMD}
{
  name=最大平均偏差,
  description={maximum mean discrepancy},
  sort={maximum mean discrepancy},
  symbol={MMD}
}

\newglossaryentry{linear_auto_regressive_network}
{
  name=线性自回归网络,
  description={linear auto-regressive network},
  sort={linear auto-regressive network}
}

\newglossaryentry{neural_auto_regressive_network}
{
  name=神经自回归网络,
  description={neural auto-regressive network},
  sort={neural auto-regressive network}
}

\newglossaryentry{NADE}
{
  name=神经自回归密度估计器,
  description={neural auto-regressive density estimator},
  sort={neural auto-regressive density estimator},
  symbol={NADE}
}

\newglossaryentry{detailed_balance}
{
  name=细致平衡,
  description={detailed balance},
  sort={detailed balance},
}

\newglossaryentry{ABC}
{
  name=近似贝叶斯计算,
  description={approximate Bayesian computation},
  sort={approximate Bayesian computationA},
  symbol={ABC}
}

\newglossaryentry{visible_layer}
{
  name=可见层,
  description={visible layer},
  sort={visible layer},
}

\newglossaryentry{infinite}
{
  name=无限,
  description={infinite},
  sort={infinite},
}

\newglossaryentry{deep_model}
{
  name=深度模型,
  description={deep model},
  sort={deep model},
}

\newglossaryentry{deep_network}
{
  name=深度网络,
  description={deep network},
  sort={deep network},
}

\newglossaryentry{tolerance}
{
  name=容差,
  description={tolerance},
  sort={tolerance},
}

\newglossaryentry{learning_rate}
{
  name=学习率,
  description={learning rate},
  sort={learning rate},
}

\newglossaryentry{ss}
{
  name=尖峰和平板,
  description={spike and slab},
  sort={spike and slab},
}

\newglossaryentry{context_specific_independence}
{
  name=特定环境下的独立,
  description={context-specific independences},
  sort={context-specific independences},
}

\newglossaryentry{coparent}
{
  name=共父,
  description={coparent},
  sort={coparent},
}

\newglossaryentry{srbm}
{
  name=半受限波尔兹曼机,
  description={semi-restricted Boltzmann Machine},
  sort={semi-restricted Boltzmann Machine},
}

\newglossaryentry{underfit_regime}
{
  name=欠拟合机制,
  description={underfitting regime},
  sort={underfitting regime},
}

\newglossaryentry{overfit_regime}
{
  name=过拟合机制,
  description={overfitting regime},
  sort={overfitting regime},
}

\newglossaryentry{optimal_capacity}
{
  name=最佳容量,
  description={optimal capacity},
  sort={optimal capacity},
}

\newglossaryentry{error_bar}
{
  name=误差条,
  description={error bar},
  sort={error bar},
}

\newglossaryentry{vstructure}
{
  name=V-结构,
  description={V-structure},
  sort={V-structure},
}

\newglossaryentry{collider}
{
  name=碰撞情况,
  description={the collider case},
  sort={the collider case},
}

\newglossaryentry{epochs}
{
  name=轮数,
  description={epochs},
  sort={epochs},
}

\newglossaryentry{epoch}
{
  name=轮,
  description={epoch},
  sort={epoch},
}

\newglossaryentry{logarithmic_scale}
{
  name=对数尺度,
  description={logarithmic scale},
  sort={logarithmic scale}
}

\newglossaryentry{random_search}
{
  name=随机搜索,
  description={random search},
  sort={random search}
}

\newglossaryentry{closed_form_solution}
{
  name=闭式解,
  description={closed form solution},
  sort={closed form solution}
}

\newglossaryentry{object_recognition}
{
  name=对象识别,
  description={object recognition},
  sort={object recognition}
}

\newglossaryentry{piecewise}
{
  name=分段,
  description={piecewise},
  sort={piecewise},
}

\newglossaryentry{alternative_splicing_dataset}
{
  name=选择性剪接数据集,
  description={alternative splicing dataset},
  sort={alternative splicing dataset},
}

\newglossaryentry{hamming_distance}
{
  name=汉明距离,
  description={Hamming distance},
  sort={Hamming distance}
}

\newglossaryentry{visible_variable}
{
  name=可见变量,
  description={visible variable},
  sort={visible variable},
}

\newglossaryentry{approxi_inference}
{
  name=近似推断,
  description={approximate inference},
  sort={approximate inference},
}

\newglossaryentry{exact_inference}
{
  name=精确推断,
  description={exact inference},
  sort={exact inference},
}

\newglossaryentry{latent}
{
  name=潜在,
  description={latent},
  sort={latent},
}

\newglossaryentry{latent_layer}
{
  name=潜层,
  description={latent layer},
  sort={latent layer},
}

\newglossaryentry{knowledge_graph}
{
  name=知识图谱,
  description={knowledge graph},
  sort={knowledge graph},
}

\newglossaryentry{factors_of_variation}
{
  name=变差因素,
  description={factors of variation},
  sort={factors of variation},
}

\newglossaryentry{isomap}
{
  name=Isomap,
  description={Isomap},
  sort={isomap},
}




% title 
\title{\Huge\textbf{深度学习}}
\author{}
\date{\today}


\begin{document}
\frontmatter

\maketitle
\cleardoublepage

% From en book -B
\setlength{\parskip}{0.25 \baselineskip}
% Sean said to make figures 26 picas wide
\newlength{\figwidth}
\setlength{\figwidth}{26pc}
% Spacing between notation sections
\newlength{\notationgap}
\setlength{\notationgap}{1pc}
% From en book -E


\tableofcontents

\newpage
\input{acknowledgments.tex}
\input{notation.tex}
\mainmatter

% !Mode:: "TeX:UTF-8"
% Translator: Shenjian Zhao
\chapter{前言}
\label{chap:introduction}
远在古希腊时期,发明家就梦想着创造能思考的机器。
神话人物皮格马利翁(Pygmalion)、代达罗斯(Daedalus)和赫淮斯托斯(Hephaestus)可以被看作传说中的发明家,而加拉蒂亚(Galatea)、塔洛斯(Talos)和潘多拉(Pandora)则可以被视为人造生命\citep{ovid2004metamorphoses,sparkes1996red,1997works}。

当人类第一次构思可编程计算机时,就已经在思考计算机能否变得智能(尽管这距造出第一台计算机还有一百多年)\citep{Lovelace1842}。
如今,\firstall{AI}已经成为一个具有众多实际应用和活跃研究课题的领域,并且正在蓬勃发展。
我们期望通过智能软件自动地处理常规劳动、理解语音或图像、帮助医学诊断和支持基础科学研究。

在\gls{AI}的早期,那些对人类智力来说非常困难、但对计算机来说相对简单的问题得到迅速解决,比如,那些可以通过一系列形式化的数学规则来描述的问题。
\gls{AI}的真正挑战在于解决那些对人来说很容易执行、但很难形式化描述的任务,如识别人们所说的话或图像中的脸。对于这些问题,我们人类往往可以凭直觉轻易地解决。


针对这些比较直观的问题,本书讨论一种解决方案。
该方案可以让计算机从经验中学习,并根据层次化的概念体系来理解世界,而每个概念则通过与某些相对简单的概念之间的关系来定义。
让计算机从经验获取知识,可以避免由人类来给计算机形式化地指定它需要的所有知识。
层次化的概念让计算机构建较简单的概念来学习复杂概念。
如果绘制出这些概念如何建立在彼此之上的图,我们将得到一张``深''(层次很多)的图。
基于这个原因,我们称这种方法为\glssymbol{AI}\firstgls{DL}。

% -- 1 --

\glssymbol{AI}许多早期的成功发生在相对朴素且形式化的环境中, 而且不要求计算机具备很多关于世界的知识。
例如,IBM的深蓝(Deep Blue)国际象棋系统在1997年击败了世界冠军\ENNAME{Garry Kasparov}\citep{Hsu2002}。
显然国际象棋是一个非常简单的领域,因为它仅含有64个位置并只能以严格限制的方式移动32个棋子。
设计一种成功的国际象棋策略是巨大的成就,但向计算机描述棋子及其允许的走法并不是挑战的困难所在。
国际象棋完全可以由一个非常简短的、完全形式化的规则列表来描述,并可以容易地由程序员事先准备好。

讽刺的是,抽象和形式化的任务对人类而言是最困难的脑力任务之一,但对计算机而言却属于最容易的。
计算机早就能够打败人类最好的象棋选手,但直到最近计算机才在识别对象或语音任务中达到人类平均水平。
一个人的日常生活需要关于世界的巨量知识。
很多这方面的知识是主观的、直观的,因此很难通过形式化的方式表达清楚。
计算机需要获取同样的知识才能表现出智能。
\gls{AI}的一个关键挑战就是如何将这些非形式化的知识传达给计算机。

一些\gls{AI}项目力求将关于世界的知识用形式化的语言进行硬编码(hard-code)。
计算机可以使用逻辑推理规则来自动地理解这些形式化语言中的申明。
这就是众所周知的\gls{AI}的\firstgls{knowledge_base}方法。
这些项目没有导致重大的成功。
其中最著名的项目是Cyc \citep{Lenat-1989-book}。
Cyc包括一个\gls{inference}引擎和一个使用CycL语言描述的声明数据库。
这些声明是由人类监督者输入的。
这是一个笨拙的过程。
人们设法设计出足够复杂的形式化规则来精确地描述世界。
例如,Cyc不能理解一个关于名为\ENNAME{Fred}的人在早上剃须的故事\citep{MachineChangedWorld}。
它的推理引擎检测到故事中的不一致性:它知道人没有电气零件,但由于\ENNAME{Fred}正拿着一个电动剃须刀,它认为实体
``正在剃须的Fred''(``FredWhileShaving'')含有电气部件。
因此它产生了这样的疑问——\ENNAME{Fred}在刮胡子的时候是否仍然是一个人。

依靠硬编码的知识体系面对的困难表明,\glssymbol{AI}系统需要具备自己获取知识的能力,即从原始数据中提取模式的能力。
这种能力被称为\firstgls{ML}。
引入\gls{ML}使计算机能够解决涉及现实世界知识的问题,并能作出看似主观的决策。
比如,一个被称为\firstgls{logistic_regression}的简单\gls{ML}算法可以决定是否建议剖腹产\citep{MorYosef90}。
而同样是简单\gls{ML}算法的\firstgls{naive_bayes}则可以区分垃圾电子邮件和合法电子邮件。

% -- 2 --

这些简单的\gls{ML}算法的性能在很大程度上依赖于给定数据的\firstgls{representation}。
例如,当\gls{logistic_regression}被用于推荐剖腹产时,\glssymbol{AI}系统不直接检查患者。
相反,医生需要告诉系统几条相关的信息,诸如子宫疤痕是否存在。
表示患者的每条信息被称为一个特征。
\gls{logistic_regression}学习病人的这些特征如何与各种结果相关联。
然而,它丝毫不能影响该特征定义的方式。
如果将病人的MRI扫描作为\gls{logistic_regression}的输入,而不是医生正式的报告,它将无法作出有用的预测。
MRI扫描的单一像素与分娩过程中并发症之间的相关性微乎其微。

在整个计算机科学乃至日常生活中,对\gls{representation}的依赖都是一个普遍现象。
在计算机科学中,如果数据集合被精巧地结构化并被智能地索引,那么诸如搜索之类的操作的处理速度就可以成指数级地加快。
人们可以很容易地在阿拉伯数字的\gls{representation}下进行算术运算,但在罗马数字的\gls{representation}下运算会比较耗时。
因此,毫不奇怪,\gls{representation}的选择会对\gls{ML}算法的性能产生巨大的影响。
\figref{fig:chap1_polar}展示了一个简单的可视化例子。

\begin{figure}[!htb]
\ifOpenSource
\centerline{\includegraphics{figure.pdf}}
\else
\centerline{\includegraphics{Chapter1/figures/polar_color}}
\fi
\caption{不同表示的例子:假设我们想在散点图中画一条线来分隔两类数据。
在左图,我们使用笛卡尔坐标表示数据,这个任务是不可能的。 
右图中,我们用极坐标表示数据,可以用垂直线简单地解决这个任务。(与David Warde-Farley合作画出此图。)}
\label{fig:chap1_polar}
\end{figure}

许多\gls{AI}任务都可以通过以下方式解决:先提取一个合适的特征集,然后将这些特征提供给简单的\gls{ML}算法。
例如,对于通过声音鉴别说话者的任务来说,一个有用的特征是对其声道大小的估计。
这个特征为判断说话者是男性、女性还是儿童提供了有力线索。

然而,对于许多任务来说,我们很难知道应该提取哪些特征。
例如,假设我们想编写一个程序来检测照片中的车。
我们知道,汽车有轮子,所以我们可能会想用车轮的存在与否作为特征。
不幸的是,我们难以准确地根据像素值来描述车轮看上去像什么。
虽然车轮具有简单的几何形状,但它的图像可能会因场景而异,如落在车轮上的阴影、太阳照亮的车轮的金属零件、汽车的挡泥板或者遮挡的车轮一部分的前景物体等等。

% -- 3 --

解决这个问题的途径之一是使用\gls{ML}来发掘\gls{representation}本身,而不仅仅把\gls{representation}映射到输出。
这种方法我们称之为\firstgls{representation_learning}。
学习到的\gls{representation}往往比手动设计的\gls{representation}表现得更好。
并且它们只需最少的人工干预,就能让\glssymbol{AI}系统迅速适应新的任务。
\gls{representation_learning}算法只需几分钟就可以为简单的任务发现一个很好的特征集,对于复杂任务则需要几小时到几个月。
手动为一个复杂的任务设计特征需要耗费大量的人工时间和精力;甚至需要花费整个社群研究人员几十年的时间。

\gls{representation_learning}算法的典型例子是\firstgls{AE}。
\gls{AE}由一个\firstgls{encoder}函数和一个\firstgls{decoder}函数组合而成。
\gls{encoder}函数将输入数据转换为一种不同的\gls{representation},而\gls{decoder}函数则将这个新的\gls{representation}转换到原来的形式。
我们期望当输入数据经过\gls{encoder}和\gls{decoder}之后尽可能多地保留信息,同时希望新的\gls{representation}有各种好的特性,
这也是\gls{AE}的训练目标。
为了实现不同的特性,我们可以设计不同形式的\gls{AE}。

当设计特征或设计用于学习特征的算法时,我们的目标通常是分离出能解释观察数据的\firstgls{factors_of_variation}。
在此背景下,``因素''这个词仅指代影响的不同来源;因素通常不是乘性组合。
这些因素通常是不能被直接观察到的量。
相反,它们可能是现实世界中观察不到的物体或者不可观测的力,但会影响可观测的量。
为了对观察到的数据提供有用的简化解释或推断其原因,它们还可能以概念的形式存在于人类的思维中。
它们可以被看作数据的概念或者抽象,帮助我们了解这些数据的丰富多样性。
当分析语音记录时,\gls{factors_of_variation}包括说话者的年龄、性别、他们的口音和他们正在说的词语。
当分析汽车的图像时,\gls{factors_of_variation}包括汽车的位置、它的颜色、太阳的角度和亮度。

% -- 4 --

在许多现实的\gls{AI}应用中,困难主要源于很多\gls{factors_of_variation}影响着我们能够观察到的每一个数据。
比如,在一张包含红色汽车的图片中,其单个像素在夜间可能会非常接近黑色。
汽车轮廓的形状取决于视角。
大多数应用需要我们\emph{理清}\gls{factors_of_variation}并忽略我们不关心的因素。

显然,从原始数据中提取如此高层次、抽象的特征是非常困难的。
许多诸如说话口音这样的\gls{factors_of_variation},只能通过对数据进行复杂的、接近人类水平的理解来辨识。
这几乎与获得原问题的\gls{representation}一样困难,因此,乍一看,\gls{representation_learning}似乎并不能帮助我们。

\firstgls{DL}通过其他较简单的\gls{representation}来表达复杂\gls{representation},解决了\gls{representation_learning}中的核心问题。

\begin{figure}[!htb]
\ifOpenSource
\centerline{\includegraphics{figure.pdf}}
\else
\centerline{\includegraphics{Chapter1/figures/deep_learning}}
\fi
\caption{深度学习模型的示意图。 计算机难以理解原始感观输入数据的含义,如表示为像素值集合的图像。
将一组像素映射到对象标识的函数非常复杂。
如果直接处理,学习或评估此映射似乎是不可能的。
深度学习将所需的复杂映射分解为一系列嵌套的简单映射(每个由模型的不同层描述)来解决这一难题。
输入展示在\firstgls{visible_layer},这样命名的原因是因为它包含我们能观察到的变量。
然后是一系列从图像中提取越来越多抽象特征的\firstgls{hidden_layer}。
因为它们的值不在数据中给出,所以将这些层称为``隐藏''; 模型必须确定哪些概念有利于解释观察数据中的关系。
这里的图像是每个\gls{hidden_unit}表示的特征的可视化。
给定像素,第一层可以轻易地通过比较相邻像素的亮度来识别边缘。
有了第一\gls{hidden_layer}描述的边缘,第二\gls{hidden_layer}可以容易地搜索可识别为角和扩展轮廓的边集合。
给定第二\gls{hidden_layer}中关于角和轮廓的图像描述,第三\gls{hidden_layer}可以找到轮廓和角的特定集合来检测特定对象的整个部分。
最后,根据图像描述中包含的对象部分,可以识别图像中存在的对象。
经\citet{ZeilerFergus14}许可转载此图。
}
\label{fig:chap1_deep_learning}
\end{figure}

\gls{DL}让计算机通过较简单概念构建复杂的概念。
\figref{fig:chap1_deep_learning}展示了\gls{DL}系统如何通过组合较简单的概念(例如转角和轮廓,它们转而由边线定义)来表示图像中人的概念。
\gls{DL}模型的典型例子是前馈深度网络或\firstall{MLP}。
\gls{MLP}仅仅是一个将一组输入值映射到输出值的数学函数。
该函数由许多较简单的函数复合而成。
我们可以认为不同数学函数的每一次应用都为输入提供了新的\gls{representation}。

学习数据的正确\gls{representation}的想法是解释\gls{DL}的一个视角。
另一个视角是深度促使计算机学习一个多步骤的计算机程序。
每一层\gls{representation}都可以被认为是并行执行另一组指令之后计算机的存储器状态。
更深的网络可以按顺序执行更多的指令。
顺序指令提供了极大的能力,因为后面的指令可以参考早期指令的结果。
从这个角度上看,在某层激活函数里,并非所有信息都蕴涵着解释输入的\gls{factors_of_variation}。
\gls{representation}还存储着状态信息,用于帮助程序理解输入。
这里的状态信息类似于传统计算机程序中的计数器或指针。
它与具体的输入内容无关,但有助于模型组织其处理过程。

% -- 6 --

目前主要有两种度量模型深度的方式。
第一个观点是基于评估架构所需执行的顺序指令的数目。
假设我们将模型表示为给定输入后,计算对应输出的流程图,则可以将这张流程图中的最长路径视为模型的深度。
正如两个使用不同语言编写的等价程序将具有不同的长度;相同的函数可以被绘制为具有不同深度的流程图,其深度取决于我们可以用来作为一个步骤的函数。
\figref{fig:chap1_language}说明了语言的选择如何给相同的架构两个不同的衡量。

\begin{figure}[!htb]
\ifOpenSource
\centerline{\includegraphics{figure.pdf}}
\else
\centerline{\includegraphics{Chapter1/figures/language}}
\fi
\caption{将输入映射到输出的计算图表的示意图,其中每个节点执行一个操作。
深度是从输入到输出的最长路径的长度,但这取决于可能的计算步骤的定义。
这些图中所示的计算是\gls{logistic_regression}模型的输出,$\sigma(\Vw^T \Vx)$,其中$\sigma$是\ENNAME{logistic sigmoid}函数。
如果我们使用加法、乘法和\ENNAME{logistic sigmoid}作为我们计算机语言的元素,那么这个模型深度为三。
如果我们将\gls{logistic_regression}视为元素本身,那么这个模型深度为一。
}
\label{fig:chap1_language}
\end{figure}

另一种是在深度概率模型中使用的方法,它不是将计算图的深度视为模型深度,而是将描述概念彼此如何关联的图的深度视为模型深度。
在这种情况下,计算每个概念\gls{representation}的计算流程图的深度可能比概念本身的图更深。
这是因为系统对较简单概念的理解在给出更复杂概念的信息后可以进一步精细化。
例如,一个\glssymbol{AI}系统观察其中一只眼睛在阴影中的脸部图像时,它最初可能只看到一只眼睛。
但当检测到脸部的存在后,系统可以推断第二只眼睛也可能是存在的。
在这种情况下,概念的图仅包括两层(关于眼睛的层和关于脸的层),但如果我们根据每个概念给出的其他$n$次估计进行细化,计算的图将包括$2n$层。

% -- 7 --

由于并不总是清楚计算图的深度或概率模型图的深度哪一个是最有意义的,并且由于不同的人选择不同的最小元素集来构建相应的图,因此就像计算机程序的长度不存在单一的正确值一样,架构的深度也不存在单一的正确值。
另外,也不存在模型多么深才能被修饰为``深''的共识。
但相比传统\gls{ML},\gls{DL}研究的模型涉及更多学到功能或学到概念的组合,这点毋庸置疑。

总之, 这本书的主题——\gls{DL}是通向\gls{AI}的途径之一。
具体来说,它是\gls{ML}的一种,一种能够使计算机系统从经验和数据中得到提高的技术。
我们坚信\gls{ML}可以构建出在复杂实际环境下运行的\glssymbol{AI}系统,并且是唯一切实可行的方法。
\gls{DL}是一种特定类型的\gls{ML},具有强大的能力和灵活性,它将大千世界\gls{representation}为嵌套的层次概念体系
(由较简单概念间的联系定义复杂概念、从一般抽象概括到高级抽象表示)。
\figref{fig:chap1_venn}说明了这些不同的\glssymbol{AI}学科之间的关系。\figref{fig:chap1_which_part_learned}展示了每个学科如何工作的高层次原理。

\begin{figure}[!hbt]
\ifOpenSource
\centerline{\includegraphics{figure.pdf}}
\else
\centerline{\includegraphics[width=0.65\textwidth]{Chapter1/figures/venn}}
\fi
\caption{维恩图展示了深度学习是一种表示学习,也是一种机器学习,可以用于许多(但不是全部)\glssymbol{AI}方法。
维恩图的每个部分包括一个\glssymbol{AI}技术的示例。
}
\label{fig:chap1_venn}
\end{figure}

\begin{figure}[!htb]
\ifOpenSource
\centerline{\includegraphics{figure.pdf}}
\else
\centerline{\includegraphics{Chapter1/figures/which_part_learned}}
\fi
\caption{流程图展示了\glssymbol{AI}系统的不同部分如何在不同的\glssymbol{AI}学科中彼此相关。
阴影框表示能从数据中学习的组件。}
\label{fig:chap1_which_part_learned}
\end{figure}

\section{本书面向的读者}
\label{sec:who_should_read_this_book}

这本书对各类读者都有一定用处,但我们主要是为两类受众对象而写的。
其中一类受众对象是学习\gls{ML}的大学生(本科或研究生),包括那些已经开始职业生涯的\gls{DL}和\gls{AI}研究者。
另一类受众对象是没有\gls{ML}或统计背景但希望能快速地掌握这方面知识并在他们的产品或平台中使用\gls{DL}的软件工程师。
\gls{DL}在许多软件领域都已被证明是有用的,包括计算机视觉、语音和音频处理、自然语言处理、机器人技术、生物信息学和化学、电子游戏、搜索引擎、网络广告和金融。

\begin{figure}[!htb]
\ifOpenSource
\centerline{\includegraphics{figure.pdf}}
\else
\centerline{\includegraphics[width=0.65\textwidth]{Chapter1/figures/dependency}}
\fi
\caption{本书的高层组织。
从一章到另一章的箭头表示前一章是理解后一章的必备内容。}
\label{fig:chap1_dependency}
\end{figure}

% -- 8 --

为了最好地服务各类读者,这本书被组织为三个部分。
第一部分介绍基本的数学工具和\gls{ML}的概念。
第二部分介绍本质上已解决的技术和最成熟的\gls{DL}算法。
第三部分讨论某些具有展望性的想法,它们被广泛地认为是\gls{DL}未来的研究重点。

读者可以随意跳过不感兴趣或与自己背景不相关的部分。
熟悉线性代数、概率和基本\gls{ML}概念的读者可以跳过第一部分,例如,当读者只是想实现一个能工作的系统则不需要阅读超出第二部分的内容。
为了帮助读者选择章节,\figref{fig:chap1_dependency}展示了这本书的高层组织结构的流程图。

% -- 10 --

我们假设所有读者都具备计算机科学背景。
也假设读者熟悉编程,并且对计算的性能问题、复杂性理论、入门级微积分和一些图论术语有基本的了解。
% 
\section{深度学习的历史趋势}
\label{sec:historical_trends_in_deep_learning}
通过历史背景了解\gls{DL}是最简单的方式。
这里我们仅指出\gls{DL}的几个关键趋势,而不是提供其详细的历史:
\begin{itemize}
 \item \gls{DL}有着悠久而丰富的历史,但随着许多不同哲学观点的渐渐消逝,与之对应的名称也渐渐尘封。
 \item 随着可用的训练数据量不断增加,\gls{DL}变得更加有用。
 \item 随着时间的推移,针对\gls{DL}的计算机软硬件基础设施都有所改善,\gls{DL}模型的规模也随之增长。
 \item 随着时间的推移,\gls{DL}已经解决日益复杂的应用,并且精度不断提高。
\end{itemize}

\subsection{神经网络的众多名称和命运变迁}
\label{sec:the_many_names_and_changing_fortunes_of_neural_networks}

我们期待这本书的许多读者都听说过\gls{DL}这一激动人心的新技术,并对一本书提及一个新兴领域的``历史''而感到惊讶。
事实上,\gls{DL}的历史可以追溯到20世纪40年代。
\gls{DL}\emph{看似}是一个全新的领域,只不过因为在目前流行的前几年它是相对冷门的,同时也因为它被赋予了许多不同的名称(其中大部分已经不再使用),最近才成为众所周知的``\gls{DL}''。
这个领域已经更换了很多名称,它反映了不同的研究人员和不同观点的影响。

全面地讲述\gls{DL}的历史超出了本书的范围。
然而,一些基本的背景对理解\gls{DL}是有用的。
一般来说,目前为止\gls{DL}已经经历了三次发展浪潮:20世纪40年代到60年代\gls{DL}的雏形出现在\firstgls{cybernetics}中,20世纪80年代到90年代\gls{DL}表现为\firstgls{connectionism},直到2006年,才真正以\gls{DL}之名复兴。
\figref{fig:chap1_cybernetics_connectionism_ngrams_color}给出了定量的展示。

\begin{figure}[!htb]
\ifOpenSource
\centerline{\includegraphics{figure.pdf}}
\else
\centerline{\includegraphics{Chapter1/figures/cybernetics_connectionism_ngrams_color}}
\fi
\caption{根据Google图书中短语``\gls{cybernetics}''、``\gls{connectionism}''或``\gls{NN}''频率衡量的\gls{ANN}研究的历史浪潮(图中展示了三次浪潮的前两次,第三次最近才出现)。
第一次浪潮开始于20世纪40年代到20世纪60年代的\gls{cybernetics},随着生物学习理论的发展\citep{McCulloch43,Hebb49}
和第一个模型的实现(如感知机~\citep{Rosenblatt-1958}) ,能实现单个神经元的训练。
第二次浪潮开始于1980-1995年间的\gls{connectionism}方法,可以使用反向传播\citep{Rumelhart86b-small} 训练具有一两个\gls{hidden_layer}的神经网络。
当前第三次浪潮,也就是深度学习,大约始于2006年\citep{Hinton06,Bengio-nips-2006-small,ranzato-07-small},并且现在在2016年以书的形式出现。
另外两次浪潮类似地出现在书中的时间比相应的科学活动晚得多。
}
\label{fig:chap1_cybernetics_connectionism_ngrams_color}
\end{figure}

% -- 12 --

我们今天知道的一些最早的学习算法,是旨在模拟生物学习的计算模型,即大脑怎样学习或为什么能学习的模型。
其结果是\gls{DL}以\firstall{ANN}之名而淡去。
彼时,\gls{DL}模型被认为是受生物大脑(无论人类大脑或其他动物的大脑)所启发而设计出来的系统。
尽管有些\gls{ML}的\gls{NN}有时被用来理解大脑功能\citep{hinton1991lesioning},但它们一般都没有被设计成生物功能的真实模型。
\gls{DL}的神经观点受两个主要思想启发。
一个想法是大脑作为例子证明智能行为是可能的,因此,概念上,建立智能的直接途径是逆向大脑背后的计算原理,并复制其功能。
另一种看法是,理解大脑和人类智能背后的原理也非常有趣,因此\gls{ML}模型除了解决工程应用的能力, 如果能让人类对这些基本的科学问题有进一步的认识也将会很有用。

% -- 13 --
  
现代术语``\gls{DL}''超越了目前\gls{ML}模型的神经科学观点。
学习\emph{多层次组合}这一更普遍的原则更加吸引人,这可以应用于\gls{ML}框架而不必受神经系统启发。
 
 
现代\gls{DL}的最早前身是从神经科学的角度出发的简单线性模型。
这些模型被设计为使用一组$\Sn$个输入$\Sx_1, \dots ,\Sx_n$并将它们与一个输出$\Sy$相关联。 
这些模型希望学习一组权重$\Sw_1, \dots, \Sw_n $,并计算它们的输出$f(\Vx, \Vw) = \Sx_1 \Sw_1 + \dots + \Sx_n \Sw_n$。
如\figref{fig:chap1_cybernetics_connectionism_ngrams_color}所示,这第一波\gls{NN}研究浪潮被称为\gls{cybernetics}。

\ENNAME{McCulloch-Pitts}神经元\citep{McCulloch43}是脑功能的早期模型。
该线性模型通过检验函数$f(\Vx,\Vw)$的正负来识别两种不同类别的输入。
显然,模型的权重需要正确设置后才能使模型的输出对应于期望的类别。
这些权重可以由操作人员设定。
在20世纪50年代,感知机\citep{Rosenblatt-1956,Rosenblatt-1958}成为第一个能根据每个类别的输入\gls{example}来学习权重的模型。
约在同一时期,\textbf{自适应线性单元}(adaptive linear element, ADALINE)简单地返回函数$f(\Vx)$本身的值来预测一个实数\citep{Widrow60},并且它还可以学习从数据预测这些数。

这些简单的学习算法大大影响了\gls{ML}的现代景象。
用于调节ADALINE权重的训练算法是被称为\firstgls{SGD}的一种特例。
稍加改进后的\gls{SGD}算法仍然是当今\gls{DL}的主要训练算法。

基于感知机和ADALINE中使用的函数$f(\Vx, \Vw)$的模型被称为\firstgls{linear_model}。
尽管在许多情况下,这些模型以不同于原始模型的方式进行\emph{训练},但仍是目前最广泛使用的\gls{ML}模型。

\gls{linear_model}有很多局限性。
最著名的是,它们无法学习异或(XOR)函数,即$f([0,1], \Vw) = 1$和$f([1,0], \Vw)=1$,但$f([1,1], \Vw)=0$和$f([0,0],\Vw)= 0$。
观察到\gls{linear_model}这个缺陷的批评者对受生物学启发的学习普遍地产生了抵触\citep{Minsky69}。
这导致了\gls{NN}热潮的第一次大衰退。

现在,神经科学被视为\gls{DL}研究的一个重要灵感来源,但它已不再是该领域的主要指导。

% -- 14 --

如今神经科学在\gls{DL}研究中的作用被削弱,主要原因是我们根本没有足够的关于大脑的信息来作为指导去使用它。
要获得对被大脑实际使用算法的深刻理解,我们需要有能力同时监测(至少是)数千相连神经元的活动。
我们不能够做到这一点,所以我们甚至连大脑最简单、最深入研究的部分都还远远没有理解\citep{olshausen:2005}。

神经科学已经给了我们依靠单一\gls{DL}算法解决许多不同任务的理由。
神经学家们发现,如果将雪貂的大脑重新连接,使视觉信号传送到听觉区域,它们可以学会用大脑的听觉处理区域去``看''\citep{von2000visual}。
这暗示着大多数哺乳动物的大脑能够使用单一的算法就可以解决其大脑可以解决的大部分不同任务。
在这个假设之前,\gls{ML}研究是比较分散的,研究人员在不同的社群研究自然语言处理、计算机视觉、运动规划和语音识别。
如今,这些应用社群仍然是独立的,但是对于\gls{DL}研究团体来说,同时研究许多或甚至所有这些应用领域是很常见的。

我们能够从神经科学得到一些粗略的指南。
仅通过计算单元之间的相互作用而变得智能的基本思想是受大脑启发的。
新认知机\citep{Fukushima80}受哺乳动物视觉系统的结构启发,引入了一个处理图片的强大模型架构,它后来成为了现代卷积网络的基础\citep{LeCun98-small}(我们将会在\secref{sec:the_neuroscientific_basis_for_convolutional_networks}看到)。
目前大多数\gls{NN}是基于一个称为\firstgls{ReLU}的神经单元模型。
原始认知机\citep{Fukushima75}受我们关于大脑功能知识的启发, 引入了一个更复杂的版本。
简化的现代版通过吸收来自不同观点的思想而形成,\citet{Nair-2010}和\citet{Glorot+al-AI-2011-small}援引神经科学作为影响,\citet{Jarrett-ICCV2009}援引更多面向工程的影响。
虽然神经科学是灵感的重要来源,但它不需要被视为刚性指导。
我们知道,真实的神经元计算着与现代\gls{ReLU}非常不同的函数,但更接近真实神经网络的系统并没有导致\gls{ML}性能的提升。
此外,虽然神经科学已经成功地启发了一些\gls{NN}\emph{架构},但我们对用于神经科学的生物学习还没有足够多的了解,因此也就不能为训练这些架构用的\emph{学习算法}提供太多的借鉴。


媒体报道经常强调\gls{DL}与大脑的相似性。
的确,\gls{DL}研究者比其他\gls{ML}领域(如核方法或贝叶斯统计)的研究者更可能地引用大脑作为影响,但是大家不应该认为\gls{DL}在尝试模拟大脑。
现代\gls{DL}从许多领域获取灵感,特别是应用数学的基本内容如线性代数、概率论、信息论和数值优化。
尽管一些\gls{DL}的研究人员引用神经科学作为灵感的重要来源,然而其他学者完全不关心神经科学。

% -- 15 --

值得注意的是,了解大脑是如何在算法层面上工作的尝试确实存在且发展良好。
这项尝试主要被称为``计算神经科学'',并且是独立于\gls{DL}的领域。
研究人员在两个领域之间来回研究是很常见的。
\gls{DL}领域主要关注如何构建计算机系统,从而成功解决需要智能才能解决的任务,而计算神经科学领域主要关注构建大脑如何真实工作的比较精确的模型。

在20世纪80年代,神经网络研究的第二次浪潮在很大程度上是伴随一个被称为\firstgls{connectionism}或\textbf{并行分布处理}( parallel distributed processing)潮流而出现的\citep{Rumelhart86,mcclelland1995appeal}。
\gls{connectionism}是在认知科学的背景下出现的。
认知科学是理解思维的跨学科途径,即它融合多个不同的分析层次。
在20世纪80年代初期,大多数认知科学家研究符号推理模型。
尽管这很流行,但符号模型很难解释大脑如何真正使用神经元实现推理功能。 
\gls{connectionism}者开始研究真正基于神经系统实现的认知模型\citep{Touretzky1985},其中很多复苏的想法可以追溯到心理学家\ENNAME{Donald Hebb}在20世纪40年代的工作\citep{Hebb49}。

\gls{connectionism}的中心思想是,当网络将大量简单的计算单元连接在一起时可以实现智能行为。
这种见解同样适用于生物神经系统中的神经元,因为它和计算模型中\gls{hidden_unit}起着类似的作用。

在上世纪80年代的\gls{connectionism}期间形成的几个关键概念在今天的\gls{DL}中仍然是非常重要的。

其中一个概念是\firstgls{distributed_representation}\citep{Hinton-et-al-PDP1986}。
其思想是:系统的每一个输入都应该由多个特征\gls{representation},并且每一个特征都应该参与到多个可能输入的\gls{representation}。
例如,假设我们有一个能够识别红色、绿色、或蓝色的汽车、卡车和鸟类的视觉系统,
\gls{representation}这些输入的其中一个方法是将九个可能的组合:红卡车,红汽车,红鸟,绿卡车等等使用单独的神经元或\gls{hidden_unit}激活。
这需要九个不同的神经元,并且每个神经必须独立地学习颜色和对象身份的概念。
改善这种情况的方法之一是使用\gls{distributed_representation},即用三个神经元描述颜色,三个神经元描述对象身份。 
这仅仅需要6个神经元而不是9个,并且描述红色的神经元能够从汽车、卡车和鸟类的图像中学习红色,而不仅仅是从一个特定类别的图像中学习。 
\gls{distributed_representation}的概念是本书的核心,我们将在\chapref{chap:representation_learning}中更加详细地描述。

% -- 16 --

\gls{connectionism}潮流的另一个重要成就是反向传播在训练具有内部\gls{representation}的深度\gls{NN}中的成功使用以及反向传播算法的普及\citep{RHW,Lecun-these87}。
这个算法虽然曾黯然失色不再流行,但截至写书之时,它仍是训练深度模型的主导方法。% ??

在20世纪90年代,研究人员在使用\gls{NN}进行序列建模的方面取得了重要进展。
\citet{Hochreiter91}和\citet{Bengio1994ITNN}指出了对长序列进行建模的一些根本性数学难题,这将在\secref{sec:the_challenge_of_long_term_dependencies}中描述。
\citet{Hochreiter+Schmidhuber-1997}引入\firstall{LSTM}网络来解决这些难题。
如今,\glssymbol{LSTM}在许多序列建模任务中广泛应用,包括Google的许多自然语言处理任务。

\gls{NN}研究的第二次浪潮一直持续到上世纪90年代中期。
基于\gls{NN}和其他\glssymbol{AI}技术的创业公司开始寻求投资,其做法野心勃勃但不切实际。
当\glssymbol{AI}研究不能实现这些不合理的期望时,投资者感到失望。
同时,\gls{ML}的其他领域取得了进步。
比如,核方法\citep{Boser92,Cortes95,SchBurSmo99}和图模型\citep{Jordan98}都在很多重要任务上实现了很好的效果。
这两个因素导致了\gls{NN}热潮的第二次衰退,并一直持续到2007年。

在此期间,\gls{NN}继续在某些任务上获得令人印象深刻的表现\citep{LeCun98-small,Bengio-nnlm2001}。
加拿大高级研究所(CIFAR)通过其神经计算和自适应感知(NCAP)研究计划帮助维持\gls{NN}研究。
该计划联合了分别由\ENNAME{Geoffrey Hinton}、\ENNAME{Yoshua Bengio}和\ENNAME{Yann LeCun}领导的多伦多大学、蒙特利尔大学和纽约大学的\gls{ML}研究小组。
这个多学科的CIFAR NCAP研究计划还囊括了神经科学家、人类和计算机视觉专家。

% -- 17 --

在那个时候,人们普遍认为深度网络是难以训练的。
现在我们知道,20世纪80年代就存在的算法能工作得非常好,但是直到在2006年前后都没有体现出来。
这可能仅仅由于其计算代价太高,而以当时可用的硬件难以进行足够的实验。

神经网络研究的第三次浪潮始于2006年的突破。
\ENNAME{Geoffrey Hinton}表明名为\gls{DBN}的\gls{NN}可以使用一种称为贪婪逐层预训练的策略来有效地训练\citep{Hinton06},我们将在\secref{sec:greedy_layer_wise_unsupervised_pretraining}中更详细地描述。
其他CIFAR附属研究小组很快表明,同样的策略可以被用来训练许多其他类型的深度网络\citep{Bengio+Lecun-chapter2007-small,ranzato-07},并能系统地帮助提高在测试样例上的泛化能力。
\gls{NN}研究的这一次浪潮普及了``\gls{DL}''这一术语的使用,强调研究者现在有能力训练以前不可能训练的比较深的神经网络,并着力于深度的理论重要性上\citep{Bengio+Lecun-chapter2007,Delalleau+Bengio-2011-small,Pascanu-et-al-ICLR2014,Montufar-et-al-NIPS2014}。
此时,深度\gls{NN}已经优于与之竞争的基于其他\gls{ML}技术以及手工设计功能的\glssymbol{AI}系统。
在写这本书的时候,神经网络的第三次发展浪潮仍在继续,尽管深度学习的研究重点在这一段时间内发生了巨大变化。
第三次浪潮已开始着眼于新的无监督学习技术和深度模型在小数据集的泛化能力,但目前更多的兴趣点仍是比较传统的监督学习算法和深度模型充分利用大型标注数据集的能力。

\subsection{与日俱增的数据量}
\label{sec:increasing_dataset_sizes}
人们可能想问,既然人工\gls{NN}的第一个实验在20世纪50年代就完成了,但为什么\gls{DL}直到最近才被认为是关键技术。
自20世纪90年代以来,\gls{DL}就已经成功用于商业应用,但通常被视为是一种艺术而不是一种技术,且只有专家可以使用的艺术,这种观点持续到最近。
确实,要从一个\gls{DL}算法获得良好的性能需要一些技巧。
幸运的是,随着训练数据的增加,所需的技巧正在减少。
目前在复杂的任务达到人类水平的学习算法,与20世纪80年代努力解决玩具问题(toy problem)的学习算法几乎是一样的,尽管我们使用这些算法训练的模型经历了变革,即简化了极深架构的训练。
最重要的新进展是现在我们有了这些算法得以成功训练所需的资源。
\figref{fig:chap1_dataset_size_color}展示了基准数据集的大小如何随着时间的推移而显著增加。
这种趋势是由社会日益数字化驱动的。
由于我们的活动越来越多发生在计算机上,我们做什么也越来越多地被记录。
由于我们的计算机越来越多地联网在一起,这些记录变得更容易集中管理,并更容易将它们整理成适于\gls{ML}应用的数据集。
因为统计估计的主要负担(观察少量数据以在新数据上泛化)已经减轻,``大数据''时代使\gls{ML}更加容易。
截至2016年,一个粗略的经验法则是,监督\gls{DL}算法在每类给定约5000个标注样本情况下一般将达到可以接受的性能,当至少有1000万个标注样本的数据集用于训练时,它将达到或超过人类表现。
此外,在更小的数据集上获得成功是一个重要的研究领域,为此我们应特别侧重于如何通过无监督或半监督学习充分利用大量的未标注样本。

\begin{figure}[!htb]
\ifOpenSource
\centerline{\includegraphics{figure.pdf}}
\else
\centerline{\includegraphics{Chapter1/figures/dataset_size_color}}
\fi
\caption{与日俱增的数据量。
20世纪初,统计学家使用数百或数千的手动制作的度量来研究数据集\citep{garson:1900,student08ttest,IrisData1935,Fisher-1936}。
20世纪50年代到80年代,受生物启发的机器学习开拓者通常使用小的合成数据集,如低分辨率的字母位图,设计为在低计算成本下表明神经网络能够学习特定功能\citep{Widrow60,Rumelhart86c}。
20世纪80年代和90年代,机器学习变得更加统计,并开始利用包含成千上万个样本的更大数据集,如手写扫描数字的MNIST数据集(如\figref{fig:chap1_mnist})所示\citep{LeCun98-small}。
在21世纪初的第一个十年,相同大小更复杂的数据集持续出现,如CIFAR-10数据集\citep{KrizhevskyHinton2009} 。
在这十年结束和下五年,明显更大的数据集(包含数万到数千万的样例)完全改变了深度学习的可能实现的事。
这些数据集包括公共Street View House Numbers数据集 \citep{Netzer-wkshp-2011}、各种版本的ImageNet数据集\citep{imagenet_cvpr09,Deng2010,ILSVRCarxiv14}以及Sports-1M数据集\citep{KarpathyCVPR14}。
在图顶部,我们看到翻译句子的数据集通常远大于其他数据集,如根据Canadian Hansard制作的IBM数据集\citep{brown1990statistical}和WMT 2014英法数据集\citep{wmt14} 。
}
\label{fig:chap1_dataset_size_color}
\end{figure}
\begin{figure}[!htb]
\ifOpenSource
\centerline{\includegraphics{figure.pdf}}
\else
\centerline{\includegraphics[width=0.8\textwidth]{Chapter1/figures/mnist}}
\fi
\caption{MNIST数据集的输入样例。
``NIST''代表国家标准和技术研究所(National Institute of Standards and Technology),是最初收集这些数据的机构。
``M''代表``修改的(Modified)'',为更容易地与机器学习算法一起使用,数据已经过预处理。
MNIST数据集包括手写数字的扫描和相关标签(描述每个图像中包含0-9中哪个数字)。
这个简单的分类问题是深度学习研究中最简单和最广泛使用的测试之一。
尽管现代技术很容易解决这个问题,它仍然很受欢迎。
Geoffrey Hinton将其描述为``机器学习的\emph{果蝇}'',这意味着机器学习研究人员可以在受控的实验室条件下研究他们的算法,就像生物学家经常研究果蝇一样。
}
\label{fig:chap1_mnist}
\end{figure}

% -- 20 --

\subsection{与日俱增的模型规模}
\label{sec:increasing_model_sizes}


20世纪80年代,\gls{NN}只能取得相对较小的成功,而现在\gls{NN}非常成功的另一个重要原因是我们现在拥有的计算资源可以运行更大的模型。
\gls{connectionism}的主要见解之一是,当动物的许多神经元一起工作时会变得聪明。
单独神经元或小集合的神经元不是特别有用。

生物神经元不是特别稠密地连接在一起。
如\figref{fig:chap1_number_of_synapses_color}所示,几十年来,我们的\gls{ML}模型中每个神经元的连接数量已经与哺乳动物的大脑在同一数量级上。

\begin{figure}[!htb]
\ifOpenSource
\centerline{\includegraphics{figure.pdf}}
\else
\centerline{\includegraphics{Chapter1/figures/number_of_synapses_color}}
\fi
\caption{与日俱增的每神经元连接数。 % ? 可以翻成平均吗 ?
最初,\gls{ANN}中神经元之间的连接数受限于硬件能力。
而现在,神经元之间的连接数大多是出于设计考虑。
一些\gls{ANN}中每个神经元的连接数与猫一样多,并且对于其他神经网络来说,每个神经元的连接与较小哺乳动物(如小鼠)一样多是非常普遍的。
甚至人类大脑每个神经元的连接也没有过高的数量。
生物神经网络规模来自\citet{number_of_neurons}。
}
\label{fig:chap1_number_of_synapses_color}
{\tiny
\begin{enumerate}
  \itemsep0em
  \item % 1
    自适应线性单元~\citep{Widrow60}
  \item % 2
    神经认知机~\citep{Fukushima80}
  \item % 3
    GPU-加速 \gls{convolutional_network}~\citep{chellapilla:inria-00112631}
  \item % 4
    \gls{DBM}~\citep{SalHinton09}
  \item % 5
    \gls{unsupervised}\gls{convolutional_network}~\citep{Jarrett-ICCV2009-small}
  \item % 6
    GPU-加速 \gls{MLP}~\citep{Ciresan-2010}
  \item % 7
    分布式\gls{AE}~\citep{QuocLe-ICML2012}
  \item % 8
    Multi-GPU \gls{convolutional_network}~\citep{Krizhevsky-2012-small}
  \item % 9
    COTS HPC  \gls{unsupervised}\gls{convolutional_network}~\citep{icml2013_coates13}
  \item % 10
    GoogLeNet~\citep{Szegedy-et-al-arxiv2014}
\end{enumerate}
} % end tiny
\end{figure}

如\figref{fig:chap1_number_of_neurons_color}所示,就神经元的总数目而言,直到最近\gls{NN}都是惊人的小。
自从\gls{hidden_unit}引入以来,人工\gls{NN}的规模大约每2.4年扩大一倍。
这种增长是由更大内存、更快的计算机和更大的可用数据集驱动的。
更大的网络能够在更复杂的任务中实现更高的精度。
这种趋势看起来将持续数十年。
除非有能力迅速扩展的新技术,否则至少要到21世纪50年代,人工\gls{NN}将才能具备与人脑相同数量级的神经元。
生物神经元表示的功能可能比目前的人工神经元所表示的更复杂,因此生物神经网络可能比图中描绘的甚至要更大。

\begin{figure}[!htb]
\ifOpenSource
\centerline{\includegraphics{figure.pdf}}
\else
\centerline{\includegraphics{Chapter1/figures/number_of_neurons_color}}
\fi
\caption{与日俱增的神经网络规模。
自从引入\gls{hidden_unit},\gls{ANN}的大小大约每2.4年翻一倍。
生物神经网络规模来自\citet{number_of_neurons}。
}
\label{fig:chap1_number_of_neurons_color}
{\tiny
\begin{enumerate}
  \itemsep-.1em
  \item % 1
    感知机~\citep{Rosenblatt-1958,Rosenblatt62}
  \item % 2
    自适应线性单元~\citep{Widrow60}
  \item % 3
    神经认知机~\citep{Fukushima80}
  \item % 4
    早期后向传播网络~\citep{Rumelhart86c}
  \item % 5
    用于语音识别的\gls{RNN}~\citep{Robinson+Fallside91}
  \item % 6
    用于语音识别的\gls{MLP}~\citep{Bengio91z}
  \item % 7
    \gls{meanfield}sigmoid\gls{BN}~\citep{Saul+96}
  \item % 8
    LeNet-5~\citep{LeCun98-small}
  \item % 9
    \gls{ESN}~\citep{Jaeger+Haas-2004}
  \item % 10
    \gls{DBN}~\citep{Hinton06}
  \item % 11
    GPU-加速\gls{convolutional_network}~\citep{chellapilla:inria-00112631}
  \item % 12
    \gls{DBM}~\citep{SalHinton09}
  \item % 13
    GPU-加速\gls{DBN}~\citep{RainaICML09}
  \item % 14
    \gls{unsupervised}\gls{convolutional_network}~\citep{Jarrett-ICCV2009-small}
  \item % 15
    GPU-加速\gls{MLP}~\citep{Ciresan-2010}
  \item % 16
    OMP-1 网络~\citep{Coates2011b}
  \item % 17
    分布式\gls{AE}~\citep{QuocLe-ICML2012}
  \item % 18
    Multi-GPU\gls{convolutional_network}~\citep{Krizhevsky-2012-small}
  \item % 19
    COTS HPC \gls{unsupervised}\gls{convolutional_network}~\citep{icml2013_coates13}
  \item % 20
    GoogLeNet~\citep{Szegedy-et-al-arxiv2014}
\end{enumerate}
}
\end{figure}


现在看来,其神经元比一个水蛭还少的\gls{NN}不能解决复杂的\gls{AI}问题是不足为奇的。
即使现在的网络,从计算系统角度来看它可能相当大的,但实际上它比相对原始的脊椎动物如青蛙的神经系统还要小。

由于更快的CPU、通用GPU的出现(在\secref{sec:gpu_implementations}中讨论)、更快的网络连接和更好的分布式计算的软件基础设施,模型规模随着时间的推移不断增加是\gls{DL}历史中最重要的趋势之一。
普遍预计这种趋势将很好地持续到未来。

% -- 21 --

\subsection{与日俱增的精度、复杂度和对现实世界的冲击}
\label{sec:increasing_accuracy_complexity_and_real_world_impact}

20世纪80年代以来,\gls{DL}提供精确识别和预测的能力一直在提高。
而且,\gls{DL}持续成功地被应用于越来越广泛的实际问题中。

最早的深度模型被用来识别裁剪紧凑且非常小的图像中的单个对象\citep{Rumelhart86}。
此后,\gls{NN}可以处理的图像尺寸逐渐增加。
现代对象识别网络能处理丰富的高分辨率照片,并且不需要在被识别的对象附近进行裁剪\citep{Krizhevsky-2012}。
类似地,最早的网络只能识别两种对象(或在某些情况下,单类对象的存在与否),而这些现代网络通常能够识别至少\NUMTEXT{1000}个不同类别的对象。
对象识别中最大的比赛是每年举行的ImageNet大型视觉识别挑战(ILSVRC)。
\gls{DL}迅速崛起的激动人心的一幕是卷积网络第一次大幅赢得这一挑战,它将最高水准的前5错误率从\NUMTEXT{26.1\%}降到\NUMTEXT{15.3\%}\citep{Krizhevsky-2012},这意味着该卷积网络针对每个图像的可能类别生成一个顺序列表,除了15.3\%的测试样本,其他测试样本的正确类标都出现在此列表中的前5项里。
此后,深度卷积网络连续地赢得这些比赛,截至写本书时,\gls{DL}的最新结果将这个比赛中的前5错误率降到了\NUMTEXT{3.6\%}, 如\figref{fig:chap1_imagenet_color}所示。

\begin{figure}[!htb]
\ifOpenSource
\centerline{\includegraphics{figure.pdf}}
\else
\centerline{\includegraphics{Chapter1/figures/imagenet_color}}
\fi
\caption{日益降低的错误率。
由于深度网络达到了在ImageNet大规模视觉识别挑战中竞争所必需的规模,它们每年都能赢得胜利,并且产生越来越低的错误率。
数据来源于 \citet{russakovsky2014imagenet}和\citet{He-et-al-arxiv2015}。}
\label{fig:chap1_imagenet_color}
\end{figure}

% -- 23 --

\gls{DL}也对语音识别产生了巨大影响。
语音识别在20世纪90年代得到提高后,直到约2000年都停滞不前。
\gls{DL}的引入\citep{dahl2010phonerec,Deng-2010,Seide2011,Hinton-et-al-2012}使得语音识别错误率陡然下降,有些错误率甚至降低了一半。
我们将在\secref{sec:speech_recognition}更详细地探讨这个历史。

深度网络在行人检测和图像分割中也取得了引人注目的成功\citep{sermanet-cvpr-13,Farabet-et-al-2013,couprie-iclr-13},并且在交通标志分类上取得了超越人类的表现\citep{Ciresan-et-al-2012}。

在深度网络的规模和精度有所提高的同时,它们可以解决的任务也日益复杂。
\citet{Goodfellow+et+al-ICLR2014a}表明,\gls{NN}可以学习输出描述图像的整个字符序列,而不是仅仅识别单个对象。
此前,人们普遍认为,这种学习需要对序列中的单个元素进行标注\citep{Gulcehre+Bengio-arxiv-2013}。
\gls{RNN},如之前提到的\glssymbol{LSTM}序列模型,现在用于对序列和其他序列之间的关系进行建模,而不是仅仅固定输入之间的关系。
这种序列到序列的学习似乎引领着另一个应用的颠覆性发展,即机器翻译\citep{Sutskever-et-al-NIPS2014,Bahdanau-et-al-ICLR2015-small}。

% -- 24 --

这种复杂性日益增加的趋势已将其推向逻辑结论,即神经图灵机\citep{Graves-et-al-arxiv2014}的引入,它能学习读取存储单元和向存储单元写入任意内容。
这样的\gls{NN}可以从期望行为的\gls{example}中学习简单的程序。
例如,从杂乱和排好序的\gls{example}中学习对一系列数进行排序。
这种自我编程技术正处于起步阶段,但原则上未来可以适用于几乎所有的任务。


\gls{DL}的另一个最大的成就是其在\firstgls{RL}领域的扩展。
在\gls{RL}中,一个自主的智能体必须在没有人类操作者指导的情况下,通过试错来学习执行任务。
DeepMind表明,基于\gls{DL}的\gls{RL}系统能够学会玩Atari视频游戏,并在多种任务中可与人类匹敌\citep{Mnih-et-al-2015}。
\gls{DL}也显著改善了机器人\gls{RL}的性能\citep{finn2015learning}。

许多\gls{DL}应用都是高利润的。现在\gls{DL}被许多顶级的技术公司使用,包括Google、Microsoft、Facebook、IBM、Baidu、Apple、Adobe、Netflix、NVIDIA和NEC等。

\gls{DL}的进步也严重依赖于软件基础架构的进展。
软件库如Theano\citep{bergstra+al:2010-scipy,Bastien-2012}、PyLearn2\citep{pylearn2_arxiv_2013}、Torch\citep{Torch-2011}、DistBelief\citep{Dean-et-al-NIPS2012}、Caffe\citep{Jia13caffe}、MXNet\citep{chen2015mxnet}和TensorFlow\citep{tensorflow}都能支持重要的研究项目或商业产品。

\gls{DL}也为其他科学做出了贡献。
用于对象识别的现代卷积网络为神经科学家们提供了可以研究的视觉处理模型\citep{dicarlo-tutorial-2013}。
\gls{DL}也为处理海量数据以及在科学领域作出有效的预测提供了非常有用的工具。
它已成功地用于预测分子如何相互作用从而帮助制药公司设计新的药物\citep{Dahl-et-al-arxiv2014},搜索亚原子粒子\citep{baldi2014searching},以及自动解析用于构建人脑三维图的显微镜图像\citep{knowlesdeep}等。
我们期待\gls{DL}未来能够出现在越来越多的科学领域中。

% -- 25 --

总之,\gls{DL}是\gls{ML}的一种方法。在过去几十年的发展中,它大量借鉴了我们关于人脑、统计学和应用数学的知识。
近年来,得益于更强大的计算机、更大的数据集和能够训练更深网络的技术,\gls{DL}的普及性和实用性都有了极大的发展。
未来几年充满了进一步提高\gls{DL}并将它带到新领域的挑战和机遇。

% -- 26 --

% !Mode:: "TeX:UTF-8"
\part{应用数学与机器学习基础}
\label{part:applied_math_and_machine_learning_basics}

\newpage

本书这一部分将介绍理解\gls{DL}所需的基本数学概念。
我们从应用数学的一般概念开始,这能使我们定义许多变量的函数,找到这些函数的最高和最低点,并量化信念度。

接着,我们描述\gls{ML}的基本目标,并描述如何实现这些目标。
我们需要指定代表某些信念的模型、设计衡量这些信念与现实对应程度的\gls{cost_function}以及使用训练算法最小化这个\gls{cost_function}。


这个基本框架是广泛多样的\gls{ML}算法的基础,其中也包括非深度的\gls{ML}方法。
在本书的后续部分,我们将在这个框架下开发\gls{DL}算法。

% !Mode:: "TeX:UTF-8"
% Translator: Yujun Li 
\chapter{线性代数}
\label{chap:linear_algebra}

线性代数作为数学的一个分支,广泛用于科学和工程中。
然而,因为线性代数主要是面向连续数学,而非离散数学,所以很多计算机科学家很少接触它。
掌握好线性代数对于理解和从事机器学习算法相关工作是很有必要的,尤其对于深度学习算法而言。
因此,在我们开始介绍深度学习之前,我们集中探讨一些必备的线性代数知识。


如果你已经很熟悉线性代数,那么你可以轻松地跳过本章。
如果你已经了解这些概念,但是需要一份索引表来回顾一些重要公式,那么我们推荐\emph{The Matrix Cookbook} \citep{matrix-cookbook}。
如果你没有接触过线性代数,那么本章将告诉你本书所需的线性代数知识,不过我们仍然非常建议你参考其他专注于讲解线性代数的文献,例如\cite{shilov1977linear}。
最后,本章跳过了很多重要但是对于理解深度学习非必需的线性代数知识。




\section{标量、向量、矩阵和张量}
\label{sec:scalars_vectors_matrices_and_tensors}

学习线性代数,会涉及以下几类数学概念:
\begin{itemize}
    \item \firstgls{scalar}:一个标量就是一个单独的数,它不同于线性代数中研究的其他大部分对象(通常是多个数的数组)。
    我们用斜体表示标量。标量通常被赋予小写的变量名称。
    当我们介绍标量时,会明确它们是哪种类型的数。
    比如,在定义实数标量时,我们可能会说``令$\Ss \in \SetR$表示一条线的斜率'';在定义自然数标量时,我们可能会说``令$\Sn\in\SetN$表示元素的数目''。

% -- 29 --

    \item \firstgls{vector}:一个向量是一列数。
    这些数是有序排列的。
    通过次序中的\gls{index},我们可以确定每个单独的数。
    通常我们赋予向量粗体的小写变量名称,比如$\Vx$。
    向量中的元素可以通过带脚标的斜体表示。
    向量$\Vx$的第一个元素是$\Sx_1$,第二个元素是$\Sx_2$,等等。
    我们也会注明存储在向量中的元素是什么类型的。
    如果每个元素都属于$\SetR$,并且该向量有$\Sn$个元素,那么该向量属于实数集$\SetR$的$\Sn$次笛卡尔乘积构成的集合,记为$\SetR^n$。
    当我们需要明确表示向量中的元素时,我们会将元素排列成一个方括号包围的纵列:
    \begin{equation}
        \Vx=\begin{bmatrix} \Sx_1   \\  
                            \Sx_2   \\ 
                            \vdots  \\ 
                            \Sx_n 
                \end{bmatrix}.
    \end{equation}
    我们可以把向量看作空间中的点,每个元素是不同坐标轴上的坐标。
    
    有时我们需要\gls{index}向量中的一些元素。
    在这种情况下,我们定义一个包含这些元素\gls{index}的集合,然后将该集合写在脚标处。
    比如,指定$\Sx_1$,$\Sx_3$和$\Sx_6$,我们定义集合$S=\{1,3,6\}$,然后写作$\Vx_S$。我
    们用符号-表示集合的补集中的\gls{index}。
    比如$\Vx_{-1}$表示$\Vx$中除$\Sx_1$外的所有元素,$\Vx_{-S}$表示$\Vx$中除$\Sx_1$,$\Sx_3$,$\Sx_6$外所有元素构成的向量。

    \item \firstgls{matrix}:矩阵是一个二维数组,其中的每一个元素被两个\gls{index}而非一个所确定。
    我们通常会赋予矩阵粗体的大写变量名称,比如$\MA$。
    如果一个实数矩阵高度为$m$,宽度为$n$,那么我们说$\MA\in \SetR^{m\times n}$。
    我们在表示矩阵中的元素时,通常以不加粗的斜体形式使用其名称,\gls{index}用逗号间隔。
    比如,$\SA_{1,1}$表示$\MA$左上的元素,$\SA_{m,n}$表示$\MA$右下的元素。
    我们通过用``:''表示水平坐标,以表示垂直坐标$\Si$中的所有元素。
    比如,$\MA_{i,:}$表示$\MA$中垂直坐标$i$上的一横排元素。
    这也被称为$\MA$的第$i$~\firstgls{row}。
    同样地,$\MA_{:,i}$表示$\MA$的第$i$~\firstgls{column}。
    当我们需要明确表示矩阵中的元素时,我们将它们写在用方括号包围起来的数组中:
    \begin{equation}
        \begin{bmatrix}
            A_{1,1} & A_{1,2} \\
            A_{2,1} & A_{2,2} \\
        \end{bmatrix}.
    \end{equation}
    有时我们需要矩阵值表达式的\gls{index},而不是单个元素。
    在这种情况下,我们在表达式后面接下标,但不必将矩阵的变量名称小写化。
    比如,$f(\MA)_{i,j}$表示函数$f$作用在$\MA$上输出的矩阵的第$i$行第$j$列元素。

% -- 30 --

    \item \firstgls{tensor}:在某些情况下,我们会讨论坐标超过两维的数组。
    一般地,一个数组中的元素分布在若干维坐标的规则网格中,我们将其称之为张量。
    我们使用字体$\TSA$来表示张量``A''。
    张量$\TSA$中坐标为$(i,j,k)$的元素记作$\TEA_{i,j,k}$。
\end{itemize}


\firstgls{transpose}是矩阵的重要操作之一。
矩阵的转置是以对角线为轴的镜像,这条从左上角到右下角的对角线被称为\firstgls{main_diagonal}。
\figref{fig:chap2_transpose}显示了这个操作。
我们将矩阵$\MA$的转置表示为$\MA^\top$,定义如下
\begin{equation}
(\MA^\top)_{i,j}= \SA_{j,i}.
\end{equation}

向量可以看作是只有一列的矩阵。
对应地,向量的转置可以看作是只有一行的矩阵。
有时,我们通过将向量元素作为行矩阵写在文本行中,然后使用转置操作将其变为标准的列向量,来定义一个向量,比如$\Vx=[\Sx_1, \Sx_2, \Sx_3]^\top$.


标量可以看作是只有一个元素的矩阵。
因此,标量的转置等于它本身,$\Sa=\Sa^\top$。

\begin{figure}[!hbt]
\ifOpenSource
\centerline{\includegraphics{figure.pdf}}
\else
\centerline{\includegraphics{Chapter2/figures/transpose}}
\fi
\caption{矩阵的转置可以看成是以主对角线为轴的一个镜像。}
\label{fig:chap2_transpose}
\end{figure}

% -- 31 --

只要矩阵的形状一样,我们可以把两个矩阵相加。
两个矩阵相加是指对应位置的元素相加,比如$\MC=\MA+\MB$,其中$\SC_{i,j}= \SA_{i,j}+\SB_{i,j}$。


标量和矩阵相乘,或是和矩阵相加时,我们只需将其与矩阵的每个元素相乘或相加,比如$\MD = \Sa \cdot \MB + \Sc$,其中$\SD_{i,j} = \Sa\cdot  \SB_{i,j} + \Sc$。


在深度学习中,我们也使用一些不那么常规的符号。
我们允许矩阵和向量相加,产生另一个矩阵:$\MC=\MA + \Vb$,其中$\SC_{i,j}= \SA_{i,j} + \Sb_{j}$。
换言之,向量$\Vb$和矩阵$\MA$的每一行相加。
这个简写方法使我们无需在加法操作前定义一个将向量$\Vb$复制到每一行而生成的矩阵。
这种隐式地复制向量$\Vb$到很多位置的方式,被称为\firstgls{broadcasting}。




\section{矩阵和向量相乘}
\label{sec:multiplying_matrices_and_vectors}

矩阵乘法是矩阵运算中最重要的操作之一。
两个矩阵$\MA$和$\MB$的\firstgls{matrix_product}是第三个矩阵$\MC$。
为了使乘法定义良好,矩阵$\MA$的列数必须和矩阵$\MB$的行数相等。
如果矩阵$\MA$的形状是$\Sm \times \Sn$,矩阵$\MB$的形状是$\Sn\times \Sp$,那么矩阵$\MC$的形状是$\Sm\times \Sp$。
我们可以通过将两个或多个矩阵并列放置以书写矩阵乘法,例如
\begin{equation}
    \MC=\MA\MB.
\end{equation}


具体地,该乘法操作定义为
\begin{equation}
    \SC_{i,j}=\sum_k \SA_{i,k} \SB_{k,j}.
\end{equation}


需要注意的是,两个矩阵的标准乘积\emph{不是}指两个矩阵中对应元素的乘积。
不过,那样的矩阵操作确实是存在的,被称为\firstgls{element_wise_product}或者\firstgls{hadamard_product},记为$\MA\odot\MB$。


两个相同维数的向量$\Vx$和$\Vy$的\firstgls{dot_product}可看作是矩阵乘积$\Vx^\top\Vy$。
我们可以把矩阵乘积$\MC=\MA\MB$中计算$\SC_{i,j}$的步骤看作是$\MA$的第$\Si$行和$\MB$的第$\Sj$列之间的\gls{dot_product}。


矩阵乘积运算有许多有用的性质,从而使矩阵的数学分析更加方便。
比如,矩阵乘积服从分配律:
\begin{equation}
    \MA(\MB+\MC)=\MA\MB +\MA\MC.
\end{equation}
矩阵乘积也服从结合律:
\begin{equation}
\MA(\MB\MC)=(\MA\MB)\MC.
\end{equation}


% -- 32 --


不同于标量乘积,矩阵乘积\emph{并不}满足交换律($\MA\MB=\MB\MA$的情况并非总是满足)。
然而,两个向量的\firstgls{dot_product}满足交换律:
\begin{equation}
\label{eq:2.8}
\Vx^\top\Vy=\Vy^\top\Vx.
\end{equation}


矩阵乘积的转置有着简单的形式:
\begin{equation}
(\MA\MB)^\top=\MB^\top\MA^\top.
\end{equation}
利用向量乘积是标量,标量转置是自身的事实,我们可以证明\eqnref{eq:2.8}:
\begin{equation}
    \Vx^\top \Vy = \left(\Vx^\top \Vy \right)^\top = \Vy^\top \Vx.
\end{equation}


由于本书的重点不是线性代数,我们并不试图展示矩阵乘积的所有重要性质,但读者应该知道矩阵乘积还有很多有用的性质。


现在我们已经知道了足够多的线性代数符号,可以表达下列线性方程组:
\begin{equation}
\label{eq:2.11}
\MA\Vx=\Vb
\end{equation}
其中$\MA\in \SetR^{m\times n}$是一个已知矩阵,$\Vb\in\SetR^m$是一个已知向量,$\Vx\in\SetR^n$是一个我们要求解的未知向量。
向量$\Vx$的每一个元素$\Sx_i$都是未知的。
矩阵$\MA$的每一行和$\Vb$中对应的元素构成一个约束。
我们可以把\eqnref{eq:2.11}重写为
\begin{gather}
\MA_{1,:}\Vx=b_1\\
\MA_{2,:}\Vx=b_2 \\
\cdots \\
\MA_{m,:}\Vx=b_m
\end{gather}
或者,更明确地,写作
\begin{gather}
    \MA_{1,1}x_1+\MA_{1,2}x_2+\cdots \MA_{1,n}x_n = b_1\\
    \MA_{2,1}x_1+\MA_{2,2}x_2+\cdots \MA_{2,n}x_n = b_2\\
    \cdots\\
    \MA_{m,1}x_1+\MA_{m,2}x_2+\cdots \MA_{m,n}x_n = b_m.
\end{gather}


矩阵向量乘积符号为这种形式的方程提供了更紧凑的表示。



% -- 33 --


\section{单位矩阵和逆矩阵}
\label{sec:identity_and_inverse_atrices}


线性代数提供了被称为\firstgls{matrix_inverse}的强大工具。
对于大多数矩阵$\MA$,我们都能通过\gls{matrix_inverse}解析地求解\eqnref{eq:2.11}。


为了描述矩阵逆,我们首先需要定义\firstgls{identity_matrix}的概念。
任意向量和单位矩阵相乘,都不会改变。
我们将保持$\Sn$维向量不变的单位矩阵记作$\MI_{\Sn}$。
形式上,$\MI_{\Sn}\in \SetR^{\Sn\times \Sn}$,
\begin{equation}
    \forall \Vx \in \SetR^{\Sn}, \MI_{\Sn} \Vx = \Vx.
\end{equation}
单位矩阵的结构很简单:所有沿主对角线的元素都是$1$,而所有其他位置的元素都是$0$。
如\figref{fig:chap2_empty2}所示的例子。
\begin{figure}[!htb]
\ifOpenSource
\centerline{\includegraphics{figure.pdf}}
\else
\centering
\begin{equation}
\begin{bmatrix} 
1 & 0 & 0 \\
0 & 1 & 0 \\
0 & 0 & 1 \\
\end{bmatrix}
\end{equation}
\fi
\caption{单位矩阵的一个样例:这是$\MI_3$。}
\label{fig:chap2_empty2}
\end{figure}

\begin{figure}[!htb]
\ifOpenSource
\centerline{\includegraphics{figure.pdf}}
\else
\centering
\begin{equation}
\begin{bmatrix} 
1 & 0 & 0 \\
0 & 1 & 0 \\
0 & 0 & 1 \\
\end{bmatrix}
\end{equation}
\fi
\caption{单位矩阵的一个样例:这是$\MI_3$。}
\label{fig:chap2_empty2}
\end{figure}

矩阵$\MA$的\firstgls{matrix_inverse}记作$\MA^{-1}$,其定义的矩阵满足如下条件
\begin{equation} \MA^{-1}\MA = \MI_{\Sn}. \end{equation}

现在我们可以通过以下步骤求解\eqnref{eq:2.11}:
\begin{gather}
\MA\Vx=\Vb \\
\MA^{-1}\MA\Vx = \MA^{-1}\Vb \\
\MI_{\Sn} \Vx=\MA^{-1}\Vb \\
\Vx=\MA^{-1}\Vb. 
\end{gather}


当然,这取决于我们能否找到一个逆矩阵$\MA^{-1}$。
在接下来的章节中,我们会讨论逆矩阵$\MA^{-1}$存在的条件。


当逆矩阵$\MA^{-1}$存在时,有几种不同的算法都能找到它的闭解形式。
理论上,相同的逆矩阵可用于多次求解不同向量$\Vb$的方程。
然而,逆矩阵$\MA^{-1}$主要是作为理论工具使用的,并不会在大多数软件应用程序中实际使用。
这是因为逆矩阵$\MA^{-1}$在数字计算机上只能表现出有限的精度,有效使用向量$\Vb$的算法通常可以得到更精确的$\Vx$。



% -- 34 --


\section{线性相关和生成子空间}
\label{sec:linear_dependence_and_span}

如果逆矩阵$\MA^{-1}$存在,那么\eqnref{eq:2.11}肯定对于每一个向量$\Vb$恰好存在一个解。
但是,对于方程组而言,对于向量$\Vb$的某些值,有可能不存在解,或者存在无限多个解。
存在多于一个解但是少于无限多个解的情况是不可能发生的;因为如果$\Vx$和$\Vy$都是某方程组的解,则
\begin{equation}
\Vz=\alpha \Vx + (1-\alpha) \Vy
\end{equation}
(其中$\alpha$取任意实数)也是该方程组的解。


为了分析方程有多少个解,我们可以将$\MA$的列向量看作是从\firstgls{origin}(元素都是零的向量)出发的不同方向,确定有多少种方法可以到达向量$\Vb$。
在这个观点下,向量$\Vx$中的每个元素表示我们应该沿着这些方向走多远,即$\Sx_{\Si}$表示我们需要沿着第$\Si$个向量的方向走多远:
\begin{equation}
\MA \Vx = \sum_i x_i \MA_{:,i}.
\end{equation}
一般而言,这种操作被称为\firstgls{linear_combination}。
形式上,一组向量的线性组合,是指每个向量乘以对应标量系数之后的和,即:
\begin{equation}
    \sum_i \Sc_i \Vv^{(i)}.
\end{equation}
一组向量的\firstgls{span}是原始向量线性组合后所能抵达的点的集合。


确定$\MA\Vx=\Vb$是否有解相当于确定向量$\Vb$是否在$\MA$列向量的\gls{span}中。
这个特殊的\gls{span}被称为$\MA$的\firstgls{column_space}或者$\MA$的\firstgls{range}。


为了使方程$\MA \Vx=\Vb$对于任意向量$\Vb \in \SetR^m$都存在解,我们要求$\MA$的列空间构成整个$\SetR^{\Sm}$。
如果$\SetR^m$中的某个点不在$\MA$的列空间中,那么该点对应的$\Vb$会使得该方程没有解。
矩阵$\MA$的列空间是整个$\SetR^m$的要求,意味着$\MA$至少有$m$列,即$n\geq m$。
否则,$\MA$列空间的维数会小于$m$。
例如,假设$\MA$是一个$3\times 2$的矩阵。
目标$\Vb$是$3$维的,但是$\Vx$只有$2$维。
所以无论如何修改$\Vx$的值,也只能描绘出$\SetR^3$空间中的二维平面。
当且仅当向量$\Vb$在该二维平面中时,该方程有解。



% -- 35 --


不等式$n\geq m$仅是方程对每一点都有解的必要条件。
这不是一个充分条件,因为有些列向量可能是冗余的。
假设有一个$\SetR^{2\times 2}$中的矩阵,它的两个列向量是相同的。
那么它的列空间和它的一个列向量作为矩阵的列空间是一样的。
换言之,虽然该矩阵有$2$列,但是它的列空间仍然只是一条线,不能涵盖整个$\SetR^2$空间。


正式地说,这种冗余被称为\firstgls{linear_dependence}。
如果一组向量中的任意一个向量都不能表示成其他向量的线性组合,那么这组向量被称为\firstgls{linearly_independent}。
如果某个向量是一组向量中某些向量的线性组合,那么我们将这个向量加入到这组向量后不会增加这组向量的\gls{span}。
这意味着,如果一个矩阵的\gls{column_space}涵盖整个$\SetR^m$,那么该矩阵必须包含至少一组$m$个线性无关的向量。
这是\eqnref{eq:2.11}对于每一个向量$\Vb$的取值都有解的充分必要条件。
值得注意的是,这个条件是说该向量集恰好有$m$个线性无关的列向量,而不是至少$m$个。
不存在一个$m$维向量的集合具有多于$m$个彼此线性不相关的列向量,但是一个有多于$m$个列向量的矩阵却有可能拥有不止一个大小为$m$的线性无关向量集。


要想使矩阵可逆,我们还需要保证\eqnref{eq:2.11}对于每一个$\Vb$值至多有一个解。
为此,我们需要确保该矩阵至多有$m$个列向量。
否则,该方程会有不止一个解。


综上所述,这意味着该矩阵必须是一个\firstgls{square},即$m=n$,并且所有列向量都是线性无关的。一个列向量线性相关的方阵被称为\firstgls{singular}。


如果矩阵$\MA$不是一个方阵或者是一个奇异的方阵,该方程仍然可能有解。
但是我们不能使用矩阵逆去求解。


目前为止,我们已经讨论了逆矩阵左乘。我们也可以定义逆矩阵右乘:
\begin{equation}
\MA\MA^{-1}=\MI.
\end{equation}
对于方阵而言,它的左逆和右逆是相等的。




\section{范数}
\label{sec:norms}

有时我们需要衡量一个向量的大小。
在机器学习中,我们经常使用被称为\firstgls{norm}的函数衡量向量大小。
形式上,$L^p$范数定义如下
\begin{equation}
\label{eq:2.30}
    \norm{\Vx}_p = \left( \sum_i |x_i|^p \right)^{\frac{1}{p}}
\end{equation}
其中$p\in \SetR$,$p\geq 1$。



% -- 36 --


范数(包括$L^p$范数)是将向量映射到非负值的函数。
直观上来说,向量$\Vx$的范数衡量从原点到点$\Vx$的距离。
更严格地说,范数是满足下列性质的任意函数:
\begin{itemize}
\item $f(\Vx) = 0 \Rightarrow \Vx = \mathbf{0}$ 
\item $f(\Vx + \Vy) \leq f(\Vx) + f(\Vy)$ (\firstgls{triangle_inequality})
\item $\forall \alpha \in \SetR$, $f(\alpha \Vx) = |\alpha| f(\Vx)$
\end{itemize}


当$p=2$时,$L^2$范数被称为\firstgls{euclidean_norm}。
它表示从原点出发到向量$\Vx$确定的点的欧几里得距离。
$L^2$范数在机器学习中出现地十分频繁,经常简化表示为$\norm{x}$,略去了下标$2$。
平方$L^2$范数也经常用来衡量向量的大小,可以简单地通过\gls{dot_product} $\Vx^\top\Vx$计算。


平方$L^2$范数在数学和计算上都比$L^2$范数本身更方便。
例如,平方$L^2$范数对$\Vx$中每个元素的导数只取决于对应的元素,而$L^2$范数对每个元素的导数却和整个向量相关。
但是在很多情况下,平方$L^2$范数也可能不受欢迎,因为它在原点附近增长得十分缓慢。
在某些机器学习应用中,区分恰好是零的元素和非零但值很小的元素是很重要的。
在这些情况下,我们转而使用在各个位置斜率相同,同时保持简单的数学形式的函数:$L^1$范数。
$L^1$范数可以简化如下:
\begin{equation}
    \norm{\Vx}_1 = \sum_i  |x_i|.
\end{equation}
当机器学习问题中零和非零元素之间的差异非常重要时,通常会使用$L^1$范数。
每当$\Vx$中某个元素从$0$增加$\epsilon$,对应的$L^1$范数也会增加$\epsilon$。


有时候我们会统计向量中非零元素的个数来衡量向量的大小。
有些作者将这种函数称为``$L^0$范数'',但是这个术语在数学意义上是不对的。
向量的非零元素的数目不是范数,因为对向量缩放$\alpha$倍不会改变该向量非零元素的数目。
因此,$L^1$范数经常作为表示非零元素数目的替代函数。



% -- 37 --


另外一个经常在机器学习中出现的范数是$L^\infty$范数,也被称为~\firstgls{max_norm}。
这个范数表示向量中具有最大幅值的元素的绝对值:
\begin{equation}
    \norm{\Vx}_\infty = \max_i |x_i|.
\end{equation}


有时候我们可能也希望衡量矩阵的大小。
在深度学习中,最常见的做法是使用\firstgls{frobenius_norm},
\begin{equation}
    \norm{\MA}_F = \sqrt{\sum_{i,j} A_{i,j}^2}, 
%%lyj 原文是\norm{A}_F ...
\end{equation}
其类似于向量的$L^2$范数。


两个向量的\firstgls{dot_product}可以用范数来表示。
具体地,
\begin{equation}
    \Vx^\top\Vy = \norm{\Vx}_2\norm{\Vy}_2 \cos \theta
\end{equation}
其中$\theta$表示$\Vx$和$\Vy$之间的夹角。




\section{特殊类型的矩阵和向量}
\label{sec:special_kinds_of_matrices_and_vectors}

有些特殊类型的矩阵和向量是特别有用的。


\firstgls{diagonal_matrix}只在主对角线上含有非零元素,其他位置都是零。
形式上,矩阵$\MD$是对角矩阵,当且仅当对于所有的$i\neq j$,$\SD_{i,j}=0$。
我们已经看到过一个对角矩阵:单位矩阵,对角元素全部是$1$。
我们用$\text{diag}(\Vv)$表示一个对角元素由向量$\Vv$中元素给定的对角方阵。
对角矩阵受到关注的部分原因是对角矩阵的乘法计算很高效。
计算乘法$\text{diag}(\Vv)\Vx$,我们只需要将$\Vx$中的每个元素$x_i$放大$v_i$倍。
换言之,$\text{diag}(\Vv)\Vx=\Vv \odot \Vx$。
计算对角方阵的逆矩阵也很高效。
对角方阵的逆矩阵存在,当且仅当对角元素都是非零值,在这种情况下,$\text{diag}(\Vv)^{-1}=\text{diag}([1/v_1,\dots,1/v_n]^\top)$。
在很多情况下,我们可以根据任意矩阵导出一些通用的机器学习算法;但通过将一些矩阵限制为对角矩阵,我们可以得到计算代价较低的(并且简明扼要的)算法。


不是所有的对角矩阵都是方阵。
长方形的矩阵也有可能是对角矩阵。
非方阵的对角矩阵没有逆矩阵,但我们仍然可以高效地计算它们的乘法。
对于一个长方形对角矩阵$\MD$而言,乘法$\MD\Vx$会涉及到$\Vx$中每个元素的缩放,如果$\MD$是瘦长型矩阵,那么在缩放后的末尾添加一些零;如果$\MD$是胖宽型矩阵,那么在缩放后去掉最后一些元素。


% -- 38 --

\firstgls{symmetric}矩阵是转置和自己相等的矩阵:
\begin{equation}
    \MA=\MA^\top.
\end{equation}
当某些不依赖参数顺序的双参数函数生成元素时,对称矩阵经常会出现。
例如,如果$\MA$是一个距离度量矩阵,$\MA_{i,j}$表示点$i$到点$j$的距离,那么$\MA_{i,j}=\MA_{j,i}$,因为距离函数是对称的。


\firstgls{unit_vector}是具有\firstgls{unit_norm}的向量:
\begin{equation}
\norm{\Vx}_2=1.
\end{equation}


如果$\Vx^\top \Vy = 0$,那么向量$\Vx$和向量$\Vy$互相\firstgls{orthogonal}。
如果两个向量都有非零范数,那么这两个向量之间的夹角是$90$度。
在$\SetR^n$中,至多有$n$个范数非零向量互相正交。
如果这些向量不仅互相正交,并且范数都为$1$,那么我们称它们是\firstgls{orthonormal}。


\firstgls{orthogonal_matrix}是指行向量和列向量是分别标准正交的方阵:
\begin{equation}
    \MA^\top\MA=\MA\MA^\top=\MI.
\end{equation}
这意味着 
\begin{equation}
    \MA^{-1}=\MA^\top,
\end{equation}
所以正交矩阵受到关注是因为求逆计算代价小。
我们需要注意正交矩阵的定义。
反直觉地,正交矩阵的行向量不仅是正交的,还是标准正交的。
对于行向量或列向量互相正交但不是标准正交的矩阵没有对应的专有术语。




\section{\glsentrytext{eigendecomposition}}
\label{sec:eigendecomposition}

许多数学对象可以通过将它们分解成多个组成部分,或者找到它们的一些属性而更好地理解,这些属性是通用的,而不是由我们选择表示它们的方式产生的。

% -- 39 --

例如,整数可以分解为质因数。
我们可以用十进制或二进制等不同方式表示整数$12$,但是$12=2\times 3\times 3$永远是对的。
从这个表示中我们可以获得一些有用的信息,比如$12$不能被$5$整除,或者$12$的倍数可以被$3$整除。


正如我们可以通过分解质因数来发现整数的一些内在性质,我们也可以通过分解矩阵来发现矩阵表示成数组元素时不明显的函数性质。


\firstgls{eigendecomposition}是使用最广的矩阵分解之一,即我们将矩阵分解成一组特征向量和特征值。


方阵$\MA$的\firstgls{eigenvector}是指与$\MA$相乘后相当于对该向量进行缩放的非零向量$\Vv$:
\begin{equation}
    \MA\Vv=\lambda \Vv.
\end{equation}
标量$\lambda$被称为这个特征向量对应的\firstgls{eigenvalue}。
(类似地,我们也可以定义\firstgls{left_Evector} $\Vv^\top\MA=\lambda \Vv^\top$,但是通常我们更关注\firstgls{right_Evector})。


如果$\Vv$是$\MA$的特征向量,那么任何缩放后的向量$s\Vv$~($s\in \SetR$,$s\neq 0$)也是$\MA$的特征向量。
此外,$s\Vv$和$\Vv$有相同的特征值。
基于这个原因,通常我们只考虑单位特征向量。


假设矩阵$\MA$有$n$个线性无关的特征向量$\{\Vv^{(1)}, \dots, \Vv^{(n)}\}$,对应着特征值$\{\lambda_1, \dots , \lambda_n \}$。
我们将特征向量连接成一个矩阵,使得每一列是一个特征向量:$\MV=[\Vv^{(1)}, \dots, \Vv^{(n)}]$.
类似地,我们也可以将特征值连接成一个向量$\Vlambda = [\lambda_1, \dots , \lambda_n]^\top$。
因此$\MA$的\firstgls{eigendecomposition}可以记作
\begin{equation}
    \MA = \MV \text{diag}(\Vlambda) \MV^{-1}.
\end{equation}


我们已经看到了\emph{构建}具有特定特征值和特征向量的矩阵,能够使我们在目标方向上延伸空间。
然而,我们也常常希望将矩阵\firstgls{decompose}成特征值和特征向量。
这样可以帮助我们分析矩阵的特定性质,就像质因数分解有助于我们理解整数。


不是每一个矩阵都可以分解成特征值和特征向量。
在某些情况下,特征分解存在,但是会涉及到复数,而非实数。
幸运的是,在本书中我们通常只需要分解一类有简单分解的矩阵。
具体地,每个实对称矩阵都可以分解成实特征向量和实特征值:
\begin{equation}
    \MA = \MQ \VLambda \MQ^\top.
\end{equation}
其中$\MQ$是$\MA$的特征向量组成的正交矩阵,$\VLambda$是对角矩阵。
特征值$\Lambda_{i,i}$对应的特征向量是矩阵$\MQ$的第$i$列,记作$\MQ_{:,i}$。
因为$\MQ$是正交矩阵,我们可以将$\MA$看作是沿方向$\Vv^{(i)}$延展$\lambda_i$倍的空间。
如\figref{fig:chap2_eigen_ellipse}所示的例子。

% -- 40 --

虽然任意一个实对称矩阵$\MA$都有特征分解,但是特征分解可能并不唯一。
如果两个或多个特征向量拥有相同的特征值,那么在由这些特征向量产生的生成子空间中,任意一组正交向量都是该特征值对应的特征向量。
因此,我们可以等价地从这些特征向量中构成$\MQ$作为替代。
按照惯例,我们通常按降序排列$\VLambda$的元素。
在该约定下,特征分解唯一当且仅当所有的特征值都是唯一的。

\begin{figure}[!htb]
\ifOpenSource
\centerline{\includegraphics{figure.pdf}}
\else
\centerline{\includegraphics[width=0.8\textwidth]{Chapter2/figures/eigen_ellipse_color}}
\fi
\caption{特征向量和特征值的作用效果。
特征向量和特征值的作用效果的一个实例。
在这里,矩阵$\MA$有两个\gls{orthonormal}的特征向量,对应特征值为$\lambda_1$的$\Vv^{(1)}$以及对应特征值为$\lambda_2$的$\Vv^{(2)}$。
\emph{(左)}我们画出了所有的单位向量$\Vu\in\SetR^2$的集合,构成一个单位圆。
\emph{(右)}我们画出了所有的$\MA\Vu$点的集合。
通过观察$\MA$拉伸单位圆的方式,我们可以看到它将$\Vv^{(i)}$方向的空间拉伸了$\lambda_i$倍。	}
\label{fig:chap2_eigen_ellipse}
\end{figure}

% -- 41 --


矩阵的特征分解给了我们很多关于矩阵的有用信息。
矩阵是奇异的当且仅当含有零特征值。
实对称矩阵的特征分解也可以用于优化二次方程$f(\Vx) = \Vx^\top \MA \Vx$,其中限制$\norm{\Vx}_2 = 1$。
当$\Vx$等于$\MA$的某个特征向量时,$f$将返回对应的特征值。
在限制条件下,函数$f$的最大值是最大特征值,最小值是最小特征值。


所有特征值都是正数的矩阵被称为\firstgls{P_D};所有特征值都是非负数的矩阵被称为\firstgls{P_SD}。
同样地,所有特征值都是负数的矩阵被称为\firstgls{ND};所有特征值都是非正数的矩阵被称为\firstgls{NSD}。
\gls{P_SD}矩阵受到关注是因为它们保证$\forall \Vx, \Vx^\top \MA \Vx \geq 0$。
此外,\gls{P_D}矩阵还保证$\Vx^\top \MA \Vx =0 \Rightarrow \Vx = \mathbf{0}$。





\section{\glsentrytext{SVD}}
\label{sec:singular_value_decomposition}

在\secref{sec:eigendecomposition},我们探讨了如何将矩阵分解成特征向量和特征值。
还有另一种分解矩阵的方法,被称为\firstall{SVD},将矩阵分解为\firstgls{Svector}和\firstgls{Svalue}。
通过\gls{SVD},我们会得到一些与\gls{eigendecomposition}相同类型的信息。
然而,\gls{SVD}有更广泛的应用。
每个实数矩阵都有一个\gls{SVD},但不一定都有\gls{eigendecomposition}。
例如,非方阵的矩阵没有\gls{eigendecomposition},这时我们只能使用\gls{SVD}。


回想一下, 我们使用\gls{eigendecomposition}去分析矩阵$\MA$时, 得到特征向量构成的矩阵$\MV$和特征值构成的向量$\Vlambda$,我们可以重新将$\MA$写作
\begin{equation}
    \MA=\MV\text{diag}(\Vlambda)\MV^{-1}.
\end{equation}

\gls{SVD}是类似的,只不过这回我们将矩阵$\MA$分解成三个矩阵的乘积:
\begin{equation}
    \MA=\MU\MD\MV^\top.
\end{equation}


假设$\MA$是一个$m\times n$的矩阵,那么$\MU$是一个$m\times m$的矩阵,$\MD$是一个$m\times n$的矩阵,$\MV$是一个$n\times n$矩阵。


% -- 42 --

这些矩阵中的每一个经定义后都拥有特殊的结构。
矩阵$\MU$和$\MV$都被定义为正交矩阵,而矩阵$\MD$被定义为对角矩阵。
注意,矩阵$\MD$不一定是方阵。


对角矩阵$\MD$对角线上的元素被称为矩阵$\MA$的\firstgls{Svalue}。
矩阵$\MU$的列向量被称为\firstgls{left_Svector},矩阵$\MV$的列向量被称\firstgls{right_Svector}。


事实上,我们可以用与$\MA$相关的特征分解去解释$\MA$的\gls{SVD}。
$\MA$的\firstgls{left_Svector}是$\MA\MA^\top$的特征向量。
$\MA$的\firstgls{right_Svector}是$\MA^\top\MA$的特征向量。
$\MA$的非零奇异值是$\MA^\top\MA$特征值的平方根,同时也是$\MA\MA^\top$特征值的平方根。


\glssymbol{SVD}最有用的一个性质可能是拓展矩阵求逆到非方矩阵上。我们将在下一节中探讨。




\section{\glsentrytext{Moore}}
\label{sec:the_moore_penrose_pseudoinverse}


对于非方矩阵而言,其逆矩阵没有定义。
假设在下面的问题中,我们希望通过矩阵$\MA$的左逆$\MB$来求解线性方程,
\begin{equation}
    \MA\Vx=\Vy
\end{equation}
等式两边左乘左逆$\MB$后,我们得到
\begin{equation}
    \Vx=\MB\Vy.
\end{equation}
取决于问题的形式,我们可能无法设计一个唯一的映射将$\MA$映射到$\MB$。


如果矩阵$\MA$的行数大于列数,那么上述方程可能没有解。
如果矩阵$\MA$的行数小于列数,那么上述矩阵可能有多个解。


\firstgls{Moore}使我们在这类问题上取得了一定的进展。
矩阵$\MA$的伪逆定义为:
\begin{equation}
    \MA^+ = \lim_{a \searrow 0} (\MA^\top\MA + \alpha \MI)^{-1} \MA^\top.
\end{equation}
计算伪逆的实际算法没有基于这个定义,而是使用下面的公式:
\begin{equation}
    \MA^+ = \MV\MD^+\MU^\top.
\end{equation}
其中,矩阵$\MU$,$\MD$和$\MV$是矩阵$\MA$\gls{SVD}后得到的矩阵。
对角矩阵$\MD$的伪逆$\MD^+$是其非零元素取倒数之后再转置得到的。

% -- 43 --

当矩阵$\MA$的列数多于行数时,使用伪逆求解线性方程是众多可能解法中的一种。
特别地,$\Vx=\MA^+\Vy$是方程所有可行解中欧几里得范数$\norm{\Vx}_2$最小的一个。


当矩阵$\MA$的行数多于列数时,可能没有解。
在这种情况下,通过伪逆得到的$\Vx$使得$\MA\Vx$和$\Vy$的欧几里得距离$\norm{\MA\Vx-\Vy}_2$最小。





\section{迹运算}
\label{sec:the_trace_operator}

迹运算返回的是矩阵对角元素的和:
\begin{equation}
    \Tr(\MA)= \sum_i \MA_{i,i}.
\end{equation}
迹运算因为很多原因而有用。
若不使用求和符号,有些矩阵运算很难描述,而通过矩阵乘法和迹运算符号,可以清楚地表示。
例如,迹运算提供了另一种描述矩阵\gls{frobenius_norm}的方式:
\begin{equation}
\label{eq:2.49}
    \norm{A}_F = \sqrt{\text{Tr}(\MA \MA^\top)}.
\end{equation}


用迹运算表示表达式,我们可以使用很多有用的等式巧妙地处理表达式。
例如,迹运算在转置运算下是不变的:
\begin{equation}
    \Tr(\MA)=\Tr(\MA^\top).
\end{equation}


多个矩阵相乘得到的方阵的迹,和将这些矩阵中的最后一个挪到最前面之后相乘的迹是相同的。
当然,我们需要考虑挪动之后矩阵乘积依然定义良好:
\begin{equation}
\Tr(\MA\MB\MC)=\Tr(\MC\MA\MB)= \Tr(\MB\MC\MA).
\end{equation}
或者更一般地,
\begin{equation} 
\label{eq:2.52}
\Tr(\prod_{i=1}^n \MF^{(i)})= \Tr(\MF^{(n)} \prod_{i=1}^{n-1} \MF^{(i)}).
\end{equation}
即使循环置换后矩阵乘积得到的矩阵形状变了,迹运算的结果依然不变。
例如,假设矩阵$\MA\in \SetR^{m\times n}$,矩阵$\MB\in \SetR^{n\times m}$,我们可以得到
\begin{equation} 
    \Tr(\MA\MB)= \Tr(\MB\MA)
\end{equation}
尽管$\MA\MB \in \SetR^{m\times m}$和$\MB\MA \in \SetR^{n\times n}$。


% -- 44 --

另一个有用的事实是标量在迹运算后仍然是它自己:$a=\Tr(a)$。




\section{行列式}
\label{sec:the_determinant}

行列式,记作$\text{det}(\MA)$,是一个将方阵$\MA$映射到实数的函数。
行列式等于矩阵特征值的乘积。
行列式的绝对值可以用来衡量矩阵参与矩阵乘法后空间扩大或者缩小了多少。
如果行列式是$0$,那么空间至少沿着某一维完全收缩了,使其失去了所有的体积。
如果行列式是$1$,那么这个转换保持空间体积不变。




\section{实例:\glsentrytext{PCA}}
\label{sec:example_principal_components_analysis_chap2}

\firstall{PCA}是一个简单的机器学习算法,可以通过基础的线性代数知识推导。


假设在$\SetR^n$空间中我们有$m$个点$\{\Vx^{(1)}, \dots ,\Vx^{(m)}\}$,我们希望对这些点进行有损压缩。
有损压缩表示我们使用更少的内存,但损失一些精度去存储这些点。
我们希望损失的精度尽可能少。


一种编码这些点的方式是用低维表示。
对于每个点$\Vx^{(i)} \in \SetR^n$,会有一个对应的编码向量$\Vc^{(i)}\in \SetR^l$。
如果$l$比$n$小,那么我们便使用了更少的内存来存储原来的数据。
我们希望找到一个编码函数,根据输入返回编码,$f(\Vx)=\Vc$;我们也希望找到一个解码函数,给定编码重构输入,$\Vx\approx g(f(\Vx))$。


\glssymbol{PCA}由我们选择的解码函数而定。
具体地,为了简化解码器,我们使用矩阵乘法将编码映射回$\SetR^n$,即$g(\Vc)=\MD\Vc$,其中$\MD\in \SetR^{n\times l}$是定义解码的矩阵。

% -- 45 --

目前为止所描述的问题,可能会有多个解。
因为如果我们按比例地缩小所有点对应的编码向量$c_i$,那么我们只需按比例放大$\MD_{:,i}$,即可保持结果不变。
为了使问题有唯一解,我们限制$\MD$中所有列向量都有\gls{unit_norm}。


计算这个解码器的最优编码可能是一个困难的问题。
为了使编码问题简单一些,\glssymbol{PCA}限制$\MD$的列向量彼此正交(注意,除非$l=n$,否则严格意义上$\MD$不是一个正交矩阵)。


为了将这个基本想法变为我们能够实现的算法,首先我们需要明确如何根据每一个输入$\Vx$得到一个最优编码$\Vc^*$。
一种方法是最小化原始输入向量$\Vx$和重构向量$g(\Vc^*)$之间的距离。
我们使用范数来衡量它们之间的距离。
在\glssymbol{PCA}算法中,我们使用$L^2$范数:
\begin{equation}
 \Vc^* = \underset{\Vc}{\arg\min} \norm{\Vx-g(\Vc)}_2.
\end{equation}


我们可以用平方$L^2$范数替代$L^2$范数,因为两者在相同的值$\Vc$上取得最小值。
这是因为$L^2$范数是非负的,并且平方运算在非负值上是单调递增的。
\begin{equation}
\Vc^* = \argmin_{\Vc} \norm{\Vx - g(\Vc)}_2^2.
\end{equation}
该最小化函数可以简化成
\begin{equation}
(\Vx-g(\Vc))^\top(\Vx-g(\Vc))
\end{equation}
(\eqnref{eq:2.30}中$L^2$范数的定义)
\begin{equation}
    = \Vx^\top\Vx - \Vx^\top g(\Vc) - g(\Vc)^\top\Vx + g(\Vc)^\top g(\Vc)
\end{equation}
(分配律)
\begin{equation}
    = \Vx^\top \Vx - 2\Vx^\top g(\Vc) + g(\Vc)^\top g(\Vc)
\end{equation}
(因为标量$g(\Vc)^\top\Vx$的转置等于自己)


因为第一项$\Vx^\top\Vx$不依赖于$\Vc$,所以我们可以忽略它,得到如下的优化目标:
\begin{equation}
\Vc^* = \underset{\Vc}{\arg\min} - 2\Vx^\top g(\Vc) + g(\Vc)^\top g(\Vc).
\end{equation}

% -- 46 --

更进一步,我们代入$g(\Vc)$的定义:
\begin{equation}
    \Vc^* = \underset{\Vc}{\arg\min} - 2\Vx^\top\MD\Vc + \Vc^\top\MD^\top\MD\Vc
\end{equation}
\begin{equation}
    = \underset{\Vc}{\arg\min} -2\Vx^\top\MD\Vc + \Vc^\top\MI_l\Vc
\end{equation}
(矩阵$\MD$的正交性和单位范数约束)
\begin{equation}
    = \underset{\Vc}{\arg\min} -2\Vx^\top\MD\Vc + \Vc^\top\Vc
\end{equation}


我们可以通过向量微积分来求解这个最优化问题(如果你不清楚怎么做,请参考\secref{sec:gradient_based_optimization})
\begin{gather}
    \nabla_{\Vc} (-2\Vx^\top \MD \Vc + \Vc^\top\Vc) = 0\\
    -2\MD^\top\Vx + 2\Vc = 0\\
    \Vc = \MD^\top \Vx.
\end{gather}


这使得算法很高效:最优编码$\Vx$只需要一个矩阵-向量乘法操作。
为了编码向量,我们使用编码函数:
\begin{equation}
    f(\Vx)=\MD^\top\Vx.
\end{equation}
进一步使用矩阵乘法,我们也可以定义\glssymbol{PCA}重构操作:
\begin{equation}
\label{eq:2.67}
    r(\Vx)=g(f(\Vx)) = \MD\MD^\top \Vx.
\end{equation}


接下来,我们需要挑选编码矩阵$\MD$。
要做到这一点,我们回顾最小化输入和重构之间$L^2$距离的这个想法。
因为我们用相同的矩阵$\MD$对所有点进行解码,我们不能再孤立地看待每个点。
反之,我们必须最小化所有维数和所有点上的误差矩阵的\gls{frobenius_norm}:
\begin{equation}
\label{eq:2.68}
    \MD^* =  \underset{\MD}{\arg\min} \sqrt{\sum_{i,j}\left( \Vx_j^{(i)} - r(\Vx^{(i)})_j\right)^2} \text{ subject to } \MD^\top\MD = \MI_l.
\end{equation}

为了推导用于寻求$\MD^*$的算法,我们首先考虑$l=1$的情况。
在这种情况下,$\MD$是一个单一向量$\Vd$。
将\eqnref{eq:2.67}代入\eqnref{eq:2.68},简化$\MD$为$\Vd$,问题简化为
\begin{equation}
\label{eq:2.67}
    \Vd^* = \underset{\Vd}{\arg\min} \sum_i \norm{\Vx^{(i)} - \Vd\Vd^\top \Vx^{(i)}}_2^2
    \text{ subject to } \norm{\Vd}_2 = 1.
\end{equation}

% -- 47 --

上述公式是直接代入得到的,但不是文体表述最舒服的方式。
在上述公式中,我们将标量$\Vd^\top\Vx^{(i)}$放在向量$\Vd$的右边。
将该标量放在左边的写法更为传统。
于是我们通常写作如下:
\begin{equation}
\label{eq:2.68}
    \Vd^* = \underset{\Vd}{\arg\min} \sum_i \norm{\Vx^{(i)} - \Vd^\top \Vx^{(i)}\Vd}_2^2
        \text{ subject to } \norm{\Vd}_2 = 1,
\end{equation}
或者,考虑到标量的转置和自身相等,我们也可以写作:
\begin{equation}
    \Vd^* = \underset{\Vd}{\arg\min} \sum_i \norm{\Vx^{(i)} - \Vx^{(i)\top}\Vd\Vd}_2^2
        \text{ subject to } \norm{\Vd}_2 = 1.
\end{equation}
读者应该对这些重排写法慢慢熟悉起来。


此时,使用单一矩阵来重述问题,比将问题写成求和形式更有帮助。
这有助于我们使用更紧凑的符号。
将表示各点的向量堆叠成一个矩阵,记为$\MX\in\SetR^{m\times n}$,其中$\MX_{i,:}=\Vx^{(i)^\top}$。
原问题可以重新表述为:
\begin{equation}
    \Vd^* = \underset{\Vd}{\arg\min} \norm{\MX - \MX\Vd\Vd^\top}_F^2
        \text{ subject to } \Vd^\top \Vd = 1.
\end{equation}
暂时不考虑约束,我们可以将\gls{frobenius_norm}简化成下面的形式:
\begin{equation}
     \underset{\Vd}{\arg\min} \norm{\MX - \MX \Vd\Vd^\top}_F^2
\end{equation}
\begin{equation}
    = \underset{\Vd}{\arg\min} \, \Tr \left( \left( \MX - \MX \Vd\Vd^\top  \right)^\top \left( \MX - \MX \Vd\Vd^\top  \right) \right)
\end{equation}
(\eqnref{eq:2.49})
\begin{equation}
    = \underset{\Vd}{\arg\min} \, \Tr \left( \MX^\top\MX - \MX^\top\MX \Vd\Vd^\top - \Vd\Vd^\top \MX^\top\MX + \Vd\Vd^\top \MX^\top\MX\Vd\Vd^\top  \right)
\end{equation}
\begin{equation}
    = \underset{\Vd}{\arg\min} \, \Tr( \MX^\top\MX)  - \Tr(\MX^\top\MX \Vd\Vd^\top)  - \Tr(\Vd\Vd^\top \MX^\top\MX) + \Tr(\Vd\Vd^\top \MX^\top\MX\Vd\Vd^\top)
\end{equation}
\begin{equation}
    = \underset{\Vd}{\arg\min} \, - \Tr(\MX^\top\MX \Vd\Vd^\top)  - \Tr(\Vd\Vd^\top \MX^\top\MX) + \Tr(\Vd\Vd^\top \MX^\top\MX\Vd\Vd^\top)
\end{equation}
(因为与$\Vd$无关的项不影响$\arg\min$)
\begin{equation}
    = \underset{\Vd}{\arg\min} \, - 2\Tr(\MX^\top\MX \Vd\Vd^\top) + \Tr(\Vd\Vd^\top \MX^\top\MX\Vd\Vd^\top)
\end{equation}
(因为循环改变迹运算中相乘矩阵的顺序不影响结果,如\eqnref{eq:2.52}所示)
\begin{equation}
    = \underset{\Vd}{\arg\min} \, - 2\Tr(\MX^\top\MX \Vd\Vd^\top) + \Tr(\MX^\top\MX\Vd\Vd^\top\Vd\Vd^\top )
\end{equation}
(再次使用上述性质)

% -- 48 --

此时,我们再来考虑约束条件:
\begin{equation}
    \underset{\Vd}{\arg\min} \, - 2\Tr(\MX^\top\MX \Vd\Vd^\top) + \Tr(\MX^\top\MX\Vd\Vd^\top\Vd\Vd^\top )
    \text{ subject to } \Vd^\top \Vd = 1
\end{equation}
\begin{equation}
    = \underset{\Vd}{\arg\min} \, - 2\Tr(\MX^\top\MX \Vd \Vd^\top) + \Tr(\MX^\top\MX\Vd\Vd^\top )
    \text{ subject to } \Vd^\top \Vd = 1
\end{equation}
 (因为约束条件)
 \begin{equation}
     = \underset{\Vd}{\arg\min} \, - \Tr(\MX^\top\MX \Vd\Vd^\top)
     \text{ subject to } \Vd^\top \Vd = 1
 \end{equation}
 \begin{equation}
     = \underset{\Vd}{\arg\max} \, \Tr(\MX^\top\MX \Vd\Vd^\top)
     \text{ subject to } \Vd^\top \Vd = 1
 \end{equation}
 \begin{equation}
     = \underset{\Vd}{\arg\max} \, \Tr(\Vd^\top\MX^\top\MX \Vd)
     \text{ subject to } \Vd^\top \Vd = 1 .
 \end{equation}


这个优化问题可以通过特征分解来求解。
具体地,最优的$\Vd$是$\MX^\top\MX$最大特征值对应的特征向量。


以上推导特定于$l=1$的情况, 仅得到了第一个主成分。
更一般地,当我们希望得到主成分的基时,矩阵$\MD$由前$l$个最大的特征值对应的特征向量组成。
这个结论可以通过归纳法证明,我们建议将此证明作为练习。


线性代数是理解深度学习所必须掌握的基础数学学科之一。
另一门在机器学习中无处不在的重要数学学科是概率论,我们将在下章探讨。



% -- 49 --

% !Mode:: "TeX:UTF-8"
% Translator: Kai Li
\chapter{概率与信息论}
\label{chap:probability_and_information_theory}

本章我们讨论概率论和信息论。

概率论是用于表示不确定性\gls{statement}的数学框架。
%(statement要不要换一下说法)
它不仅提供了量化不确定性的方法,也提供了用于导出新的不确定性\firstgls{statement}的公理。
在\gls{AI}领域,概率论主要有两种用途。
首先,概率法则告诉我们~\glssymbol{AI}~系统如何推理,据此我们设计一些算法来计算或者估算由概率论导出的表达式。
其次,我们可以用概率和统计从理论上分析我们提出的~\glssymbol{AI}~系统的行为。

概率论是众多科学学科和工程学科的基本工具。
我们提供这一章,是为了确保那些背景偏软件工程而较少接触概率论的读者也可以理解本书的内容。

概率论使我们能够提出不确定的\gls{statement}以及在不确定性存在的情况下进行推理,而信息论使我们能够量化\gls{PD}中的不确定性总量。

如果你已经对概率论和信息论很熟悉了,那么除了\secref{sec:structured_probabilistic_models_chap3}以外的整章内容,你都可以跳过。
而在\secref{sec:structured_probabilistic_models_chap3}中,我们会介绍用来描述机器学习中\gls{structured_probabilistic_models}的图。
即使你对这些主题没有任何的先验知识,本章对于完成深度学习的研究项目来说也已经足够,尽管如此我们还是建议你能够参考一些额外的资料,例如~\cite{Jaynes03}。

% -- 51 --

\section{为什么要使用概率?}
\label{sec:why_probability}

计算机科学的许多分支处理的实体大部分都是完全确定且必然的。
程序员通常可以安全地假定CPU将完美地执行每条机器指令。
虽然硬件错误确实会发生,但它们足够罕见,以致于大部分软件应用在设计时并不需要考虑这些因素的影响。
鉴于许多计算机科学家和软件工程师在一个相对干净和确定的环境中工作,机器学习对于概率论的大量使用是很令人吃惊的。

这是因为机器学习通常必须处理不确定量,有时也可能需要处理随机(非确定性的)量。
%(这里uncertain和stochastic有什么区别?)
不确定性和随机性可能来自多个方面。
至少从20世纪80年代开始,研究人员就对使用概率论来量化不确定性提出了令人信服的论据。
这里给出的许多论据都是根据~\cite{Pearl88}的工作总结或启发得到的。

几乎所有的活动都需要一些在不确定性存在的情况下进行推理的能力。
事实上,除了那些被定义为真的数学\gls{statement},我们很难认定某个命题是千真万确的或者确保某件事一定会发生。

不确定性有三种可能的来源:

\begin{enumerate}
\item 被建模系统内在的随机性。
例如,大多数\gls{quantum_mechanics}的解释,都将\gls{subatomic}粒子的动力学描述为概率的。
我们还可以创建一些我们假设具有随机动态的理论情境,例如一个假想的纸牌游戏,在这个游戏中我们假设纸牌被真正混洗成了随机顺序。

\item 不完全观测。
即使是确定的系统,当我们不能观测到所有驱动系统行为的变量时,该系统也会呈现随机性。
例如,在Monty Hall问题中,一个游戏节目的参与者被要求在三个门之间选择,并且会赢得放置在选中门后的奖品。
其中两扇门通向山羊,第三扇门通向一辆汽车。
选手的每个选择所导致的结果是确定的,但是站在选手的角度,结果是不确定的。

\item 不完全建模。
当我们使用一些必须舍弃某些观测信息的模型时,舍弃的信息会导致模型的预测出现不确定性。
例如,假设我们制作了一个机器人,它可以准确地观察周围每一个对象的位置。
在对这些对象将来的位置进行预测时,如果机器人采用的是离散化的空间,那么离散化的方法将使得机器人无法确定对象们的精确位置:因为每个对象都可能处于它被观测到的离散单元的任何一个角落。
\end{enumerate}

% -- 52 --  

在很多情况下,使用一些简单而不确定的规则要比复杂而确定的规则更为实用,即使真正的规则是确定的并且我们建模的系统可以足够精确地容纳复杂的规则。
例如,``多数鸟儿都会飞''这个简单的规则描述起来很简单很并且使用广泛,而正式的规则——``除了那些还没学会飞翔的幼鸟,因为生病或是受伤而失去了飞翔能力的鸟,包括食火鸟(cassowary)、鸵鸟(ostrich)、几维(kiwi,一种新西兰产的无翼鸟)等不会飞的鸟类……以外,鸟儿会飞'',很难应用、维护和沟通,即使经过这么多的努力,这个规则还是很脆弱而且容易失效。

尽管我们的确需要一种用以对不确定性进行表示和推理的方法,但是概率论并不能明显地提供我们在\gls{AI}领域需要的所有工具。
概率论最初的发展是为了分析事件发生的频率。
我们可以很容易地看出概率论,对于像在扑克牌游戏中抽出一手特定的牌这种事件的研究中,是如何使用的。
这类事件往往是可以重复的。
当我们说一个结果发生的概率为$p$,这意味着如果我们反复实验(例如,抽取一手牌)无限次,有$p$的比例可能会导致这样的结果。
这种推理似乎并不立即适用于那些不可重复的命题。
如果一个医生诊断了病人,并说该病人患流感的几率为40\%,这意味着非常不同的事情——我们既不能让病人有无穷多的副本,也没有任何理由去相信病人的不同副本在具有不同的潜在条件下表现出相同的症状。
在医生诊断病人的例子中,我们用概率来表示一种\firstgls{degree_of_belief},其中1表示非常肯定病人患有流感,而0表示非常肯定病人没有流感。
前面那种概率,直接与事件发生的频率相联系,被称为\firstgls{frequentist_probability};而后者,涉及到确定性水平,被称为\firstgls{bayesian_probability}。

关于不确定性的常识推理,如果我们已经列出了若干条我们期望它具有的性质,那么满足这些性质的唯一一种方法就是将\gls{bayesian_probability}和\gls{frequentist_probability}视为等同的。
例如,如果我们要在扑克牌游戏中根据玩家手上的牌计算她能够获胜的概率,我们使用和医生情境完全相同的公式,就是我们依据病人的某些症状计算她是否患病的概率。
为什么一小组常识性假设蕴含了必须是相同的公理控制两种概率?更多的细节参见~\cite{Ramsey1926}。

% -- 53 --

概率可以被看作是用于处理不确定性的逻辑扩展。
逻辑提供了一套形式化的规则,可以在给定某些命题是真或假的假设下,判断另外一些命题是真的还是假的。
概率论提供了一套形式化的规则,可以在给定一些命题的\gls{likelihood}后,计算其他命题为真的\gls{likelihood}。

\section{\glsentrytext{RV}}
\label{sec:random_variables}

\firstgls{RV}是可以随机地取不同值的变量。
我们通常用无格式字体(plain typeface)中的小写字母来表示\gls{RV}本身,而用手写体中的小写字母来表示\gls{RV}能够取到的值。
例如,$x_1$和$x_2$都是\gls{RV}~$\RSx$可能的取值。
对于向量值变量,我们会将\gls{RV}写成$\RVx$,它的一个可能取值为$\Vx$。
就其本身而言,一个\gls{RV}只是对可能的状态的描述;它必须伴随着一个\gls{PD}来指定每个状态的可能性。

\gls{RV}可以是离散的或者连续的。
离散\gls{RV}拥有有限或者可数无限多的状态。
注意这些状态不一定非要是整数;它们也可能只是一些被命名的状态而没有数值。
连续\gls{RV}伴随着实数值。

\section{\glsentrytext{PD}}
\label{sec:probability_distributions}

\firstgls{PD}用来描述\gls{RV}或一簇\gls{RV}在每一个可能取到的状态的可能性大小。
我们描述\gls{PD}的方式取决于\gls{RV}是离散的还是连续的。

\subsection{离散型变量和\glsentrytext{PMF}}
\label{sec:discrete_variables_and_probability_mass_functions}

离散型变量的\gls{PD}可以用\firstall{PMF}\footnote{译者注:国内有些教材也将它翻译成概率分布律。}来描述。
我们通常用大写字母$P$来表示\gls{PMF}。
通常每一个\gls{RV}都会有一个不同的\gls{PMF},并且读者必须根据\gls{RV}来推断所使用的~\glssymbol{PMF},而不是根据函数的名称来推断;例如,$P(\RSx)$ 通常和$P(\RSy)$不一样。

% -- 54 --

\gls{PMF}将\gls{RV}能够取得的每个状态映射到\gls{RV}取得该状态的概率。
$\RSx = x$的概率用$P(x)$来表示,概率为1表示$\RSx = x$是确定的,概率为0表示$\RSx=x$是不可能发生的。
有时为了使得\glssymbol{PMF}的使用不相互混淆,我们会明确写出\gls{RV}的名称:$P(\RSx = x)$。
有时我们会先定义一个\gls{RV},然后用$\sim$符号来说明它遵循的分布:$\RSx\sim P(\RSx)$。

\gls{PMF}可以同时作用于多个\gls{RV}。
这种多个变量的\gls{PD}被称为\firstgls{joint_probability_distribution}。
$P(\RSx = x, \RSy = y)$表示$\RSx = x$和$\RSy = y$同时发生的概率。我们也可以简写为$P(x, y)$。

如果一个函数$P$是\gls{RV} $\RSx$的~\glssymbol{PMF},必须满足下面这几个条件:
\begin{itemize}
\item $P$的定义域必须是$\RSx$所有可能状态的集合。

\item $\forall x \in \RSx, 0\le P(x)\le 1.$不可能发生的事件概率为0,并且不存在比这概率更低的状态。
类似的,能够确保一定发生的事件概率为1,而且不存在比这概率更高的状态。

\item $\sum_{x \in \RSx} P(x) = 1.$我们把这条性质称之为\firstgls{normalized}。
如果没有这条性质,当我们计算很多事件其中之一发生的概率时可能会得到大于1的概率。
\end{itemize}

例如,考虑一个离散型\gls{RV} $\RSx$有$k$个不同的状态。
我们可以假设$\RSx$是\firstgls{uniform_distribution}的(也就是将它的每个状态视为等可能的),通过将它的\glssymbol{PMF}设为
\begin{equation}
P(\RSx = x_i) = \frac{1}{k}
\end{equation}
对于所有的$i$都成立。
我们可以看出这满足上述成为\gls{PMF}的条件。
因为$k$是一个正整数,所以$\frac{1}{k}$是正的。我们也可以看出
\begin{equation}
\sum_i P(\RSx = x_i) = \sum_i \frac{1}{k} = \frac{k}{k} = 1,
\end{equation}
因此分布也满足归一化条件。

% -- 55 --

\subsection{连续型变量和\glsentrytext{PDF}}
\label{sec:continuous_variables_and_probability_density_functions}

当我们研究的对象是连续型\gls{RV}时,我们用\firstall{PDF}而不是\gls{PMF}来描述它的\gls{PD}。
如果一个函数$p$是\gls{PDF},必须满足下面这几个条件:
\begin{itemize}
\item $p$的定义域必须是$\RSx$所有可能状态的集合。

\item $\forall x \in \RSx, p(x)\ge 0.$注意,我们并不要求$p(x)\le 1$。

\item $\int p(x) dx = 1.$
\end{itemize}

\gls{PDF} $p(x)$并没有直接对特定的状态给出概率,相对的,它给出了落在面积为$\delta x$ 的无限小的区域内的概率为$p(x)\delta x$。

我们可以对\gls{PDF}求积分来获得点集的真实概率质量。
特别地,$x$落在集合$\SetS$中的概率可以通过$p(x)$对这个集合求积分来得到。
在单变量的例子中,$x$落在区间$[a, b]$的概率是$\int_{[a,b]} p(x)dx$。

为了给出一个连续型\gls{RV}的~\glssymbol{PDF}~的例子,我们可以考虑实数区间上的\gls{uniform_distribution}。
我们可以使用函数$u(x; a, b)$,其中$a$和$b$是区间的端点且满足$b>a$。
符号``$;$''表示``以什么为参数'';我们把$x$作为函数的自变量,$a$和$b$作为定义函数的参数。
为了确保区间外没有概率,我们对所有的$x\not\in[a,b]$,令$u(x; a, b)=0$。
在$[a,b]$内,有$u(x; a, b)= \frac{1}{b-a}$。
我们可以看出任何一点都非负。
另外,它的积分为1。我们通常用$\RSx \sim U(a, b)$表示$x$在$[a, b]$上是\gls{uniform_distribution}的。

\section{边缘概率}
\label{sec:marginal_probability}

有时候,我们知道了一组变量的\gls{joint_probability_distribution},但想要了解其中一个子集的\gls{PD}。
这种定义在子集上的\gls{PD}被称为\firstgls{marginal_probability_distribution}。

例如,假设有离散型\gls{RV} $\RSx$和$\RSy$,并且我们知道$P(\RSx, \RSy)$。
我们可以依据下面的\firstgls{sum_rule}来计算$P(\RSx)$:
\begin{equation}
\forall x \in \RSx, P(\RSx = x) = \sum_y P(\RSx = x, \RSy = y).
\end{equation}

% -- 56 --

``边缘概率''的名称来源于手算边缘概率的计算过程。
当$P(\RSx, \RSy)$的每个值被写在由每行表示不同的$x$值,每列表示不同的$y$值形成的网格中时,对网格中的每行求和是很自然的事情,然后将求和的结果$P(x)$写在每行右边的纸的边缘处。

对于连续型变量,我们需要用积分替代求和:
\begin{equation}
p(x) = \int p(x, y)dy.
\end{equation}

\section{\glsentrytext{conditional_probability}}
\label{sec:conditional_probability}

在很多情况下,我们感兴趣的是某个事件,在给定其他事件发生时出现的概率。
这种概率叫做\gls{conditional_probability}。
我们将给定$\RSx = x$,$\RSy = y$发生的\gls{conditional_probability}记为$P(\RSy = y\mid \RSx =x)$。
这个\gls{conditional_probability}可以通过下面的公式计算:
\begin{equation}
P(\RSy = y\mid \RSx = x) = \frac{P(\RSy = y, \RSx = x)}{P(\RSx = x)} .
\label{eq:3.5}
\end{equation}
\gls{conditional_probability}只在$P(\RSx = x)>0$时有定义。
我们不能计算给定在永远不会发生的事件上的\gls{conditional_probability}。

这里需要注意的是,不要把\gls{conditional_probability}和计算当采用某个动作后会发生什么相混淆。
假定某个人说德语,那么他是德国人的\gls{conditional_probability}是非常高的,但是如果随机选择的一个人会说德语,他的国籍不会因此而改变。
计算一个行动的后果被称为\firstgls{intervention_query}。
\gls{intervention_query}属于\firstgls{causal_modeling}的范畴,我们不会在本书中讨论。

\section{\glsentrytext{conditional_probability}的\glsentrytext{chain_rule}}
\label{sec:the_chain_rule_of_conditional_probabilities}

任何多维\gls{RV}的\gls{joint_probability_distribution},都可以分解成只有一个变量的\gls{conditional_probability}相乘的形式:
\begin{equation}
P(\RSx^{(1)}, \ldots, \RSx^{(n)}) = P(\RSx^{(1)}) \Pi_{i=2}^n P(\RSx^{(i)} \mid \RSx^{(1)}, \ldots, \RSx^{(i-1)}) .
\label{eq:chap3_chain_rule}
\end{equation}

% -- 57 --

这个规则被称为概率的\firstgls{chain_rule}或者\firstgls{product_rule}。
它可以直接从\eqnref{eq:3.5}\gls{conditional_probability}的定义中得到。
例如,使用两次定义可以得到
\begin{eqnarray*}
P(\RSa, \RSb, \RSc) &=& P(\RSa \mid \RSb, \RSc) P(\RSb, \RSc)\\
P(\RSb, \RSc) &=& P(\RSb \mid \RSc) P(\RSc)\\
P(\RSa, \RSb, \RSc) &=& P(\RSa \mid \RSb, \RSc) P(\RSb \mid \RSc) P(\RSc).
\end{eqnarray*}

\section{独立性和条件独立性}
\label{sec:independence_and_conditional_independence}

两个\gls{RV} $\RSx$和$\RSy$,如果它们的\gls{PD}可以表示成两个因子的乘积形式,并且一个因子只包含$\RSx$另一个因子只包含$\RSy$,我们就称这两个\gls{RV}是\firstgls{independent}:
\begin{equation}
\forall x \in \RSx, y \in \RSy, p(\RSx = x, \RSy = y) = p(\RSx = x)p(\RSy = y).
\end{equation}

如果关于$\RSx$和$\RSy$的\gls{conditional_probability}分布对于$z$的每一个值都可以写成乘积的形式,那么这两个\gls{RV} $\RSx$和$\RSy$在给定\gls{RV}~$z$时是\firstgls{conditionally_independent}:
\begin{equation}
\forall x \in \RSx, y \in \RSy, z \in \RSz, p( \RSx=x, \RSy=y \mid \RSz=z) =
p(\RSx = x \mid \RSz = z) p(\RSy = y \mid \RSz = z).
\end{equation}

我们可以采用一种简化形式来表示独立性和条件独立性:$\RSx \bot \RSy$表示$\RSx$和$\RSy$相互独立,$\RSx \bot \RSy \mid \RSz$表示$\RSx$和$\RSy$在给定$\RSz$时条件独立。

\section{\glsentrytext{expectation}、\glsentrytext{variance}和\glsentrytext{covariance}}
\label{sec:expectation_variance_and_covariance}

函数$f(x)$关于某分布$P(\RSx)$的\firstgls{expectation}或者\firstgls{expected_value}是指,当$x$由$P$产生,$f$作用于$x$时,$f(x)$的平均值。
对于离散型\gls{RV},这可以通过求和得到:
\begin{equation}
\SetE_{\RSx \sim P}[f(x)] = \sum_x P(x)f(x),
\end{equation}
对于连续型\gls{RV}可以通过求积分得到:
\begin{equation}
\SetE_{\RSx \sim p}[f(x)] = \int p(x)f(x)dx.
\end{equation}
当\gls{PD}在上下文中指明时,我们可以只写出\gls{expectation}作用的\gls{RV}的名称来进行简化,例如$\SetE_{\RSx}[f(x)]$。
如果\gls{expectation}作用的\gls{RV}也很明确,我们可以完全不写脚标,就像$\SetE[f(x)]$。
默认地,我们假设$\SetE[\cdot]$表示对方括号内的所有\gls{RV}的值求平均。
类似的,当没有歧义时,我们还可以省略方括号。

% -- 58 --

\gls{expectation}是线性的,例如,
\begin{equation}
\SetE_{\RSx}[\alpha f(x) + \beta g(x)] = \alpha \SetE_{\RSx}[f(x)] + \beta \SetE_{\RSx}[g(x)],
\end{equation}
其中$\alpha$和$\beta$不依赖于$x$。

\firstgls{variance}衡量的是当我们对$x$依据它的\gls{PD}进行采样时,\gls{RV} $\RSx$的函数值会呈现多大的差异:
\begin{equation}
\Var(f(x)) = \SetE \left [(f(x) - \SetE[f(x)])^2 \right].
\end{equation}
当\gls{variance}很小时,$f(x)$的值形成的簇比较接近它们的\gls{expected_value}。\gls{variance}的平方根被称为\firstgls{standard_deviation}。

\firstgls{covariance}在某种意义上给出了两个变量线性相关性的强度以及这些变量的尺度:
\begin{equation}
\Cov(f(x), g(y)) = \SetE[ ( f(x)-\SetE[f(x)] )( g(y)-\SetE[g(y)] ) ].
\end{equation}
\gls{covariance}的绝对值如果很大则意味着变量值变化很大并且它们同时距离各自的均值很远。
如果\gls{covariance}是正的,那么两个变量都倾向于同时取得相对较大的值。
如果\gls{covariance}是负的,那么其中一个变量倾向于取得相对较大的值的同时,另一个变量倾向于取得相对较小的值,反之亦然。
其他的衡量指标如\firstgls{correlation}将每个变量的贡献归一化,为了只衡量变量的相关性而不受各个变量尺度大小的影响。

\gls{covariance}和相关性是有联系的,但实际上不同的概念。
它们是有联系的,因为两个变量如果相互独立那么它们的\gls{covariance}为零,如果两个变量的\gls{covariance}不为零那么它们一定是相关的。
然而,独立性又是和\gls{covariance}完全不同的性质。
两个变量如果\gls{covariance}为零,它们之间一定没有线性关系。
独立性是比零\gls{covariance}的要求更强,因为独立性还排除了非线性的关系。
两个变量相互依赖但是具有零\gls{covariance}是可能的。
例如,假设我们首先从区间$[-1, 1]$上的\gls{uniform_distribution}中采样出一个实数$x$。
然后我们对一个\gls{RV} $s$进行采样。
$s$以$\frac{1}{2}$的概率值为1,否则为-1。
我们可以通过令$y=sx$来生成一个\gls{RV} $y$。
显然,$x$和$y$不是相互独立的,因为$x$完全决定了$y$的尺度。
然而,$\Cov(x,y)=0$。

% -- 59 --

随机向量$\Vx \in \SetR^n$的\firstgls{covariance_matrix}是一个$n\times n$的矩阵,并且满足
\begin{equation}
\Cov(\RVx)_{i,j} = \Cov(\RSx_i, \RSx_j).
\end{equation}
\gls{covariance_matrix}的对角元是方差:
\begin{equation}
\Cov(\RSx_i, \RSx_i) = \Var(\RSx_i).
\end{equation}

\section{常用\glsentrytext{PD}}
\label{sec:common_probability_distributions}

许多简单的\gls{PD}在机器学习的众多领域中都是有用的。

\subsection{\glsentrytext{bernoulli_distribution}}
\label{sec:bernoulli_distribution}

\firstgls{bernoulli_distribution}是单个二值\gls{RV}的分布。
它由单个参数$\phi \in [0, 1]$控制,$\phi$给出了\gls{RV}等于1的概率。
它具有如下的一些性质:
\begin{gather}
P(\RSx =1) = \phi\\
P(\RSx =0) = 1-\phi\\
P(\RSx = x) = \phi^x (1-\phi)^{1-x}\\
\SetE_{\RSx}[\RSx] = \phi\\
\Var_{\RSx}(\RSx) = \phi(1-\phi)
\end{gather}

\subsection{\glsentrytext{multinoulli_distribution}}
\label{sec:multinoulli_distribution}

\firstgls{multinoulli_distribution}或者\firstgls{categorical_distribution}是指在具有$k$个不同状态的单个离散型\gls{RV}上的分布,其中$k$是一个有限值。\footnote{``multinoulli''这个术语是最近被Gustavo Lacerdo发明、被\cite{MurphyBook2012}推广的。
\gls{multinoulli_distribution}是\firstgls{multinomial_distribution}的一个特例。
\gls{multinomial_distribution}是$\{0,\ldots, n\}^k$中的向量的分布,用于表示当对~\gls{multinoulli_distribution}采样$n$次时$k$个类中的每一个被访问的次数。
很多文章使用``\gls{multinomial_distribution}''而实际上说的是~\gls{multinoulli_distribution},但是他们并没有说是对$n=1$的情况,这点需要注意。}
\gls{multinoulli_distribution}由向量$\Vp \in [0, 1]^{k-1}$参数化,其中每一个分量$p_i$表示第$i$个状态的概率。
最后的第$k$个状态的概率可以通过$1-\Vone^\top \Vp$给出。
注意我们必须限制$\Vone^\top \Vp \le 1$。
\gls{multinoulli_distribution}经常用来表示对象分类的分布,所以我们很少假设状态1具有数值1之类的。
因此,我们通常不需要去计算~\gls{multinoulli_distribution}的\gls{RV}的\gls{expectation}和\gls{variance}。

% -- 60 --

\gls{bernoulli_distribution}和~\gls{multinoulli_distribution}足够用来描述在它们领域内的任意分布。
它们能够描述这些分布,不是因为它们特别强大,而是因为它们的领域很简单;
它们可以对那些,能够将所有的状态进行枚举的离散型\gls{RV}进行建模。
当处理的是连续型\gls{RV}时,会有不可数无限多的状态,所以任何通过少量参数描述的\gls{PD}都必须在分布上加以严格的限制。

\subsection{\glsentrytext{gaussian_distribution}}
\label{sec:gaussian_distribution}


实数上最常用的分布就是\firstgls{normal_distribution},也称为\firstgls{gaussian_distribution}:
\begin{equation}
\CalN(x; \mu, \sigma^2) = \sqrt{\frac{1}{2\pi \sigma^2}} \exp \left ( -\frac{1}{2\sigma^2} (x-\mu)^2 \right ).
\end{equation}

\figref{fig:chap3_normal_color}画出了\gls{normal_distribution}的\gls{PDF}。
% fig 3.1
\begin{figure}[!htb]
\ifOpenSource
\centerline{\includegraphics{figure.pdf}}
\else
\centerline{\includegraphics{Chapter3/figures/normal_color}}
\fi
\captionsetup{singlelinecheck=off}
\caption[.]{\gls{normal_distribution}。
\gls{normal_distribution} $\CalN(x; \mu, \sigma^2)$呈现经典的``钟形曲线''的形状,其中中心峰的$x$坐标由$\mu$给出,峰的宽度受$\sigma$控制。
在这个示例中,我们展示的是\firstgls{standard_normal_distribution},其中$\mu=0, \sigma = 1$。}
\label{fig:chap3_normal_color}
\end{figure}

\gls{normal_distribution}由两个参数控制,$\mu \in \SetR$和$\sigma \in (0, \infty)$。
参数$\mu$给出了中心峰值的坐标,这也是分布的均值:$\SetE[\RSx] = \mu$。
分布的\gls{standard_error}用$\sigma$表示,\gls{variance}用$\sigma^2$表示。

当我们要对\gls{PDF}求值时,我们需要对$\sigma$平方并且取倒数。
当我们需要经常对不同参数下的\gls{PDF}求值时,一种更高效的参数化分布的方式是使用参数$\beta \in (0, \infty)$,来控制分布的\firstgls{precision}(或方差的倒数):
\begin{equation}
\CalN(x; \mu, \beta^{-1}) = \sqrt{\frac{\beta}{2\pi}} \exp \left(  -\frac{1}{2}\beta (x-\mu)^2 \right).
\label{eq:3.22}
\end{equation}

采用\gls{normal_distribution}在很多应用中都是一个明智的选择。
当我们由于缺乏关于某个实数上分布的先验知识而不知道该选择怎样的形式时,\gls{normal_distribution}是默认的比较好的选择,其中有两个原因。

% -- 61 --

第一,我们想要建模的很多分布的真实情况是比较接近\gls{normal_distribution}的。
\firstgls{central_limit_theorem}说明很多独立\gls{RV}的和近似服从\gls{normal_distribution}。
这意味着在实际中,很多复杂系统都可以被成功地建模成\gls{normal_distribution}的噪声,即使系统可以被分解成一些更结构化的部分。

第二,在具有相同方差的所有可能的\gls{PD}中,\gls{normal_distribution}在实数上具有最大的不确定性。
因此,我们可以认为\gls{normal_distribution}是对模型加入的先验知识量最少的分布。
充分利用和证明这个想法需要更多的数学工具,我们推迟到\secref{sec:calculus_of_variations}进行讲解。

\gls{normal_distribution}可以推广到$\SetR^n$空间,这种情况下被称为\firstgls{multivariate_normal_distribution}。
它的参数是一个正定对称矩阵$\VSigma$:
\begin{equation}
\label{eq:3.23}
\CalN(\Vx; \Vmu, \VSigma) = \sqrt{ \frac{1}{ (2\pi)^n \det(\VSigma)}}  \exp \left ( -\frac{1}{2} (\Vx-\Vmu)^\top \VSigma^{-1} (\Vx- \Vmu) \right).
\end{equation}

参数$\Vmu$仍然表示分布的均值,只不过现在是向量值。
参数$\VSigma$给出了分布的\gls{covariance_matrix}。
和单变量的情况类似,当我们希望对很多不同参数下的\gls{PDF}多次求值时,\gls{covariance_matrix}并不是一个很高效的参数化分布的方式,因为对\gls{PDF}求值时需要对$\VSigma$求逆。
我们可以使用一个\firstgls{precision_matrix} $\Vbeta$ 进行替代:
\begin{equation}
\CalN(\Vx; \Vmu, \Vbeta^{-1}) = \sqrt{ \frac{\det(\Vbeta)}{ (2\pi)^n}}  \exp \left ( -\frac{1}{2} (\Vx-\Vmu)^\top \Vbeta (\Vx- \Vmu) \right).
\end{equation}

% -- 62 --

我们常常把\gls{covariance_matrix}固定成一个对角阵。
一个更简单的版本是\firstgls{isotropic}\gls{gaussian_distribution},它的\gls{covariance_matrix}是一个标量乘以单位阵。

\subsection{\glsentrytext{exponential_distribution}和\glsentrytext{laplace_distribution}}
\label{sec:exponential_and_laplace_distributions}

在深度学习中,我们经常会需要一个在$x=0$点处取得边界点(sharp point)的分布。
为了实现这一目的,我们可以使用\firstgls{exponential_distribution}:
\begin{equation}
p(x; \lambda) = \lambda \Vone_{x\ge 0} \exp(-\lambda x).
\end{equation}
指数分布使用\gls{indicator_function}(indicator function)$\Vone_{x\ge 0}$来使得当$x$取负值时的概率为零。

一个联系紧密的\gls{PD}是\firstgls{laplace_distribution},它允许我们在任意一点$\mu$处设置概率质量的峰值
\begin{equation}
\label{eq:chap3_laplace}
\text{Laplace}(x; \mu, \gamma) = \frac{1}{2\gamma} \exp \left( -\frac{|x-\mu|}{\gamma}  \right).
\end{equation}

\subsection{\glsentrytext{dirac_distribution}和\glsentrytext{empirical_distribution}}
\label{sec:the_dirac_distribution_and_empirical_distribution}

在一些情况下,我们希望概率分布中的所有质量都集中在一个点上。
这可以通过\firstgls{dirac_delta_function} $\delta(x)$定义\gls{PDF}来实现:
\begin{equation}
p(x) = \delta(x-\mu).
\end{equation}
\gls{dirac_delta_function}被定义成在除了0以外的所有点的值都为0,但是积分为1。
\gls{dirac_delta_function}不像普通函数一样对$x$的每一个值都有一个实数值的输出,它是一种不同类型的数学对象,被称为\firstgls{generalized_function},\gls{generalized_function}是依据积分性质定义的数学对象。
我们可以把~\gls{dirac_delta_function}想成一系列函数的极限点,这一系列函数把除0以外的所有点的概率密度越变越小。

% -- 63 --

通过把$p(x)$定义成$\delta$函数左移$-\mu$个单位,我们得到了一个在$x=\mu$ 处具有无限窄也无限高的峰值的概率质量。

\gls{dirac_distribution}经常作为\firstgls{empirical_distribution}的一个组成部分出现:
\begin{equation}
\hat{p}(\Vx) = \frac{1}{m} \sum_{i=1}^m \delta(\Vx - \Vx^{(i)})
\end{equation}
\gls{empirical_distribution}将概率密度$\frac{1}{m}$赋给$m$个点$\Vx^{(1)}, \ldots, \Vx^{(m)}$中的每一个,这些点是给定的数据集或者采样的集合。
只有在定义连续型\gls{RV}的经验分布时,\gls{dirac_delta_function}才是必要的。
对于离散型\gls{RV},情况更加简单:\gls{empirical_distribution}可以被定义成一个~\gls{multinoulli_distribution},对于每一个可能的输入,其概率可以简单地设为在训练集上那个输入值的\firstgls{empirical_frequency}。

当我们在训练集上训练模型时,我们可以认为从这个训练集上得到的\gls{empirical_distribution}指明了我们采样来源的分布。
关于\gls{empirical_distribution}另外一种重要的观点是,它是训练数据的似然最大的那个概率密度函数(见\secref{sec:maximum_likelihood_estimation})。

\subsection{分布的混合}
\label{sec:mixtures_of_distributions}

通过组合一些简单的\gls{PD}来定义新的\gls{PD}也是很常见的。
一种通用的组合方法是构造\firstgls{mixture_distribution}。
混合分布由一些组件(component)分布构成。
每次实验,样本是由哪个组件分布产生的取决于从一个~\gls{multinoulli_distribution}中采样的结果:
\begin{equation}
P(\RSx) = \sum_i P(\RSc = i) P(\RSx \mid \RSc = i),
\end{equation}
这里$P(\RSc)$是对各组件的一个~\gls{multinoulli_distribution}。

我们已经看过一个混合分布的例子了:实值变量的\gls{empirical_distribution}对于每一个训练实例来说,就是以~\gls{dirac_distribution}为组件的混合分布。

% -- 64 --

混合模型是组合简单\gls{PD}来生成更丰富的分布的一种简单策略。
在\chapref{chap:structured_probabilistic_models_for_deep_learning}中,我们更加详细地探讨从简单\gls{PD}构建复杂模型的技术。

混合模型使我们能够一瞥以后会用到的一个非常重要的概念——\firstgls{latent_variable}。
\gls{latent_variable}是我们不能直接观测到的\gls{RV}。
混合模型的组件标识变量$\RSc$就是其中一个例子。
\gls{latent_variable}在联合分布中可能和$\RSx$有关,在这种情况下,$P(\RSx, \RSc) = P(\RSx \mid \RSc)P(\RSc)$。
\gls{latent_variable}的分布$P(\RSc)$以及关联\gls{latent_variable}和观测变量的条件分布$P(\RSx \mid\RSc)$,共同决定了分布$P(\RSx)$的形状,尽管描述$P(\RSx)$时可能并不需要\gls{latent_variable}。
\gls{latent_variable}将在\secref{sec:learning_about_dependencies}中深入讨论。

一个非常强大且常见的混合模型是\firstgls{GMM},它的组件$p(\RVx \mid \RSc= i)$是\gls{gaussian_distribution}。
每个组件都有各自的参数,均值$\Vmu^{(i)}$和\gls{covariance_matrix} $\VSigma^{(i)}$。
有一些混合可以有更多的限制。
例如,\gls{covariance_matrix}可以通过$\VSigma^{(i)} = \VSigma, \forall i$的形式在组件之间共享参数。
和单个\gls{gaussian_distribution}一样,\gls{GMM}有时会限制每个组件的\gls{covariance_matrix}为对角的或者各向同性的(标量乘以单位矩阵)。

除了均值和\gls{covariance}以外,\gls{GMM}的参数指明了给每个组件$i$的\firstgls{prior_probability} $\alpha_i = P(\RSc = i)$。
``先验''一词表明了在观测到$\RVx$~\emph{之前}传递给模型关于$\RSc$的信念。
作为对比,$P(\RSc \mid \Vx)$是\firstgls{posterior_probability},因为它是在观测到$\RSx$\emph{之后}进行计算的。
\gls{GMM}是概率密度的\firstgls{universal_approximator},在这种意义下,任何平滑的概率密度都可以用具有足够多组件的\gls{GMM}以任意精度来逼近。

\figref{fig:chap3_mog_color}演示了某个\gls{GMM}生成的样本。
% fig 3.2
\begin{figure}[!htb]
\ifOpenSource
\centerline{\includegraphics{figure.pdf}}
\else
\centerline{\includegraphics{Chapter3/figures/mog_color}}
\fi
\caption{来自\gls{GMM}的样本。
在这个示例中,有三个组件。
从左到右,第一个组件具有\gls{isotropic}的协方差矩阵,这意味着它在每个方向上具有相同的方差。 第二个组件具有对角的协方差矩阵,这意味着它可以沿着每个轴的对齐方向单独控制方差。
该示例中,沿着$x_2$轴的方差要比沿着$x_1$轴的方差大。 第三个组件具有满秩的协方差矩阵,使它能够沿着任意基的方向单独地控制方差。}
\label{fig:chap3_mog_color}
\end{figure}


\section{常用函数的有用性质}
\label{sec:useful_properties_of_common_functions}

某些函数在处理\gls{PD}时经常会出现,尤其是深度学习的模型中用到的\gls{PD}。

% -- 65 --

其中一个函数是~\textbf{\gls{logistic_sigmoid}}~函数:
\begin{equation}
\sigma(x) = \frac{1}{1+\exp(-x)}.
\end{equation}
\gls{logistic_sigmoid}~函数通常用来产生~\gls{bernoulli_distribution}中的参数$\phi$,因为它的范围是$(0,1)$,处在$\phi$的有效取值范围内。
\figref{fig:chap3_sigmoid_color}给出了sigmoid函数的图示。
sigmoid函数在变量取绝对值非常大的正值或负值时会出现\firstgls{saturate}现象,意味着函数会变得很平,并且对输入的微小改变会变得不敏感。
% fig 3.3
\begin{figure}[!htb]
\ifOpenSource
\centerline{\includegraphics{figure.pdf}}
\else
\centerline{\includegraphics{Chapter3/figures/sigmoid_color}}
\fi
\caption{\gls{logistic_sigmoid}函数。}
\label{fig:chap3_sigmoid_color}
\end{figure}

另外一个经常遇到的函数是\firstgls{softplus_function}\citep{secondorder:2001:nips}:
\begin{equation}
\zeta(x) = \log(1+\exp(x)).
\end{equation}
\gls{softplus_function}可以用来产生\gls{normal_distribution}的$\beta$和$\sigma$参数,因为它的范围是$(0,\infty)$。
当处理包含sigmoid函数的表达式时它也经常出现。
\gls{softplus_function}名来源于它是另外一个函数的平滑(或``软化'')形式,这个函数是
\begin{equation}
x^+ = \max(0, x).
\end{equation}
\figref{fig:chap3_softplus_color}给出了~\gls{softplus_function}的图示。
% fig 3.4
\begin{figure}[!htb]
\ifOpenSource
\centerline{\includegraphics{figure.pdf}}
\else
\centerline{\includegraphics{Chapter3/figures/softplus_color}}
\fi
\caption{\gls{softplus_function}。}
\label{fig:chap3_softplus_color}
\end{figure}


% -- 66 --

下面一些性质非常有用,你可能要记下来:
\begin{gather}
\sigma(x) = \frac{\exp(x)}{\exp(x)+\exp(0)}\\
\frac{d}{dx} \sigma(x) = \sigma(x)(1 - \sigma(x))\\
1-\sigma(x) = \sigma(-x)\\
\log \sigma(x) = -\zeta(-x)\\
\frac{d}{dx} \zeta(x) = \sigma(x)\\
\forall x \in (0, 1), \sigma^{-1}(x) = \log \left (  \frac{x}{1-x} \right)\\
\forall x>0, \zeta^{-1}(x) = \log(\exp(x) - 1)\\
\zeta(x) = \int_{-\infty}^x \sigma(y) dy\\
\zeta(x) - \zeta(-x) = x
\label{eq:3.41}
\end{gather}
函数$\sigma^{-1}(x)$在统计学中被称为\firstgls{logit},但这个函数在机器学习中很少用到。

% -- 67 --

\eqnref{eq:3.41}为函数名``softplus''提供了其他的正当理由。
\gls{softplus_function}被设计成\firstgls{positive_part_function}的平滑版本,这个\gls{positive_part_function}是指$x^+ = \max \{ 0, x\}$。
与\gls{positive_part_function}相对的是\firstgls{negative part function} $x^- = \max\{ 0, -x\}$。
为了获得类似\gls{negative part function}的一个平滑函数,我们可以使用$\zeta(-x)$。
就像$x$可以用它的正部和负部通过等式$x^+ - x^- = x$恢复一样,我们也可以用同样的方式对$\zeta(x)$和$\zeta(-x)$进行操作,就像\eqnref{eq:3.41}中那样。

\section{\glsentrytext{bayes_rule}}
\label{sec:bayes_rule}

我们经常会需要在已知$P(\RSy \mid \RSx)$时计算$P(\RSx \mid \RSy)$。
幸运的是,如果还知道$P(\RSx)$,我们可以用\firstgls{bayes_rule}来实现这一目的:
\begin{equation}
P(\RSx \mid \RSy) = \frac{P(\RSx) P(\RSy \mid \RSx)}{P(\RSy)}.
\end{equation}
注意到$P(\RSy)$出现在上面的公式中,它通常使用$P(\RSy) = \sum_x P(\RSy \mid x) P(x)$来计算,所以我们并不需要事先知道$P(\RSy)$的信息。

\gls{bayes_rule}可以从\gls{conditional_probability}的定义直接推导得出,但我们最好记住这个公式的名字,因为很多文献通过名字来引用这个公式。
这个公式是以Reverend Thomas Bayes来命名的,他是第一个发现这个公式特例的人。
这里介绍的一般形式由Pierre-Simon Laplace独立发现。

\section{连续型变量的技术细节}
\label{sec:technical_details_of_continuous_variables}

连续型\gls{RV}和\gls{PDF}的深入理解需要用到数学分支\firstgls{measure_theory}的相关内容来扩展概率论。
\gls{measure_theory}超出了本书的范畴,但我们可以简要勾勒一些\gls{measure_theory}用来解决的问题。

在\secref{sec:continuous_variables_and_probability_density_functions}中,我们已经看到连续型向量值\gls{RV} $\RVx$落在某个集合$\SetS$ 中的概率是通过$p(\Vx)$对集合$\SetS$积分得到的。
对于集合$\SetS$的一些选择可能会引起悖论。
例如,构造两个集合$\SetS_1$和$\SetS_2$使得$p(\Vx\in \SetS_1) + p(\Vx\in \SetS_2)>1$并且$\SetS_1 \cap \SetS_2 = \emptyset$是可能的。
这些集合通常是大量使用了实数的无限精度来构造的,例如通过构造分形形状(fractal-shaped)的集合或者是通过有理数相关集合的变换定义的集合。\footnote{Banach-Tarski定理给出了这类集合的一个有趣的例子。
译者注:我们这里把``the set of rational numbers''翻译成``有理数相关集合'',理解为``一些有理数组成的集合'',如果直接用后面的翻译读起来会比较拗口。}
\gls{measure_theory}的一个重要贡献就是提供了一些集合的特征使得我们在计算概率时不会遇到悖论。
在本书中,我们只对相对简单的集合进行积分,所以\gls{measure_theory}的这个方面不会成为一个相关考虑。

% -- 68 --

对于我们的目的,\gls{measure_theory}更多的是用来描述那些适用于$\SetR^n$上的大多数点,却不适用于一些边界情况的定理。
\gls{measure_theory}提供了一种严格的方式来描述那些非常微小的点集。
这种集合被称为``\firstgls{measure_zero}''的。
我们不会在本书中给出这个概念的正式定义。
然而,直观地理解这个概念是有用的,我们可以认为零测度集在我们的度量空间中不占有任何的体积。
例如,在$\SetR^2$空间中,一条直线的测度为零,而填充的多边形具有正的测度。
类似的,一个单独的点的测度为零。
可数多个零测度集的并仍然是零测度的(所以所有有理数构成的集合测度为零)。

另外一个有用的\gls{measure_theory}中的术语是``\firstgls{almost_everywhere}''。
某个性质如果是几乎处处都成立的,那么它在整个空间中除了一个测度为零的集合以外都是成立的。
因为这些例外只在空间中占有极其微小的量,它们在多数应用中都可以被放心地忽略。
概率论中的一些重要结果对于离散值成立但对于连续值只能是``几乎处处''成立。

连续型\gls{RV}的另一技术细节,涉及到处理那种相互之间有确定性函数关系的连续型变量。
假设我们有两个\gls{RV} $\RVx$和$\RVy$满足$\Vy = g(\Vx)$,其中$g$是可逆的、连续可微的函数。
可能有人会想$p_y(\Vy) = p_x(g^{-1}(\Vy))$。
但实际上这并不对。

举一个简单的例子,假设我们有两个标量值\gls{RV} $\RSx$和$\RSy$,并且满足$\RSy= \frac{\RSx}{2}$以及$\RSx \sim U(0, 1)$。
如果我们使用$p_y(y) = p_x(2y)$,那么$p_y$ 除了区间$[0, \frac{1}{2}]$以外都为0,并且在这个区间上的值为1。
这意味着
\begin{equation}
\int p_y(y)dy = \frac{1}{2},
\end{equation}
而这违背了概率密度的定义(积分为1)。
这个常见错误之所以错是因为它没有考虑到引入函数$g$后造成的空间变形。
回忆一下,$\Vx$落在无穷小的体积为$\delta \Vx$的区域内的概率为$p(\Vx)\delta\Vx$。
因为$g$可能会扩展或者压缩空间,在$\Vx$空间内的包围着$\Vx$ 的无穷小体积在$\Vy$空间中可能有不同的体积。

% -- 69 --

为了看出如何改正这个问题,我们回到标量值的情况。
我们需要保持下面这个性质:
\begin{equation}
|p_y(g(x))dy| = |p_x(x)dx|.
\end{equation}
求解上式,我们得到
\begin{equation}
p_y(y) = p_x(g^{-1}(y)) \left \vert \frac{\partial x}{\partial y} \right \vert
\end{equation}
或者等价地,
\begin{equation}
p_x(x) = p_y(g(x)) \left | \frac{\partial g(x)}{\partial x} \right |.
\end{equation}
在高维空间中,微分运算扩展为\firstgls{jacobian_matrix}的行列式——矩阵的每个元素为$J_{i, j} = \frac{\partial x_i}{\partial y_j}$。
因此,对于实值向量$\Vx$和$\Vy$,
\begin{equation}
\label{eqn:3.47}
p_x(\Vx) = p_y(g(\Vx)) \left | \det \left ( \frac{\partial g(\Vx)}{\partial \Vx} \right) \right |.
\end{equation}

\section{信息论}
\label{sec:information_theory}

信息论是应用数学的一个分支,主要研究的是对一个信号包含信息的多少进行量化。
它最初被发明是用来研究在一个含有噪声的信道上用离散的字母表来发送消息,例如通过无线电传输来通信。
在这种情况下,信息论告诉我们如何设计最优编码,以及计算从一个特定的\gls{PD}上采样得到、使用多种不同编码机制的消息的期望长度。
在机器学习中,我们也可以把信息论应用在连续型变量上,而信息论中一些消息长度的解释不怎么使用。
信息论是电子工程和计算机科学中许多领域的基础。
在本书中,我们主要使用信息论的一些关键思想来描述\gls{PD}或者量化\gls{PD}之间的相似性。
有关信息论的更多细节,参见~\cite{cover-book2006}或者~\cite{MacKay03}。

信息论的基本想法是一个不太可能的事件居然发生了,要比一个非常可能的事件发生,能提供更多的信息。
消息说:``今天早上太阳升起''信息量是如此之少以至于没有必要发送,但一条消息说:``今天早上有日食''信息量就很丰富。

% -- 70 --

我们想要通过这种基本想法来量化信息。
特别地,
\begin{itemize}
\item 非常可能发生的事件信息量要比较少,并且极端情况下,确保能够发生的事件应该没有信息量。

\item 较不可能发生的事件具有更高的信息量。

\item 独立事件应具有增量的信息。
例如,投掷的硬币两次正面朝上传递的信息量,应该是投掷一次硬币正面朝上的信息量的两倍。
\end{itemize}

为了满足上述三个性质,我们定义一个事件$\RSx = x$的\firstgls{self_information}为
\begin{equation}
I(x) = -\log P(x).
\end{equation}
在本书中,我们总是用$\log$来表示自然对数,其底数为$e$。
因此我们定义的$I(x)$单位是\firstgls{nats}。
一奈特是以$\frac{1}{e}$的概率观测到一个事件时获得的信息量。
其他的材料中使用底数为2的对数,单位是\firstgls{bits}或者\firstgls{shannons};通过比特度量的信息只是通过奈特度量信息的常数倍。

当$\RSx$是连续的,我们使用类似的关于信息的定义,但有些来源于离散形式的性质就丢失了。
例如,一个具有单位密度的事件信息量仍然为0,但是不能保证它一定发生。

\gls{self_information}只处理单个的输出。
我们可以用\firstgls{Shannon_entropy}来对整个\gls{PD}中的不确定性总量进行量化:
\begin{equation}
H(\RSx) = \SetE_{\RSx \sim P}[I(x)] = -\SetE_{\RSx \sim P}[\log P(x)],
\end{equation}
也记作$H(P)$。
换言之,一个分布的\gls{Shannon_entropy}是指遵循这个分布的事件所产生的期望信息总量。
它给出了对依据\gls{PD} $P$生成的符号进行编码所需的比特数在平均意义上的下界(当对数底数不是2时,单位将有所不同)。
那些接近确定性的分布(输出几乎可以确定)具有较低的熵;那些接近\gls{uniform_distribution}的\gls{PD}具有较高的熵。
\figref{fig:chap3_entropy_demo_color}给出了一个说明。
当$\RSx$是连续的,\gls{Shannon_entropy}被称为\firstgls{differential_entropy}。
% fig 3.5
\begin{figure}[!htb]
\ifOpenSource
\centerline{\includegraphics{figure.pdf}}
\else
\centerline{\includegraphics{Chapter3/figures/entropy_demo_color}}
\fi
\captionsetup{singlelinecheck=off}
\caption{二值随机变量的\gls{Shannon_entropy}。%此处不翻译为``二元随机变量'',是因为元表示自变量
该图说明了更接近确定性的分布是如何具有较低的\gls{Shannon_entropy},而更接近\gls{uniform_distribution}的分布是如何具有较高的\gls{Shannon_entropy}。
水平轴是$p$,表示二值随机变量等于1的概率。
熵由$(p-1)\log(1-p) - p\log p$给出。
当$p$接近0时,分布几乎是确定的,因为随机变量几乎总是0。
当$p$接近1时,分布也几乎是确定的,因为随机变量几乎总是1。
当$p = 0.5$时,熵是最大的,因为分布在两个结果(0和1)上是均匀的。}
\label{fig:chap3_entropy_demo_color}
\end{figure}


% -- 71 --

如果我们对于同一个\gls{RV} $\RSx$有两个单独的\gls{PD} $P(\RSx)$和$Q(\RSx)$,我们可以使用\firstgls{KL_divergence}来衡量这两个分布的差异:
\begin{equation}
D_{\text{KL}}(P||Q) = \SetE_{\RSx \sim P} \left [  \log \frac{P(x)}{Q(x)} \right ] = \SetE_{\RSx \sim P} [\log P(x) - \log Q(x)].
\end{equation}

在离散型变量的情况下,\gls{KL_divergence}衡量的是,当我们使用一种被设计成能够使得\gls{PD} $Q$ 产生的消息的长度最小的编码,发送包含由\gls{PD} $P$产生的符号的消息时,所需要的额外信息量(如果我们使用底数为2的对数时,信息量用比特衡量,但在机器学习中,我们通常用奈特和自然对数。)

\gls{KL_divergence}有很多有用的性质,最重要的是它是非负的。
\gls{KL_divergence}为0当且仅当$P$和$Q$在离散型变量的情况下是相同的分布,或者在连续型变量的情况下是``几乎处处''相同的。
因为~\gls{KL_divergence}是非负的并且衡量的是两个分布之间的差异,它经常被用作分布之间的某种距离。
然而,它并不是真的距离因为它不是对称的:对于某些$P$和$Q$,$D_\text{KL}(P||Q) \ne D_\text{KL}(Q||P)$。
这种非对称性意味着选择$D_\text{KL}(P||Q)$还是$D_\text{KL}(Q||P)$影响很大。
更多细节可以看\figref{fig:chap3_kl_direction_color}。
% fig 3.6
\begin{figure}[!htb]
\ifOpenSource
\centerline{\includegraphics{figure.pdf}}
\else
\centerline{\includegraphics{Chapter3/figures/kl_direction_color}}
\fi
\caption{\gls{KL_divergence}是不对称的。
假设我们有一个分布$p(x)$,并且希望用另一个分布$q(x)$来近似它。
我们可以选择最小化$D_\text{KL}(p||q)$或最小化$D_\text{KL}(q||p)$。
为了说明每种选择的效果,我们令$p$是两个\gls{gaussian_distribution}的混合,令$q$为单个\gls{gaussian_distribution}。
选择使用~\gls{KL_divergence}的哪个方向是取决于问题的。
一些应用需要这个近似分布$q$在真实分布$p$放置高概率的所有地方都放置高概率,而其他应用需要这个近似分布$q$在真实分布$p$放置低概率的所有地方都很少放置高概率。 
\gls{KL_divergence}方向的选择反映了对于每种应用,优先考虑哪一种选择。
\emph{(左)}最小化$D_\text{KL}(p||q)$的效果。
在这种情况下,我们选择一个$q$使得它在$p$具有高概率的地方具有高概率。
当$p$具有多个峰时,$q$选择将这些峰模糊到一起,以便将高概率质量放到所有峰上。 
\emph{(右)}最小化$D_\text{KL}(q||p)$的效果。
在这种情况下,我们选择一个$q$使得它在$p$具有低概率的地方具有低概率。
当$p$具有多个峰并且这些峰间隔很宽时,如该图所示,最小化~\gls{KL_divergence}会选择单个峰,以避免将概率质量放置在$p$的多个峰之间的低概率区域中。
这里,我们说明当$q$被选择成强调左边峰时的结果。
我们也可以通过选择右边峰来得到~\gls{KL_divergence}相同的值。
如果这些峰没有被足够强的低概率区域分离,那么~\gls{KL_divergence}的这个方向仍然可能选择模糊这些峰。}
\label{fig:chap3_kl_direction_color}
\end{figure}


% -- 72 --

一个和~\gls{KL_divergence}密切联系的量是\firstgls{cross_entropy} $H(P, Q) = H(P) + D_\text{KL}(P||Q)$,它和~\gls{KL_divergence}很像但是缺少左边一项:
\begin{equation}
H(P, Q) = -\SetE_{\RSx\sim P} \log Q(x).
\end{equation}
针对$Q$最小化\gls{cross_entropy}等价于最小化~\gls{KL_divergence},因为$Q$并不参与被省略的那一项。

当我们计算这些量时,经常会遇到$0\log 0$这个表达式。
按照惯例,在信息论中,我们将这个表达式处理为$\lim_{x \to 0} x\log x = 0$。

% -- 73 --

\section{\glsentrytext{structured_probabilistic_models}}
\label{sec:structured_probabilistic_models_chap3}

机器学习的算法经常会涉及到在非常多的\gls{RV}上的\gls{PD}。
通常,这些\gls{PD}涉及到的直接相互作用都是介于非常少的变量之间的。
使用单个函数来描述整个\gls{joint_probability_distribution}是非常低效的(无论是计算上还是统计上)。

我们可以把\gls{PD}分解成许多因子的乘积形式,而不是使用单一的函数来表示\gls{PD}。
例如,假设我们有三个\gls{RV} $\RSa, \RSb$和$\RSc$,并且$\RSa$影响$\RSb$的取值,$\RSb$影响$\RSc$的取值,但是$\RSa$和$\RSc$在给定$\RSb$时是条件独立的。
我们可以把全部三个变量的\gls{PD}重新表示为两个变量的\gls{PD}的连乘形式:
\begin{equation}
p(\RSa, \RSb, \RSc) = p(\RSa)p(\RSb\mid \RSa)p(\RSc\mid\RSb).
\end{equation}

这种\gls{factorization}可以极大地减少用来描述一个分布的参数数量。
每个因子使用的参数数目是它的变量数目的指数倍。
这意味着,如果我们能够找到一种使每个因子分布具有更少变量的分解方法,我们就能极大地降低表示联合分布的成本。

我们可以用图来描述这种\gls{factorization}。
这里我们使用的是图论中的``图''的概念:由一些可以通过边互相连接的顶点的集合构成。
当我们用图来表示这种\gls{PD}的\gls{factorization},我们把它称为\firstgls{structured_probabilistic_model}或者\firstgls{graphical_model}。

有两种主要的\gls{structured_probabilistic_models}:有向的和无向的。
两种图模型都使用图$\CalG$,其中图的每个节点对应着一个\gls{RV},连接两个\gls{RV}的边意味着\gls{PD}可以表示成这两个\gls{RV}之间的直接作用。

\firstgls{directed}模型使用带有有向边的图,它们用\gls{conditional_probability}分布来表示\gls{factorization},就像上面的例子。
特别地,有向模型对于分布中的每一个\gls{RV} $\RSx_i$都包含着一个影响因子,这个组成$\RSx_i$\gls{conditional_probability}的影响因子被称为$\RSx_i$的父节点,记为$Pa_\CalG(\RSx_i)$:
\begin{equation}
p(\RVx) = \prod_i p(\RSx_i \mid Pa_\CalG(\RSx_i)).
\end{equation}
\figref{fig:chap3_directed}给出了一个有向图的例子以及它表示的\gls{PD}的\gls{factorization}。
% fig 3.7
\begin{figure}[!htb]
\ifOpenSource
\centerline{\includegraphics{figure.pdf}}
\else
\centerline{\includegraphics{Chapter3/figures/directed}}
\fi
\captionsetup{singlelinecheck=off}
\caption[.]{关于随机变量$\RSa, \RSb, \RSc, \RSd$和$\RSe$的有向图模型。
这幅图对应的概率分布可以分解为
\begin{equation}
p(\RSa, \RSb, \RSc, \RSd, \RSe) = p(\RSa) p(\RSb \mid \RSa) p(\RSc \mid \RSa, \RSb) p(\RSd \mid \RSb) p(\RSe \mid \RSc).
\end{equation}
该图模型使我们能够快速看出此分布的一些性质。
例如,$\RSa$和$\RSc$直接相互影响,但$\RSa$和$\RSe$只有通过$\RSc$间接相互影响。}
\label{fig:chap3_directed}
\end{figure}


% -- 74 --

\firstgls{undirected}模型使用带有无向边的图,它们将\gls{factorization}表示成一组函数;不像有向模型那样,这些函数通常不是任何类型的\gls{PD}。
$\CalG$中任何满足两两之间有边连接的顶点的集合被称为团。
无向模型中的每个团$\CalC^{(i)}$都伴随着一个因子$\phi^{(i)}(\CalC^{(i)})$。
 这些因子仅仅是函数,并不是\gls{PD}。
 每个因子的输出都必须是非负的,但是并没有像\gls{PD}中那样要求因子的和或者积分为1。

\gls{RV}的联合概率与所有这些因子的乘积\firstgls{proportional}——意味着因子的值越大则可能性越大。
当然,不能保证这种乘积的求和为1。
所以我们需要除以一个归一化常数$Z$来得到归一化的\gls{PD},归一化常数$Z$被定义为$\phi$函数乘积的所有状态的求和或积分。
\gls{PD}为:
\begin{equation}
p(\RVx) = \frac{1}{Z} \prod_i \phi^{(i)} \left (\CalC^{(i)} \right).
\end{equation}
\figref{fig:chap3_undirected}给出了一个无向图的例子以及它表示的\gls{PD}的\gls{factorization}。
% fig 3.8
\begin{figure}[!htb]
\ifOpenSource
\centerline{\includegraphics{figure.pdf}}
\else
\centerline{\includegraphics{Chapter3/figures/undirected}}
\fi
\captionsetup{singlelinecheck=off}
\caption[.]{关于随机变量$\RSa, \RSb, \RSc, \RSd$和$\RSe$的无向图模型。
这幅图对应的概率分布可以分解为
\begin{equation}
p(\RSa, \RSb, \RSc, \RSd, \RSe) = \frac{1}{Z} \phi^{(1)} (\RSa, \RSb, \RSc) \phi^{(2)}(\RSb, \RSd) \phi^{(3)} (\RSc, \RSe).
\end{equation}
该图模型使我们能够快速看出此分布的一些性质。
例如,$\RSa$和$\RSc$直接相互影响,但$\RSa$和$\RSe$只有通过$\RSc$间接相互影响。}
\label{fig:chap3_undirected}
\end{figure}


% -- 75 --

请记住,这些图模型表示的\gls{factorization}仅仅是描述\gls{PD}的一种语言。
它们不是互相排斥的\gls{PD}族。
有向或者无向不是\gls{PD}的特性;它是\gls{PD}的一种特殊\firstgls{description}所具有的特性,而任何\gls{PD}都可以用这两种方式进行描述。

在本书第\ref{part:applied_math_and_machine_learning_basics}部分和第\ref{part:deep_networks_modern_practices}部分中, 我们仅仅将\gls{structured_probabilistic_models}视作一门语言,来描述不同的机器学习算法选择表示的直接的概率关系。
在讨论研究课题之前,读者不需要更深入地理解\gls{structured_probabilistic_models}。
在第\ref{part:deep_learning_research}部分的研究课题中,我们将更为详尽地探讨\gls{structured_probabilistic_models}。

本章复习了概率论中与深度学习最为相关的一些基本概念。
我们还剩下一些基本的数学工具需要讨论:数值方法。

% -- 76 --

% !Mode:: "TeX:UTF-8"
% Translator: Shenjian Zhao
\chapter{数值计算}
\label{chap:numerical_computation}

\gls{ML}算法通常需要大量的数值计算。
这通常是指通过迭代过程更新解的估计值来解决数学问题的算法,而不是通过解析过程推导出公式来提供正确解的方法。
常见的操作包括优化(找到最小化或最大化函数值的参数)和线性方程组的求解。
对数字计算机来说实数无法在有限内存下精确表示,因此仅仅是计算涉及实数的函数也是困难的。

\section{\glsentrytext{overflow}和\glsentrytext{underflow}}
\label{sec:overflow_and_underflow}
连续数学在数字计算机上的根本困难是,我们需要通过有限数量的位模式来表示无限多的实数。
这意味着我们在计算机中表示实数时,几乎总会引入一些近似误差。
在许多情况下,这仅仅是舍入误差。
舍入误差会导致一些问题,特别是当许多操作复合时,即使是理论上可行的算法,如果在设计时没有考虑最小化舍入误差的累积,在实践时也可能会导致算法失效。

一种极具毁灭性的舍入误差是\firstgls{underflow}。
当接近零的数被四舍五入为零时发生\gls{underflow}。
许多函数在其参数为零而不是一个很小的正数时才会表现出质的不同。
例如,我们通常要避免被零除(一些软件环境将在这种情况下抛出异常,有些会返回一个非数字(not-a-number, NaN)的占位符)或避免取零的对数(这通常被视为$-\infty$,进一步的算术运算会使其变成非数字)。

% -- 77 --

另一个极具破坏力的数值错误形式是\firstgls{overflow}。
当大量级的数被近似为$\infty$或$-\infty$时发生\gls{overflow}。
进一步的运算通常会导致这些无限值变为非数字。

必须对\gls{overflow}和\gls{underflow}进行数值稳定的一个例子是\firstgls{softmax}。
\gls{softmax}经常用于预测与~\gls{multinoulli}相关联的概率,定义为
\begin{align}
 \text{softmax}(\Vx)_i = \frac{\exp(\Sx_i)}{\sum_{j=1}^n \exp(\Sx_j)} .
\end{align}
考虑一下当所有$\Sx_i$都等于某个常数$\Sc$时会发生什么。
从理论分析上说,我们可以发现所有的输出都应该为$\frac{1}{n}$。
从数值计算上说,当$\Sc$量级很大时,这可能不会发生。
如果$\Sc$是很小的负数,$\exp(c)$就会\gls{underflow}。
这意味着~\gls{softmax}的分母会变成0,所以最后的结果是未定义的。
当$\Sc$是非常大的正数时,$\exp(c)$的\gls{overflow}再次导致整个表达式未定义。
这两个困难能通过计算$\text{softmax}(\Vz)$同时解决,其中$\Vz = \Vx - \max_i \Sx_i$。
简单的代数计算表明,$\text{softmax}$解析上的函数值不会因为从输入向量减去或加上标量而改变。
减去$\max_i x_i$导致$\exp$的最大参数为$0$,这排除了\gls{overflow}的可能性。
同样地,分母中至少有一个值为1的项,这就排除了因分母\gls{underflow}而导致被零除的可能性。

还有一个小问题。
分子中的\gls{underflow}仍可以导致整体表达式被计算为零。
这意味着,如果我们在计算$\log ~\text{softmax}(\Vx)$时,先计算$\text{softmax}$再把结果传给$\log$函数,会错误地得到$-\infty$。
相反,我们必须实现一个单独的函数,并以数值稳定的方式计算$\log \text{softmax}$。
我们可以使用相同的技巧来稳定$\log \text{softmax}$函数。

在大多数情况下,我们没有明确地对本书描述的各种算法所涉及的数值考虑进行详细说明。
底层库的开发者在实现\gls{DL}算法时应该牢记数值问题。
本书的大多数读者可以简单地依赖保证数值稳定的底层库。
在某些情况下,我们有可能在实现一个新的算法时自动保持数值稳定。
Theano~\citep{bergstra+al:2010-scipy,Bastien-2012}就是这样软件包的一个例子,它能自动检测并稳定\gls{DL}中许多常见的数值不稳定的表达式。

% -- 78 --

\section{\glsentrytext{poor_conditioning}}
\label{sec:poor_conditioning}

条件数表征函数相对于输入的微小变化而变化的快慢程度。
输入被轻微扰动而迅速改变的函数对于科学计算来说可能是有问题的,因为输入中的舍入误差可能导致输出的巨大变化。

考虑函数$f(\Vx) = \MA^{-1} \Vx$。
当$\MA \in \SetR^{n \times n}$ 具有特征值分解时,其条件数为
\begin{align}
 \underset{i,j}{\max}~ \Bigg| \frac{\lambda_i}{ \lambda_j} \Bigg|.
\end{align}
这是最大和最小特征值的模之比\footnote{译者注:与通常的条件数定义有所不同。}。
当该数很大时,矩阵求逆对输入的误差特别敏感。

这种敏感性是矩阵本身的固有特性,而不是矩阵求逆期间舍入误差的结果。
即使我们乘以完全正确的矩阵逆,\gls{poor_conditioning}的矩阵也会放大预先存在的误差。
在实践中,该错误将与求逆过程本身的数值误差进一步复合。



\section{基于梯度的优化方法}
\label{sec:gradient_based_optimization}

大多数\gls{DL}算法都涉及某种形式的优化。
优化指的是改变$\Vx$以最小化或最大化某个函数$f(\Vx)$的任务。
我们通常以最小化$f(\Vx)$指代大多数最优化问题。
最大化可经由最小化算法最小化$-f(\Vx)$来实现。

我们把要最小化或最大化的函数称为\firstgls{objective_function}或\firstgls{criterion}。
当我们对其进行最小化时,我们也把它称为\firstgls{cost_function}、\firstgls{loss_function}或\firstgls{error_function}。
虽然有些机器学习著作赋予这些名称特殊的意义,但在这本书中我们交替使用这些术语。

我们通常使用一个上标$*$表示最小化或最大化函数的$\Vx$值。
如我们记$\Vx^*=\argmin f(\Vx)$。

我们假设读者已经熟悉微积分,这里简要回顾微积分概念如何与优化联系。

% -- 79 --

假设我们有一个函数$\Sy = f(\Sx)$, 其中$\Sx$和$\Sy$是实数。
这个函数的\firstgls{derivative}记为$f^\prime(x)$或$\frac{dy}{dx}$。
导数$f^\prime(\Sx)$代表$f(\Sx)$在点$x$处的斜率。
换句话说,它表明如何缩放输入的小变化才能在输出获得相应的变化:
$f(\Sx+\epsilon) \approx f(\Sx) + \epsilon f^\prime(\Sx) $。

因此\gls{derivative}对于最小化一个函数很有用,因为它告诉我们如何更改$x$来略微地改善$y$。
例如,我们知道对于足够小的$\epsilon$来说,$f(\Sx-\epsilon \text{sign}(f^\prime(\Sx)) )$是比$f(\Sx)$小的。
因此我们可以将$\Sx$往\gls{derivative}的反方向移动一小步来减小$f(\Sx)$。
这种技术被称为\firstgls{GD}\citep{cauchy1847}。
\figref{fig:chap4_gradient_descent_color}展示了一个例子。
\begin{figure}[!htb]
\ifOpenSource
\centerline{\includegraphics{figure.pdf}}
\else
\centerline{\includegraphics{Chapter4/figures/gradient_descent_color}}
\fi
\caption{\gls{GD}。 
\gls{GD}算法如何使用函数导数的示意图,即沿着函数的下坡方向(导数反方向)直到最小。}
\label{fig:chap4_gradient_descent_color}
\end{figure}

% -- 80 --

当$f^\prime(\Sx)=0$,\gls{derivative}无法提供往哪个方向移动的信息。
$ f^\prime(\Sx)=0 $的点称为\firstgls{critical_points}或\firstgls{stationary_point}。
一个\firstgls{local_minimum}意味着这个点的$f(\Sx)$小于所有邻近点,因此不可能通过移动无穷小的步长来减小$f(\Sx)$。
一个\firstgls{local_maximum}意味着这个点的$f(\Sx)$大于所有邻近点,因此不可能通过移动无穷小的步长来增大$f(\Sx)$。
有些\gls{critical_points}既不是最小点也不是最大点。这些点被称为\firstgls{saddle_points}。
见\figref{fig:chap4_critical_color}给出的各种临界点的例子。
\begin{figure}[!htb]
\ifOpenSource
\centerline{\includegraphics{figure.pdf}}
\else
\centerline{\includegraphics{Chapter4/figures/critical_color}}
\fi
\caption{\gls{critical_points}的类型。 
一维情况下,三种\gls{critical_points}的示例。
\gls{critical_points}是斜率为零的点。
这样的点可以是\firstgls{local_minimum},其值低于相邻点; \firstgls{local_maximum},其值高于相邻点; 或\gls{saddle_points},同时存在更高和更低的相邻点。
}
\label{fig:chap4_critical_color}
\end{figure}

使$f(x)$取得绝对的最小值(相对所有其他值)的点是\firstgls{global_minimum}。
函数可能只有一个\gls{global_minimum}或存在多个\gls{global_minimum},
还可能存在不是全局最优的\gls{local_minimum}。
在\gls{DL}的背景下,我们要优化的函数可能含有许多不是最优的\gls{local_minimum},或者还有很多处于非常平坦的区域内的\gls{saddle_points}。
尤其是当输入是多维的时候,所有这些都将使优化变得困难。
因此,我们通常寻找使$f$非常小的点,但这在任何形式意义下并不一定是最小。
见\figref{fig:chap4_approx_opt_color}的例子。
\begin{figure}[!htb]
\ifOpenSource
\centerline{\includegraphics{figure.pdf}}
\else
\centerline{\includegraphics{Chapter4/figures/approx_opt_color}}
\fi
\caption{近似最小化。 
当存在多个\gls{local_minimum}或平坦区域时,优化算法可能无法找到\gls{global_minimum}。
在深度学习的背景下,即使找到的解不是真正最小的,但只要它们对应于\gls{cost_function}显著低的值,我们通常就能接受这样的解。
}
\label{fig:chap4_approx_opt_color}
\end{figure}

我们经常最小化具有多维输入的函数:$f: \SetR^n \rightarrow \SetR $。 
为了使``最小化''的概念有意义,输出必须是一维的(标量)。

% -- 81 --

针对具有多维输入的函数,我们需要用到\firstgls{partial_derivatives}的概念。
\gls{partial_derivatives} $\frac{\partial}{\partial \Sx_i}f(\Vx)$衡量点$\Vx$处只有$x_i$增加时$f(\Vx)$如何变化。
\firstgls{gradient}是相对一个向量求导的\gls{derivative}:$f$的导数是包含所有\gls{partial_derivatives}的向量,记为$\nabla_{\Vx} f(\Vx)$。
\gls{gradient}的第$i$个元素是$f$关于$x_i$的\gls{partial_derivatives}。
在多维情况下,\gls{critical_points}是\gls{gradient}中所有元素都为零的点。

在$\Vu$(单位向量)方向的\firstgls{directional_derivative}是函数$f$在$\Vu$方向的斜率。
换句话说,\gls{directional_derivative}是函数$f(\Vx + \alpha \Vu)$关于$\alpha$的\gls{derivative}(在$\alpha = 0$时取得)。
使用\gls{chain_rule},我们可以看到当$\alpha=0$时,$\frac{\partial}{\partial \alpha} f(\Vx + \alpha \Vu) = \Vu^\Tsp \nabla_{\Vx} f(\Vx)$。

为了最小化$f$,我们希望找到使$f$下降得最快的方向。
计算\gls{directional_derivative}:
\begin{align}
 \underset{\Vu, \Vu^\Tsp\Vu = 1}{\min} \Vu^\Tsp & \nabla_{\Vx} f(\Vx) \\
 = \underset{\Vu, \Vu^\Tsp\Vu = 1}{\min} \| \Vu \|_2 \| &\nabla_{\Vx}f(\Vx) \|_2 \cos \theta
\end{align}
其中$\theta$是$\Vu$与\gls{gradient}的夹角。
将$ \| \Vu \|_2 = 1$代入,并忽略与$\Vu$无关的项,就能简化得到$ \underset{\Vu}{\min} \cos \theta $。 
这在$\Vu$与\gls{gradient}方向相反时取得最小。
换句话说,\gls{gradient}向量指向上坡,负\gls{gradient}向量指向下坡。
我们在负\gls{gradient}方向上移动可以减小$f$。
这被称为\textbf{最速下降法}(method of steepest descent)或\firstgls{GD}。

最速下降建议新的点为
\begin{align}
  \Vx' = \Vx - \epsilon \nabla_{\Vx} f(\Vx)
\end{align}
其中$\epsilon$为\firstgls{learning_rate},是一个确定步长大小的正标量。
我们可以通过几种不同的方式选择$\epsilon$。
普遍的方式是选择一个小常数。
有时我们通过计算,选择使\gls{directional_derivative}消失的步长。
还有一种方法是根据几个$\epsilon$计算$f(\Vx - \epsilon \nabla_{\Vx} f(\Vx))$, 并选择其中能产生最小\gls{objective_function}值的$\epsilon$。
这种策略被称为\gls{line_search}。

最速下降在\gls{gradient}的每一个元素为零时收敛(或在实践中,很接近零时)。
在某些情况下,我们也许能够避免运行该迭代算法,并通过解方程$\nabla_{\Vx} f(\Vx)= 0$直接跳到\gls{critical_points}。

% -- 82 --

虽然\gls{GD}被限制在连续空间中的优化问题,但不断向更好的情况移动一小步(即近似最佳的小移动)的一般概念可以推广到离散空间。
递增带有离散参数的\gls{objective_function}被称为\firstgls{hill_climbing}算法\citep{Russel+Norvig-book2003}。

\subsection{\glsentrytext{gradient}之上:\glsentrytext{jacobian}和\glsentrytext{hessian}矩阵}
\label{sec:beyond_the_gradient_jacobian_and_hessian_matrices}
有时我们需要计算输入和输出都为向量的函数的所有\gls{partial_derivatives}。
包含所有这样的偏导数的矩阵被称为\firstgls{jacobian}矩阵。
具体来说,如果我们有一个函数:$\Vf: \SetR^m \rightarrow \SetR^n$,$\Vf$的~\gls{jacobian}~矩阵$\MJ \in \SetR^{n \times m}$定义为$J_{i,j} = \frac{\partial}{\partial \Sx_j} f(\Vx)_i$。

有时,我们也对\gls{derivative}的\gls{derivative}感兴趣,即\firstgls{second_derivative}。
例如,有一个函数$f: \SetR^m \rightarrow \SetR$,$f$的一阶\gls{derivative}(关于$\Sx_j$)关于$x_i$的\gls{derivative}记为$\frac{\partial^2}{\partial \Sx_i \partial \Sx_j} f$。
在一维情况下,我们可以将$\frac{\partial^2}{\partial \Sx^2} f$为$f''(\Sx)$。
\gls{second_derivative}告诉我们,一阶\gls{derivative}将如何随着输入的变化而改变。
它表示只基于\gls{gradient}信息的\gls{GD}步骤是否会产生如我们预期的那样大的改善,因此它是重要的。
我们可以认为,\gls{second_derivative}是对\gls{curvature}的衡量。
假设我们有一个二次函数(虽然很多实践中的函数都不是二次的,但至少在局部可以很好地用二次近似)。
如果这样的函数具有零\gls{second_derivative},那就没有\gls{curvature}。
也就是一条完全平坦的线,仅用\gls{gradient}就可以预测它的值。
我们使用沿负\gls{gradient}方向大小为$\epsilon$的下降步,当该\gls{gradient}是$1$时,\gls{cost_function}将下降$\epsilon$。
如果\gls{second_derivative}是负的,函数曲线向下凹陷(向上凸出),因此\gls{cost_function}将下降的比$\epsilon$多。
如果\gls{second_derivative}是正的,函数曲线是向上凹陷(向下凸出),
因此\gls{cost_function}将下降的比$\epsilon$少。
从\figref{fig:chap4_curvature_color}可以看出不同形式的\gls{curvature}如何影响基于\gls{gradient}的预测值与真实的\gls{cost_function}值的关系。
\begin{figure}[!htb]
\ifOpenSource
\centerline{\includegraphics{figure.pdf}}
\else
\centerline{\includegraphics{Chapter4/figures/curvature_color}}
\fi
\caption{\gls{second_derivative}确定函数的\gls{curvature}。
这里我们展示具有各种\gls{curvature}的二次函数。
虚线表示我们仅根据梯度信息进行\gls{GD}后预期的\gls{cost_function}值。
对于负\gls{curvature},\gls{cost_function}实际上比梯度预测下降得更快。
没有\gls{curvature}时,梯度正确预测下降值。
对于正\gls{curvature},函数比预期下降得更慢,并且最终会开始增加,因此太大的步骤实际上可能会无意地增加函数值。
}
\label{fig:chap4_curvature_color}
\end{figure}

当我们的函数具有多维输入时,\gls{second_derivative}也有很多。
我们可以将这些导数合并成一个矩阵,称为\firstgls{hessian}矩阵。
\gls{hessian}~矩阵$\MH(f)(\Vx)$定义为
\begin{align}
 \MH(f)(\Vx)_{i,j} = \frac{\partial^2}{\partial \Sx_i \partial  \Sx_j} f(\Vx).
\end{align}
\gls{hessian}~等价于\gls{gradient}的~\gls{jacobian}~矩阵。

% -- 83 --

微分算子在任何二阶偏导连续的点处可交换,也就是它们的顺序可以互换:
\begin{align}
 \frac{\partial^2}{\partial \Sx_i \partial \Sx_j} f(\Vx) = \frac{\partial^2}{\partial \Sx_j\partial \Sx_i} f(\Vx) .
\end{align}
这意味着$H_{i,j} = H_{j,i}$, 因此~\gls{hessian}~矩阵在这些点上是对称的。
在\gls{DL}背景下,我们遇到的大多数函数的~\gls{hessian}~几乎处处都是对称的。
因为~\gls{hessian}~矩阵是实对称的,我们可以将其分解成一组实特征值和一组特征向量的正交基。
在特定方向$\Vd$上的\gls{second_derivative}可以写成$\Vd^\Tsp \MH \Vd$。
当$\Vd$是$\MH$的一个特征向量时,这个方向的\gls{second_derivative}就是对应的特征值。
对于其他的方向$\Vd$,方向\gls{second_derivative}是所有特征值的加权平均,权重在0和1之间,且与$\Vd$夹角越小的特征向量的权重越大。
最大特征值确定最大\gls{second_derivative},最小特征值确定最小\gls{second_derivative}。

% -- 84 --

我们可以通过(方向)\gls{second_derivative}预期一个\gls{GD}步骤能表现得多好。
我们在当前点$\Vx^{(0)}$处作函数$f(\Vx)$的近似二阶\gls{taylor}级数:
\begin{align}
 f(\Vx) \approx f(\Vx^{(0)}) + (\Vx - \Vx^{(0)})^\Tsp \Vg + 
 \frac{1}{2}  (\Vx - \Vx^{(0)})^\Tsp \MH  (\Vx - \Vx^{(0)}),
\end{align}
其中$\Vg$是梯度,$\MH$是$ \Vx^{(0)}$点的~\gls{hessian}。
如果我们使用\gls{learning_rate} $\epsilon$,那么新的点$\Vx$将会是$\Vx^{(0)}-\epsilon \Vg$。
代入上述的近似,可得
\begin{align}
\label{eq:4.9}
 f(\Vx^{(0)} - \epsilon \Vg ) \approx f(\Vx^{(0)})  - \epsilon \Vg^\Tsp \Vg + \frac{1}{2} \epsilon^2 \Vg^\Tsp \MH  \Vg.
\end{align}
其中有3项:函数的原始值、函数斜率导致的预期改善、函数\gls{curvature}导致的校正。
当最后一项太大时,\gls{GD}实际上是可能向上移动的。
当$\Vg^\Tsp \MH  \Vg$为零或负时,近似的\gls{taylor}级数表明增加$\epsilon$将永远使$f$下降。
在实践中,\gls{taylor}级数不会在$\epsilon$大的时候也保持准确,因此在这种情况下我们必须采取更启发式的选择。
当$\Vg^\Tsp \MH  \Vg$为正时,通过计算可得,使近似\gls{taylor}级数下降最多的最优步长为
\begin{align}
 \epsilon^* = \frac{ \Vg^\Tsp \Vg}{ \Vg^\Tsp \MH  \Vg} .
\end{align}
最坏的情况下,$\Vg$与$\MH$最大特征值$\lambda_{\max}$对应的特征向量对齐,则最优步长是$\frac{1}{\lambda_{\max}}$。
我们要最小化的函数能用二次函数很好地近似的情况下,\gls{hessian}~的特征值决定了\gls{learning_rate}的量级。

\gls{second_derivative}还可以被用于确定一个\gls{critical_points}是否是\gls{local_maximum}、\gls{local_minimum}或\gls{saddle_points}。
回想一下,在\gls{critical_points}处$f'(x) = 0$。
而$f''(x) > 0$意味着$f'(x)$会随着我们移向右边而增加,移向左边而减小,也就是 $f'(x - \epsilon) < 0$ 和 $f'(x+\epsilon)>0$对足够小的$\epsilon$成立。 换句话说,当我们移向右边,斜率开始指向右边的上坡,当我们移向左边,斜率开始指向左边的上坡。
因此我们得出结论,当$f'(x) = 0$且$f''(x) > 0$时,$\Vx$是一个\gls{local_minimum}。
同样,当$f'(x) = 0$且$f''(x) < 0$时,$\Vx$是一个\gls{local_maximum}。
这就是所谓的\firstgls{second_derivative_test}。
不幸的是,当$f''(x) = 0$时测试是不确定的。
在这种情况下,$\Vx$可以是一个\gls{saddle_points}或平坦区域的一部分。

% -- 85 --

在多维情况下,我们需要检测函数的所有\gls{second_derivative}。
利用~\gls{hessian}~的特征值分解,我们可以将\gls{second_derivative_test}扩展到多维情况。
在\gls{critical_points}处($\nabla_{\Vx} f(\Vx) = 0$),我们通过检测~\gls{hessian}~的特征值来判断该\gls{critical_points}是一个\gls{local_maximum}、\gls{local_minimum}还是\gls{saddle_points}。
当~\gls{hessian}~是正定的(所有特征值都是正的),则该\gls{critical_points}是\gls{local_minimum}。
因为方向\gls{second_derivative}在任意方向都是正的,参考单变量的\gls{second_derivative_test}就能得出此结论。
同样的,当~\gls{hessian}~是负定的(所有特征值都是负的),这个点就是\gls{local_maximum}。
在多维情况下,实际上我们可以找到确定该点是否为\gls{saddle_points}的积极迹象(某些情况下)。
如果~\gls{hessian}~的特征值中至少一个是正的且至少一个是负的,那么$\Vx$是$f$某个横截面的\gls{local_maximum},却是另一个横截面的\gls{local_minimum}。
见\figref{fig:chap4_saddle_3d_color}中的例子。
最后,多维\gls{second_derivative_test}可能像单变量版本那样是不确定的。
当所有非零特征值是同号的且至少有一个特征值是$0$时,这个检测就是不确定的。
这是因为单变量的\gls{second_derivative_test}在零特征值对应的横截面上是不确定的。
\begin{figure}[!htb]
\ifOpenSource
\centerline{\includegraphics{figure.pdf}}
\else
\centerline{\includegraphics{Chapter4/figures/saddle_3d_color}}
\fi
\caption{既有正\gls{curvature}又有负\gls{curvature}的\gls{saddle_points}。
示例中的函数是$f(\Vx) = x_1^2 - x_2^2$。
函数沿$x_1$轴向上弯曲。
$x_1$轴是~\gls{hessian}~的一个特征向量,并且具有正特征值。
函数沿$x_2$轴向下弯曲。
该方向对应于~\gls{hessian}~负特征值的特征向量。
名称``\gls{saddle_points}''源自该处函数的鞍状形状。
这是具有\gls{saddle_points}函数的典型示例。
维度多于一个时,\gls{saddle_points}不一定要具有0特征值:仅需要同时具有正特征值和负特征值。
我们可以想象这样一个鞍点(具有正负特征值)在一个横截面内是\gls{local_maximum},而在另一个横截面内是\gls{local_minimum}。
}
\label{fig:chap4_saddle_3d_color}
\end{figure}

多维情况下,单个点处每个方向上的\gls{second_derivative}是不同。
\gls{hessian}~的条件数衡量这些\gls{second_derivative}的变化范围。
当~\gls{hessian}~的条件数很差时,\gls{GD}法也会表现得很差。
这是因为一个方向上的\gls{derivative}增加得很快,而在另一个方向上增加得很慢。
\gls{GD}不知道导数的这种变化,所以它不知道应该优先探索导数长期为负的方向。
\gls{poor_conditioning}也导致很难选择合适的步长。
步长必须足够小,以免冲过最小而向具有较强正\gls{curvature}的方向上升。
这通常意味着步长太小,以致于在其他较小\gls{curvature}的方向上进展不明显。
见\figref{fig:chap4_poor_conditioning_color}的例子。
\begin{figure}[!htb]
\ifOpenSource
\centerline{\includegraphics{figure.pdf}}
\else
\centerline{\includegraphics{Chapter4/figures/poor_conditioning_color}}
\fi
\caption{\gls{GD}无法利用包含在~\gls{hessian}~矩阵中的\gls{curvature}信息。
这里我们使用\gls{GD}来最小化~\gls{hessian}~矩阵条件数为5的二次函数$f(\Vx)$。
这意味着最大\gls{curvature}方向具有比最小\gls{curvature}方向多五倍的\gls{curvature}。
在这种情况下,最大\gls{curvature}在$[1,1]^\top$方向上,最小\gls{curvature}在$[1,-1]^\top$方向上。
红线表示\gls{GD}的路径。
这个非常细长的二次函数类似一个长峡谷。
\gls{GD}把时间浪费于在峡谷壁反复下降,因为它们是最陡峭的特征。
由于步长有点大,有超过函数底部的趋势,因此需要在下一次迭代时在对面的峡谷壁下降。
与指向该方向的特征向量对应的~\gls{hessian}~的大的正特征值表示该方向上的导数快速增加,因此基于~\gls{hessian}~的优化算法可以预测,在此情况下最陡峭方向实际上不是有前途的搜索方向。
}
\label{fig:chap4_poor_conditioning_color}
\end{figure}

% -- 86 --

我们可以使用~\gls{hessian}~矩阵的信息来指导搜索,以解决这个问题。
其中最简单的方法是\firstgls{newton_method}。
\gls{newton_method}基于一个二阶\gls{taylor}展开来近似$\Vx^{(0)}$附近的$f(\Vx)$:
\begin{align}
 f(\Vx) \approx f(\Vx^{(0)}) + (\Vx - \Vx^{(0)})^\Tsp \nabla_{\Vx} f(\Vx^{(0)}) + 
 \frac{1}{2}  (\Vx - \Vx^{(0)})^\Tsp \MH(f)(\Vx^{(0)})  (\Vx - \Vx^{(0)}).
\end{align}
接着通过计算,我们可以得到这个函数的\gls{critical_points}:
\begin{align} \label{eq:newtonstep}
 \Vx^* =  \Vx^{(0)} -  \MH(f)(\Vx^{(0)})^{-1}  \nabla_{\Vx} f(\Vx^{(0)}) .
\end{align}
当$f$是一个正定二次函数时,\gls{newton_method}只要应用一次\eqnref{eq:newtonstep}就能直接跳到函数的最小点。
如果$f$不是一个真正二次但能在局部近似为正定二次,\gls{newton_method}则需要多次迭代应用\eqnref{eq:newtonstep}。
迭代地更新近似函数和跳到近似函数的最小点可以比\gls{GD}更快地到达临界点。
这在接近\gls{local_minimum}时是一个特别有用的性质,但是在\gls{saddle_points}附近是有害的。
如\eqnref{sec:plateaus_saddle_points_and_other_flat_regions}所讨论的,当附近的\gls{critical_points}是最小点(\gls{hessian}~的所有特征值都是正的)时\gls{newton_method}才适用,而\gls{GD}不会被吸引到\gls{saddle_points}(除非\gls{gradient}指向\gls{saddle_points})。

仅使用\gls{gradient}信息的优化算法被称为\,\textbf{一阶优化算法}(first-order optimization algorithms),如\gls{GD}。
使用~\gls{hessian}~矩阵的优化算法被称为\,\textbf{二阶最优化算法}(second-order optimization algorithms)\citep{NumOptBook},如\gls{newton_method}。

在本书大多数上下文中使用的优化算法适用于各种各样的函数,但几乎都没有保证。
因为在\gls{DL}中使用的函数族是相当复杂的,所以\gls{DL}算法往往缺乏保证。
在许多其他领域,优化的主要方法是为有限的函数族设计优化算法。

在\gls{DL}的背景下,限制函数满足\firstgls{lipschitz_continuous}或其导数\gls{lipschitz}连续可以获得一些保证。
\gls{lipschitz}~连续函数的变化速度以\firstgls{lipschitz_constant} $\CalL$为界:
\begin{align}
 \forall \Vx,~\forall \Vy, ~| f(\Vx) - f(\Vy)|  \leq \CalL \| \Vx - \Vy \|_2 .
\end{align}
这个属性允许我们量化我们的假设——\gls{GD}等算法导致的输入的微小变化将使输出只产生微小变化,因此是很有用的。
\gls{lipschitz}~连续性也是相当弱的约束,并且\gls{DL}中很多优化问题经过相对较小的修改后就能变得~\gls{lipschitz_continuous}。

% -- 88 --

最成功的特定优化领域或许是\firstgls{convex_optimization}。
\gls{convex_optimization}通过更强的限制提供更多的保证。
\gls{convex_optimization}算法只对凸函数适用,即~\gls{hessian}~处处半正定的函数。
因为这些函数没有\gls{saddle_points}而且其所有\gls{local_minimum}必然是\gls{global_minimum},所以表现很好。
然而,\gls{DL}中的大多数问题都难以表示成\gls{convex_optimization}的形式。
\gls{convex_optimization}仅用作一些\gls{DL}算法的子程序。
\gls{convex_optimization}中的分析思路对证明\gls{DL}算法的收敛性非常有用,然而一般来说,\gls{DL}背景下\gls{convex_optimization}的重要性大大减少。 
有关\gls{convex_optimization}的详细信息,详见~\cite{Boyd04}或~\cite{rockafellar1997convex}。


\section{\glsentrytext{constrained_optimization}}
\label{sec:constrained_optimization}
有时候,在$\Vx$的所有可能值下最大化或最小化一个函数$f(x)$不是我们所希望的。
相反,我们可能希望在$\Vx$的某些集合$\SetS$中找$f(\Vx)$的最大值或最小值。
这被称为\firstgls{constrained_optimization}。
在\gls{constrained_optimization}术语中,集合$\SetS$内的点$\Vx$被称为\firstgls{feasible}点。

我们常常希望找到在某种意义上小的解。
针对这种情况下的常见方法是强加一个范数约束,如$\| \Vx \| \leq 1$。

\gls{constrained_optimization}的一个简单方法是将约束考虑在内后简单地对\gls{GD}进行修改。
如果我们使用一个小的恒定步长$\epsilon$,我们可以先取\gls{GD}的单步结果,然后将结果投影回$\SetS$。
如果我们使用\gls{line_search},我们只能在步长为$\epsilon$范围内搜索\gls{feasible}的新$\Vx$点,或者我们可以将线上的每个点投影到约束区域。
如果可能的话,在\gls{GD}或\gls{line_search}前将\gls{gradient}投影到\gls{feasible}域的切空间会更高效\citep{rosen1960}。

一个更复杂的方法是设计一个不同的、无约束的优化问题,其解可以转化成原始\gls{constrained_optimization}问题的解。
例如,我们要在$\Vx \in \SetR^2$中最小化$f(\Vx)$,其中$\Vx$约束为具有单位$L^2$范数。
我们可以关于$\theta$最小化$g(\theta) = f([\cos \theta, \sin \theta]^\Tsp)$,最后返回$[\cos \theta, \sin \theta]$作为原问题的解。
这种方法需要创造性;优化问题之间的转换必须专门根据我们遇到的每一种情况进行设计。

% -- 89 -- 

\firstacr{KKT}方法\footnote{\glssymbol{KKT}~方法是\textbf{Lagrange乘子法}(只允许等式约束)的推广。}是针对\gls{constrained_optimization}非常通用的解决方案。
为介绍\glssymbol{KKT}方法,我们引入一个称为\firstgls{generalized_lagrangian}或\firstgls{generalized_lagrange_function}的新函数。

为了定义\ENNAME{Lagrangian},我们先要通过等式和不等式的形式描述$\SetS$。 
我们希望通过$m$个函数$g^{(i)}$和$n$个函数$h^{(j)}$描述$\SetS$,那么$\SetS$可以表示为$\SetS = \{ \Vx \mid \forall i, g^{(i)}(\Vx) = 0 ~\text{and}~ \forall j, h^{(j)}(\Vx) \leq 0  \}$。
其中涉及$g^{(i)}$的等式称为\firstgls{equality_constraints},涉及$h^{(j)}$的不等式称为\firstgls{inequality_constraints}。

我们为每个约束引入新的变量$\lambda_i$和$\alpha_j$,这些新变量被称为~\glssymbol{KKT}~乘子。\gls{generalized_lagrangian}~可以如下定义:
\begin{align}
 L(\Vx, \Vlambda, \Valpha) = f(\Vx) + \sum_i \lambda_i g^{(i)}(\Vx)  + \sum_j \alpha_j h^{(j)}(\Vx).
\end{align}

现在,我们可以通过优化无约束的\gls{generalized_lagrangian}~解决约束最小化问题。
只要存在至少一个\gls{feasible}点且$f(\Vx)$不允许取$\infty$,那么
\begin{align}
 \underset{\Vx}{\min}~  \underset{\Vlambda}{\max}~
 \underset{\Valpha, \Valpha \geq 0}{\max}   L(\Vx, \Vlambda, \Valpha) 
\end{align}
与如下函数有相同的最优\gls{objective_function}值和最优点集$\Vx$
\begin{align}
 \underset{\Vx \in \SetS}{\min}~ f(\Vx).
\end{align}
这是因为当约束满足时,
\begin{align}
  \underset{\Vlambda}{\max}~
 \underset{\Valpha, \Valpha \geq 0}{\max}   L(\Vx, \Vlambda, \Valpha)  = f(\Vx) ,
\end{align}
而违反任意约束时,
\begin{align}
  \underset{\Vlambda}{\max}  
 \underset{\Valpha, \Valpha \geq 0}{\max}   L(\Vx, \Vlambda, \Valpha)  = \infty .
\end{align}
这些性质保证不可行点不会是最佳的,并且\gls{feasible}点范围内的最优点不变。

% -- 90 --

要解决约束最大化问题,我们可以构造$-f(\Vx)$的\gls{generalized_lagrange_function},从而导致以下优化问题:
\begin{align}
 \underset{\Vx}{\min}~ \underset{\Vlambda}{\max}  ~
 \underset{\Valpha, \Valpha \geq 0}{\max} 
  -f(\Vx) + \sum_i \lambda_i g^{(i)}(\Vx)  + \sum_j \alpha_j h^{(j)}(\Vx).
\end{align}
我们也可将其转换为在外层最大化的问题:
\begin{align}
 \underset{\Vx}{\max}~ \underset{\Vlambda}{\min}~
 \underset{\Valpha, \Valpha \geq 0}{\min} 
  f(\Vx) + \sum_i \lambda_i g^{(i)}(\Vx) - \sum_j \alpha_j h^{(j)}(\Vx).
\end{align}
\gls{equality_constraints}对应项的符号并不重要;因为优化可以自由选择每个$\lambda_i$的符号,我们可以随意将其定义为加法或减法。

\gls{inequality_constraints}特别有趣。
如果$h^{(i)}(\Vx^*)= 0$,我们就说说这个约束$h^{(i)}(\Vx)$是\textbf{活跃}(active)的。
如果约束不是活跃的,则有该约束的问题的解与去掉该约束的问题的解至少存在一个相同的局部解。
一个不活跃约束有可能排除其他解。
例如,整个区域(代价相等的宽平区域)都是全局最优点的的凸问题可能因约束消去其中的某个子区域,或在非凸问题的情况下,收敛时不活跃的约束可能排除了较好的局部\gls{stationary_point}。
然而,无论不活跃的约束是否被包括在内,收敛时找到的点仍然是一个\gls{stationary_point}。
因为一个不活跃的约束$h^{(i)}$必有负值,那么$
 \underset{\Vx}{\min}~  \underset{\Vlambda}{\max}~
 \underset{\Valpha, \Valpha \geq 0}{\max}   L(\Vx, \Vlambda, \Valpha) 
$中的$\alpha_i = 0$。
因此,我们可以观察到在该解中$\Valpha \odot \Vh(\Vx) = 0$。
换句话说,对于所有的$i$, $\alpha_i \geq 0$或$ h^{(j)}(\Vx) \leq 0$在收敛时必有一个是活跃的。
为了获得关于这个想法的一些直观解释,我们可以说这个解是由不等式强加的边界,我们必须通过对应的~\glssymbol{KKT}~乘子影响$\Vx$的解,或者不等式对解没有影响,我们则归零~\glssymbol{KKT}~乘子。

我们可以使用一组简单的性质来描述\gls{constrained_optimization}问题的最优点。
这些性质称为\firstacr{KKT}条件\citep{Karush39,kuhn1951}。
这些是确定一个点是最优点的必要条件,但不一定是充分条件。
这些条件是:
\begin{itemize}
 \item \gls{generalized_lagrangian}~的\gls{gradient}为零。
 \item 所有关于$\Vx$和~\glssymbol{KKT}~乘子的约束都满足。
 \item \gls{inequality_constraints}显示的``互补松弛性'':$\Valpha \odot \Vh(\Vx) = 0$。
\end{itemize}
有关~\glssymbol{KKT}~方法的详细信息,请参阅~\cite{NumOptBook}。

% -- 91 --

\section{实例:线性最小二乘}
\label{sec:example_linear_least_squares}
假设我们希望找到最小化下式的$\Vx$值
\begin{align}
 f(\Vx) = \frac{1}{2}\| \MA \Vx - \Vb \|_2^2 .
\end{align}
存在专门的线性代数算法能够高效地解决这个问题;但是,我们也可以探索如何使用基于\gls{gradient}的优化来解决这个问题,这可以作为这些技术是如何工作的一个简单例子。

首先,我们计算\gls{gradient}:
\begin{align}
 \nabla_{\Vx} f(\Vx) = \MA^\Tsp (\MA \Vx - \Vb) = \MA^\Tsp \MA \Vx - \MA^\Tsp \Vb .
\end{align}

然后,我们可以采用小的步长,并按照这个\gls{gradient}下降。见\algref{alg:gdlsq}中的详细信息。

\begin{algorithm}[ht]
\caption{从任意点$\Vx$开始,使用\gls{GD}关于$\Vx$最小化
$ f(\Vx) = \frac{1}{2} || \MA \Vx - \Vb ||_2^2$的算法。
}
\label{alg:gdlsq}
\begin{algorithmic}
\STATE 将步长 ($\epsilon$) 和\gls{tolerance} ($\delta$)设为小的正数。
\WHILE{ $ || \MA^\top \MA \Vx - \MA^\top \Vb ||_2 > \delta $}
\STATE $\Vx \leftarrow \Vx - \epsilon \left( \MA^\top \MA \Vx - \MA^\top \Vb \right)$
\ENDWHILE
\end{algorithmic}
\end{algorithm}


我们也可以使用\gls{newton_method}解决这个问题。
因为在这个情况下,真实函数是二次的,\gls{newton_method}所用的二次近似是精确的,该算法会在一步后收敛到\gls{global_minimum}。

现在假设我们希望最小化同样的函数,但受 $\Vx^\Tsp \Vx \leq 1$ 的约束。 
要做到这一点,我们引入~\ENNAME{Lagrangian}
\begin{align}
 L(\Vx, \lambda) = f(\Vx) + \lambda (\Vx^\Tsp \Vx - 1).
\end{align}
现在,我们解决以下问题
\begin{align}
  \underset{\Vx}{\min}~
 \underset{\lambda, \lambda \geq 0}{\max}~ L(\Vx, \lambda) .
\end{align}

% -- 92 --

我们可以用~\ENNAME{Moore-Penrose}~伪逆:$\Vx = \MA^+ \Vb$找到无约束最小二乘问题的最小范数解。
如果这一点是\gls{feasible},那么这也是约束问题的解。
否则,我们必须找到约束是活跃的解。
关于$\Vx$对~\ENNAME{Lagrangian}~微分,我们得到方程
\begin{align}
 \MA^\Tsp \MA \Vx - \MA^\Tsp \Vb + 2 \lambda \Vx = 0.
\end{align}
这就告诉我们,该解的形式将会是
\begin{align}
\Vx =  (\MA^\Tsp \MA + 2 \lambda \MI )^{-1} \MA^\Tsp \Vb.
\end{align}
$\lambda$的选择必须使结果服从约束。
我们可以关于$\lambda$进行\gls{gradient}上升找到这个值。
为了做到这一点,观察
\begin{align}
 \frac{\partial}{\partial \lambda} L(\Vx, \lambda)  = \Vx^\Tsp \Vx - 1.
\end{align}
当$\Vx$的范数超过1时,该导数是正的,所以为了跟随\gls{derivative}上坡并相对$\lambda$增加~\ENNAME{Lagrangian},我们需要增加$\lambda$。
因为$\Vx^\Tsp \Vx$的惩罚系数增加了,求解关于$\Vx$的线性方程现在将得到具有较小范数的解。
求解线性方程和调整$\lambda$的过程将一直持续到$\Vx$具有正确的范数并且关于$\lambda$的\gls{derivative}是$0$。

本章总结了开发\gls{ML}算法所需的数学基础。
现在,我们已经准备好建立和分析一些成熟的学习系统。

% -- 93 --


% !Mode:: "TeX:UTF-8"
% Translator: Yujun Li 
\chapter{\glsentrytext{ML}基础}
\label{chap:machine_learning_basics}
\gls{DL}是\gls{ML}的一个特定分支。
我们要想充分理解\gls{DL},必须对\gls{ML}的基本原理有深刻的理解。
本章将探讨贯穿本书其余部分的一些\gls{ML}重要原理。
我们建议新手读者或是希望更全面了解的读者参考一些更全面覆盖基础知识的\gls{ML}参考书,例如~\cite{MurphyBook2012}或者~\cite{bishop-book2006}。
如果你已经熟知\gls{ML},可以跳过前面的部分,前往\secref{sec:challenges_motivating_deep_learning}。
\secref{sec:challenges_motivating_deep_learning}涵盖了一些传统\gls{ML}技术观点,这些技术对\gls{DL}的发展有着深远影响。

首先,我们将介绍学习算法的定义,并介绍一个简单的示例:\gls{linear_regression}算法。
接下来,我们会探讨拟合训练数据与寻找能够泛化到新数据的模式存在哪些不同的挑战。
大部分\gls{ML}算法都有\emph{超参数}(必须在学习算法外设定);我们将探讨如何使用额外的数据设置超参数。
\gls{ML}本质上属于应用统计学,更多地关注于如何用计算机统计地估计复杂函数,不太关注为这些函数提供置信区间;因此我们会探讨两种统计学的主要方法:频率派估计和\gls{bayesian_inference}。
大部分\gls{ML}算法可以分成\gls{supervised_learning}和\gls{unsupervised_learning}两类;我们将探讨不同的分类,并为每类提供一些简单的\gls{ML}算法作为示例。
大部分\gls{DL}算法都是基于被称为\gls{SGD}的算法求解的。
我们将介绍如何组合不同的算法部分,例如优化算法、\gls{cost_function}、模型和\gls{dataset},来建立一个\gls{ML}算法。
最后在\secref{sec:challenges_motivating_deep_learning},我们会介绍一些限制传统\gls{ML}泛化能力的因素。
这些挑战促进了解决这些问题的\gls{DL}算法的发展。

% -- 95 --

\section{学习算法}
\label{sec:learning_algorithms}
\gls{ML}算法是一种能够从数据中学习的算法。
然而,我们所谓的``学习''是什么意思呢?
\cite{Mitchell:1997:ML}提供了一个简洁的定义:``对于某类任务$T$和\gls{performance_measures} $P$,一个计算机程序被认为可以从\gls{experience} $E$中学习是指,通过\gls{experience} $E$改进后,它在任务$T$上由\gls{performance_measures} $P$衡量的性能有所提升。''
\gls{experience} $E$,任务$T$和\gls{performance_measures} $P$的定义范围非常宽广,在本书中我们并不会试图去解释这些定义的具体意义。
相反,我们会在接下来的章节中提供直观的解释和示例来介绍不同的任务、\gls{performance_measures}和\gls{experience},这些将被用来构建\gls{ML}算法。

\subsection{任务 $T$}
\label{sec:the_task_t}
\gls{ML}可以让我们解决一些人为设计和使用确定性程序很难解决的问题。
从科学和哲学的角度来看,\gls{ML}受到关注是因为提高我们对\gls{ML}的认识需要提高我们对智能背后原理的理解。


从``任务''的相对正式的定义上说,学习过程本身不能算是任务。
学习是我们所谓的获取完成任务的能力。
例如,我们的目标是使机器人能够行走,那么行走便是任务。
我们可以编程让机器人学会如何行走,或者可以人工编写特定的指令来指导机器人如何行走。


通常\gls{ML}任务定义为\gls{ML}系统应该如何处理\firstgls{example:chap5}。
\gls{example:chap5}是指我们从某些希望\gls{ML}系统处理的对象或事件中收集到的已经量化的\firstgls{feature}的集合。
我们通常会将\gls{example:chap5}表示成一个向量$\Vx\in\SetR^n$,其中向量的每一个元素$x_i$是一个\gls{feature}。
例如,一张图片的\gls{feature}通常是指这张图片的像素值。

% -- 96 --

\gls{ML}可以解决很多类型的任务。
一些非常常见的\gls{ML}任务列举如下:
\begin{itemize}
    \item \textbf{分类}:
    在这类任务中,计算机程序需要指定某些输入属于$k$类中的哪一类。
    为了完成这个任务,学习算法通常会返回一个函数$f:\SetR^n \to \{1,\dots,k\}$。
    当$y=f(\Vx)$时,模型将向量$\Vx$所代表的输入分类到数字码$y$所代表的类别。
    还有一些其他的分类问题,例如,$f$输出的是不同类别的概率分布。
    分类任务中有一个任务是对象识别,其中输入是图片(通常由一组像素亮度值表示),输出是表示图片物体的数字码。
    例如,Willow Garage PR2机器人能像服务员一样识别不同饮料,并送给点餐的顾客\citep{Goodfellow2010}。
    目前,最好的对象识别工作正是基于\gls{DL}的\citep{Krizhevsky-2012-small,Ioffe+Szegedy-2015}。
    对象识别同时也是计算机识别人脸的基本技术,可用于标记相片合辑中的人脸\citep{Taigman-et-al-CVPR2014},有助于计算机更自然地与用户交互。
    
    \item \textbf{输入缺失分类}:
    当输入向量的每个度量不被保证的时候,分类问题将会变得更有挑战性。
    为了解决分类任务,学习算法只需要定义\emph{一个}从输入向量映射到输出类别的函数。
    当一些输入可能丢失时,学习算法必须学习\emph{一组}函数,而不是单个分类函数。
    每个函数对应着分类具有不同缺失输入子集的$\Vx$。
    这种情况在医疗诊断中经常出现,因为很多类型的医学测试是昂贵的,对身体有害的。
    有效地定义这样一个大集合函数的方法是学习所有相关变量的概率分布,然后通过边缘化缺失变量来解决分类任务。 
    使用$n$个输入变量,我们现在可以获得每个可能的缺失输入集合所需的所有$2^n$个不同的分类函数,但是计算机程序仅需要学习一个描述联合概率分布的函数。
    参见~\cite{Goodfellow-et-al-NIPS2013}了解以这种方式将深度概率模型应用于这类任务的示例。 
    本节中描述的许多其他任务也可以推广到缺失输入的情况; 缺失输入分类只是\gls{ML}能够解决的问题的一个示例。
    
% -- 97 --

    \item \textbf{回归}:在这类任务中,计算机程序需要对给定输入预测数值。
    为了解决这个任务,学习算法需要输出函数$f:\SetR^n \to \SetR$。
    除了返回结果的形式不一样外,这类问题和分类问题是很像的。
    这类任务的一个示例是预测投保人的索赔金额(用于设置保险费),或者预测证券未来的价格。
    这类预测也用在算法交易中。
    
    \item \textbf{\gls{transcribe}}:
    这类任务中,\gls{ML}系统观测一些相对非结构化表示的数据,并\gls{transcribe}信息为离散的文本形式。
    例如,光学字符识别要求计算机程序根据文本图片返回文字序列(ASCII码或者Unicode码)。
    谷歌街景以这种方式使用\gls{DL}处理街道编号\citep{Goodfellow+et+al-ICLR2014a}。
    另一个例子是语音识别,计算机程序输入一段音频波形,输出一序列音频记录中所说的字符或单词ID的编码。
    \gls{DL}是现代语音识别系统的重要组成部分,被各大公司广泛使用,包括微软,IBM和谷歌\citep{Hinton-et-al-2012}。

    \item \textbf{机器翻译}:在机器翻译任务中,输入是一种语言的符号序列,计算机程序必须将其转化成另一种语言的符号序列。
    这通常适用于自然语言,如将英语译成法语。
    最近,\gls{DL}已经开始在这个任务上产生重要影响\citep{Sutskever-et-al-NIPS2014,Bahdanau-et-al-ICLR2015-small}。

    \item \textbf{结构化输出}:结构化输出任务的输出是向量或者其他包含多个值的数据结构,并且构成输出的这些不同元素间具有重要关系。
    这是一个很大的范畴,包括上述\gls{transcribe}任务和翻译任务在内的很多其他任务。
    例如语法分析——映射自然语言句子到语法结构树,并标记树的节点为动词、名词、副词等等。
    参考~\cite{Collobert-AISTATS2011}将\gls{DL}应用到语法分析的示例。
    另一个例子是图像的像素级分割,将每一个像素分配到特定类别。
    例如,\gls{DL}可用于标注航拍照片中的道路位置\citep{MnihHinton2010}。
    在这些标注型的任务中,输出的结构形式不需要和输入尽可能相似。
    例如,在为图片添加描述的任务中,计算机程序观察到一幅图,输出描述这幅图的自然语言句子\citep{Kiros-et-al-ICML2014,Kiros-et-al-arxiv2014,Mao-et-al-2014,Vinyals-et-al-CVPR2015,Donahue-et-al-arxiv2014,Karpathy+Li-CVPR2015,Fang-et-al-CVPR2015,Xu-et-al-ICML2015}。
    这类任务被称为\emph{结构化输出任务}是因为输出值之间内部紧密相关。
    例如,为图片添加标题的程序输出的单词必须组合成一个通顺的句子。

% -- 98 --

    \item \textbf{异常检测}:在这类任务中,计算机程序在一组事件或对象中筛选,并标记不正常或非典型的个体。
    异常检测任务的一个示例是信用卡欺诈检测。
    通过对你的购买习惯建模,信用卡公司可以检测到你的卡是否被滥用。
    如果窃贼窃取你的信用卡或信用卡信息,窃贼采购物品的分布通常和你的不同。
    当该卡发生了不正常的购买行为时,信用卡公司可以尽快冻结该卡以防欺诈。
    参考~\cite{chandola2009anomaly}了解欺诈检测方法。

    \item \textbf{合成和采样}:在这类任务中,\gls{ML}程序生成一些和训练数据相似的新\gls{example:chap5}。
    通过\gls{ML},合成和采样可能在媒体应用中非常有用,可以避免艺术家大量昂贵或者乏味费时的手动工作。
    例如,视频游戏可以自动生成大型物体或风景的纹理,而不是让艺术家手动标记每个像素\citep{Luo+al-AISTATS2013-small}。
    在某些情况下,我们希望采样或合成过程可以根据给定的输入生成一些特定类型的输出。
    例如,在语音合成任务中,我们提供书写的句子,要求程序输出这个句子语音的音频波形。
    这是一类\emph{结构化输出任务},但是多了每个输入并非只有一个正确输出的条件,并且我们明确希望输出有很多变化,这可以使结果看上去更加自然和真实。

    \item \textbf{缺失值填补}:在这类任务中,\gls{ML}算法给定一个新\gls{example:chap5} $\Vx\in\SetR^n$,$\Vx$中某些元素$x_i$缺失。
    算法必须填补这些缺失值。

% -- 99 --

    \item \textbf{\gls{denoising}}:在这类任务中,\gls{ML}算法的输入是,\emph{干净\gls{example:chap5}}~$\Vx \in \SetR^n$经过未知损坏过程后得到的\emph{损坏\gls{example:chap5}}~$\tilde{\Vx} \in \SetR^n$。
    算法根据损坏后的\gls{example:chap5} $\tilde{\Vx}$预测干净的\gls{example:chap5} $\Vx$,或者更一般地预测条件概率分布$p(\Vx\mid\tilde{\Vx})$。
    
    \item \textbf{密度估计}或\textbf{\gls{PMF}估计}:在密度估计问题中,\gls{ML}算法学习函数$p_{\text{model}}:\SetR^n \to \SetR$,其中$p_{\text{model}}(\Vx)$可以解释成\gls{example:chap5}采样空间的概率密度函数(如果$\RVx$是连续的)或者\gls{PMF}(如果$\RVx$是离散的)。
    要做好这样的任务(当我们讨论\gls{performance_measures} $P$时,我们会明确定义任务是什么),算法需要学习观测到的数据的结构。
    算法必须知道什么情况下\gls{example:chap5}聚集出现,什么情况下不太可能出现。
    以上描述的大多数任务都要求学习算法至少能隐式地捕获概率分布的结构。
    密度估计可以让我们显式地捕获该分布。
    原则上,我们可以在该分布上计算以便解决其他任务。
    例如,如果我们通过密度估计得到了概率分布$p(\Vx)$,我们可以用该分布解决缺失值填补任务。
    如果$x_i$的值是缺失的,但是其他的变量值$\Vx_{-i}$已知,那么我们可以得到条件概率分布$p(x_i\mid\Vx_{-i})$。
    实际情况中,密度估计并不能够解决所有这类问题,因为在很多情况下$p(\Vx)$是难以计算的。
\end{itemize}

当然,还有很多其他同类型或其他类型的任务。
这里我们列举的任务类型只是用来介绍\gls{ML}可以做哪些任务,并非严格地定义\gls{ML}任务分类。

\subsection{\glsentrytext{performance_measures} $P$}
\label{sec:the_performance_measure_p}
为了评估\gls{ML}算法的能力,我们必须设计其性能的定量度量。
通常\gls{performance_measures} $P$是特定于系统执行的任务$T$而言的。

对于诸如分类、缺失输入分类和\gls{transcribe}任务,我们通常度量模型的\firstgls{accuracy}。
\gls{accuracy}是指该模型输出正确结果的\gls{example:chap5}比率。
我们也可以通过\firstgls{error_rate}得到相同的信息。
\gls{error_rate}是指该模型输出错误结果的\gls{example:chap5}比率。
我们通常把\gls{error_rate}称为$0-1$\gls{loss}的期望。
在一个特定的\gls{example:chap5}上,如果结果是对的,那么$0-1$\gls{loss}是$0$;否则是$1$。
但是对于密度估计这类任务而言,度量准确率,错误率或者其他类型的$0-1$\gls{loss}是没有意义的。
反之,我们必须使用不同的性能度量,使模型对每个\gls{example:chap5}都输出一个连续数值的得分。
最常用的方法是输出模型在一些\gls{example:chap5}上概率对数的平均值。


% -- 100 --

通常,我们会更加关注\gls{ML}算法在未观测数据上的性能如何,因为这将决定其在实际应用中的性能。
因此,我们使用\firstgls{test_set}数据来评估系统性能,将其与训练机器学习系统的训练集数据分开。


\gls{performance_measures}的选择或许看上去简单且客观,但是选择一个与系统理想表现对应的\gls{performance_measures}通常是很难的。


在某些情况下,这是因为很难确定应该度量什么。
例如,在执行\gls{transcribe}任务时,我们是应该度量系统\gls{transcribe}整个序列的准确率,还是应该用一个更细粒度的指标,对序列中正确的部分元素以正面评价?
在执行回归任务时,我们应该更多地惩罚频繁犯一些中等错误的系统,还是较少犯错但是犯很大错误的系统?
这些设计的选择取决于应用。


还有一些情况,我们知道应该度量哪些数值,但是度量它们不太现实。
这种情况经常出现在密度估计中。
很多最好的概率模型只能隐式地表示概率分布。
在许多这类模型中,计算空间中特定点的概率是不可行的。
在这些情况下,我们必须设计一个仍然对应于设计对象的替代标准,或者设计一个理想标准的良好近似。

\subsection{\glsentrytext{experience} $E$}
\label{sec:the_experience_e}
根据学习过程中的不同\gls{experience},\gls{ML}算法可以大致分类为\firstgls{unsupervised}算法和\firstgls{supervised}算法。

本书中的大部分学习算法可以被理解为在整个\firstgls{dataset}上获取\gls{experience}。
\gls{dataset}是指很多\gls{example:chap5}组成的集合,如\secref{sec:the_task_t}所定义的。
有时我们也将\gls{example:chap5}称为\firstgls{data_points}。

% -- 101 --

Iris(鸢尾花卉)\gls{dataset}~\citep{Fisher-1936}是统计学家和\gls{ML}研究者使用了很久的\gls{dataset}。
它是$150$个鸢尾花卉植物不同部分测量结果的集合。
每个单独的植物对应一个\gls{example:chap5}。
每个\gls{example:chap5}的\gls{feature}是该植物不同部分的测量结果:萼片长度、萼片宽度、花瓣长度和花瓣宽度。
这个\gls{dataset}也记录了每个植物属于什么品种,其中共有三个不同的品种。

\firstgls{unsupervised_learning_algorithm}训练含有很多\gls{feature}的\gls{dataset},然后学习出这个\gls{dataset}上有用的结构性质。
在\gls{DL}中,我们通常要学习生成\gls{dataset}的整个概率分布,显式地,比如密度估计,或是隐式地,比如合成或\gls{denoising}。
还有一些其他类型的\gls{unsupervised_learning}任务,例如聚类,将\gls{dataset}分成相似\gls{example:chap5}的集合。

\firstgls{supervised_learning_algorithm}训练含有很多\gls{feature}的\gls{dataset},不过\gls{dataset}中的\gls{example:chap5}都有一个\firstgls{label}或\firstgls{target}。
例如,Iris~\gls{dataset}注明了每个鸢尾花卉\gls{example:chap5}属于什么品种。
\gls{supervised_learning}算法通过研究Iris~\gls{dataset},学习如何根据测量结果将\gls{example:chap5}划分为三个不同品种。

大致说来,\gls{unsupervised_learning}涉及到观察随机向量$\RVx$的好几个\gls{example:chap5},试图显式或隐式地学习出概率分布$p(\RVx)$,或者是该分布一些有意思的性质;
而\gls{supervised_learning}包含观察随机向量$\RVx$及其相关联的值或向量$\RVy$,然后从$\RVx$预测$\RVy$,通常是估计$p(\RVy\mid\RVx)$。
术语\firstgls{supervised_learning}源自这样一个视角,教员或者老师提供\gls{target} $\RVy$给\gls{ML}系统,指导其应该做什么。
在\gls{unsupervised_learning}中,没有教员或者老师,算法必须学会在没有指导的情况下理解数据。

\gls{unsupervised_learning}和\gls{supervised_learning}不是严格定义的术语。
它们之间界线通常是模糊的。
很多\gls{ML}技术可以用于这两个任务。
例如,概率的链式法则表明对于向量$\RVx\in\SetR^n$,联合分布可以分解成
\begin{equation}
    p(\RVx) = \prod_{i=1}^n p(\RSx_i \mid \RSx_1,\dots,\RSx_{i-1}) .
\end{equation}
该分解意味着我们可以将其拆分成$n$个\gls{supervised_learning}问题,来解决表面上的\gls{unsupervised_learning} $p(\Vx)$。
另外,我们求解\gls{supervised_learning}问题$p(y\mid\RVx)$时,也可以使用传统的\gls{unsupervised_learning}策略学习联合分布$p(\RVx,y)$,然后推断
\begin{equation}
    p(y\mid\RVx) = \frac{p(\RVx,y)}{\sum_{y'}p(\RVx,y')}.
\end{equation}
尽管\gls{unsupervised_learning}和\gls{supervised_learning}并非完全没有交集的正式概念,它们确实有助于粗略分类我们研究\gls{ML}算法时遇到的问题。
传统地,人们将回归、分类或者结构化输出问题称为\gls{supervised_learning}。
支持其他任务的密度估计通常被称为\gls{unsupervised_learning}。

% -- 102 --

学习范式的其他变种也是有可能的。
例如,半监督学习中,一些\gls{example:chap5}有监督\gls{target},但其他\gls{example:chap5}没有。
在多实例学习中,\gls{example:chap5}的整个集合被标记为含有或者不含有该类的\gls{example:chap5},但是集合中单独的样本是没有标记的。
参考~\cite{Kotzias2015}了解最近\gls{deep_model}进行多实例学习的示例。

有些\gls{ML}算法并不是训练于一个固定的\gls{dataset}上。
例如,\firstgls{reinforcement_learning}算法会和环境进行交互,所以学习系统和它的训练过程会有反馈回路。
这类算法超出了本书的范畴。
请参考~\cite{Sutton+Barto-98}或~\cite{Bertsekas+Tsitsiklis-book1996}了解强化学习相关知识,\cite{Mnih2013}介绍了强化学习方向的\gls{DL}方法。

大部分\gls{ML}算法简单地训练于一个\gls{dataset}上。
\gls{dataset}可以用很多不同方式来表示。
在所有的情况下,\gls{dataset}都是\gls{example:chap5}的集合,而\gls{example:chap5}是\gls{feature}的集合。

表示\gls{dataset}的常用方法是\firstgls{design_matrix}。
\gls{design_matrix}的每一行包含一个不同的\gls{example:chap5}。
每一列对应不同的\gls{feature}。
例如,Iris~\gls{dataset}包含$150$个\gls{example:chap5},每个\gls{example:chap5}有4个\gls{feature}。
这意味着我们可以将该\gls{dataset}表示为\gls{design_matrix} $\MX\in\SetR^{150\times 4}$,其中$X_{i,1}$表示第$i$个植物的萼片长度,$X_{i,2}$表示第$i$个植物的萼片宽度等等。
我们在本书中描述的大部分学习算法都是讲述它们是如何运行在\gls{design_matrix}\gls{dataset}上的。

当然,每一个\gls{example:chap5}都能表示成向量,并且这些向量的大小相同,才能将一个\gls{dataset}表示成\gls{design_matrix}。
这一点并非永远可能。
例如,你有不同宽度和高度的照片的集合,那么不同的照片将会包含不同数量的像素。
因此不是所有的照片都可以表示成相同长度的向量。
\secref{sec:data_types}和\chapref{chap:sequence_modeling_recurrent_and_recursive_nets}将会介绍如何处理这些不同类型的异构数据。
在上述这类情况下,我们不会将\gls{dataset}表示成$m$行的矩阵,而是表示成$m$个元素的结合:$\{\Vx^{(1)},\Vx^{(2)},\dots,\Vx^{(m)}\}$。
这种表示方式意味着\gls{example:chap5}向量$\Vx^{(i)}$和$\Vx^{(j)}$可以有不同的大小。

% -- 103 --

在\gls{supervised_learning}中,\gls{example:chap5}包含一个\gls{label}或\gls{target}和一组\gls{feature}。
例如,我们希望使用学习算法从照片中识别对象。
我们需要明确哪些对象会出现在每张照片中。
我们或许会用数字编码表示,如$0$表示人、$1$表示车、$2$表示猫等等。
通常在处理包含观测\gls{feature}的\gls{design_matrix} $\MX$的\gls{dataset}时,我们也会提供一个\gls{label}向量$\Vy$,其中$y_i$表示\gls{example:chap5} $i$的\gls{label}。


当然,有时\gls{label}可能不止一个数。
例如,如果我们想要训练语音模型\gls{transcribe}整个句子,那么每个句子\gls{example:chap5}的\gls{label}是一个单词序列。


正如\gls{supervised_learning}和\gls{unsupervised_learning}没有正式的定义,\gls{dataset}或者\gls{experience}也没有严格的区分。
这里介绍的结构涵盖了大多数情况,但始终有可能为新的应用设计出新的结构。

\subsection{示例:\glsentrytext{linear_regression}}
\label{sec:example_linear_regression}
我们将\gls{ML}算法定义为,通过经验以提高计算机程序在某些任务上性能的算法。
这个定义有点抽象。
为了使这个定义更具体点,我们展示一个简单的\gls{ML}示例:\firstgls{linear_regression}。
当我们介绍更多有助于理解\gls{ML}特性的概念时,我们会反复回顾这个示例。

顾名思义,\gls{linear_regression}解决回归问题。
换言之,我们的目标是建立一个系统,将向量$\Vx\in\SetR^n$作为输入,预测标量$y\in\SetR$作为输出。
\gls{linear_regression}的输出是其输入的线性函数。
令$\hat{y}$表示模型预测$y$应该取的值。
我们定义输出为
\begin{equation}
    \hat{y} = \Vw^\Tsp \Vx ,
\end{equation}
其中$\Vw\in\SetR^n$是\firstgls{parameters}向量。

\gls{parameters}是控制系统行为的值。
在这种情况下,$w_i$是系数,会和\gls{feature} $x_i$相乘之后全部相加起来。
我们可以将$\Vw$看作是一组决定每个\gls{feature}如何影响预测的\firstgls{weights}。
如果\gls{feature} $x_i$对应的权重$w_i$是正的,那么\gls{feature}的值增加,我们的预测值$\hat{y}$也会增加。
如果\gls{feature} $x_i$对应的权重$w_i$是负的,那么\gls{feature}的值增加,我们的预测值$\hat{y}$会减少。
如果\gls{feature}权重的大小很大,那么它对预测有很大的影响;如果\gls{feature}权重的大小是零,那么它对预测没有影响。

% -- 104 --

因此,我们可以定义任务$T$:通过输出$\hat{y} = \Vw^\Tsp \Vx$从$\Vx$预测$y$。
接下来我们需要定义\gls{performance_measures}——$P$。

假设我们有$m$个输入\gls{example:chap5}组成的\gls{design_matrix},我们不用它来训练模型,而是评估模型性能如何。
我们也有每个\gls{example:chap5}对应的正确值$y$组成的回归\gls{target}向量。
因为这个\gls{dataset}只是用来评估性能,我们称之为\firstgls{test_set}。
我们将输入的\gls{design_matrix}记作$\MX^{\text{(test)}}$,回归\gls{target}向量记作$\Vy^{(\text{test})}$。

度量模型性能的一种方法是计算模型在\gls{test_set}上的\firstgls{mean_squared_error}。
如果$\hat{\Vy}^{(\text{test})}$表示模型在\gls{test_set}上的预测值,那么\gls{mean_squared_error}表示为:
\begin{equation}
    \text{MSE}_{\text{test}} = \frac{1}{m} \sum_i ( \hat{\Vy}^{(\text{test})} - \Vy^{(\text{test})})_i^2.
\end{equation}
直观上,当$\hat{\Vy}^{(\text{test})} = \Vy^{(\text{test})}$时,我们会发现误差降为$0$。
我们也可以看到
\begin{equation}
    \text{MSE}_{\text{test}} = \frac{1}{m} \norm{ \hat{\Vy}^{(\text{test})} - \Vy^{(\text{test})}}_2^2,
\end{equation}
所以当预测值和\gls{target}值之间的欧几里得距离增加时,误差也会增加。

为了构建一个\gls{ML}算法,我们需要设计一个算法,通过观察训练集$(\MX^{(\text{train})},\Vy^{(\text{train})})$获得\gls{experience},减少$\text{MSE}_{\text{test}}$以改进权重$\Vw$。
一种直观方式(我们将在后续的\secref{sec:conditional_log_likelihood_and_mean_squared_error}说明其合法性)是最小化训练集上的\gls{mean_squared_error},即$\text{MSE}_{\text{train}}$。

最小化$\text{MSE}_{\text{train}}$,我们可以简单地求解其导数为$\mathbf{0}$的情况:
\begin{equation}
\nabla_{\Vw} \text{MSE}_{\text{train}} = 0
\end{equation}
\begin{equation}
\Rightarrow \nabla_{\Vw} \frac{1}{m} \norm{ \hat{\Vy}^{(\text{train})} - \Vy^{(\text{train})}}_2^2 = 0
\end{equation}
\begin{equation}
\Rightarrow \frac{1}{m} \nabla_{\Vw} \norm{ \MX^{(\text{train})}\Vw - \Vy^{(\text{train})}}_2^2 = 0
\end{equation}
\begin{equation}
\Rightarrow \nabla_{\Vw} \left( \MX^{(\text{train})}\Vw - \Vy^{(\text{train})} \right)^\Tsp \left( \MX^{(\text{train})}\Vw - \Vy^{(\text{train})} \right) = 0
\end{equation}
\begin{equation}
\Rightarrow \nabla_{\Vw} \left( 
    \Vw^\Tsp \MX^{(\text{train})\Tsp}\MX^{(\text{train})}\Vw - 2\Vw^\Tsp\MX^{(\text{train})\Tsp} \Vy^{(\text{train})} + \Vy^{(\text{train})\Tsp}\Vy^{(\text{train})}  
  \right) = 0
\end{equation}
\begin{equation}
    \Rightarrow 2\MX^{(\text{train})\Tsp}\MX^{(\text{train})} \Vw  -
    2\MX^{(\text{train})\Tsp} \Vy^{(\text{train})}  = 0
\end{equation}
\begin{equation}
\label{eq:5.12}
    \Rightarrow \Vw =  \left(\MX^{(\text{train})\Tsp}\MX^{(\text{train})}
     \right)^{-1} \MX^{(\text{train})\Tsp} \Vy^{(\text{train})}
\end{equation}

% -- 105 --

通过\eqnref{eq:5.12}给出解的系统方程被称为\firstgls{normal_equations}。
计算\eqnref{eq:5.12}构成了一个简单的机器学习算法。
\figref{fig:chap5_linreg}展示了\gls{linear_regression}算法的使用示例。

\begin{figure}[!htb]
\ifOpenSource
\centerline{\includegraphics{figure.pdf}}
\else
\centerline{\includegraphics{Chapter5/figures/linreg_color}}
\fi
\caption{一个\gls{linear_regression}问题,其中训练集包括十个数据点,每个数据点包含一个特征。因为只有一个特征,权重向量$\Vw$也只有一个要学习的参数$w_1$。\emph{(左)}我们可以观察到\gls{linear_regression}学习$w_1$,从而使得直线$y=w_1x$能够尽量接近穿过所有的训练点。\emph{(右)}标注的点表示由\gls{normal_equations}学习到的$w_1$的值,我们发现它可以最小化训练集上的\gls{mean_squared_error}。}
\label{fig:chap5_linreg}
\end{figure}

值得注意的是,术语\firstgls{linear_regression}通常用来指稍微复杂一些,附加额外参数(截距项$b$)的模型。
在这个模型中,
\begin{equation}
    \hat{y} = \Vw^\Tsp \Vx + b,
\end{equation}
因此从参数到预测的映射仍是一个线性函数,而从\gls{feature}到预测的映射是一个仿射函数。
如此扩展到仿射函数意味着模型预测的曲线仍然看起来像是一条直线,只是这条直线没必要经过原点。
除了通过添加偏置参数$b$,我们还可以使用仅含权重的模型,但是$\Vx$需要增加一项永远为$1$的元素。
对应于额外$1$的权重起到了偏置参数的作用。
当我们在本书中提到仿射函数时,我们会经常使用术语``线性''。

% -- 106 --

截距项$b$通常被称为仿射变换的\,\textbf{\gls{bias_aff}}(bias)参数。
这个术语的命名源自该变换的输出在没有任何输入时会偏移$b$。
它和统计偏差中指代统计估计算法的某个量的期望估计偏离真实值的意思是不一样的。

\gls{linear_regression}当然是一个极其简单且有局限的学习算法,但是它提供了一个说明学习算法如何工作的例子。
在接下来的小节中,我们将会介绍一些设计学习算法的基本原则,并说明如何使用这些原则来构建更复杂的学习算法。

\section{\glsentrytext{capacity}、\glsentrytext{overfitting}和\glsentrytext{underfitting}}
\label{sec:capacity_overfitting_and_underfitting}
\gls{ML}的主要挑战是我们的算法必须能够在\emph{先前未观测的新}输入上表现良好,而不只是在训练集上表现良好。 %?? space in book
在先前未观测到的输入上表现良好的能力被称为\firstgls{generalization}。

通常情况下,当我们训练\gls{ML}模型时,我们可以使用某个训练集,在训练集上计算一些被称为\firstgls{training_error}的度量误差,目标是降低训练误差。
目前为止,我们讨论的是一个简单的优化问题。
\gls{ML}和优化不同的地方在于,我们也希望\firstgls{generalization_error}(也被称为\firstgls{test_error})很低。
泛化误差被定义为新输入的误差期望。
这里,期望的计算基于不同的可能输入,这些输入采自于系统在现实中遇到的分布。

通常,我们度量模型在训练集中分出来的\firstgls{test_set}\gls{example:chap5}上的性能,来评估\gls{ML}模型的泛化误差。

在我们的\gls{linear_regression}示例中,我们通过最小化\gls{training_error}来训练模型,
\begin{equation}
    \frac{1}{m^{(\text{train})}} \norm{\MX^{(\text{train})}\Vw - \Vy^{(\text{train})}}_2^2,
\end{equation}
但是我们真正关注的是\gls{test_error}~$\frac{1}{m^{(\text{test})}} \norm{\MX^{(\text{test})}\Vw - \Vy^{(\text{test})}}_2^2$。

% -- 107 --

当我们只能观测到训练集时,我们如何才能影响\gls{test_set}的性能呢?
\firstgls{SLT}提供了一些答案。
如果训练集和\gls{test_set}的数据是任意收集的,那么我们能够做的确实很有限。
如果我们可以对\gls{training_set}和\gls{test_set}数据的收集方式有些假设,那么我们能够对算法做些改进。

\gls{training_set}和\gls{test_set}数据通过\gls{dataset}上被称为\firstgls{DGP}的概率分布生成。
通常,我们会做一系列被统称为\firstgls{iid}的假设。
该假设是说,每个\gls{dataset}中的\gls{example:chap5}都是彼此\firstgls{independent},并且\gls{training_set}和\gls{test_set}是\firstgls{id},采样自相同的分布。
这个假设使我们能够在单个样本的概率分布描述数据生成过程。
然后相同的分布可以用来生成每一个训练\gls{example:chap5}和每一个测试\gls{example:chap5}。
我们将这个共享的潜在分布称为\firstgls{DGD},记作$p_{\text{data}}$。
这个概率框架和独立同分布假设允许我们从数学上研究\gls{training_error}和\gls{test_error}之间的关系。

我们能观察到\gls{training_error}和\gls{test_error}之间的直接联系是,随机模型\gls{training_error}的期望和该模型\gls{test_error}的期望是一样的。
假设我们有概率分布$p(\Vx,y)$,从中重复采样生成\gls{training_set}和\gls{test_set}。
对于某个固定的$\Vw$,\gls{training_set}误差的期望恰好和\gls{test_set}误差的期望一样,这是因为这两个期望的计算都使用了相同的数据集生成过程。
这两种情况的唯一区别是\gls{dataset}的名字不同。

当然,当我们使用\gls{ML}算法时,我们不会提前固定参数,然后从\gls{dataset}中采样。
我们会在训练集上采样,然后挑选参数去降低训练集误差,然后再在\gls{test_set}上采样。
在这个过程中,测试误差期望会大于或等于训练误差期望。
以下是决定\gls{ML}算法效果是否好的因素:
\begin{enumerate}
    \item 降低训练误差。
    \item 缩小训练误差和测试误差的差距。
\end{enumerate}

这两个因素对应\gls{ML}的两个主要挑战:\firstgls{underfitting}和\firstgls{overfitting}。
\gls{underfitting}是指模型不能在训练集上获得足够低的误差。
而\gls{overfitting}是指训练误差和和测试误差之间的差距太大。

% -- 108 --

通过调整模型的\firstgls{capacity},我们可以控制模型是否偏向于\gls{overfitting}或者\gls{underfitting}。
通俗地,模型的\gls{capacity}是指其拟合各种函数的能力。
\gls{capacity}低的模型可能很难拟合训练集。
\gls{capacity}高的模型可能会过拟合,因为记住了不适用于\gls{test_set}的\gls{training_set}性质。

一种控制训练算法容量的方法是选择\firstgls{hypothesis_space},即学习算法可以选择为解决方案的函数集。
例如,\gls{linear_regression}函数将关于其输入的所有线性函数作为假设空间。
广义\gls{linear_regression}的假设空间包括多项式函数,而非仅有线性函数。
这样做就增加了模型的容量。

一次多项式提供了我们已经熟悉的\gls{linear_regression}模型,其预测如下:
\begin{equation}
    \hat{y} = b + wx.
\end{equation}
通过引入$x^2$作为\gls{linear_regression}模型的另一个\gls{feature},我们能够学习关于$x$的二次函数模型:
\begin{equation}
    \hat{y} = b + w_1x + w_2x^2.
\end{equation}
尽管该模型是\emph{输入}的二次函数,但输出仍是\emph{参数}的线性函数。
因此我们仍然可以用\gls{normal_equations}得到模型的闭解。
我们可以继续添加$x$的更高幂作为额外\gls{feature},例如下面的$9$次多项式:
\begin{equation}
    \hat{y} = b + \sum_{i=1}^9 w_i x^i.
\end{equation}

当\gls{ML}算法的\gls{capacity}适合于所执行任务的复杂度和所提供训练数据的数量时,算法效果通常会最佳。
\gls{capacity}不足的模型不能解决复杂任务。
\gls{capacity}高的模型能够解决复杂的任务,但是当其容量高于任务所需时,有可能会过拟合。

\figref{fig:chap5_underfit_just_right_overfit}展示了这个原理的使用情况。
我们比较了线性,二次和$9$次预测器拟合真实二次函数的效果。
线性函数无法刻画真实函数的曲率,所以欠拟合。
$9$次函数能够表示正确的函数,但是因为训练参数比训练\gls{example:chap5}还多,所以它也能够表示无限多个刚好穿越训练\gls{example:chap5}点的很多其他函数。
我们不太可能从这很多不同的解中选出一个泛化良好的。
在这个问题中,二次模型非常符合任务的真实结构,因此它可以很好地泛化到新数据上。

\begin{figure}[!htb]
\ifOpenSource
\centerline{\includegraphics{figure.pdf}}
\else
\centerline{\includegraphics{Chapter5/figures/underfit_just_right_overfit_color}}
\fi
\caption{我们用三个模型拟合了这个训练集的样本。训练数据是通过随机抽取$x$然后用二次函数确定性地生成$y$来合成的。\emph{(左)}用一个线性函数拟合数据会导致\gls{underfitting}——它无法捕捉数据中的\gls{curvature}信息。\emph{(中)}用二次函数拟合数据在未观察到的点上\gls{generalization}得很好。这并不会导致明显的\gls{underfitting}或者\gls{overfitting}。\emph{(右)}一个$9$阶的多项式拟合数据会导致\gls{overfitting}。在这里我们使用\,\gls{Moore}来解这个欠定的\gls{normal_equations}。得出的解能够精确地穿过所有的训练点,但可惜我们无法提取有效的结构信息。在两个数据点之间它有一个真实的函数所不包含的深谷。在数据的左侧,它也会急剧增长,而在这一区域真实的函数却是下降的。}
\label{fig:chap5_underfit_just_right_overfit}
\end{figure}

% -- 109 --

目前为止,我们探讨了通过改变输入\gls{feature}的数目和加入这些\gls{feature}对应的参数,改变模型的容量。
事实上,还有很多方法可以改变模型的容量。
容量不仅取决于模型的选择。
模型规定了调整参数降低训练目标时,学习算法可以从哪些函数族中选择函数。
这被称为模型的\firstgls{representational_capacity}。
在很多情况下,从这些函数中挑选出最优函数是非常困难的优化问题。
实际中,学习算法不会真的找到最优函数,而仅是找到一个可以大大降低训练误差的函数。
额外的限制因素,比如优化算法的不完美,意味着学习算法的\firstgls{effective_capacity}可能小于模型族的\gls{representational_capacity}。

% -- 110 --

提高\gls{ML}模型泛化的现代思想可以追溯到早在托勒密时期的哲学家的思想。
许多早期的学者提出一个简约原则,现在广泛被称为\firstgls{OR}(c. 1287-1387)。
该原则指出,在同样能够解释已知观测现象的假设中,我们应该挑选``最简单''的那一个。
这个想法是在20世纪,由统计学习理论创始人形式化并精确化的\citep{Vapnik71,Vapnik82,Blumer-et-al-1989,Vapnik95}。

统计学习理论提供了量化模型容量的不同方法。
在这些中,最有名的是\firstall{VC}。
\glssymbol{VC}\,维度量二元分类器的容量。
\glssymbol{VC}\,维定义为该分类器能够分类的训练\gls{example:chap5}的最大数目。
假设存在$m$个不同$\Vx$点的训练集,分类器可以任意地标记该$m$个不同的$\Vx$点,\glssymbol{VC}\,维被定义为$m$的最大可能值。

量化模型的容量使得统计学习理论可以进行量化预测。
统计学习理论中最重要的结论阐述了训练误差和泛化误差之间差异的上界随着模型容量增长而增长,但随着训练\gls{example:chap5}增多而下降\citep{Vapnik71,Vapnik82,Blumer-et-al-1989,Vapnik95}。
这些边界为\gls{ML}算法可以有效解决问题提供了理论验证,但是它们很少应用于实际中的\gls{DL}算法。
一部分原因是边界太松,另一部分原因是很难确定\gls{DL}算法的容量。
由于有效容量受限于优化算法的能力,确定\gls{DL}模型容量的问题特别困难。
而且对于\gls{DL}中的一般非凸优化问题,我们只有很少的理论分析。

我们必须记住虽然更简单的函数更可能泛化(训练误差和测试误差的差距小),但我们仍然需要选择一个充分复杂的假设以达到低的\gls{training_error}。
通常,当模型容量上升时,训练误差会下降,直到其渐近最小可能误差(假设误差度量有最小值)。
通常,\gls{generalization_error}是一个关于模型容量的U形曲线函数。
如\figref{fig:chap5_generalization_vs_capacity}所示。

\begin{figure}[!htb]
\ifOpenSource
\centerline{\includegraphics{figure.pdf}}
\else
\centerline{\includegraphics{Chapter5/figures/generalization_vs_capacity_color}}
\fi
\caption{\gls{capacity}和误差之间的典型关系。\gls{training_error}和\gls{test_error}表现得非常不同。在图的左端,\gls{training_error}和\gls{generalization_error}都非常高。这是\firstgls{underfit_regime}。当我们增加\gls{capacity}时,\gls{training_error}减小,但是\gls{training_error}和\gls{generalization_error}之间的间距却不断扩大。最终,这个间距的大小超过了\gls{training_error}的下降,我们进入到了\firstgls{overfit_regime},其中\gls{capacity}过大,超过了\firstgls{optimal_capacity}。}
\label{fig:chap5_generalization_vs_capacity}
\end{figure}

% -- 111 --

为考虑容量任意高的极端情况,我们介绍\firstgls{nonparametric}\emph{模型}的概念。
至此,我们只探讨过参数模型,例如\gls{linear_regression}。
参数模型学习到的函数在观测新数据前,参数是有限且固定的向量。
非参数模型没有这些限制。

有时,非参数模型仅是一些不能实际实现的理论抽象(比如搜索所有可能概率分布的算法)。
然而,我们也可以设计一些实用的非参数模型,使它们的复杂度和训练集大小有关。
这种算法的一个示例是\firstgls{nearest_neighbor_regression}。
不像\gls{linear_regression}有固定长度的向量作为权重,\gls{nearest_neighbor_regression}模型存储了训练集中所有的$\MX$和$\Vy$。
当需要为测试点$\Vx$分类时,模型会查询训练集中离该点最近的点,并返回相关的回归\gls{target}。
换言之,$\hat{y}=y_i$其中$i=\argmin \norm{\MX_{i,:}-\Vx}_2^2$。
该算法也可以扩展成$L^2$范数以外的距离度量,例如学成距离度量\citep{RoweisNCA2005}。
如果允许该算法通过平均$\MX_{i,:}$中所有邻近的向量对应的$y_i$来打破联系,那么该算法会在任意回归\gls{dataset}上达到最小可能的训练误差(如果存在两个相同的输入对应不同的输出,那么训练误差可能会大于零)。

最后,我们也可以将参数学习算法嵌入另一个增加参数数目的算法来创建非参数学习算法。
例如,我们可以想象这样一个算法,外层循环调整多项式的次数,内层循环通过\gls{linear_regression}学习模型。

% -- 112 --

理想模型假设我们能够预先知道生成数据的真实概率分布。
然而这样的模型仍然会在很多问题上发生一些错误,因为分布中仍然会有一些\gls{noise}。
在\gls{supervised_learning}中,从$\Vx$到$y$的映射可能内在是随机的,或者$y$可能是其他变量(包括$\Vx$在内)的确定性函数。
从预先知道的真实分布$p(\Vx,y)$预测而出现的误差被称为\firstgls{bayes_error}。

\gls{training_error}和\gls{generalization_error}会随训练集的大小发生变化。
泛化误差的期望从不会因训练\gls{example:chap5}数目的增加而增加。
对于非参数模型而言,更多的数据会得到更好的泛化能力,直到达到最佳可能的泛化误差。
任何模型容量小于最优容量的固定参数模型会渐近到大于\gls{bayes_error}的误差值。
如\figref{fig:chap5_training_size_grows}所示。
值得注意的是,具有最优容量的模型仍然有可能在\gls{training_error}和\gls{generalization_error}之间存在很大的差距。
在这种情况下,我们可以通过收集更多的训练\gls{example:chap5}来缩小差距。

\begin{figure}[!htb]
\ifOpenSource
\centerline{\includegraphics{figure.pdf}}
\else
\centerline{\includegraphics{Chapter5/figures/training_size_grows}}
\fi
\caption{训练集大小对\gls{training_error},\gls{test_error}以及\gls{optimal_capacity}的影响。通过给一个$5$阶多项式添加适当大小的噪声,我们构造了一个合成的\gls{regression}问题,生成单个测试集,然后生成一些不同尺寸的训练集。为了描述$95\%$置信区间的\gls{error_bar},对于每一个尺寸,我们生成了$40$个不同的训练集。\emph{(上)}两个不同的模型上训练集和测试集的\,\glssymbol{mean_squared_error},一个二次模型,另一个模型的阶数通过最小化\gls{test_error}来选择。两个模型都是用闭式解来拟合。对于二次模型来说,当训练集增加时\gls{training_error}也随之增大。这是由于越大的数据集越难以拟合。同时,\gls{test_error}随之减小,因为关于训练数据的不正确的假设越来越少。二次模型的容量并不足以解决这个问题,所以它的\gls{test_error}趋近于一个较高的值。\gls{optimal_capacity}点处的\gls{test_error}趋近于\gls{bayes_error}。\gls{training_error}可以低于\gls{bayes_error},因为训练算法有能力记住训练集中特定的样本。当训练集趋向于无穷大时,任何固定容量的模型(在这里指的是二次模型)的\gls{training_error}都至少增至\gls{bayes_error}。\emph{(下)}当训练集大小增大时,\gls{optimal_capacity}(在这里是用最优多项式回归器的阶数衡量的)也会随之增大。\gls{optimal_capacity}在达到足够捕捉模型复杂度之后就不再增长了。}
\label{fig:chap5_training_size_grows}
\end{figure}

\subsection{\glsentrytext{no_free_lunch_theorem}}
\label{sec:the_no_free_lunch_theorem}
学习理论表明\gls{ML}算法能够在有限个训练集\gls{example:chap5}中很好地泛化。
这似乎违背一些基本的逻辑原则。
归纳推理,或是从一组有限的\gls{example:chap5}中推断一般的规则,在逻辑上不是很有效。
为了逻辑地推断一个规则去描述集合中的元素,我们必须具有集合中每个元素的信息。

在一定程度上,\gls{ML}仅通过概率法则就可以避免这个问题,而无需使用纯逻辑推理整个确定性法则。
\gls{ML}保证找到一个在所关注的\emph{大多数}\gls{example:chap5}上\emph{可能}正确的规则。

可惜,即使这样也不能解决整个问题。
\gls{ML}的\firstgls{no_free_lunch_theorem}表明,在所有可能的数据生成分布上平均之后,每一个分类算法在未事先观测的点上都有相同的错误率。
换言之,在某种意义上,没有一个\gls{ML}算法总是比其他的要好。
我们能够设想的最先进的算法和简单地将所有点归为同一类的简单算法有着相同的平均性能(在所有可能的任务上)。

% -- 113 --

幸运的是,这些结论仅在我们考虑\emph{所有}可能的数据生成分布时才成立。
在真实世界应用中,如果我们对遇到的概率分布进行假设的话,那么我们可以设计在这些分布上效果良好的学习算法。

这意味着\gls{ML}研究的\gls{target}不是找一个通用学习算法或是绝对最好的学习算法。
反之,我们的\gls{target}是理解什么样的分布与人工智能获取经验的``真实世界''相关,什么样的学习算法在我们关注的数据生成分布上效果最好。

\subsection{\glsentrytext{regularization}}
\label{sec:regularization}
没有免费午餐定理暗示我们必须在特定任务上设计性能良好的\gls{ML}算法。
我们建立一组学习算法的偏好来达到这个要求。
当这些偏好和我们希望算法解决的学习问题相吻合时,性能会更好。

至此,我们具体讨论修改学习算法的方法只有,通过增加或减少学习算法可选假设空间的函数来增加或减少模型的容量。
我们列举的一个具体示例是\gls{linear_regression}增加或减少多项式的次数。
目前为止讨论的观点都是过度简化的。

算法的效果不仅很大程度上受影响于假设空间的函数数量,也取决于这些函数的具体形式。
我们已经讨论的学习算法(\gls{linear_regression})具有包含其输入的线性函数集的假设空间。
对于输入和输出确实接近线性相关的问题,这些线性函数是很有用的。
对于完全非线性的问题它们不太有效。
例如,我们用\gls{linear_regression},从$x$预测$\sin(x)$,效果不会好。
因此我们可以通过两种方式控制算法的性能,一是允许使用的函数种类,二是这些函数的数量。

在假设空间中,相比于某一个学习算法,我们可能更偏好另一个学习算法。
这意味着两个函数都是符合条件的,但是我们更偏好其中一个。
只有非偏好函数比偏好函数在训练\gls{dataset}上效果明显好很多时,我们才会考虑非偏好函数。

% -- 115 --

例如,我们可以加入\firstgls{weight_decay}来修改\gls{linear_regression}的训练标准。
带权重衰减的\gls{linear_regression}最小化训练集上的\gls{mean_squared_error}和正则项的和$J(\Vw)$,其偏好于平方$L^2$范数较小的权重。
具体如下:
\begin{equation}
    J(\Vw) = \text{MSE}_{\text{train}} + \lambda \Vw^\Tsp \Vw,
\end{equation}
其中$\lambda$是提前挑选的值,控制我们偏好小范数权重的程度。
当$\lambda =0$,我们没有任何偏好。
越大的$\lambda$偏好范数越小的权重。
最小化$J(\Vw)$可以看作是拟合训练数据和偏好小权重范数之间的权衡。
这会使得解决方案的斜率较小,或是将权重放在较少的\gls{feature}上。
我们可以训练具有不同$\lambda$值的高次多项式回归模型,来举例说明如何通过权重衰减控制模型欠拟合或过拟合的趋势。
如\figref{fig:chap5_underfit_just_right_overfit_wd_color}所示。

\begin{figure}[!htb]
\ifOpenSource
\centerline{\includegraphics{figure.pdf}}
\else
\centerline{\includegraphics{Chapter5/figures/underfit_just_right_overfit_wd_color}}
\fi
\caption{我们使用高阶多项式回归模型来拟合\figref{fig:chap5_underfit_just_right_overfit}中训练样本。真实函数是二次的,但是在这里我们只使用$9$阶多项式。我们通过改变\gls{weight_decay}的量来避免高阶模型的过拟合问题。\emph{(左)}当$\lambda$非常大时,我们可以强迫模型学习到了一个没有斜率的函数。由于它只能表示一个常数函数,所以会导致\gls{underfitting}。\emph{(中)}取一个适当的$\lambda$时,学习算法能够用一个正常的形状来恢复曲率。即使模型能够用更复杂的形状来来表示函数,\gls{weight_decay}鼓励用一个带有更小参数的更简单的模型来描述它。\emph{(右)}当\gls{weight_decay}趋近于$0$(即使用\,\gls{Moore}来解这个带有最小\gls{regularization}的欠定问题)时,这个$9$阶多项式会导致严重的\gls{overfitting},这和我们在图\ref{fig:chap5_underfit_just_right_overfit}中看到的一样。}
\label{fig:chap5_underfit_just_right_overfit_wd_color}
\end{figure}

更一般地,\gls{regularization}一个学习函数$f(\Vx;\Vtheta)$的模型,我们可以给\gls{cost_function}添加被称为\firstgls{regularizer}的惩罚。
在权重衰减的例子中,\gls{regularizer}是$\Omega(\Vw) = \Vw^\Tsp \Vw$。
在\chapref{chap:regularization_for_deep_learning},我们将看到很多其他可能的\gls{regularizer}。

% -- 116 --

表示对函数的偏好是比增减假设空间的成员函数更一般的控制模型\gls{capacity}的方法。
我们可以将去掉假设空间中的某个函数看作是对不赞成这个函数的无限偏好。

在我们权重衰减的示例中,通过在最小化的\gls{target}中额外增加一项,我们明确地表示了偏好权重较小的线性函数。
有很多其他方法隐式或显式地表示对不同解的偏好。
总而言之,这些不同的方法都被称为\firstgls{regularization}。
\emph{\gls{regularization}是指我们修改学习算法,使其降低泛化误差而非训练误差}。
\gls{regularization}是\gls{ML}领域的中心问题之一,只有优化能够与其重要性相媲。

\gls{no_free_lunch_theorem}已经清楚地阐述了没有最优的学习算法,特别地,没有最优的\gls{regularization}形式。
反之,我们必须挑选一个非常适合于我们所要解决的任务的正则形式。
\gls{DL}中普遍的(特别是本书中的)理念是大量任务(例如所有人类能做的智能任务)也许都可以使用非常通用的\gls{regularization}形式来有效解决。

\section{超参数和验证集}
\label{sec:hyperparameters_and_validation_sets}
大多数\gls{ML}算法都有超参数,可以设置来控制算法行为。
超参数的值不是通过学习算法本身学习出来的(尽管我们可以设计一个嵌套的学习过程,一个学习算法为另一个学习算法学出最优超参数)。

在\figref{fig:chap5_underfit_just_right_overfit}所示的多项式回归示例中,有一个超参数:多项式的次数,作为\textbf{\gls{capacity}}超参数。
控制\gls{weight_decay}程度的$\lambda$是另一个超参数。

有时一个选项被设为学习算法不用学习的超参数,是因为它太难优化了。
更多的情况是,该选项必须是超参数,因为它不适合在训练集上学习。
这适用于控制模型\gls{capacity}的所有超参数。
如果在训练集上学习超参数,这些超参数总是趋向于最大可能的模型容量,导致过拟合(参考\figref{fig:chap5_generalization_vs_capacity})。
例如,相比低次多项式和正的权重衰减设定,更高次的多项式和权重衰减参数设定$\lambda=0$总能在训练集上更好地拟合。

% -- 117 --

为了解决这个问题,我们需要一个训练算法观测不到的\firstgls{validation_set}\gls{example:chap5}。

早先我们讨论过和训练数据相同分布的\gls{example:chap5}组成的\gls{test_set},它可以用来估计学习过程完成之后的\gls{learner}的泛化误差。
其重点在于测试\gls{example:chap5}不能以任何形式参与到模型的选择中,包括设定超参数。
基于这个原因,\gls{test_set}中的\gls{example:chap5}不能用于验证集。
因此,我们总是从\emph{训练}数据中构建验证集。
特别地,我们将训练数据分成两个不相交的子集。
其中一个用于学习参数。
另一个作为验证集,用于估计训练中或训练后的泛化误差,更新超参数。
用于学习参数的数据子集通常仍被称为训练集,尽管这会和整个训练过程用到的更大的\gls{dataset}相混。
用于挑选超参数的数据子集被称为\firstgls{validation_set}。
通常,$80\%$的训练数据用于训练,$20\%$用于验证。
由于验证集是用来``训练''超参数的,尽管验证集的误差通常会比训练集误差小,验证集会低估泛化误差。
所有超参数优化完成之后,泛化误差可能会通过\gls{test_set}来估计。

在实际中,当相同的\gls{test_set}已在很多年中重复地用于评估不同算法的性能,并且考虑学术界在该\gls{test_set}上的各种尝试,我们最后可能也会对\gls{test_set}有着乐观的估计。
\gls{benchmarks}会因之变得陈旧,而不能反映系统的真实性能。
值得庆幸的是,学术界往往会移到新的(通常会更巨大、更具挑战性)基准\gls{dataset}上。

\subsection{交叉验证}
\label{sec:cross_validation}
将\gls{dataset}分成固定的训练集和固定的\gls{test_set}后,若\gls{test_set}的误差很小,这将是有问题的。
一个小规模的\gls{test_set}意味着平均测试误差估计的统计不确定性,使得很难判断算法$A$是否比算法$B$在给定的任务上做得更好。

% -- 118 --

当\gls{dataset}有十万计或者更多的\gls{example:chap5}时,这不会是一个严重的问题。
当\gls{dataset}太小时,也有替代方法允许我们使用所有的\gls{example:chap5}估计平均测试误差,代价是增加了计算量。
这些过程是基于在原始数据上随机采样或分离出的不同\gls{dataset}上重复训练和测试的想法。
最常见的是$k$-折交叉验证过程,如\algref{alg:xv}所示,将\gls{dataset}分成$k$个不重合的子集。
测试误差可以估计为$k$次计算后的平均测试误差。
在第$i$次测试时,数据的第$i$个子集用于\gls{test_set},其他的数据用于训练集。
带来的一个问题是不存在平均误差方差的无偏估计\citep{Bengio-Grandvalet-JMLR-04},但是我们通常会使用近似来解决。

\begin{algorithm}
  \caption{$k$-折交叉验证算法。
当给定数据集$\SetD$对于简单的训练/测试或训练/验证分割而言太小难以产生泛化误差的准确估计时(因为在小的测试集上,$L$可能具有过高的方差),$k$-折交叉验证算法可以用于估计学习算法$A$的泛化误差。
数据集$\SetD$包含的元素是抽象的样本 $\Vz^{(i)}$(对于第$i$个样本),在\gls{supervised_learning}的情况代表(输入,目标)对$\Vz^{(i)} = (\Vx^{(i)}, y^{(i)})$ ,或者\gls{unsupervised_learning}的情况下仅用于输入$\Vz^{(i)} = \Vx^{(i)}$。
该算法返回$\SetD$中每个示例的误差向量$\Ve$,其均值是估计的泛化误差。
单个样本上的误差可用于计算平均值周围的置信区间(\eqnref{eq:confidence_interval})。
虽然这些置信区间在使用交叉验证之后不能很好地证明,但是通常的做法是只有当算法$A$误差的置信区间低于并且不与算法$B$的置信区间相交时,我们才声明算法$A$比算法$B$更好。}
\label{alg:xv}
\begin{algorithmic}
\item[] \hspace*{-4.2mm}{\bf Define} {\tt KFoldXV}($\SetD,A,L,k$):
\REQUIRE $\SetD$为给定数据集,其中元素为 $\Vz^{(i)}$
\REQUIRE $A$ 为学习算法,可视为一个函数(使用数据集作为输入,输出一个学好的函数)
\REQUIRE $L$ 为\gls{loss_function},可视为来自学好的函数$f$,将样本 $\Vz^{(i)} \in \SetD$ 映射到$\SetR$中标量的函数
\REQUIRE $k$为折数
\STATE 将 $\SetD$ 分为 $k$个互斥子集 $\SetD_i$,它们的并集为$\SetD$
\FOR{$i$ from $1$ to $k$}
  \STATE $f_i = A(\SetD \backslash \SetD_i)$ 
  \FOR{$\Vz^{(j)}$ in $\SetD_i$}
    \STATE $e_j = L(f_i, \Vz^{(j)})$
  \ENDFOR
\ENDFOR
\STATE {\bf Return} $\Ve$
\end{algorithmic}
\end{algorithm}

\section{估计、偏差和方差}
\label{sec:estimators_bias_and_variance}
统计领域为我们提供了很多工具来实现\gls{ML}目标,不仅可以解决训练集上的任务,还可以泛化。
基本的概念,例如参数估计、偏差和方差,对于正式地刻画泛化、欠拟合和过拟合都非常有帮助。

\subsection{点估计}
\label{sec:point_estimation}
点估计试图为一些感兴趣的量提供单个``最优''预测。
一般地,感兴趣的量可以是单个参数,或是某些参数模型中的一个向量参数,例如\secref{sec:example_linear_regression}\gls{linear_regression}中的权重,但是也有可能是整个函数。

为了区分参数估计和真实值,我们习惯将参数$\Vtheta$的点估计表示为$\hat{\Vtheta}$。

令$\{\Vx^{(1)},\dots,\Vx^{(m)}\}$是$m$个独立同分布(i.i.d.)的数据点。
\firstgls{point_estimator}或\firstgls{statistics}是这些数据的任意函数:
\begin{equation}
    \hat{\Vtheta}_m = g(\Vx^{(1)}, \dots, \Vx^{(m)}) .
\end{equation}
这个定义不要求$g$返回一个接近真实$\Vtheta$的值,或者$g$的值域恰好是$\Vtheta$的允许取值范围。
点估计的定义非常宽泛,给了\gls{estimator:chap5}的设计者极大的灵活性。
虽然几乎所有的函数都可以称为\gls{estimator:chap5},但是一个良好的\gls{estimator:chap5}的输出会接近生成训练数据的真实参数$\Vtheta$。

% -- 119 --

现在,我们采取频率派在统计上的观点。
换言之,我们假设真实参数$\Vtheta$是固定但未知的,而点估计$\hat{\Vtheta}$是数据的函数。
由于数据是随机过程采样出来的,数据的任何函数都是随机的。
因此$\hat{\Vtheta}$是一个随机变量。

% -- 120 --

点估计也可以指输入和\gls{target}变量之间关系的估计。
我们将这种类型的点估计称为函数估计。

\paragraph{函数估计} 有时我们会关注函数估计(或函数近似)。
这时我们试图从输入向量$\Vx$预测变量$\Vy$。
我们假设有一个函数$f(\Vx)$表示$\Vy$和$\Vx$之间的近似关系。
例如,我们可能假设$\Vy = f(\Vx) + \Vepsilon$,其中$\Vepsilon$是$\Vy$中未能从$\Vx$预测的一部分。
在函数估计中,我们感兴趣的是用模型估计去近似$f$,或者估计$\hat{f}$。
函数估计和估计参数$\Vtheta$是一样的;函数估计$\hat{f}$是函数空间中的一个点估计。
\gls{linear_regression}示例(\secref{sec:example_linear_regression}中讨论的)和多项式回归示例(\secref{sec:capacity_overfitting_and_underfitting}中讨论的)都既可以被解释为估计参数$\Vw$,又可以被解释为估计从$\Vx$到$y$的函数映射$\hat{f}$。

现在我们回顾点估计最常研究的性质,并探讨这些性质说明了估计的哪些特点。

\subsection{偏差}
\label{sec:bias}
\gls{estimator}的偏差被定义为:
\begin{equation}
\label{eq:5.20}
    \text{bias}(\hat{\Vtheta}_m) = \SetE(\hat{\Vtheta}_m) - \Vtheta,
\end{equation}
其中期望作用在所有数据(看作是从随机变量采样得到的)上,$\Vtheta$是用于定义数据生成分布的$\Vtheta$的真实值。
如果$\text{bias}(\hat{\Vtheta}_m)=0$,那么\gls{estimator:chap5} $\hat{\Vtheta}_m$被称为是\firstgls{unbiased},这意味着$\SetE(\hat{\Vtheta}_m) = \Vtheta$。
如果$\lim_{m\to\infty} \text{bias}(\hat{\Vtheta}_m)=0$,那么\gls{estimator:chap5} $\hat{\Vtheta}_m$被称为是\firstgls{asymptotically_unbiased},这意味着$\lim_{m\to\infty} \SetE(\hat{\Vtheta}_m) = \Vtheta$。

% -- 121 --

\paragraph{示例:伯努利分布}
考虑一组服从均值为$\theta$的伯努利分布的独立同分布的样本$\{x^{(1)}, \dots , x^{(m)}\}$:
\begin{equation}
    P(x^{(i)}; \theta) = \theta^{x^{(i)}} (1-\theta)^{(1 - x^{(i)})}.
\end{equation}
这个分布中参数$\theta$的常用\gls{estimator:chap5}是训练\gls{example:chap5}的均值:
\begin{equation}
\label{eq:5.22}
    \hat{\theta}_m = \frac{1}{m} \sum_{i=1}^m x^{(i)}.
\end{equation}
判断这个\gls{estimator:chap5}是否有偏,我们将\eqnref{eq:5.22}代入\eqnref{eq:5.20}:
\begin{align}
    \text{bias}(\hat{\theta}_m)     &= \SetE[\hat{\theta}_m] - \theta  \\
            &= \SetE \left[ \frac{1}{m} \sum_{i=1}^m x^{(i)} \right] - \theta \\
            &= \frac{1}{m} \sum_{i=1}^m \SetE \left[x^{(i)} \right] - \theta \\
            &= \frac{1}{m} \sum_{i=1}^m \sum_{x^{(i)} = 0}^1 \left( x^{(i)} \theta^{x^{(i)}} (1-\theta)^{(1-x^{(i)})} \right) - \theta \\
            &= \frac{1}{m} \sum_{i=1}^m (\theta) - \theta \\
            &= \theta - \theta = 0
\end{align}

因为$\text{bias}(\hat{\theta})=0$,我们称估计$\hat{\theta}$是无偏的。

\paragraph{示例:均值的高斯分布估计}
现在,考虑一组独立同分布的\gls{example:chap5} $\{x^{(1)}, \dots , x^{(m)}\}$服从高斯分布$p(x^{(i)}) = \mathcal{N}(x^{(i)}; \mu, \sigma^2)$,其中$i\in\{1, \dots, m\}$。
回顾高斯概率密度函数如下:
\begin{equation}
    p(x^{(i)}; \mu, \sigma^2) = \frac{1}{\sqrt{2\pi\sigma^2}} \exp\left( -\frac{1}{2} \frac{(x^{(i)} - \mu)^2}{\sigma^2}  \right).
\end{equation}

高斯均值参数的常用\gls{estimator:chap5}被称为\firstgls{sample_mean}:
\begin{equation}
    \hat{\mu}_m = \frac{1}{m} \sum_{i=1}^m x^{(i)}
\end{equation}
判断\gls{sample_mean}是否有偏,我们再次计算它的期望:
\begin{align}
\text{bias} (\hat{\mu}_m) &= \SetE[ \hat{\mu}_m ]  - \mu \\
    &= \SetE \left[ \frac{1}{m} \sum_{i=1}^m x^{(i)}  \right] - \mu \\
    &= \left( \frac{1}{m}\sum_{i=1}^m \SetE \left[ x^{(i)} \right] \right) - \mu \\
    &= \left( \frac{1}{m}\sum_{i=1}^m \mu \right) - \mu \\
    &= \mu - \mu = 0
\end{align}
因此我们发现\gls{sample_mean}是高斯均值参数的无偏\gls{estimator:chap5}。

% -- 122 --

\paragraph{示例:高斯分布方差估计}
本例中,我们比较高斯分布方差参数$\sigma^2$的两个不同估计。
我们探讨是否有一个是有偏的。

我们考虑的第一个方差估计被称为\firstgls{sample_variance}:
\begin{equation}
    \hat{\sigma}_m^2 = \frac{1}{m} \sum_{i=1}^m \left( x^{(i)} - \hat{\mu}_m \right)^2,
\end{equation}
其中$\hat{\mu}_m$是\gls{sample_mean}。
更形式地,我们对计算感兴趣
\begin{equation}
\label{eq:5.37}
    \text{bias} (\hat{\sigma}_m^2) = \SetE [ \hat{\sigma}_m^2 ]  - \sigma^2.
\end{equation}
我们首先估计项$\SetE [ \hat{\sigma}_m^2 ]$:
\begin{align}
    \SetE [ \hat{\sigma}_m^2 ]  &= \SetE \left[ \frac{1}{m} \sum_{i=1}^m \left( x^{(i)} - \hat{\mu}_m \right)^2  \right] \\
    &= \frac{m-1}{m} \sigma^2
\end{align}
回到\eqnref{eq:5.37},我们可以得出$\hat{\sigma}^2_m$的偏差是$-\sigma^2/m$。
因此\gls{sample_variance}是有偏估计。

\firstgls{unbiased_sample_variance}估计
\begin{equation}
    \tilde{\sigma}_m^2 = \frac{1}{m-1} \sum_{i=1}^m \left( x^{(i)} - \hat{\mu}_m \right)^2
\end{equation}
提供了另一种可选方法。
正如名字所言,这个估计是无偏的。
换言之,我们会发现$\SetE[\tilde{\sigma}_m^2] = \sigma^2$:
\begin{align}
    \SetE[\tilde{\sigma}_m^2] &= \SetE \left[ \frac{1}{m-1} \sum_{i=1}^m \left( x^{(i)} - \hat{\mu}_m \right)^2 \right] \\
        &= \frac{m}{m-1} \SetE[ \hat{\sigma}_m^2 ]  \\
        &= \frac{m}{m-1} \left( \frac{m-1}{m} \sigma^2 \right) \\
        &= \sigma^2.
\end{align}

% -- 123 --

我们有两个\gls{estimator:chap5}:一个是有偏的,另一个是无偏的。
尽管无偏估计显然是令人满意的,但它并不总是``最好''的估计。
我们将看到,经常会使用其他具有重要性质的有偏估计。

\subsection{方差和\glsentrytext{standard_error}}
\label{sec:variance_and_standard_error}
我们有时会考虑\gls{estimator:chap5}的另一个性质是它作为数据样本的函数,期望的变化程度是多少。
正如我们可以计算\gls{estimator:chap5}的期望来决定它的偏差,我们也可以计算它的方差。
\gls{estimator:chap5}的\firstgls{variance}就是一个\gls{variance}
\begin{equation}
    \text{Var}(\hat{\theta})
\end{equation}
其中随机变量是训练集。
另外,方差的平方根被称为\firstgls{standard_error},记作$\text{SE}(\hat{\theta})$。

\gls{estimator:chap5}的方差或\gls{standard_error}告诉我们,当独立地从潜在的数据生成过程中重采样数据集时,如何期望估计的变化。
正如我们希望估计的偏差较小,我们也希望其方差较小。

当我们使用有限的样本计算任何统计量时,真实参数的估计都是不确定的,在这个意义下,从相同的分布得到其他样本时,它们的统计量也会不一样。
任何方差估计量的期望程度是我们想量化的误差的来源。

均值的\gls{standard_error}被记作
\begin{equation}
\label{eq:5.46}
    \text{SE}(\hat{\mu}_m) = \sqrt{ \text{Var} \left[ \frac{1}{m} \sum_{i=1}^m x^{(i)} \right] } = \frac{\sigma}{\sqrt{m}},
\end{equation}
其中$\sigma^2$是\gls{example:chap5} $x^{(i)}$的真实方差。
\gls{standard_error}通常被记作$\sigma$。
可惜,样本方差的平方根和方差无偏估计的平方根都不是\gls{standard_deviation}的无偏估计。
这两种计算方法都倾向于低估真实的\gls{standard_deviation},但仍用于实际中。
相较而言,方差无偏估计的平方根较少被低估。
对于较大的$m$,这种近似非常合理。

% -- 124 --

均值的\gls{standard_error}在\gls{ML}实验中非常有用。
我们通常用\gls{test_set}样本的误差均值来估计泛化误差。
\gls{test_set}中\gls{example:chap5}的数量决定了这个估计的精确度。
中心极限定理告诉我们均值会接近一个高斯分布,我们可以用\gls{standard_error}计算出真实期望落在选定区间的概率。
例如,以均值$\hat{\mu}_m$为中心的$95\%$置信区间是
\begin{equation}
\label{eq:confidence_interval}
    ( \hat{\mu}_m - 1.96\text{SE}(\hat{\mu}_m), \hat{\mu}_m + 1.96 \text{SE}(\hat{\mu}_m) ),
\end{equation}
以上区间是基于均值$\hat{\mu}_m$和方差$\text{SE}(\hat{\mu}_m)^2$的高斯分布。
在\gls{ML}实验中,我们通常说算法$A$比算法$B$好,是指算法$A$的误差的$95\%$置信区间的上界小于算法$B$的误差的$95\%$置信区间的下界。

\paragraph{示例:伯努利分布} 我们再次考虑从伯努利分布(回顾$P(x^{(i)}; \theta) = \theta^{x^{(i)}} (1-\theta)^{1 - x^{(i)}}$)中独立同分布采样出来的一组\gls{example:chap5} $\{ x^{(1)}, \dots, x^{(m)} \}$。
这次我们关注估计$\hat{\theta}_m = \frac{1}{m} \sum_{i=1}^m x^{(i)}$的方差:
\begin{align}
    \text{Var}\left( \hat{\theta}_m \right) &= \text{Var}\left( \frac{1}{m} \sum_{i=1}^m x^{(i)} \right) \\
    &= \frac{1}{m^2} \sum_{i=1}^m \text{Var} \left( x^{(i)} \right) \\
    &= \frac{1}{m^2} \sum_{i=1}^m \theta (1 - \theta) \\
    &= \frac{1}{m^2} m\theta(1-\theta) \\
    &= \frac{1}{m} \theta(1-\theta)
\end{align} 
\gls{estimator:chap5}方差的下降速率是关于\gls{dataset}\gls{example:chap5}数目$m$的函数。
这是常见估计量的普遍性质,在探讨一致性(参考\secref{sec:consistency})时,我们会继续讨论。

% -- 125 --

\subsection{权衡偏差和方差以最小化均方误差}
\label{sec:trading_off_bias_and_variance_to_minimize_mean_squared_error}
偏差和方差度量着估计量的两个不同误差来源。
偏差度量着偏离真实函数或参数的误差期望。
而方差度量着数据上任意特定采样可能导致的估计期望的偏差。

当我们可以在一个偏差更大的估计和一个方差更大的估计中进行选择时,会发生什么呢?我们该如何选择?
例如,想象我们希望近似\figref{fig:chap5_underfit_just_right_overfit}中的函数,我们只可以选择一个偏差较大的估计或一个方差较大的估计,我们该如何选择呢?

判断这种权衡最常用的方法是交叉验证。
经验上,交叉验证在真实世界的许多任务中都非常成功。
另外,我们也可以比较这些估计的\firstall{mean_squared_error}:
\begin{align}
    \text{MSE} &= \SetE[ ( \hat{\theta}_m - \theta  )^2 ] \\
        &= \text{Bias}( \hat{\theta}_m )^2 + \text{Var}( \hat{\theta}_m ) \label{eq:5.54}
\end{align}
\glssymbol{mean_squared_error}\,度量着估计和真实参数$\theta$之间平方误差的总体期望偏差。
如\eqnref{eq:5.54}所示,\glssymbol{mean_squared_error}\,估计包含了偏差和方差。
理想的估计具有较小的\,\glssymbol{mean_squared_error}\,或是在检查中会稍微约束它们的偏差和方差。

偏差和方差的关系和\gls{ML}容量、欠拟合和过拟合的概念紧密相联。
用\,\glssymbol{mean_squared_error}\,度量泛化误差(偏差和方差对于泛化误差都是有意义的)时,增加容量会增加方差,降低偏差。
如\figref{fig:chap5_bias_variance_tradeoff}所示,我们再次在关于容量的函数中,看到泛化误差的U形曲线。

\begin{figure}[!htb]
\ifOpenSource
\centerline{\includegraphics{figure.pdf}}
\else
\centerline{\includegraphics{Chapter5/figures/bias_variance_tradeoff}}
\fi
\caption{当\gls{capacity}增大($x$轴)时,\gls{BIAS}(用点表示)随之减小,而方差(虚线)随之增大,使得\gls{generalization_error}(加粗曲线)产生了另一种U形。如果我们沿着轴改变\gls{capacity},会发现\gls{optimal_capacity},当\gls{capacity}小于\gls{optimal_capacity}会呈现\gls{underfitting},大于时导致\gls{overfitting}。这种关系与\secref{sec:capacity_overfitting_and_underfitting}以及\figref{fig:chap5_generalization_vs_capacity}中讨论的\gls{capacity}、\gls{underfitting}和\gls{overfitting}之间的关系类似。}
\label{fig:chap5_bias_variance_tradeoff}
\end{figure}

% -- 126 --

\subsection{一致性}
\label{sec:consistency}
目前我们已经探讨了固定大小训练集下不同\gls{estimator:chap5}的性质。
通常,我们也会关注训练数据增多后\gls{estimator:chap5}的效果。
特别地,我们希望当\gls{dataset}中数据点的数量$m$增加时,点估计会收敛到对应参数的真实值。
更形式地,我们想要
\begin{equation}
\label{eq:5.55}
    \plim_{m\to\infty} \hat{\theta}_m = \theta.
\end{equation}
符号$\plim$表示依概率收敛,即对于任意的$\epsilon > 0$,当$m\to\infty$时,有$P(|\hat{\theta}_m - \theta| > \epsilon) \to 0$。
\eqnref{eq:5.55}表示的条件被称为\firstgls{consistency}。
有时它是指弱一致性,强一致性是指\firstgls{almost_sure}从$\hat{\theta}$收敛到$\theta$。
\firstgls{almost_sure_convergence}是指当$p(\lim_{m\to\infty} \RVx^{(m)} = \Vx) = 1 $时,随机变量序列$\RVx^{(1)}$,$\RVx^{(2)}$,$\dots$收敛到$\Vx$。

一致性保证了\gls{estimator:chap5}的偏差会随数据\gls{example:chap5}数目的增多而减少。
然而,反过来是不正确的——渐近无偏并不意味着一致性。
例如,考虑用包含$m$个样本的\gls{dataset} $\{x^{(1)},\dots,x^{(m)}\}$估计正态分布$\mathcal{N}(x;\mu,\sigma^2)$的均值参数$\mu$。
我们可以使用\gls{dataset}的第一个\gls{example:chap5} $x^{(1)}$作为无偏估计量:$\hat{\theta} = x^{(1)}$。
在该情况下,$\SetE(\hat{\theta}_m) = \theta$,所以不管观测到多少数据点,该\gls{estimator:chap5}都是无偏的。
然而,这不是一个一致估计,因为它\emph{不}满足当$m\to\infty$时,$\hat{\theta}_m \to \theta$。

% -- 127 --

\section{\glsentrytext{maximum_likelihood_estimation}}
\label{sec:maximum_likelihood_estimation}
之前,我们已经看过常用估计的定义,并分析了它们的性质。
但是这些估计是从哪里来的呢?
我们希望有些准则可以让我们从不同模型中得到特定函数作为好的估计,而不是猜测某些函数可能是好的估计,然后分析其偏差和方差。

最常用的准则是\gls{maximum_likelihood_estimation}。

考虑一组含有$m$个\gls{example:chap5}的\gls{dataset} $\SetX=\{ \Vx^{(1)},\dots, \Vx^{(m)} \}$,独立地由未知的真实数据生成分布$p_{\text{data}}(\RVx)$生成。

令$p_{\text{model}}( \RVx; \Vtheta )$是一族由$\Vtheta$确定在相同空间上的概率分布。
换言之,$p_{\text{model}}(\Vx;\Vtheta)$将任意输入$\Vx$映射到实数来估计真实概率$p_{\text{data}}(\Vx)$。

对$\Vtheta$的最大似然估计被定义为:
\begin{align}
    \Vtheta_{\text{ML}} &= \underset{\Vtheta}{\argmax} \, p_{\text{model}} (\SetX; \Vtheta), \\
        &= \underset{\Vtheta}{\argmax} \prod_{i=1}^m p_{\text{model}} (\Vx^{(i)}; \Vtheta).
\end{align}

多个概率的乘积会因很多原因不便于计算。
例如,计算中很可能会出现数值下溢。
为了得到一个便于计算的等价优化问题,我们观察到似然对数不会改变其$\argmax$但是将乘积转化成了便于计算的求和形式:
\begin{equation}
    \Vtheta_{\text{ML}} = \underset{\Vtheta}{\argmax} \sum_{i=1}^m \log p_{\text{model}} (\Vx^{(i)}; \Vtheta) .
\end{equation}
因为当我们重新缩放\gls{cost_function}时$\argmax$不会改变,我们可以除以$m$得到和训练数据经验分布$\hat{p}_{\text{data}}$相关的期望作为准则:
\begin{equation}
\label{eq:5.59}
    \Vtheta_{\text{ML}} = \underset{\Vtheta}{\argmax} \,\SetE_{\RVx \sim \hat{p}_{\text{data}}} \log p_{\text{model}} (\Vx; \Vtheta) .
\end{equation}

一种解释\gls{maximum_likelihood_estimation}的观点是将它看作最小化训练集上的经验分布$\hat{p}_{\text{data}}$和模型分布之间的差异,两者之间的差异程度可以通过~\gls{KL}度量。
\gls{KL}被定义为
\begin{equation}
    D_{\text{KL}}(\hat{p}_{\text{data}} \| p_{\text{model}}) = \SetE_{\RVx \sim \hat{p}_{\text{data}}} [ \log \hat{p}_{\text{data}}(\Vx) - \log p_{\text{model}}(\Vx) ] .
\end{equation}
左边一项仅涉及到数据生成过程,和模型无关。
这意味着当我们训练模型最小化~\gls{KL}时,我们只需要最小化
\begin{equation}
    -\SetE_{\RVx \sim \hat{p}_{\text{data}}} [ \log p_{\text{model}}(\Vx)  ] ,
\end{equation}
当然,这和\eqnref{eq:5.59}中最大化是相同的。

% -- 128 --

最小化~\gls{KL}其实就是在最小化分布之间的\gls{cross_entropy}。
许多作者使用术语``\gls{cross_entropy}''特定表示伯努利或softmax分布的负对数似然,但那是用词不当的。
任何一个由负对数似然组成的\gls{loss}都是定义在训练集上的经验分布和定义在模型上的概率分布之间的\gls{cross_entropy}。
例如,\gls{mean_squared_error}是经验分布和高斯模型之间的\gls{cross_entropy}。

我们可以将最大似然看作是使模型分布尽可能地和经验分布$\hat{p}_{\text{data}}$相匹配的尝试。
理想情况下,我们希望匹配真实的数据生成分布$p_{\text{data}}$,但我们没法直接知道这个分布。

虽然最优$\Vtheta$在最大化似然或是最小化~\gls{KL}时是相同的,但目标函数值是不一样的。
在软件中,我们通常将两者都称为最小化\gls{cost_function}。
因此最大化似然变成了最小化负对数似然(NLL),或者等价的是最小化交叉熵。
将最大化似然看作最小化~\gls{KL}的视角在这个情况下是有帮助的,因为已知~\gls{KL}最小值是零。
当$\Vx$取实数时,负对数似然是负值。

\subsection{条件对数似然和\glsentrytext{mean_squared_error}}
\label{sec:conditional_log_likelihood_and_mean_squared_error}
\gls{maximum_likelihood_estimation}很容易扩展到估计条件概率$P(\RVy \mid \RVx;\Vtheta)$,从而给定$\RVx$预测$\RVy$。
实际上这是最常见的情况,因为这构成了大多数\gls{supervised_learning}的基础。
如果$\MX$表示所有的输入,$\MY$表示我们观测到的\gls{target},那么条件\gls{maximum_likelihood_estimation}是
\begin{equation}
    \Vtheta_{\text{ML}} = \underset{\Vtheta}{\argmax} P(\MY \mid \MX; \Vtheta).
\end{equation}
如果假设\gls{example:chap5}是独立同分布的,那么这可以分解成
\begin{equation}
\label{eq:5.63}
    \Vtheta_{\text{ML}} = \underset{\Vtheta}{\argmax} \sum_{i=1}^m \log P(\Vy^{(i)} \mid \Vx^{(i)}; \Vtheta).
\end{equation}

% -- 129 --

\paragraph{示例:\gls{linear_regression}作为最大似然} \secref{sec:example_linear_regression}介绍的\gls{linear_regression},可以被看作是最大似然过程。
之前,我们将\gls{linear_regression}作为学习从输入$\Vx$映射到输出$\hat{y}$的算法。
从$\Vx$到$\hat{y}$的映射选自最小化\gls{mean_squared_error}(我们或多或少介绍的一个标准)。
现在,我们以\gls{maximum_likelihood_estimation}的角度重新审视\gls{linear_regression}。
我们现在希望模型能够得到条件概率$p(y \mid \Vx)$,而不只是得到一个单独的预测$\hat{y}$。
想象有一个无限大的训练集,我们可能会观测到几个训练\gls{example:chap5}有相同的输入$\Vx$但是不同的$y$。
现在学习算法的目标是拟合分布$p(y \mid \Vx)$到和$\Vx$相匹配的不同的$y$。
为了得到我们之前推导出的相同的\gls{linear_regression}算法,我们定义$p(y \mid \Vx) = \mathcal{N}(y; \hat{y}(\Vx; \Vw), \sigma^2)$。
函数$\hat{y}(\Vx; \Vw)$预测高斯的均值。
在这个例子中,我们假设方差是用户固定的某个常量$\sigma^2$。
这种函数形式$p(y \mid \Vx)$会使得\gls{maximum_likelihood_estimation}得出和之前相同的学习算法。
由于假设\gls{example:chap5}是独立同分布的,条件对数似然(\eqnref{eq:5.63})如下
\begin{align}
     & \sum_{i=1}^m \log p(y^{(i)} \mid \Vx^{(i)}; \Vtheta) \\
    =& -m \log\sigma - \frac{m}{2} \log(2\pi) - \sum_{i=1}^m \frac{ \norm{\hat{y}^{(i)} - y^{(i)} }^2 }{2\sigma^2},
\end{align}
其中$\hat{y}^{(i)}$是\gls{linear_regression}在第$i$个输入$\Vx^{(i)}$上的输出,$m$是训练\gls{example:chap5}的数目。
对比于\gls{mean_squared_error}的对数似然,
\begin{equation}
    \text{MSE}_{\text{train}} = \frac{1}{m} \sum_{i=1}^m \norm{\hat{y}^{(i)} - y^{(i)}}^2,
\end{equation}
我们立刻可以看出最大化关于$\Vw$的对数似然和最小化\gls{mean_squared_error}会得到相同的参数估计$\Vw$。
但是对于相同的最优$\Vw$,这两个准则有着不同的值。
这验证了\,\glssymbol{mean_squared_error}\,可以用于\gls{maximum_likelihood_estimation}。
正如我们将看到的,\gls{maximum_likelihood_estimation}有几个理想的性质。

% -- 130 --

\subsection{最大似然的性质}
\label{sec:properties_of_maximum_likelihood}
\gls{maximum_likelihood_estimation}最吸引人的地方在于,它被证明当\gls{example:chap5}数目$m\to\infty$时,就收敛率而言是最好的渐近估计。

在合适的条件下,\gls{maximum_likelihood_estimation}具有一致性(参考\secref{sec:consistency}),意味着训练\gls{example:chap5}数目趋向于无穷大时,参数的\gls{maximum_likelihood_estimation}会收敛到参数的真实值。
这些条件是:
\begin{itemize}
    \item 真实分布$p_{\text{data}}$必须在模型族$p_{\text{model}}(\cdot; \Vtheta)$ 中。
    否则,没有估计可以还原$p_{\text{data}}$。
    
    \item 真实分布$p_{\text{data}}$必须刚好对应一个$\Vtheta$值。
    否则,\gls{MLE}恢复出真实分布$p_{\text{data}}$后,也不能决定数据生成过程使用哪个$\Vtheta$。
\end{itemize}

除了\gls{maximum_likelihood_estimation},还有其他的归纳准则,其中许多共享一致估计的性质。
然而,一致估计的\firstgls{statistical_efficiency}可能区别很大。
某些一致估计可能会在固定数目的样本上获得一个较低的泛化误差,或者等价地,可能只需要较少的\gls{example:chap5}就能达到一个固定程度的泛化误差。

统计效率通常用于\firstgls{parametric_case}的研究中(例如\gls{linear_regression})。有参情况中我们的目标是估计参数值(假设有可能确定真实参数),而不是函数值。
一种度量我们和真实参数相差多少的方法是计算\gls{mean_squared_error}的期望,即计算$m$个从数据生成分布中出来的训练\gls{example:chap5}上的估计参数和真实参数之间差值的平方。
有参\gls{mean_squared_error}估计随着$m$的增加而减少,当$m$较大时,Cram\'er-Rao下界\citep{Rao-1945,Cramer-1946}表明不存在\gls{mean_squared_error}低于\gls{MLE}的一致估计。

因为这些原因(一致性和统计效率),最大似然通常是\gls{ML}中的首选估计。
当\gls{example:chap5}数目小到会发生过拟合时,\gls{regularization}策略如权重衰减可用于获得训练数据有限时方差较小的最大似然有偏版本。

% -- 131 --

\section{贝叶斯统计}
\label{sec:bayesian_statistics}
至此我们已经讨论了\firstgls{frequentist_statistics}方法和基于估计单一值$\Vtheta$的方法,然后基于该估计作所有的预测。
另一种方法是在做预测时会考虑所有可能的$\Vtheta$。
后者属于\firstgls{bayesian_statistics}的范畴。

正如\secref{sec:point_estimation}中讨论的,频率派的视角是真实参数$\Vtheta$是未知的定值,而点估计$\hat{\Vtheta}$是考虑\gls{dataset}上函数(可以看作是随机的)的随机变量。

贝叶斯统计的视角完全不同。
贝叶斯用概率反映知识状态的确定性程度。
\gls{dataset}能够被直接观测到,因此不是随机的。
另一方面,真实参数$\Vtheta$是未知或不确定的,因此可以表示成随机变量。

在观察到数据前,我们将$\Vtheta$的已知知识表示成\firstgls{prior_probability_distribution},$p(\Vtheta)$(有时简单地称为``先验'')。
一般而言,\gls{ML}实践者会选择一个相当宽泛的(即,高熵的)先验分布,反映在观测到任何数据前参数$\Vtheta$的高度不确定性。
例如,我们可能会假设先验$\Vtheta$在有限区间中均匀分布。
许多先验偏好于``更简单''的解(如小幅度的系数,或是接近常数的函数)。

现在假设我们有一组数据样本$\{x^{(1)},\dots,x^{(m)}\}$。
通过\gls{bayes_rule}结合数据似然$p(x^{(1)},\dots,x^{(m)} \mid \Vtheta)$和先验,我们可以恢复数据对我们关于$\Vtheta$信念的影响:
\begin{equation}
        p(\Vtheta \mid x^{(1)},\dots,x^{(m)}) = 
        \frac{p(x^{(1)},\dots,x^{(m)} \mid \Vtheta) p(\Vtheta)}
            {p(x^{(1)},\dots,x^{(m)})}
\end{equation}
在贝叶斯估计常用的情景下,先验开始是相对均匀的分布或高熵的高斯分布,观测数据通常会使后验的熵下降,并集中在参数的几个可能性很高的值。

相对于\gls{maximum_likelihood_estimation},贝叶斯估计有两个重要区别。
第一,不像最大似然方法预测时使用$\Vtheta$的点估计,贝叶斯方法使用$\Vtheta$的全分布。
例如,在观测到$m$个\gls{example:chap5}后,下一个数据\gls{example:chap5}~$x^{(m+1)}$的预测分布如下:
\begin{equation}
    p(x^{(m+1)} \mid x^{(1)},\dots,x^{(m)}) = 
    \int p(x^{(m+1)} \mid \Vtheta) p(\Vtheta \mid x^{(1)},\dots,x^{(m)})~d\Vtheta .
\end{equation}
这里,每个具有正概率密度的$\Vtheta$的值有助于下一个\gls{example:chap5}的预测,其中贡献由后验密度本身加权。
在观测到\gls{dataset} $\{ x^{(1)},\dots,x^{(m)}\}$之后,如果我们仍然非常不确定$\Vtheta$的值,那么这个不确定性会直接包含在我们所做的任何预测中。

% -- 132 --

在\secref{sec:estimators_bias_and_variance}中,我们已经探讨频率派方法解决给定点估计$\Vtheta$的不确定性的方法是评估方差,估计的方差评估了观测数据重新从观测数据中采样后,估计可能如何变化。
对于如何处理估计不确定性的这个问题,贝叶斯派的答案是积分,这往往会防止过拟合。
当然,积分仅仅是概率法则的应用,使贝叶斯方法容易验证,而频率派\gls{ML}基于相当特别的决定构建了一个估计,将\gls{dataset}里的所有信息归纳到一个单独的点估计。

贝叶斯方法和最大似然方法的第二个最大区别是由贝叶斯先验分布造成的。
先验能够影响概率质量密度朝参数空间中偏好先验的区域偏移。
实践中,先验通常表现为偏好更简单或更光滑的模型。
对贝叶斯方法的批判认为先验是人为主观判断影响预测的来源。

当训练数据很有限时,贝叶斯方法通常泛化得更好,但是当训练\gls{example:chap5}数目很大时,通常会有很大的计算\gls{cost}。


\paragraph{示例:贝叶斯\gls{linear_regression}}  我们使用贝叶斯估计方法学习\gls{linear_regression}的参数。
在\gls{linear_regression}中,我们学习从输入向量$\Vx\in\SetR^n$预测标量$y\in\SetR$的线性映射。
该预测由向量$\Vw \in \SetR^n$参数化:
\begin{equation}
    \hat{y} = \Vw^\Tsp \Vx .
\end{equation}
给定一组$m$个训练样本$(\MX^{(\text{train})}, \Vy^{(\text{train})})$,
我们可以表示整个训练集对$y$的预测:
\begin{equation}
    \hat{\Vy}^{(\text{train})} = \MX^{(\text{train})} \Vw .
\end{equation}
表示为$\Vy^{(\text{train})}$上的高斯条件分布,我们得到
\begin{align}
    p(\Vy^{(\text{train})} \mid \MX^{(\text{train})}, \Vw) &= 
    \mathcal{N}( \Vy^{(\text{train})}; \MX^{(\text{train})}\Vw, \MI ) \\
    & \propto \exp\left( 
        -\frac{1}{2}( \Vy^{(\text{train})} - \MX^{(\text{train})}\Vw )^\Tsp
        ( \Vy^{(\text{train})} - \MX^{(\text{train})}\Vw )
    \right),
\end{align}
其中,我们根据标准的\,\glssymbol{mean_squared_error}\,公式假设$y$上的高斯方差为$1$。
在下文中,为减少符号负担,我们将$(\MX^{(\text{train})}, \Vy^{(\text{train})})$简单表示为$(\MX, \Vy)$。

% -- 133 --

为确定模型参数向量$\Vw$的后验分布,我们首先需要指定一个先验分布。
先验应该反映我们对这些参数取值的信念。
虽然有时将我们的先验信念表示为模型的参数很难或很不自然,但在实践中我们通常假设一个相当广泛的分布来表示$\Vtheta$的高度不确定性。
实数值参数通常使用高斯作为先验分布:
\begin{equation}
    p(\Vw) = \mathcal{N}( \Vw; \Vmu_0, \VLambda_0 ) 
    \propto \exp\left( 
    -\frac{1}{2}( \Vw-\Vmu_0 )^\Tsp \VLambda_0^{-1} ( \Vw-\Vmu_0 )
    \right),
\end{equation}
其中,$\Vmu_0$和$\VLambda_0$分别是先验分布的均值向量和协方差矩阵。
\footnote{除非有理由使用协方差矩阵的特定结构,我们通常假设其为对角协方差矩阵
$\VLambda_0=\text{diag}(\Vlambda_0)$。
}

确定好先验后,我们现在可以继续确定模型参数的\textbf{后验}\emph{分布}。
\begin{align}
    p(\Vw \mid \MX, \Vy) &\propto p(\Vy \mid \MX, \Vw) p(\Vw) \\
    & \propto 
        \exp\left( 
            -\frac{1}{2} ( \Vy - \MX\Vw )^\Tsp( \Vy - \MX\Vw )
        \right)
        \exp\left(
    -\frac{1}{2} ( \Vw - \Vmu_0)^\Tsp \VLambda_0^{-1} ( \Vw - \Vmu_0)
        \right) \\
    & \propto \exp
    \left(
    \frac{1}{2}\left(
    -2\Vy^\Tsp\MX\Vw + \Vw^\Tsp\MX^\Tsp\MX\Vw + \Vw^\Tsp\VLambda_0^{-1}\Vw - 
    2\Vmu_0^\Tsp \VLambda_0^{-1}\Vw
    \right)
    \right).
\end{align}
现在我们定义$\VLambda_m = (\MX^\Tsp\MX + \VLambda_0^{-1})^{-1}$和$\Vmu_m = \VLambda_m ( \MX^\Tsp \Vy + \VLambda_0^{-1}\Vmu_0 )$。
使用这些新的变量,我们发现后验可改写为高斯分布:
\begin{align}
    p(\Vw \mid \MX, \Vy) &\propto \exp \left(
    -\frac{1}{2} (\Vw - \Vmu_m )^\Tsp \VLambda_m^{-1}  (\Vw - \Vmu_m ) 
    + \frac{1}{2} \Vmu_m^\Tsp \VLambda_m^{-1}  \Vmu_m 
    \right) \\
    &\propto \exp\left(
    -\frac{1}{2} (\Vw - \Vmu_m)^\Tsp \VLambda_m^{-1} (\Vw - \Vmu_m)
    \right).
\end{align}
所有不包括的参数向量$\Vw$的项都已经被删去了;它们意味着分布的积分必须归一这个事实。
\eqnref{eq:3.23}显示了如何标准化多元高斯分布。

% -- 134 --

检查此后验分布可以让我们获得\gls{bayesian_inference}效果的一些直觉。
大多数情况下,我们设置$\Vmu_0 = 0$。
如果我们设置$\VLambda_0 = \frac{1}{\alpha}\MI$,那么$\mu_m$对$\Vw$的估计就和频率派带权重衰减惩罚$\alpha\Vw^\Tsp\Vw$的\gls{linear_regression}的估计是一样的。
一个区别是若$\alpha$设为$0$则贝叶斯估计是未定义的——我们不能将贝叶斯学习过程初始化为一个无限宽的$\Vw$先验。
更重要的区别是贝叶斯估计会给出一个协方差矩阵,表示$\Vw$所有不同值的可能范围,而不仅是估计$\mu_m$。

\subsection{\glsentrytext{MAP}(\glssymbol{MAP})估计}
\label{sec:maximum_a_posteriori_map_estimation}
原则上,我们应该使用参数$\Vtheta$的完整贝叶斯后验分布进行预测,但单点估计常常也是需要的。
希望使用点估计的一个常见原因是,对于大多数有意义的模型而言,大多数涉及到贝叶斯后验的计算是非常棘手的,点估计提供了一个可行的近似解。
我们仍然可以让先验影响点估计的选择来利用贝叶斯方法的优点,而不是简单地回到\gls{MLE}。
一种能够做到这一点的合理方式是选择\firstall{MAP}点估计。
\glssymbol{MAP}\,估计选择后验概率最大的点(或在$\Vtheta$是连续值的更常见情况下,概率密度最大的点):
\begin{equation}
\label{eq:5.79}
    \Vtheta_{\text{MAP}} = \underset{\Vtheta}{\argmax} \, p(\Vtheta\mid\Vx)
    = \underset{\Vtheta}{\argmax} \, \log p(\Vx \mid \Vtheta) + \log p(\Vtheta) .
\end{equation}
我们可以认出上式右边的$\log p(\Vx \mid \Vtheta)$对应着标准的对数似然项,$\log p(\Vtheta)$对应着先验分布。

% -- 135 --

例如,考虑具有高斯先验权重$\Vw$的\gls{linear_regression}模型。
如果先验是$\mathcal{N}(\Vw;\mathbf{0},\frac{1}{\lambda}I^2)$,那么\eqnref{eq:5.79}的对数先验项正比于熟悉的权重衰减惩罚$\lambda \Vw^\Tsp\Vw$,加上一个不依赖于$\Vw$也不会影响学习过程的项。
因此,具有高斯先验权重的\glssymbol{MAP}~\gls{bayesian_inference}对应着权重衰减。

正如全\gls{bayesian_inference},\glssymbol{MAP}\,\gls{bayesian_inference}的优势是能够利用来自先验的信息,这些信息无法从训练数据中获得。
该附加信息有助于减少\gls{MAP}点估计的方差(相比于ML估计)。
然而,这个优点的代价是增加了偏差。

许多正规化估计方法,例如权重衰减\gls{regularization}的最大似然学习,可以被解释为\gls{bayesian_inference}的\,\glssymbol{MAP}\,近似。
这个适应于\gls{regularization}时加到\gls{target}函数的附加项对应着$\log p(\Vtheta)$。
并非所有的正则化惩罚都对应着~\glssymbol{MAP}~\gls{bayesian_inference}。
例如,有些\gls{regularizer}可能不是一个概率分布的对数。
还有些\gls{regularizer}依赖于数据,当然也不会是一个先验概率分布。

\glssymbol{MAP}\,\gls{bayesian_inference}提供了一个直观的方法来设计复杂但可解释的\gls{regularizer}。
例如,更复杂的惩罚项可以通过混合高斯分布作为先验得到,而不是一个单独的高斯分布\citep{Nowlan-nips92}。

\section{\glsentrytext{supervised_learning}算法}
\label{sec:supervised_learning_algorithms}
回顾\secref{sec:the_experience_e},粗略地说,\gls{supervised_learning}算法是给定一组输入$\Vx$和输出$\Vy$的训练集,学习如何关联输入和输出。
在许多情况下,输出$\Vy$很难自动收集,必须由人来提供``监督'',不过该术语仍然适用于训练集目标可以被自动收集的情况。

% -- 136 --

\subsection{概率监督学习}
\label{sec:probabilistic_supervised_learning}
本书的大部分\gls{supervised_learning}算法都是基于估计概率分布$p(y\mid\Vx)$的。
我们可以使用\gls{maximum_likelihood_estimation}找到对于有参分布族$p(y\mid\Vx;\Vtheta)$最好的参数向量$\Vtheta$。 

我们已经看到,\gls{linear_regression}对应于分布族
\begin{equation}
    p(y \mid \Vx; \Vtheta) = \mathcal{N}( y; \Vtheta^\Tsp \Vx, \MI).
\end{equation}
通过定义一族不同的概率分布,我们可以将\gls{linear_regression}扩展到分类情况中。
如果我们有两个类,类$0$和类$1$,那么我们只需要指定这两类之一的概率。
类$1$的概率决定了类$0$的概率,因为这两个值加起来必须等于$1$。

我们用于\gls{linear_regression}的实数正态分布是用均值参数化的。
我们提供这个均值的任何值都是有效的。
二元变量上的的分布稍微复杂些,因为它的均值必须始终在$0$和$1$之间。
解决这个问题的一种方法是使用~\gls{logistic_sigmoid}~函数将线性函数的输出压缩进区间$(0,1)$。
该值可以解释为概率:
\begin{equation}
    p(y = 1 \mid \Vx; \Vtheta) = \sigma(\Vtheta^\Tsp \Vx).
\end{equation}
这个方法被称为\firstgls{logistic_regression},这个名字有点奇怪,因为该模型用于分类而非回归。

\gls{linear_regression}中,我们能够通过求解正规方程以找到最佳权重。
相比而言,逻辑回归会更困难些。
其最佳权重没有闭解。
反之,我们必须最大化对数似然来搜索最优解。
我们可以通过\gls{GD}算法最小化负对数似然来搜索。

通过确定正确的输入和输出变量上的有参条件概率分布族,相同的策略基本上可以用于任何\gls{supervised_learning}问题。

\subsection{\glsentrytext{SVM}}
\label{sec:support_vector_machines}
\firstall{SVM}是\gls{supervised_learning}中最有影响力的方法之一\citep{Boser92,Cortes95}。
类似于逻辑回归,这个模型也是基于线性函数$\Vw^\Tsp \Vx + b$的。
不同于逻辑回归的是,支持向量机不输出概率,只输出类别。
当$\Vw^\Tsp\Vx + b$为正时,\gls{SVM}预测属于正类。
类似地,当$\Vw^\Tsp\Vx + b$为负时,\gls{SVM}预测属于负类。

% -- 137 --

\gls{SVM}的一个重要创新是\firstgls{kernel_trick}。
\gls{kernel_trick}观察到许多\gls{ML}算法都可以写成\gls{example:chap5}间\gls{dot_product}的形式。
例如,\gls{SVM}中的线性函数可以重写为
\begin{equation}
    \Vw^\Tsp \Vx + b = b + \sum_{i=1}^m \alpha_i \Vx^\Tsp \Vx^{(i)} ,
\end{equation}
其中,$\Vx^{(i)}$是训练\gls{example:chap5},$\Valpha$是系数向量。
学习算法重写为这种形式允许我们将$\Vx$替换为\gls{feature}函数$\phi(\Vx)$的输出,\gls{dot_product}替换为被称为\firstgls{kernel}的函数$k(\Vx, \Vx^{(i)}) = \phi(\Vx)\cdot \phi(\Vx^{(i)})$。
运算符$\cdot$表示类似于$\phi(\Vx)^\Tsp \phi(\Vx^{(i)})$的\gls{dot_product}。
对于某些\gls{feature}空间,我们可能不会书面地使用向量\gls{inner_product}。
在某些无限维空间中,我们需要使用其他类型的\gls{inner_product},如基于积分而非加和的\gls{inner_product}。
这种类型\gls{inner_product}的完整介绍超出了本书的范围。

使用核估计替换\gls{dot_product}之后,我们可以使用如下函数进行预测
\begin{equation}
    f(\Vx) = b + \sum_i \alpha_i k(\Vx, \Vx^{(i)}) .
\end{equation}
这个函数关于$\Vx$是非线性的,关于$\phi(\Vx)$是线性的。
$\Valpha$和$f(\Vx)$之间的关系也是线性的。
核函数完全等价于用$\phi(\Vx)$预处理所有的输入,然后在新的转换空间学习线性模型。

\gls{kernel_trick}十分强大有两个原因。
首先,它使我们能够使用保证有效收敛的凸优化技术来学习非线性模型(关于$\Vx$的函数)。
这是可能的,因为我们可以认为$\phi$是固定的,仅优化$\alpha$,即优化算法可以将决策函数视为不同空间中的线性函数。
其二,核函数$k$的实现方法通常有比直接构建$\phi(\Vx)$再算\gls{dot_product}高效很多。

在某些情况下,$\phi(\Vx)$甚至可以是无限维的,对于普通的显式方法而言,这将是无限的计算代价。
在很多情况下,即使$\phi(\Vx)$是难算的,$k(\Vx,\Vx')$却会是一个关于$\Vx$非线性的、易算的函数。
举个无限维空间易算的核的例子,我们构建一个作用于非负整数$x$上的\gls{feature}映射$\phi(x)$。
假设这个映射返回一个由开头$x$个$1$,随后是无限个$0$的向量。
我们可以写一个核函数$k(x,x^{(i)}) = \min(x, x^{(i)})$,完全等价于对应的无限维\gls{dot_product}。

% -- 138 --

最常用的核函数是\firstgls{gaussian_kernel},
\begin{equation}
    k(\Vu, \Vv) = \mathcal{N} (\Vu - \Vv; \mathbf{0}, \sigma^2 I) ,
\end{equation}
其中$\mathcal{N}(x; \Vmu, \VSigma)$是标准正态密度。
这个核也被称为\firstall{RBF}核,因为其值沿$\Vv$中从$\Vu$向外辐射的方向减小。
高斯核对应于无限维空间中的\gls{dot_product},但是该空间的推导没有整数上最小核的示例那么直观。

我们可以认为高斯核在执行一种\textbf{模板匹配}(template matching)。
训练\gls{label} $y$相关的训练\gls{example:chap5} $\Vx$变成了类别$y$的模版。
当测试点$\Vx'$到$\Vx$的欧几里得距离很小,对应的高斯核响应很大时,表明$\Vx'$和模版$\Vx$非常相似。
该模型进而会赋予相对应的训练\gls{label} $y$较大的权重。
总的来说,预测将会组合很多这种通过训练样本相似度加权的训练标签。

\gls{SVM}不是唯一可以使用\gls{kernel_trick}来增强的算法。
许多其他的线性模型也可以通过这种方式来增强。
使用\gls{kernel_trick}的算法类别被称为\firstgls{kernel_machines}或\firstgls{kernel_methods}\citep{Williams+Rasmussen-nips8,Scholkopf99}。    

核机器的一个主要缺点是计算决策函数的成本关于训练\gls{example:chap5}的数目是线性的。
因为第$i$个\gls{example:chap5}贡献$\alpha_i k(\Vx, \Vx^{(i)})$到决策函数。
\gls{SVM}能够通过学习主要包含零的向量$\Valpha$,以缓和这个缺点。
那么判断新\gls{example:chap5}的类别仅需要计算非零$\alpha_i$对应的训练\gls{example:chap5}的核函数。
这些训练\gls{example:chap5}被称为\firstgls{support_vectors}。

当\gls{dataset}很大时,\gls{kernel_machines}的计算量也会很大。
我们将会在\secref{sec:stochastic_gradient_descent_chap5}回顾这个想法。
带通用核的核机器致力于泛化得更好。
我们将在\secref{sec:challenges_motivating_deep_learning}解释原因。
现代\gls{DL}的设计旨在克服核机器的这些限制。
当前\gls{DL}的复兴始于~\cite{Hinton06-small}表明神经网络能够在MNIST基准数据上胜过RBF核的\gls{SVM}。

% -- 139 --

\subsection{其他简单的\glsentrytext{supervised_learning}算法}
\label{sec:other_simple_supervised_learning_algorithms}
我们已经简要介绍过另一个非概率\gls{supervised_learning}算法,最近邻回归。
更一般地,k-最近邻是一类可用于分类或回归的技术。
作为一个非参数学习算法,$k$-最近邻并不局限于固定数目的参数。
我们通常认为$k$-最近邻算法没有任何参数,而是使用训练数据的简单函数。
事实上,它甚至也没有一个真正的训练阶段或学习过程。
反之,在测试阶段我们希望在新的测试输入$\Vx$上产生$y$,我们需要在训练数据$\MX$上找到$\Vx$的$k$-最近邻。
然后我们返回训练集上对应的$y$值的平均值。
这几乎适用于任何类型可以确定$y$值平均值的\gls{supervised_learning}。
在分类情况中,我们可以关于~\gls{one_hot}~编码向量$\Vc$求平均,其中$c_y = 1$,其他的$i$值取$c_i=0$。
然后,我们可以解释这些~\gls{one_hot}~编码的均值为类别的概率分布。
作为一个非参数学习算法,$k$-近邻能达到非常高的容量。
例如,假设我们有一个用$0$-$1$误差度量性能的多分类任务。
在此设定中,当训练\gls{example:chap5}数目趋向于无穷大时,$1$-最近邻收敛到两倍贝叶斯误差。
超出贝叶斯误差的原因是它会随机从等距离的临近点中随机挑一个。
而存在无限的训练数据时,所有测试点$\Vx$周围距离为零的邻近点有无限多个。
如果我们使用所有这些临近点投票的决策方式,而不是随机挑选一个,那么该过程将会收敛到贝叶斯错误率。
$k$-最近邻的高容量使其在训练\gls{example:chap5}数目大时能够获取较高的精度。
然而,它的计算成本很高,另外在训练集较小时泛化能力很差。
$k$-最近邻的一个弱点是它不能学习出哪一个\gls{feature}比其他更具识别力。
例如,假设我们要处理一个的回归任务,其中$\Vx\in\SetR^{100}$是从各向同性的高斯分布中抽取的,但是只有一个变量$x_1$和结果相关。
进一步假设该\gls{feature}直接决定了输出,即在所有情况中$y=x_1$。
\gls{nearest_neighbor_regression}不能检测到这个简单模式。
大多数点$\Vx$的最近邻将取决于$x_2$到$x_{100}$的大多数\gls{feature},
而不是单独取决于\gls{feature} $x_1$。
因此,小训练集上的输出将会非常随机。

% -- 140 --

\firstgls{decision_tree}及其变种是另一类将输入空间分成不同的区域,每个区域有独立参数的算法\citep{Breiman84}。
如\figref{fig:chap5_decision_tree}所示,决策树的每个节点都与输入空间的一个区域相关联,并且内部节点继续将区域分成子节点下的子区域(通常使用坐标轴拆分区域)。
空间由此细分成不重叠的区域,叶节点和输入区域之间形成一一对应的关系。
每个叶结点将其输入区域的每个点映射到相同的输出。
决策树通常有特定的训练算法,超出了本书的范围。
如果允许学习任意大小的决策树,那么它可以被视作非参数算法。
然而实践中通常有大小限制,作为\gls{regularizer}将其转变成有参模型。
由于决策树通常使用坐标轴相关的拆分,并且每个子节点关联到常数输出,因此有时解决一些对于逻辑回归很简单的问题很费力。
例如,假设有一个二分类问题,当$x_2>x_1$时分为正类,则决策树的分界不是坐标轴对齐的。
因此,决策树将需要许多节点近似决策边界,坐标轴对齐使其算法步骤不断地来回穿梭于真正的决策函数。

\begin{figure}[!htb]
\ifOpenSource
\centerline{\includegraphics{figure.pdf}}
\else
\centerline{\includegraphics{Chapter5/figures/decision_tree}}
\fi
\caption{描述一个\gls{decision_tree}如何工作的示意图。\emph{(上)}树中每个节点都选择将输入样本送到左子节点($0$)或者右子节点($1$)。内部的节点用圆圈表示,叶节点用方块表示。每一个节点可以用一个二值的字符串识别并对应树中的位置,这个字符串是通过给起父亲节点的字符串添加一个位元来实现的($0$表示选择左或者上,$1$表示选择右或者下)。\emph{(下)}这个树将空间分为区域。这个二维平面说明\gls{decision_tree}可以分割$\SetR^2$。这个平面中画出了树的节点,每个内部点穿过分割线并用来给样本分类,叶节点画在样本所属区域的中心。结果是一个分块常数函数,每一个叶节点一个区域。每个叶需要至少一个训练样本来定义,所以\gls{decision_tree}不可能用来学习一个\gls{local_maxima}比训练样本数量还多的函数。}
\label{fig:chap5_decision_tree}
\end{figure}

正如我们已经看到的,最近邻预测和决策树都有很多的局限性。
尽管如此,在计算资源受限制时,它们都是很有用的学习算法。
通过思考复杂算法和$k$-最近邻或决策树之间的相似性和差异,我们可以建立对更复杂学习算法的直觉。

读者可以参考~\cite{MurphyBook2012,bishop-book2006,Hastie2001}或其他\gls{ML}教科书了解更多的传统\gls{supervised_learning}算法。

\section{\glsentrytext{unsupervised_learning}算法}
\label{sec:unsupervised_learning_algorithms}
回顾\secref{sec:the_experience_e},无监督算法只处理``\gls{feature}'',不操作监督信号。
监督和无监督算法之间的区别没有规范严格的定义,因为没有客观的判断来区分监督者提供的值是\gls{feature}还是\gls{target}。
通俗地说,\gls{unsupervised_learning}的大多数尝试是指从不需要人为注释的\gls{example:chap5}的分布中抽取信息。
该术语通常与密度估计相关,学习从分布中采样、学习从分布中\gls{denoise}、寻找数据分布的流形或是将数据中相关的\gls{example:chap5}聚类。

一个经典的\gls{unsupervised_learning}任务是找到数据的``最佳''表示。
``最佳''可以是不同的表示,但是一般来说,是指该表示在比本身表示的信息\emph{更简单}或更易访问而受到一些惩罚或限制的情况下,尽可能地保存关于$\Vx$更多的信息。 

% -- 142 --

有很多方式定义较简单的表示。最常见的三种包括低维表示、稀疏表示和独立表示。
低维表示尝试将$\Vx$中的信息尽可能压缩在一个较小的表示中。
稀疏表示将\gls{dataset}嵌入到输入项大多数为零的表示中\citep{Barlow89,Olshausen+Field-1996,Hinton+Ghahramani-97}。
稀疏表示通常用于需要增加表示维数的情况,使得大部分为零的表示不会丢失很多信息。
这会使得表示的整体结构倾向于将数据分布在表示空间的坐标轴上。
独立表示试图\emph{分开}数据分布中变化的来源,使得表示的维度是统计独立的。

当然这三个标准并非相互排斥的。
低维表示通常会产生比原始的高维数据具有较少或较弱依赖关系的元素。
这是因为减少表示大小的一种方式是找到并消除冗余。
识别并去除更多的冗余使得降维算法在丢失更少信息的同时显现更大的压缩。

表示的概念是\gls{DL}核心主题之一,因此也是本书的核心主题之一。
本节会介绍表示学习算法中的一些简单示例。
总的来说,这些示例算法会说明如何实施上面的三个标准。
剩余的大部分章节会介绍额外的表示学习算法,它们以不同方式处理这三个标准或是引入其他标准。

\subsection{\glsentrytext{PCA}}
\label{sec:principal_components_analysis_chap5}
在\secref{sec:example_principal_components_analysis_chap2}中,我们看到\,\glssymbol{PCA}\,算法提供了一种压缩数据的方式。
我们也可以将\,\glssymbol{PCA}\,视为学习数据表示的\gls{unsupervised_learning}算法。
这种表示基于上述简单表示的两个标准。
\,\glssymbol{PCA}\,学习一种比原始输入维数更低的表示。
它也学习了一种元素之间彼此没有线性相关的表示。
这是学习表示中元素统计独立标准的第一步。
要实现完全独立性,表示学习算法也必须去掉变量间的非线性关系。

% -- 143 --

如\figref{fig:chap5_pca}所示,\glssymbol{PCA}\,将输入$\Vx$投影表示成$\Vz$,学习数据的正交线性变换。
在\secref{sec:example_principal_components_analysis_chap2}中,我们看到了如何学习重建原始数据的最佳一维表示(就\gls{mean_squared_error}而言),这种表示其实对应着数据的第一个主要成分。
因此,我们可以用\,\glssymbol{PCA}\,作为保留数据尽可能多信息的降维方法(再次就最小重构误差平方而言)。
在下文中,我们将研究\,\glssymbol{PCA}\,表示如何使原始数据表示$\MX$去相关的.

\begin{figure}[!htb]
\ifOpenSource
\centerline{\includegraphics{figure.pdf}}
\else
\centerline{\includegraphics{Chapter5/figures/pca_color}}
\fi
\caption{\glssymbol{PCA}\,学习一种线性投影,使最大方差的方向和新空间的轴对齐。\emph{(左)}原始数据包含了$\Vx$的样本。在这个空间中,方差的方向与轴的方向并不是对齐的。\emph{(右)}变换过的数据$\Vz = \Vx^{\top}\MW$在轴$z_1$的方向上有最大的变化。第二大变化方差的方向沿着轴$z_2$。}
\label{fig:chap5_pca}
\end{figure}

假设有一个$m\times n$的\gls{design_matrix} $\MX$,数据的均值为零,$\SetE[\Vx\,] = 0$。
若非如此,通过预处理步骤使所有\gls{example:chap5}减去均值,数据可以很容易地中心化。

$\MX$对应的无偏样本协方差矩阵给定如下
\begin{equation}
    \text{Var}[\Vx\,] = \frac{1}{m-1} \MX^\Tsp \MX .
\end{equation}
\glssymbol{PCA}\,通过线性变换找到一个$\text{Var}[\Vz]$是对角矩阵的表示$\Vz=\MW^\Tsp\Vx$。

在\secref{sec:example_principal_components_analysis_chap2},我们已知\gls{design_matrix} $\MX$的主成分由$\MX^\Tsp\MX$的\gls{feature}向量给定。
从这个角度,我们有
\begin{equation}
    \MX^\Tsp\MX = \MW \VLambda \MW^\Tsp .
\end{equation}
本节中,我们会探索主成分的另一种推导。
主成分也可以通过奇异值分解(SVD)得到。
具体来说,它们是$\MX$的右奇异向量。
为了说明这点,假设$\MW$是奇异值分解$\MX = \MU\VSigma\MW^\Tsp$的右奇异向量。
以$\MW$作为特征向量基,我们可以得到原来的特征向量方程:
\begin{equation}
    \MX^\Tsp\MX = \left( \MU\VSigma \MW^\Tsp \right)^\Tsp \MU\VSigma \MW^\Tsp = 
    \MW \VSigma^2 \MW^\Tsp .
\end{equation}

% -- 144 --

\glssymbol{SVD}\,有助于说明\,\glssymbol{PCA}\,后的$\text{Var}[\Vz]$是对角的。
使用$\MX$的\,\glssymbol{SVD}\,分解,$\MX$的方差可以表示为
\begin{align}
    \text{Var}[\Vx] &= \frac{1}{m-1} \MX^\Tsp\MX \\
    &= \frac{1}{m-1} \left( \MU\VSigma \MW^\Tsp \right)^\Tsp \MU\VSigma \MW^\Tsp \\
    &= \frac{1}{m-1} \MW \VSigma^\Tsp \MU^\Tsp \MU\VSigma \MW^\Tsp \\
    &= \frac{1}{m-1} \MW \VSigma^2 \MW^\Tsp ,
\end{align}
其中,我们使用$\MU^\Tsp\MU = \MI$,因为根据奇异值的定义矩阵$\MU$是正交的。
这表明$\Vz$的协方差满足对角的要求:
\begin{align}
    \text{Var}[\Vz] &= \frac{1}{m-1} \MZ^\Tsp\MZ \\
    &= \frac{1}{m-1} \MW^\Tsp \MX^\Tsp \MX^\Tsp \MW \\
    &= \frac{1}{m-1} \MW^\Tsp\MW \VSigma^2 \MW^\Tsp\MW \\
    &= \frac{1}{m-1} \VSigma^2 ,
\end{align}
其中,再次使用\,\glssymbol{SVD}\,的定义有$\MW^\Tsp\MW = \MI$。

以上分析指明当我们通过线性变换$\MW$将数据$\Vx$投影到$\Vz$时,得到的数据表示的协方差矩阵是对角的(即$\VSigma^2$),立刻可得$\Vz$中的元素是彼此无关的。

\glssymbol{PCA}\,这种将数据变换为元素之间彼此不相关表示的能力是\,\glssymbol{PCA}\,的一个重要性质。
它是\emph{消除数据中未知变化因素}的简单表示示例。
在\,\glssymbol{PCA}\,中,这个消除是通过寻找输入空间的一个旋转(由$\MW$确定),
使得方差的主坐标和$\Vz$相关的新表示空间的基对齐。

% -- 145 --

虽然相关性是数据元素间依赖关系的一个重要范畴,但我们对于能够消除更复杂形式的\gls{feature}依赖的表示学习也很感兴趣。
对此,我们需要比简单线性变换更强的工具。

\subsection{$k$-均值聚类}
\label{sec:k_means_clustering}
另外一个简单的表示学习算法是$k$-均值聚类。
$k$-均值聚类算法将训练集分成$k$个靠近彼此的不同\gls{example:chap5}聚类。
因此我们可以认为该算法提供了$k$-维的~\gls{one_hot}~编码向量$\Vh$以表示输入$\Vx$。
当$\Vx$属于聚类$i$时,有$h_i=1$,$\Vh$的其他项为零。

$k$-均值聚类提供的~\gls{one_hot}~编码也是一种稀疏表示,因为每个输入的表示中大部分元素为零。
之后,我们会介绍能够学习更灵活的稀疏表示的一些其他算法(表示中每个输入$\Vx$不只一个非零项)。
\gls{one_hot}~编码是稀疏表示的一个极端示例,丢失了很多分布式表示的优点。
\gls{one_hot}~编码仍然有一些统计优点(自然地传达了相同聚类中的\gls{example:chap5}彼此相似的观点),
也具有计算上的优势,因为整个表示可以用一个单独的整数表示。

$k$-均值聚类初始化$k$个不同的中心点$\{\Vmu^{(1)},\dots,\Vmu^{(k)}\}$,然后迭代交换两个不同的步骤直到收敛。
步骤一,每个训练\gls{example:chap5}分配到最近的中心点$\Vmu^{(i)}$所代表的聚类$i$。
步骤二,每一个中心点$\Vmu^{(i)}$更新为聚类$i$中所有训练\gls{example:chap5} $\Vx^{(j)}$的均值。

关于聚类的一个问题是聚类问题本身是病态的。
这是说没有单一的标准去度量聚类的数据在真实世界中效果如何。
我们可以度量聚类的性质,例如类中元素到类中心点的欧几里得距离的均值。
这使我们可以判断从聚类分配中重建训练数据的效果如何。
然而我们不知道聚类的性质是否很好地对应到真实世界的性质。
此外,可能有许多不同的聚类都能很好地对应到现实世界的某些属性。
我们可能希望找到和一个\gls{feature}相关的聚类,但是得到了一个和任务无关的,同样是合理的不同聚类。
例如,假设我们在包含红色卡车图片、红色汽车图片、灰色卡车图片和灰色汽车图片的\gls{dataset}上运行两个聚类算法。
如果每个聚类算法聚两类,那么可能一个算法将汽车和卡车各聚一类,另一个根据红色和灰色各聚一类。
假设我们还运行了第三个聚类算法,用来决定类别的数目。
这有可能聚成了四类,红色卡车、红色汽车、灰色卡车和灰色汽车。
现在这个新的聚类至少抓住了属性的信息,但是丢失了相似性信息。
红色汽车和灰色汽车在不同的类中,正如红色汽车和灰色卡车也在不同的类中。
该聚类算法没有告诉我们灰色汽车和红色汽车的相似度比灰色卡车和红色汽车的相似度更高。
我们只知道它们是不同的。

% -- 146 --

这些问题说明了一些我们可能更偏好于分布式表示(相对于~\gls{one_hot}~表示而言)的原因。
分布式表示可以对每个车辆赋予两个属性——一个表示它颜色,一个表示它是汽车还是卡车。
目前仍然不清楚什么是最优的分布式表示(学习算法如何知道我们关心的两个属性是颜色和是否汽车或卡车,而不是制造商和车龄?),
但是多个属性减少了算法去猜我们关心哪一个属性的负担,允许我们通过比较很多属性而非测试一个单一属性来细粒度地度量相似性。

\section{\glsentrytext{SGD}}
\label{sec:stochastic_gradient_descent_chap5}
几乎所有的\gls{DL}算法都用到了一个非常重要的算法:\firstall{SGD}。
\gls{SGD}是\secref{sec:gradient_based_optimization}介绍的\gls{GD}算法的一个扩展。

\gls{ML}中反复出现的一个问题是好的泛化需要大的训练集,但大的训练集的计算代价也更大。

% -- 147 --

\gls{ML}算法中的\gls{cost_function}通常可以分解成每个\gls{example:chap5}的\gls{cost_function}的总和。
例如,训练数据的负条件对数似然可以写成
\begin{equation}
    J(\Vtheta) = \SetE_{\RVx,\RSy \sim \hat{p}_{\text{data}}}
    L(\Vx, y, \Vtheta) = 
    \frac{1}{m} \sum_{i=1}^m  L(\Vx^{(i)}, y^{(i)}, \Vtheta) ,
\end{equation}
其中$L$是每个\gls{example:chap5}的\gls{loss}$L(\Vx, y, \Vtheta) = -\log p(y\mid\Vx;\Vtheta)$。

对于这些相加的\gls{cost_function},\gls{GD}需要计算
\begin{equation}
    \nabla_{\Vtheta} J(\Vtheta)
    = \frac{1}{m} \sum_{i=1}^m  
    \nabla_{\Vtheta} L(\Vx^{(i)}, y^{(i)}, \Vtheta) .
\end{equation}
这个运算的计算代价是$O(m)$。
随着训练集规模增长为数十亿的\gls{example:chap5},计算一步梯度也会消耗相当长的时间。

\gls{SGD}的核心是,梯度是期望。
期望可使用小规模的样本近似估计。
具体而言,在算法的每一步,我们从训练集中均匀抽出一\firstgls{minibatch}\gls{example:chap5} $\SetB=\{\Vx^{(1)},\dots,\Vx^{(m')}\}$。
\gls{minibatch}的数目$m'$通常是一个相对较小的数,从一到几百。
重要的是,当训练集大小$m$增长时,$m'$通常是固定的。
我们可能在拟合几十亿的\gls{example:chap5}时,每次更新计算只用到几百个\gls{example:chap5}。

梯度的估计可以表示成
\begin{equation}
    \Vg = \frac{1}{m'} \nabla_{\Vtheta} \sum_{i=1}^{m'}
    L(\Vx^{(i)}, y^{(i)}, \Vtheta).
\end{equation}
使用来自\gls{minibatch} $\SetB$的\gls{example:chap5}。
然后,\gls{SGD}算法使用如下的\gls{GD}估计:
\begin{equation}
    \Vtheta \leftarrow \Vtheta - \epsilon \Vg,
\end{equation}
其中,$\epsilon$是\gls{learning_rate}。

\gls{GD}往往被认为很慢或不可靠。
以前,将\gls{GD}应用到非凸优化问题被认为很鲁莽或没有原则。
现在,我们知道\gls{GD}用于本书第二部分中的训练时效果不错。
优化算法不一定能保证在合理的时间内达到一个局部最小值,但它通常能及时地找到\gls{cost_function}一个很小的值,并且是有用的。

% -- 148 --

\gls{SGD}在\gls{DL}之外有很多重要的应用。
它是在大规模数据上训练大型线性模型的主要方法。
对于固定大小的模型,每一步\gls{SGD}更新的计算量不取决于训练集的大小$m$。
在实践中,当训练集大小增长时,我们通常会使用一个更大的模型,但这并非是必须的。
达到收敛所需的更新次数通常会随训练集规模增大而增加。
然而,当$m$趋向于无穷大时,该模型最终会在\gls{SGD}抽样完训练集上的所有\gls{example:chap5}之前收敛到可能的最优测试误差。
继续增加$m$不会延长达到模型可能的最优测试误差的时间。
从这点来看,我们可以认为用\,\glssymbol{SGD}\,训练模型的渐近代价是关于$m$的函数的$O(1)$级别。

在\gls{DL}兴起之前,学习非线性模型的主要方法是结合\gls{kernel_trick}的线性模型。
很多核学习算法需要构建一个$m\times m$的矩阵$G_{i,j}=k(\Vx^{(i)}, \Vx^{(j)})$。
构建这个矩阵的计算量是$O(m^2)$。
当\gls{dataset}是几十亿个\gls{example:chap5}时,这个计算量是不能接受的。
在学术界,\gls{DL}从2006年开始收到关注的原因是,在数以万计\gls{example:chap5}的中等规模\gls{dataset}上,\gls{DL}在新\gls{example:chap5}上比当时很多热门算法泛化得更好。
不久后,\gls{DL}在工业界受到了更多的关注,因为其提供了一种训练大\gls{dataset}上的非线性模型的可扩展方式。

我们将会在\chapref{chap:optimization_for_training_deep_models}继续探讨\gls{SGD}及其很多改进方法。

\section{构建\glsentrytext{ML}算法}
\label{sec:building_a_machine_learning_algorithm}
几乎所有的深度学习算法都可以被描述为一个相当简单的配方:特定的\gls{dataset}、\gls{cost_function}、优化过程和模型。

例如,\gls{linear_regression}算法由以下部分组成:$\MX$和$\Vy$构成的\gls{dataset},\gls{cost_function}
\begin{equation}
    J(\Vw, b) = -\SetE_{\RVx,\RSy\sim\hat{p}_{\text{data}}}
    \log p_{\text{model}} (y \mid \Vx) ,
\end{equation}
模型是$p_{\text{model}} (y \mid \Vx) = \mathcal{N}(y; \Vx^\Tsp \Vw + b, 1)$,
在大多数情况下,优化算法可以定义为求解\gls{cost_function}梯度为零的\gls{normal_equations}。

意识到我们可以替换独立于其他组件的大多数组件,因此我们能得到很多不同的算法。

% -- 149 --

通常\gls{cost_function}至少含有一项使学习过程进行统计估计的成分。
最常见的\gls{cost_function}是负对数似然,最小化\gls{cost_function}导致的\gls{maximum_likelihood_estimation}。

\gls{cost_function}也可能含有附加项,如\gls{regularizer}。
例如,我们可以将权重衰减加到\gls{linear_regression}的\gls{cost_function}中
\begin{equation}
    J(\Vw, b) = \lambda \norm{\Vw}_2^2 - \SetE_{\RVx,\RSy\sim \hat{p}_{\text{data}}}
    \log p_{\text{model}} (y \mid \Vx) .
\end{equation}
该优化仍然有闭解。

如果我们将该模型变成非线性的,那么大多数\gls{cost_function}不再能通过闭解优化。
这就要求我们选择一个迭代数值优化过程,如\gls{GD}等。

组合模型、\gls{cost}和优化算法来构建学习算法的配方同时适用于\gls{supervised_learning}和\gls{unsupervised_learning}。
\gls{linear_regression}示例说明了如何适用于\gls{supervised_learning}的。
\gls{unsupervised_learning}时,我们需要定义一个只包含$\MX$的\gls{dataset}、一个合适的无监督\gls{cost}和一个模型。
例如,通过指定如下\gls{loss_function}可以得到\,\glssymbol{PCA}\,的第一个主向量
\begin{equation}
    J(\Vw) = \SetE_{\RVx \sim \hat{p}_{\text{data}}} \norm{\Vx - r(\Vx; \Vw)}_2^2
\end{equation}
模型定义为重构函数$r(\Vx) = \Vw^\Tsp\Vx \,\Vw$,并且$\Vw$有范数为$1$的限制。

在某些情况下,由于计算原因,我们不能实际计算\gls{cost_function}。
在这种情况下,只要我们有近似其梯度的方法,那么我们仍然可以使用迭代数值优化近似最小化\gls{target}。

尽管有时候不显然,但大多数学习算法都用到了上述配方。
如果一个\gls{ML}算法看上去特别独特或是手动设计的,那么通常需要使用特殊的优化方法进行求解。
有些模型,如决策树或$k$-均值,需要特殊的优化,因为它们的\gls{cost_function}有平坦的区域,
使其不适合通过基于梯度的优化去最小化。
在我们认识到大部分\gls{ML}算法可以使用上述配方描述之后,我们可以将不同算法视为出于相同原因解决相关问题的一类方法,而不是一长串各个不同的算法。

% -- 150 --

\section{促使\glsentrytext{DL}发展的挑战}
\label{sec:challenges_motivating_deep_learning}
本章描述的简单\gls{ML}算法在很多不同的重要问题上效果都良好。
但是它们不能成功解决人工智能中的核心问题,如语音识别或者对象识别。

\gls{DL}发展动机的一部分原因是传统学习算法在这类人工智能问题上泛化能力不足。

本节介绍为何处理高维数据时在新\gls{example:chap5}上泛化特别困难,以及为何在传统\gls{ML}中实现泛化的机制不适合学习高维空间中复杂的函数。
这些空间经常涉及巨大的计算代价。
\gls{DL}旨在克服这些以及其他一些难题。

\subsection{\glsentrytext{curse_of_dimensionality}}
\label{sec:the_curse_of_dimensionality}
当数据的维数很高时,很多\gls{ML}问题变得相当困难。
这种现象被称为\firstgls{curse_of_dimensionality}。
特别值得注意的是,一组变量不同的可能配置数量会随着变量数目的增加而指数级增长。

维数灾难发生在计算机科学的许多地方,在\gls{ML}中尤其如此。

由\gls{curse_of_dimensionality}带来的一个挑战是统计挑战。
如\figref{fig:chap5_curse}所示,统计挑战产生于$\Vx$的可能配置数目远大于训练\gls{example:chap5}的数目。
为了充分理解这个问题,我们假设输入空间如图所示被分成网格。
低维时我们可以用由数据占据的少量网格去描述这个空间。
泛化到新数据点时,通过检测和新输入在相同网格中的训练\gls{example:chap5},我们可以判断如何处理新数据点。
例如,如果要估计某点$\Vx$处的概率密度,我们可以返回$\Vx$处单位体积内训练\gls{example:chap5}的数目除以训练\gls{example:chap5}的总数。
如果我们希望对一个\gls{example:chap5}进行分类,我们可以返回相同网格中训练\gls{example:chap5}最多的类别。
如果我们是做回归分析,我们可以平均该网格中\gls{example:chap5}对应的的\gls{target}值。
但是,如果该网格中没有\gls{example:chap5},该怎么办呢?  
因为在高维空间中参数配置数目远大于\gls{example:chap5}数目,大部分配置没有相关的\gls{example:chap5}。 %?? 配置
我们如何能在这些新配置中找到一些有意义的东西呢?
许多传统\gls{ML}算法只是简单地假设在一个新点的输出应大致和最接近的训练点的输出相同。



\begin{figure}[!htb]
\ifOpenSource
\centerline{\includegraphics{figure.pdf}}
\else
\begin{tabular}{ccc}
    \includegraphics[width=0.3\textwidth]{Chapter5/figures/curse_1d_color} & \includegraphics[width=0.3\textwidth]{Chapter5/figures/curse_2d_color} & \includegraphics[width=0.3\textwidth]{Chapter5/figures/curse_3d_color}
\end{tabular}
\fi
\caption{当数据的相关维度增大时(从左向右),我们感兴趣的配置数目会随之指数级增长。\emph{(左)}在这个一维的例子中,我们用一个变量来区分所感兴趣的仅仅$10$个区域。当每个区域都有足够的样本数时(图中每个样本对应了一个细胞),学习算法能够轻易地\gls{generalization}得很好。\gls{generalization}的一个直接方法是估计目标函数在每个区域的值(可能是在相邻区域之间插值)。\emph{(中)}在二维情况下,对每个变量区分$10$个不同的值更加困难。我们需要追踪$10\times10=100$个区域,至少需要很多样本来覆盖所有的区域。\emph{(右)}三维情况下,区域数量增加到了$10^3=1000$,至少需要那么多的样本。对于需要区分的$d$维以及$v$个值来说,我们需要$O(v^d)$个区域和样本。这就是\gls{curse_of_dimensionality}的一个示例。感谢由Nicolas Chapados提供的图片。}
\label{fig:chap5_curse}
\end{figure}

% -- 151 --

\subsection{局部不变性和平滑\glsentrytext{regularization}}
\label{sec:local_constancy_and_smoothness_regularization}
为了更好地泛化,\gls{ML}算法需要由先验信念引导应该学习什么类型的函数。
此前,我们已经看到过由模型参数的概率分布形成的先验。
通俗地讲,我们也可以说先验信念直接影响\emph{函数}本身,而仅仅通过它们对函数的影响来间接改变参数。 
此外,我们还能通俗地说,先验信念还间接地体现在选择一些偏好某类函数的算法,尽管这些偏好并没有通过我们对不同函数置信程度的概率分布表现出来(也许根本没法表现)。

% -- 152 --

其中最广泛使用的隐式``先验''是\firstgls{smoothness_prior},或\firstgls{local_constancy_prior}。
这个先验表明我们学习的函数不应在小区域内发生很大的变化。

许多简单算法完全依赖于此先验达到良好的泛化,其结果是不能推广去解决\gls{AI}级别任务中的统计挑战。
本书中,我们将介绍\gls{DL}如何引入额外的(显式或隐式的)先验去降低复杂任务中的泛化误差。
这里,我们解释为什么仅依靠平滑先验不足以应对这类任务。

有许多不同的方法来显式或隐式地表示学习函数应该具有光滑或局部不变的先验。
所有这些不同的方法都旨在鼓励学习过程能够学习出函数$f^*$对于大多数设置$\Vx$和小变动$\epsilon$,都满足条件
\begin{equation}
    f^*(\Vx) \approx f^*(\Vx + \epsilon).
\end{equation}
换言之,如果我们知道对应输入$\Vx$的答案(例如,$\Vx$是个有\gls{label}的训练\gls{example:chap5}),那么该答案对于$\Vx$的邻域应该也适用。
如果在有些邻域中我们有几个好答案,那么我们可以组合它们(通过某种形式的平均或插值法)以产生一个尽可能和大多数输入一致的答案。

局部不变方法的一个极端例子是$k$-最近邻系列的学习算法。
当一个区域里的所有点$\Vx$在训练集中的$k$个最近邻是一样的,那么对这些点的预测也是一样的。
当$k=1$时,不同区域的数目不会比训练\gls{example:chap5}还多。

虽然$k$-最近邻算法复制了附近训练\gls{example:chap5}的输出,大部分核机器也是在和附近训练\gls{example:chap5}相关的训练集输出上插值。
一类重要的核函数是\firstgls{local_kernel},其核函数$k(\Vu,\Vv)$在$\Vu=\Vv$时很大,
当$\Vu$和$\Vv$距离拉大时而减小。
局部核可以看作是执行模版匹配的相似函数,用于度量测试\gls{example:chap5} $\Vx$和每个训练\gls{example:chap5} $\Vx^{(i)}$有多么相似。
近年来深度学习的很多推动力源自研究局部模版匹配的局限性,以及\gls{DL}如何克服这些局限性\citep{NIPS2005_424}。

决策树也有平滑学习的局限性,因为它将输入空间分成和叶节点一样多的区间,并在每个区间使用单独的参数(或者有些决策树的拓展有多个参数)。
如果\gls{target}函数需要至少拥有$n$个叶节点的树才能精确表示,那么至少需要$n$个训练\gls{example:chap5}去拟合。
需要几倍于$n$的\gls{example:chap5}去达到预测输出上的某种统计置信度。

% -- 153 --

总的来说,区分输入空间中$O(k)$个区间,所有的这些方法需要$O(k)$个\gls{example:chap5}。
通常会有$O(k)$个参数,$O(1)$参数对应于$O(k)$区间之一。
最近邻算法中,每个训练\gls{example:chap5}至多用于定义一个区间,如\figref{fig:chap5_non_distributed}所示。


\begin{figure}[!htb]
\ifOpenSource
\centerline{\includegraphics{figure.pdf}}
\else
\centerline{\includegraphics{Chapter5/figures/non_distributed}}
\fi
\caption{\gls{nearest_neighbor}算法如何划分输入空间的示例。每个区域内的一个样本(这里用圆圈表示)定义了区域边界(这里用线表示)。每个样本相关的$y$值定义了对应区域内所有数据点的输出。由\gls{nearest_neighbor}定义并且匹配几何模式的区域被称为Voronoi图。这些连续区域的数量不会比训练样本的数量增加得更快。尽管此图具体说明了\gls{nearest_neighbor}算法的效果,其他的单纯依赖局部光滑先验的机器学习算法也表现出了类似的\gls{generalization}能力:每个训练样本仅仅能告诉学习者如何在其周围的相邻区域\gls{generalization}。}
\label{fig:chap5_non_distributed}
\end{figure}


有没有什么方法能表示区间数目比训练\gls{example:chap5}数目还多的复杂函数?
显然,只是假设函数的平滑性不能做到这点。
例如,想象\gls{target}函数作用在西洋跳棋盘上。
棋盘包含许多变化,但只有一个简单的结构。
想象一下,如果训练\gls{example:chap5}数目远小于棋盘上的黑白方块数目,那么会发生什么。
基于局部泛化和平滑性或局部不变性先验,如果新点和某个训练\gls{example:chap5}位于相同的棋盘方块中,那么我们能够保证正确地预测新点的颜色。
但如果新点所在的方块没有训练\gls{example:chap5},\gls{learner}不一定能举一反三。
如果仅依靠这个先验,一个\gls{example:chap5}只能告诉我们它所在的方块的颜色。
获得整个棋盘颜色的唯一方法是其上的每个方块至少要有一个\gls{example:chap5}。

% -- 154 --

只要在要学习的真实函数的峰值和谷值处有足够多的\gls{example:chap5},那么平滑性假设和相关的无参数学习算法的效果都非常好。
当要学习的函数足够平滑,并且只在少数几维变化,这样做一般没问题。
在高维空间中,即使是非常平滑的函数,也会在不同维度上有不同的变化方式。
如果函数在不同的区间中表现不一样,那么就非常难用一组训练\gls{example:chap5}去刻画函数。
如果函数是复杂的(我们想区分多于训练\gls{example:chap5}数目的大量区间),有希望很好地泛化么?

这些问题,即是否可以有效地表示复杂的函数以及所估计的函数是否可以很好地泛化到新的输入,答案是有。
关键观点是,只要我们通过额外假设生成数据的分布来建立区域间的依赖关系,那么$O(k)$个\gls{example:chap5}足以描述多如$O(2^k)$的大量区间。
通过这种方式,我们确实能做到非局部的泛化\citep{Bengio+Monperrus-2005,NIPS2005_539}。
为了利用这些优势,许多不同的\gls{DL}算法都提出了一些适用于多种\,\glssymbol{AI}\,任务的隐式或显式的假设。


一些其他的\gls{ML}方法往往会提出更强的,针对特定问题的假设。
例如,假设\gls{target}函数是周期性的,我们很容易解决棋盘问题。
通常,神经网络不会包含这些很强的(针对特定任务的)假设,因此神经网络可以泛化到更广泛的各种结构中。
人工智能任务的结构非常复杂,很难限制到简单的、人工手动指定的性质,如周期性,因此我们希望学习算法具有更通用的假设。
\gls{DL}的核心思想是假设数据由\emph{因素或特征组合}产生,这些因素或特征可能来自一个层次结构的多个层级。
许多其他类似的通用假设进一步提高了\gls{DL}算法。
这些很温和的假设允许了\gls{example:chap5}数目和可区分区间数目之间的指数增益。
这类指数增益将在\secref{sec:universal_approximation_properties_and_depth}、\secref{sec:distributed_representation}和\secref{sec:exponential_gains_from_depth}中更详尽地介绍。
深度的\gls{distributed_representation}带来的指数增益有效地解决了\gls{curse_of_dimensionality}带来的挑战。

% -- 155 --

\subsection{\glsentrytext{manifold_learning}}
\label{sec:manifold_learning}
\gls{manifold}是一个\gls{ML}中很多想法内在的重要概念。

\firstgls{manifold}指连接在一起的区域。
数学上,它是指一组点,且每个点都有其邻域。
给定一个任意的点,其流形局部看起来像是欧几里得空间。
日常生活中,我们将地球视为二维平面,但实际上它是三维空间中的球状\gls{manifold}。

每个点周围邻域的定义暗示着存在变换能够从一个位置移动到其邻域位置。
例如在地球表面这个流形中,我们可以朝东南西北走。

尽管术语``\gls{manifold}''有正式的数学定义,但是\gls{ML}倾向于更松散地定义一组点,只需要考虑少数嵌入在高维空间中的自由度或维数就能很好地近似。
每一维都对应着局部的变化方向。
如\figref{fig:chap5_one_dim_manifold_and_data}所示,训练数据位于二维空间中的一维流形中。
在\gls{ML}中,我们允许流形的维数从一个点到另一个点有所变化。
这经常发生于流形和自身相交的情况中。
例如,数字``8''形状的流形在大多数位置只有一维,但在中心的相交处有两维。

\begin{figure}[!htb]
\ifOpenSource
\centerline{\includegraphics{figure.pdf}}
\else
\centerline{\includegraphics{Chapter5/figures/one_dim_manifold_and_data_color}}
\fi
\caption{从一个二维空间的分布中抽取的数据样本,这些样本实际上聚集在一维\gls{manifold}附近,像一个缠绕的带子。实线代表学习器应该推断的隐式\gls{manifold}。}
\label{fig:chap5_one_dim_manifold_and_data}
\end{figure}

如果我们希望\gls{ML}算法学习整个$\SetR^n$上有趣变化的函数,那么很多\gls{ML}问题看上去都是无望的。
\firstgls{manifold_learning}算法通过一个假设来克服这个障碍,该假设认为$\SetR^n$中大部分区域都是无效的输入,有意义的输入只分布在包含少量数据点的子集构成的一组流形中,而学习函数的输出中,有意义的变化都沿着流形的方向或仅发生在我们切换到另一流形时。
流形学习最初用于连续数值和无监督学习的环境,尽管这个概率集中的想法也能够泛化到离散数据和\gls{supervised_learning}的设定下:关键假设仍然是概率质量高度集中。


% -- 156 --

\begin{figure}[!htb]
\ifOpenSource
\centerline{\includegraphics{figure.pdf}}
\else
\centerline{\includegraphics[width=0.7\textwidth]{Chapter5/figures/noise}}
\fi
\caption{随机地均匀抽取图像(根据均匀分布随机地选择每一个像素)会得到噪声图像。
尽管在人工智能应用中以这种方式生成一个脸或者其他物体的图像是非零概率的,但是实际上我们从来没有观察到这种现象。
这也意味着人工智能应用中遇到的图像在所有图像空间中的占比可以是忽略不计的。}
\label{fig:chap5_noise}
\end{figure}


数据位于低维流形的假设并不总是对的或者有用的。
我们认为在人工智能的一些场景中,如涉及到处理图像、声音或者文本时,流形假设至少是近似对的。
这个假设的支持证据包含两类观察结果。

第一个支持\firstgls{manifold_hypothesis}的观察是现实生活中的图像、文本、声音的概率分布都是高度集中的。
均匀的\gls{noise}从来不会与这类领域的结构化输入类似。
\figref{fig:chap5_noise}显示均匀采样的点看上去像是没有信号时模拟电视上的静态模式。
同样,如果我们均匀地随机抽取字母来生成文件,能有多大的概率得到一个有意义的英语文档?
几乎是零。
因为大部分字母长序列不对应着自然语言序列:
自然语言序列的分布只占了字母序列的总空间里非常小的一部分。


当然,集中的概率分布不足以说明数据位于一个相当小的流形中。
我们还必须确保,我们遇到的\gls{example:chap5}和其他\gls{example:chap5}相互连接,每个\gls{example:chap5}被其他高度相似的\gls{example:chap5}包围,而这些高度相似的样本可以通过变换来遍历该流形得到。
支持流形假设的第二个论点是,我们至少能够非正式地想象这些邻域和变换。
在图像中,我们当然会认为有很多可能的变换仍然允许我们描绘出图片空间的流形:
我们可以逐渐变暗或变亮光泽、逐步移动或旋转图中对象、逐渐改变对象表面的颜色等等。
在大多数应用中很有可能会涉及到多个流形。
例如,人脸图像的\gls{manifold}不太可能连接到猫脸图像的\gls{manifold}。

% -- 157 --

这些支持流形假设的思维实验传递了一些支持它的直观理由。
更严格的实验\citep{Cayton-2005,Narayanan+Mitter-NIPS2010,Scholkopf98-book,Roweis2000-lle-small,Tenenbaum2000-isomap,Brand2003,Belkin+Niyogi-nips2003,Donoho+Carrie-03,Weinberger04a}在\gls{AI}中备受关注的一大类\gls{dataset}上支持了这个假设。

当数据位于低维流形中时,使用流形中的坐标而非$\SetR^n$中的坐标表示\gls{ML}数据更为自然。
日常生活中,我们可以认为道路是嵌入在三维空间的一维流形。
我们用一维道路中的地址号码确定地址,而非三维空间中的坐标。
提取这些流形中的坐标是非常具有挑战性的,但是很有希望改进许多\gls{ML}算法。
这个一般性原则能够用在很多情况中。
\figref{fig:chap5_QMUL-facedataset}展示了包含人脸的\gls{dataset}的流形结构。
在本书的最后,我们会介绍一些学习这样的流形结构的必备方法。
在\figref{fig:chap20_kingma-vae-2d-faces-manifold}中,我们将看到\gls{ML}算法如何成功完成这个\gls{target}。

\begin{figure}[!htb]
\ifOpenSource
\centerline{\includegraphics{figure.pdf}}
\else
\centerline{\includegraphics[width=0.8\textwidth]{Chapter5/figures/QMUL-facedataset}}
\fi
\caption{QMUL Multiview Face数据集中的训练样本\citep{Gong-et-al-2000},其中的物体是移动的从而覆盖对应两个旋转角度的二维\gls{manifold}。
我们希望学习算法能够发现并且理出这些\gls{manifold}坐标。
\figref{fig:chap20_kingma-vae-2d-faces-manifold}提供了这样一个示例。}
\label{fig:chap5_QMUL-facedataset}
\end{figure}

第一部分介绍了数学和\gls{ML}中的基本概念,这将用于本书其他章节中。
至此,我们已经做好了研究\gls{DL}的准备。

% -- 159 --





\input{deep_networks_modern_practices.tex}
\input{deep_learning_research.tex}


\backmatter
\appendix

\small{
\bibliography{dlbook_cn}
\bibliographystyle{natbib}
\clearpage
}

\printglossary[title=术语]
\end{document}
